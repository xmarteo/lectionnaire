La solennité de la Passion du Seigneur,
	désirée par nous, et désirable pour le monde entier, est venue:
	et elle ne nous permet point de garder le silence
	parmi les transports des joies spirituelles qu’elle répand dans nos âmes.
Car bien qu’il soit difficile
	de parler très souvent d’une manière digne et juste sur le même sujet,
	un Évêque n’est cependant pas libre
		de refuser au peuple fidèle le discours qu’il lui doit,
	sur ce grand mystère de la divine miséricorde.
La matière, par cela même qu’elle est ineffable,
		fournit abondamment de quoi parler,
	et les paroles ne peuvent faire défaut,
	puisque jamais ce qu’on dira sur ce sujet ne sera suffisant.
Que la faiblesse humaine se reconnaisse vaincue par la gloire de Dieu
	et toujours incapable d’expliquer les œuvres de sa miséricorde,
	que notre intelligence fasse effort, que notre esprit reste en suspens,
	que l’expression nous manque;
	il nous est bon de voir combien les idées les plus hautes
		que nous puissions avoir de la majesté du Seigneur,
	sont encore peu de chose auprès de la réalité.
