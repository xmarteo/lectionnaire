Vous savez quelles étaient ces assemblées de méchants:
	c’étaient celles des Juifs;
	vous connaissez l’iniquité de cette multitude de pécheurs:
	cette iniquité,
		c’était leur volonté de faire mourir notre Seigneur Jésus-Christ.
«J’ai opéré sous vos yeux un si grand nombre de bonnes œuvres:
	pour laquelle voulez-vous me mettre à mort?»
Il a soulagé tous leurs infirmes, guéri tous leurs infirmes,
	leur a prêché le Royaume des Cieux.
Il n’a pas tu leurs vices, voulant qu’ils détestent ces vices
	et non pas le médecin qui les guérissait.
Sans gratitude pour toutes ces guérisons, comme de grands fiévreux en délire,
	en rage folle contre le médecin qui était venu les guérir,
	ils ont formé le dessein de le perdre,
	comme s’ils avalent voulu éprouver ainsi
		s’il était vraiment un homme qui puisse mourir,
	ou quelque chose de supérieur aux hommes, qui ne se laisse pas mourir.
Leur parole, nous la reconnaissons dans la Sagesse de Salomon:
	«À une mort ignominieuse, disent-ils, condamnons-le.
	Interrogeons-le, on verra ce que seront ses paroles.
	Car s’il est vraiment le Fils de Dieu, que Dieu le délivre.»
