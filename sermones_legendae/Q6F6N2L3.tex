«Il ont aiguisé leur langue comme une épée.»
Que les Juifs ne disent pas: Nous n’avons pas tué le Christ.
C’est qu’en effet ils l’ont livré au jugement de Pilate,
	pour paraître comme innocents de sa mort.
Car Pilate leur ayant dit: «Tuez-le vous-mêmes»,
	ils répondirent: «Il ne nous est pas permis de mettre quelqu’un à mort.»
Ils voulaient faire retomber l’iniquité de leur crime sur un juge humain;
	mais trompaient-ils le Dieu juge?
Ce que Pilate a fait l’a rendu leur complice,
	dans la mesure de son intervention personnelle;
	mais en comparaison d’eux, il est beaucoup plus innocent.
Il insista autant qu’il put pour le délivrer de leurs mains;
	c’est pour cela qu’il le leur présenta déjà flagellé.
Ce n’est pas en persécuteur qu’il flagella le Seigneur,
	mais pour donner quelque satisfaction à leur fureur,
	pensant qu’ils s’adouciraient et cesseraient de vouloir le tuer,
	s’ils le voyaient flagellé.
C’est ce qu’il fit.
Mais comme ils persévéraient dans leur dessein,
	vous savez qu’il s’est lavé les mains
	et qu’il a dit être pur de la mort de Jésus,
	parce que, de lui-même, il ne l’aurait pas fait.
Il l’a fait cependant.
	Mais s’il est coupable pour l’avoir fait, bien que de mauvais gré,
	ceux-là seront-ils innocents, qui l’ont forcé de le faire? Nullement.
C’est bien lui qui a prononcé la sentence et qui a ordonné de le crucifier
	et pour ainsi dire l’a tué lui-même.
Mais vous aussi, Juifs, vous l’avez tué.
Comment l’avez-vous tué? Par le glaive de la langue;
	vous avez aiguisé vos langues.
Et quand l’avez-vous frappé,
	si ce n’est quand vous avez crié: «Crucifie-le, crucifie-le!»
