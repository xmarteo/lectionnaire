Depuis la bienheureuse et glorieuse Résurrection de notre Seigneur Jésus-Christ,
	dans laquelle le vrai temple de Dieu, détruit par l’impiété juive,
	a été relevé en trois jours par la divine puissance,
	voici aujourd’hui, mes bien-aimés, le quarantième jour.
Le nombre de ces saints jours
	s’est accompli en vertu d’une très sainte disposition,
	et a été employé utilement à notre instruction.
L’intention du Seigneur, en prolongeant pendant ce temps sa présence corporelle,
	était de fortifier, par des preuves indubitables, la foi en sa résurrection.
Car la mort du Christ avait beaucoup troublé les cœurs des disciples;
	son supplice sur la croix, son dernier soupir,
	l’ensevelissement de son cadavre
	avaient accablé leurs esprits d’une telle tristesse,
	qu’une certaine torpeur de défiance s’y était glissée.
