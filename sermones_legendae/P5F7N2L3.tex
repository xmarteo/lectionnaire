C’est alors que la foi mieux instruite
		apprit à s’élever par une ascension spirituelle
	jusqu’au Fils de Dieu égal au Père,
	et à ne plus avoir besoin de toucher dans le Christ
		cette substance corporelle en laquelle il est moins grand que le Père.
En effet, la substance de ce corps glorifié demeurant la même,
	c’est là que la foi des croyants a été appelée,
	là où le Fils unique égal à son Père peut être atteint,
	non plus par une main de chair, mais par l’intelligence spirituelle.
C’est pourquoi le Seigneur, après sa Résurrection,
	dit à Marie-Madeleine, représentant l’Église,
	qui se hâtait de s’approcher pour le toucher:
	«Ne me touche pas; car je ne suis pas monté vers mon Père»,
	c’est-à-dire: Je ne veux plus que tu cherches ma présence corporelle,
	je ne veux plus me faire reconnaître par les sens charnels.
Par ces délais, je t’appelle plus haut,
	je te prépare des dons plus grands.
Lorsque je serai monté vers mon Père,
	c’est alors que tu me toucheras d’une manière plus parfaite et plus vraie,
	devant saisir ce que tu ne toucheras pas,
	devant croire ce que tu ne verras pas.
