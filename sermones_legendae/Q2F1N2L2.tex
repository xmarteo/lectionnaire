Ce qui est signifié, voilà ce qui précisément est affirmé,
	et si l’on pense qu’il y a ici des mensonges,
	c’est qu’on ne comprend pas dans les paroles
		ce qu’elles signifient en vérité
	et qu’on croit que ce sont des faussetés.
Pour que cela devienne plus clair par des exemples,
	examinez précisément ce qu’a fait Jacob.
Sans doute, il s’est couvert les membres de peaux de chevreaux.
Si nous en cherchons la cause immédiate, nous penserons qu’il a menti,
	puisqu’il a fait cela afin d’être pris pour celui qu’il n’était pas.
Mais si l’on rapporte ce fait à ce que sa réalisation devait signifier,
	les peaux de bouc signifiaient les péchés
	et celui qui s’en est couvert signifiait celui qui a porté
		non pas ses propres péchés,
	mais ceux des autres.
