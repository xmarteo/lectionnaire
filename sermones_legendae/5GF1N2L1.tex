C’est un homme vraiment grand qu’Abraham,
	glorieux par la parure de beaucoup de vertus,
	et que la philosophie n’a jamais pu égaler, même par ses aspirations;
	car en somme l’idéal de ses rêves
		est au-dessous de ce que cet homme a fait,
	et la foi en la simple vérité
		est plus grande que l’ambitieux mensonge de l’éloquence.
Considérons donc ce que fut la dévotion dans cet homme.
Car cette vertu tient le premier rang, en tant que fondement des autres,
	et c’est à bon droit que Dieu l’a exigée de lui tout d’abord en disant:
	Sors de ton pays et de ta parenté et de la maison de ton père.
Il eût suffi de dire: De ton pays.
	Car c’était sortir en même temps de la parenté,
	sortir de la maison paternelle.
