Pour nous, frères bien-aimés,
	qui sommes philosophes, non dans nos paroles, mais dans nos actions;
	qui professons la sagesse, non par notre vêtement, mais dans sa réalité;
	qui connaissons mieux la pratique des vertus que leur ostentation;
	qui ne disons pas de grandes choses,
	mais qui vivons comme des serviteurs et adorateurs de Dieu;
	montrons par une soumission spirituelle cette patience
	que de divins enseignements nous ont apprise.
Car cette vertu nous est commune avec Dieu.
	C’est de lui qu’elle vient, qu’elle tire son éclat et sa gloire.
	L’origine et la grandeur de la patience viennent de Dieu.
L’homme doit aimer ce qui est cher à Dieu,
	car ce qu’aime la Majesté divine, elle le recommande.
Si Dieu est notre Seigneur et notre Père,
	imitons la patience de qui est en même temps notre Seigneur et notre Père;
	puisqu’il convient que des serviteurs soient obéissants
	et que des fils ne soient pas dégénérés.
