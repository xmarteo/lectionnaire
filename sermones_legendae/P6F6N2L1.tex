Mes bien chers frères,
	si ce n’est pas dans notre chair que le Sauveur a triomphé du démon,
	il a combattu, mais il n’a pas vaincu pour nous.
Si ce n’est pas dans notre corps qu’il est ressuscité,
	il n’a rien changé à notre condition en ressuscitant.
Celui qui parle ainsi
	ne comprend pas pourquoi le Sauveur s’est revêtu de notre chair
		et l’a élevée au ciel,
	il confond l’ordre de la rédemption, et en perd l’utilité.
Si ce n’est pas dans notre chair que le Christ a opéré notre guérison,
	il n’a pris de la nature humaine que la bassesse de la naissance.
Chassons loin de notre esprit une croyance aussi dangereuse,
	ce qu’il a pris vient de nous, ce qu’il a donné vient de lui.
J’atteste que ce qui a succombé est mien,
	afin que ce qui est ressuscité soit mien.
Je confesse que ce qui a été enseveli dans le tombeau est à moi,
	afin que ce qui est monté au ciel soit à moi.
