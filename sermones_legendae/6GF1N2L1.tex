On nous dit que le Seigneur fut irrité,
	car bien que Dieu eût pensé, c’est- à-dire eût su que l’homme,
		placé en ce bas monde et chargé du poids de la chair,
	ne peut être sans péché (car la terre est comme un lieu de tentation
		et la chair comme un appât de corruption),
	cependant les hommes ont un esprit doué de raison
	et une force d’âme infuse à leur corps,
	et ils avaient dû écarter toute réflexion,
	pour se précipiter dans une déchéance d’où ils ne voulaient pas revenir.
Dieu ne pense point à la manière des hommes,
	en sorte qu’un sentiment nouveau
		puisse succéder pour lui à une opinion précédente,
	il ne s’irrite pas non plus comme s’il était sujet au changement;
	mais ces expressions se trouvent dans l’Écriture
	afin de marquer la malice de nos péchés, qui a mérité la disgrâce divine.
C’est comme si l’écrivain sacré nous disait
	que nos fautes sont montées jusqu’à un tel excès
	qu’elles ont même paru provoquer Dieu à la colère,
	tout incapable qu’il soit, par sa nature,
	d’être ému de colère, de haine ou de quelque autre passion.
