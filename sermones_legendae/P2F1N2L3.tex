Devant ces miracles et d’autres encore,
	les disciples étaient agités de pensées craintives,
	bien que le Seigneur eût apparu au milieu d’eux et leur eût dit:
	«Paix à vous.»
Pour chasser le doute qui flottait dans leur cœur,
	car ils croyaient voir un esprit et non un corps,
	le Sauveur confond des pensées si peu conformes à la vérité;
	il met sous les yeux des disciples, qui doutaient encore,
	les marques de son crucifiement demeurées dans ses mains et dans ses pieds;
	il invite à les examiner attentivement et à les toucher.
Les traces des blessures faites par la lance et par les clous
	étaient conservées pour guérir les plaies des cœurs infidèles,
	et pour que l’on crût, non d’une fois chancelante,
	mais par une connaissance très sûre,
	que cette même nature, qui avait été gisante dans le tombeau,
	devait s’asseoir sur le trône de Dieu le Père.
