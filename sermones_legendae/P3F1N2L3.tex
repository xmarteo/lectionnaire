Chez l’homme charnel, tout ce qui règle l’intelligence,
	c’est ce qu’il a coutume de voir.
Ce qu’ils ont coutume de voir, ils le croient;
	ce qui est inaccoutumé, ils ne le croient pas.
Dieu fait des miracles contraires à ce qu’on voit d’habitude,
	car il est Dieu.
En vérité, la naissance, chaque jour, de tant d’hommes qui n’existaient pas,
	est une plus grande merveille
		que la résurrection de quelques hommes qui existaient;
	et pourtant ces merveilles de la naissance
	ne sont pas prises en considération.
Leur fréquence en a fait une banalité.
Le Christ est ressuscité;
	la chose est certaine.
Il était corps, il était chair, il a été pendu à la croix,
	il a rendu son âme, sa chair a été déposée dans le sépulcre.
	Or il a montré cette chair vivante, lui qui vivait en elle.
Pourquoi s’étonner? Pourquoi ne pas croire?
	Il est Dieu, celui qui a fait cela.
