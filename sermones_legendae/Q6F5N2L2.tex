Puissent ceux qui nous persécutent aujourd’hui,
	revenir au bien et partager nos épreuves:
	néanmoins, aussi longtemps qu’ils nous tourmentent,
	puissions-nous à notre tour ne pas les prendre en haine!
En effet, nous ignorons si chacun d’eux
		persévérera jusqu’à la fin dans sa mauvaise voie:
	bien souvent il arrive que, au lieu de haïr un ennemi comme tu le crois,
	tu détestes sans le savoir un de tes frères.
Les saintes Écritures nous l’attestent:
	le démon et ses anges sont condamnés au feu éternel:
	eux seuls ne nous laissent aucun espoir de les voir revenir au bien:
	nous avons à soutenir contre eux une lutte invisible
	et c’est pour cette lutte que l’Apôtre nous arme en disant:
	«Nous n’avons pas à combattre contre la chair et le sang,»
	c’est-à-dire, contre des hommes que l’on peut voir;
	«mais contre les principautés et les puissances,
	contre les princes de ce monde, de ces ténèbres.»
À l’entendre s’exprimer de la sorte et dire: «les princes de ce monde»,
	tu croirais peut-être
		que le gouvernement du ciel et de la terre appartient au démon,
	mais ne t’y trompe pas;
	aux mots «de ce monde», il a ajouté: de ces ténèbres.
Par «le monde», il a entendu les amis du monde.
«Le monde» selon lui, ce sont les impies et les pécheurs:
	c’est ce monde dont l’Évangile a dit: «Le monde ne l’a pas connu».
