Les gardiens du ciel courent à sa rencontre et font ouvrir les portes,
	afin qu’il entre de nouveau en possession de sa gloire.
Mais ils ne le reconnaissent pas,
	lui qui a revêtu la robe abjecte de notre vie,
	et dont les vêtements sont rougis par le pressoir des douleurs humaines.
C’est pourquoi ils interrogent de nouveau leurs compagnons par ces paroles:
	«Quel est ce Roi de gloire?»
	Or, on ne répond plus: «Celui qui est fort et puissant dans le combat»,
	mais: «C’est le Seigneur des armées»,
	qui a obtenu la principauté du monde,
	qui a tout réuni en lui-même comme en un abrégé,
	qui a rétabli toutes choses dans leur premier état:
	«C’est lui, le Roi de gloire.»
