La comparaison de l’aigle convient donc en tous points au Sauveur.
Mais que faisons-nous de ce que l’aigle s’empare fréquemment d’une proie,
	et enlève souvent le bien d’autrui?
	En cela encore, le Sauveur ne diffère pas de l’aigle.
Car il a pour ainsi dire ravi une proie,
	lorsqu’il a porté au ciel l’homme, dont il a pris la nature,
	l’a arraché au gouffre de l’enfer et l’a emmené captif au ciel,
	après avoir délivré de la servitude cet esclave d’une domination étrangère,
	c’est-à-dire de la puissance du démon,
	selon qu’il a été écrit par le Prophète:
	«Le Christ montant au ciel, a conduit la captivité captive;
	il a donné des dons aux hommes».
