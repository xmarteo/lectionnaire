«Il est monté au ciel, dit-il, il a conduit la captivité captive.»
Que ce Prophète décrit bien le triomphe du Seigneur!
	C’était, dit-on, la coutume des rois dans leur triomphe
	de faire marcher devant leur char un cortège de captifs.
Voici que la captivité glorieuse ne précède pas le Seigneur allant au ciel,
	mais l’accompagne;
	elle n’est pas traînée devant son char,
	mais elle-même sert de char au Sauveur.
Par un grand mystère,
	tandis que que le Fils de Dieu élève au ciel le fils de l’homme,
	la captivité elle-même y est portée et y porte tout à la fois.
