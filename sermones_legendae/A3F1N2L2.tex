En effet, le jeûne a toujours été un aliment pour la vertu.
L’abstinence produit les pensées chastes,
	les résolutions sages, les conseils salutaires;
	et, par les mortifications volontaires, la chair meurt à ses convoitises,
	tandis que l’esprit reçoit une nouvelle vigueur pour pratiquer les vertus.
Mais, parce que le salut de nos âmes ne s’acquiert pas uniquement par le jeûne,
	ajoutons au jeûne, des œuvres de miséricorde envers les pauvres.
Faisons servir à la vertu ce que nous retranchons à la sensualité,
	et que l’abstinence de celui qui jeûne devienne le repas du pauvre.
