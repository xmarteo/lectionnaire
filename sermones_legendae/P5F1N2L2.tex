Les prémices sont de la même espèce
		et de la même nature que le reste des fruits,
	dont on offre à Dieu la première récolte,
	en reconnaissance d’une production abondante:
	présent sacré pour tous ses dons,
	offrande pour ainsi dire de la nature renouvelée.
Les prémices donc de ceux qui sont dans le repos, c’est le Christ.
Mais l’est-il seulement de ceux qui reposent en lui,
	qui, débarrassés de la mort, sont sous l’empire d’un doux sommeil,
	ou l’est-il de tous les morts?
	«Tous meurent en Adam, tous aussi recevront la vie dans le Christ.»
C’est pourquoi de même que les prémices de la mort se trouvaient en Adam,
	de même, les prémices de la résurrection sont dans le Christ:
	tous ressusciteront.
Que personne donc ne désespère,
	et que le juste ne s’afflige pas de cette résurrection commune,
	alors qu’il a à attendre une récompense toute spéciale de sa vertu.
«Tous ressusciteront, dit l’Apôtre,
	mais chacun en son rang.»
Le fruit de la clémence divine est commun à tous,
	mais on distinguera l’ordre des mérites.
