Votre sainteté se rappelle que j’ai comparé le Sauveur à cet aigle des Psaumes,
	duquel nous lisons que sa jeunesse se renouvelle.
Cette similitude a un sens fort étendu.
En effet, comme l’aigle quitte ce qui est bas,
	recherche les hauteurs et monte jusqu’au voisinage des cieux;
	de même ainsi le Sauveur a quitté les profondeurs de l’enfer,
	a gagné les hauteurs du paradis, et a pénétré jusqu’au faîte des cieux.
Et comme l’aigle, fuyant les souillures du sol terrestre,
	volant haut, jouit de la salubrité d’un air plus pur;
ainsi le Seigneur,
	abandonnant la fange des pécheurs de la terre et volant parmi ses saints,
	se réjouit en la simplicité d’une vie plus pure.
