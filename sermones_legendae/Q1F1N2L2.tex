On devait, en vérité, à de si grands mystères,
	une dévotion continuelle, une révérence ininterrompue
	et telle que, devant Dieu, nous demeurions en l’état
		où il convient de nous trouver en la fête même de Pâques.
Mais cette constance est le fait d’un petit nombre;
	la fragilité de la chair met du relâchement
		dans l’observance d’une telle austérité,
	et à travers les occupations variées de cette vie
		où se détend notre attention,
	il est inévitable que même les cœurs religieux
		se souillent de la poussière du monde.
C’est à ce grand besoin de purification
	qu’a pourvu la divine institution d’un exercice de quarante jours,
	pendant lesquels nous renouvelons la pureté de nos âmes,
	en rachetant par des bonnes œuvres les fautes des autres temps
	et en les consumant en des jeûnes faits en toute pureté d’intention.
