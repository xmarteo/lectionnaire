Aussi faut-il détester en tous points le poison de cette erreur orientale,
	qui, par une nouveauté impie,
	ose affirmer que le Fils de Dieu et le fils de l’homme
		sont d’une même nature.
Il y a là un double mensonge:
	prétendre que le Christ n’a été qu’un homme,
	c’est nier la gloire du Créateur;
	dire qu’il a été Dieu seulement,
	c’est nier la miséricorde du Rédempteur.
Aussi n’est-il pas facile à un arien de comprendre la vérité de l’Évangile,
	où nous lisons tantôt que le Fils de Dieu est égal à son Père,
	et tantôt qu’il lui est inférieur.
En effet, celui qui, par suite d’une persuasion mortelle,
	aura cru que notre Sauveur n’a qu’une seule nature,
	sera forcé de dire que celui qui a été crucifié
		était seulement Dieu ou seulement homme.
	Mais il n’en est pas ainsi.
Car s’il eût été Dieu seulement, le Christ n’aurait pu souffrir la mort;
	et s’il n’eût été qu’homme, il n’aurait pu la vaincre.
