Écoutons l'ordre qu'il leur	 
29	 	donna, après leur avoir reproché leur dureté de coeur :	 
30	 	« Allez dans le monde entier, prêchez l'Évangile à toute	 
31	 	créature. »
Est-ce que, mes frères, le saint Évangile devait être prê-	 
2	 	ché aux êtres inanimés ou aux animaux sans raison, pour qu'il	 
3	 	soit dit aux disciples : « Prêchez à toute créature » ? Mais l'ex-	 
4	 	pression toute créature désigne l'homme.
L'homme a	 
14	 	quelque chose de toutes les créatures. Il a l'être en commun	 
15	 	avec les pierres, la vie avec les arbres, la sensibilité avec les	 
16	 	animaux, l'intelligence avec les anges. Si donc l'homme a	 
17	 	quelque chose de commun avec toute créature, en un sens	 
18	 	l'homme est toute créature. L'Évangile est donc prêché à	 
19	 	toute créature, alors même qu'il est prêché à l'homme seule-	 
20	 	ment