Ce que le Seigneur avait dit en figure, en maudissant le figuier stérile,
	il le manifeste bientôt plus ouvertement
	en chassant du temple les voleurs.
Car l’arbre n’avait péché en rien,
	du fait qu’il n’avait pas eu de fruits pour le Seigneur affamé,
	car la saison n’en était pas encore venue;
	mais ils ont péché, les prêtres qui géraient des affaires séculières
	et avaient négligé de porter le fruit de piété qu’ils auraient dû
	et dont le Seigneur avait faim en eux.
Par sa malédiction, le Seigneur dessécha l’arbre,
	pour que les hommes témoins ou informés de cette malédiction
	comprissent beaucoup mieux
		que leur condamnation au jugement divin était inévitable,
	si sans le fruit des œuvres,
		ils n’avaient à s’applaudir que d’un langage religieux,
	comme du son et du revêtement d’un verdoyant feuillage.
