Quel est donc le sens de ces paroles:
	«Bienheureux celui qui ne sera point scandalisé de moi»?
N’est-ce pas une déclaration manifeste
	de l’abjection et de l’humiliation de sa mort?
Comme s’il disait ouvertement:
	II est vrai que je fais des choses admirables;
	mais je ne dédaigne pas d’en souffrir d’abjectes.
Puisque donc, en mourant, je me fais semblable à toi,
	que les hommes veillent à ne pas mépriser ma mort,
	eux qui vénèrent mes miracles.
