A la vue de	 
4	 	tant de signes et de si grands prodiges, personne n'avait pu	 
5	 	être scandalisé, on n'avait pu qu'admirer. Mais l'âme des	 
6	 	incroyants fut gravement scandalisée lorsqu'après tant de	 
7	 	miracles ils le virent mourir. Aussi Paul dit-il : « Nous prê-	 
8	 	chons, nous, un Christ crucifié, scandale pour les Juifs, folie	 
9	 	pour les Gentils. » Aux hommes il parut en effet chose folle	 
10	 	que l'auteur de la vie meure pour les hommes ; et contre lui	 
11	 	l'homme fit une cause de scandale de ce qui aurait dû le	 
12	 	rendre plus reconnaissant. Dieu, en effet, était d'autant plus	 
13	 	digne d'être honoré par les hommes que pour les hommes	 
14	 	il supportait plus d'indignités.