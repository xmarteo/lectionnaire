En effet, gouverner l’univers est chose bien autrement merveilleuse
		que rassasier cinq mille hommes avec cinq pains.
Et pourtant, personne ne prête attention à l’un, tandis que tous admirent l’autre:
	cette différence d’appréciation vient de ce que le second fait
		est, sinon plus admirable, du moins plus rare.
Car celui qui nourrit maintenant tout le monde,
	n’est-il pas le même qui donne à quelques grains
		la vertu de produire nos récoltes?
Dieu a donc agi de la même manière:
	c’est la même puissance qui transforme, tous les jours,
		en riches moissons, quelques grains de blé,
	et qui a multiplié cinq pains entre ses mains.
Cette puissance se trouvait à la disposition du Christ:
	pour les pains, ils étaient comme une semence,
	et cette semence, au lieu d’être jetée en terre,
	a été directement multipliée par Celui qui a créé la terre.
