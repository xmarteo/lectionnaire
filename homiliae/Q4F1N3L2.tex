Car c’est un plus grand miracle de gouverner le monde entier
	que de rassasier de cinq pains cinq mille personnes.
Et pourtant, nul ne s’étonne du premier prodige,
	tandis que l’on est rempli d’admiration pour le second,
	non parce qu’il est plus grand, mais parce qu’il est rare.
Qui, en effet, maintenant encore, nourrit le monde entier,
	sinon celui qui, de quelques grains, fait sortir les moissons?
	Jésus a donc agi à la manière de Dieu.
En effet, par cette même puissance
		qui d’un petit nombre de grains multiplie les moissons,
	il a multiplié entre ses mains les cinq pains.
Car la puissance était entre les mains du Christ.
	Ces cinq pains étaient comme des semences non plus confiées à la terre,
	mais multipliées par celui qui a fait la terre.
