Rappelez-vous ma proposition:
	je veux vous parler des trente-huit ans du paralytique de l’Evangile;
	je veux vous expliquer comment il se fait
		que le nombre trente-huit indique plutôt la maladie que la santé;
	je l’ai dit: la charité accomplit la loi:
	et à l’entier accomplissement de la loi, en n’importe quelles œuvres,
		se rapporte le nombre quarante.
Mais, relativement à la charité, nous avons reçu deux commandements:
	«Tu aimeras le Seigneur ton Dieu de tout ton cœur,
		de toute ton âme, de tout ton esprit,
	et tu aimeras ton prochain comme toi-même.
Ces deux commandements renferment toute la loi et les Prophètes».
La veuve de l’Évangile n’a-t-elle pas fait don à Dieu
		de deux misérables pièces d’argent qui composaient tout son avoir?
Est-ce que l’hôtelier n’a pas reçu deux deniers
		pour veiller à la guérison du malheureux blessé
	que des voleurs avaient laissé à moitié mort sur le chemin?
Jésus n’a-t-il point passé deux jours chez les Samaritains,
	pour les affermir dans la charité?
Lorsqu’il s’agit de quelque bonne œuvre,
	le nombre deux a donc trait au double précepte de la charité:
	de là il suit que le nombre quarante indique l’entier accomplissement de la loi,
	et que la loi n’est accomplie
		que par l’observation du double précepte de la charité:
	alors, pourquoi s’étonner
		si celui à qui le nombre deux manquait pour parvenir à quarante,
	gisait sous le poids de la maladie?
