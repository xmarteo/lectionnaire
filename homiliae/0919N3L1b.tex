Les disciples	 
15	 	demandent au Seigneur quand cela aurait lieu	 
16	 	et à quel signe ils reconnaîtraient son avènement et la fin	 
17	 	du monde. Et de fait, parce qu'il y a là trois questions	 
18	 	réunies en une seule, il les sépare en faisant des distinc-	 
19	 	tions de chronologie et de sens. Ainsi c'est d'abord la	 
20	 	ruine de la cité qui est l'objet d'une réponse : il les	 
21	 	confirme dans la vérité enseignée par une mise en garde	 
22	 	contre le risque d'être surpris, si on l'ignore, par un	 
23	 	imposteur, car, à l'époque des apôtres encore, devaient	 
24	 	venir des hommes qui se donneraient le nom de Christ.	 
25	 	Ainsi l'avertissement que la foi pouvait être arrachée	 
 	--- 183 ---	 
1	 	par un mensonge funeste a précédé l'événement.