Quant à la foule,	 
6	 	n'est-ce pas la mêlée d'une multitude ignorante, qui ne	 
7	 	pouvait voir les hauteurs de la Sagesse ? Donc Zachée,	 
8	 	tant qu'il est dans la foule, ne voit pas le Christ ; il s'est	 
9	 	élevé au-dessus de la foule, et il a vu : autrement dit, en	 
10	 	dépassant l'ignorance populaire il a réussi à contempler	 
11	 	Celui qu'Il désirait. 89. On a ajouté à propos : « Parce	 
12	 	que le Seigneur devait passer en cet endroit », où était	 
13	 	soit le sycomore, soit le futur croyant : Il observait	 
14	 	ainsi le mystère et semait la grâce ; car Il était venu	 
15	 	pour passer des Juifs aux Gentils.