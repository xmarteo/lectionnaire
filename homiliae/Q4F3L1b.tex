Celui qui s’était caché, enseignait,
	il parlait en public et on ne s’emparait pas de sa personne.
Il s’était caché pour nous donner l’exemple;
	et il ne permettait pas qu’on s’emparât de lui pour montrer sa puissance.
Quand il enseignait, «les juifs s’étonnaient»;
	autant que je puis en juger, tous s’étonnaient,
	mais tous ne se convertissaient pas.
Et d’où venait leur surprise?
	De ce que beaucoup savaient où il était né,
	comment il avait été élevé.
Jamais ils ne l’avaient vu apprendre les Écritures,
	pourtant ils l’entendaient disserter sur la loi,
	citer à l’appui de ses paroles des passages de la loi,
	que personne ne pouvait citer sans les avoir lus,
	et que personne n’aurait pu lire sans avoir étudié,
	et c’est pourquoi ils s’étonnaient.
Leur surprise fut, pour le divin Maître,
	l’occasion de leur insinuer profondément la vérité.
