Celui qui se cachait, enseignait:
	il parlait en public, et personne ne mettait la main sur lui.
Il ne se faisait pas connaître, afin de nous servir d’exemple;
	et si personne ne s’emparait de lui, c’était l’effet de sa puissance.
Quand il enseignait, «les Juifs s’étonnaient».
	À mon avis, tous s’étonnaient; mais tous ne se convertissaient pas.
D’où venait leur étonnement?
Beaucoup savaient où il était né, comment il avait été élevé;
	jamais ils ne l’avaient vu apprendre les Écritures;
	pourtant, ils l’entendaient disserter sur la loi,
	citer à l’appui des passages de la loi,
	que personne ne pouvait citer sans les avoir lus,
	et que personne ne pouvait lire sans avoir appris la lecture.
Ils s’étonnaient donc.
Leur étonnement fut, pour le divin Maître,
	l’occasion de leur insinuer des vérités plus hautes.
