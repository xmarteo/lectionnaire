Lorsque	 
29	 	le Père dit : « Celui-ci est mon Fils bien-aimé, en qui je me	 
30	 	suis complu ; écoutez-le », est-ce qu'on n'entendit pas	 
31	 	clairement : « Celui-ci est mon Fils », pour qui être de moi	 
32	 	et être avec moi est une réalité qui échappe au temps ?	 
33	 	Car ni Celui qui engendre n'est antérieur à l'Engendré, ni	 
34	 	l'Engendré postérieur à Celui qui l'engendre. « Celui-ci est	 
35	 	mon Fils », que de moi ne sépare pas la divinité, ne divise	 
36	 	pas la puissance, ne distingue pas l'éternité. « Celui-ci est	 
37	 	mon Fils », non adoptif, mais propre ; non créé d'ailleurs	 
 	--- 33 ---	 
1	 	mais engendré de moi ; non d'une autre nature et fait	 
2	 	comparable à moi, mais de mon essence et né égal à moi.