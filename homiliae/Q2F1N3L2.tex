Lorsque le Père dit:
	Celui-ci est mon Fils bien-aimé en qui je me suis complu, écoutez-le,
	n’est-il pas évident qu’on doit comprendre:
	Celui-ci est mon Fils
		auquel il appartient d’être de moi et avec moi, éternellement?
Car celui qui engendre n’est pas antérieur à celui qui est engendré,
	et celui qui est engendré n’est pas postérieur à celui qui engendre.
Celui-ci est mon Fils que la divinité ne sépare pas de moi,
	que la puissance ne divise pas et que l’éternité ne distingue pas.
Celui-ci est mon Fils, non pas adoptif, mais au sens propre,
	non pas créé d’un autre, mais engendré de moi;
	non pas d’une autre nature et devenu comparable à moi,
	mais né de mon essence et égal à moi.
