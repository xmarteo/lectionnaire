mais comme, inclinée par la convoitise au	 
26	 	péché, la faiblesse humaine, corps et âme, est enlacée par	 
27	 	les liens inextricables des vices, le huitième jour assigné	 
28	 	pour la circoncision figurait que la purification de toutes	 
29	 	fautes devait s'accomplir au temps de la Résurrection.	 
30	 	C'est le sens du texte « tout mâle qui le premier ouvre	 
31	 	le sein (maternel) sera appelé saint pour le Seigneur »	 
32	 	(Ex., XIII, 12) : ces paroles de la Loi promettaient le	 
33	 	fruit de la Vierge, vraiment saint, parce que sans tache.	 
34	 	Au reste, qu'Il soit bien Celui que la Loi désigne, la reprise	 
35	 	par l'ange des mêmes expressions le manifeste : « L'enfant	 
36	 	qui va naître, dit-il, sera appelé saint, Fils de Dieu »