Quand il voulut guérir le sourd-muet,	 
19	 	pourquoi le Créateur de tout lui mit-il ses doigts dans les oreil-	 
20	 	les et de sa salive lui toucha-t-il la langue ? Que désignent les	 
21	 	doigts du Rédempteur, sinon les dons du Saint-Esprit ? Dans	 
22	 	une autre circonstance, ayant chassé un démon, le Seigneur a	 
23	 	dit : « Si moi je chasse les démons par le doigt de Dieu, c'est	 
24	 	assurément que le royaume de Dieu est arrivé jusqu'à vous. »	 
25	 	Or un autre évangéliste rapporte qu'il a dit à cette occasion :	 
26	 	« Si moi je chasse les démons par l'Esprit de Dieu, c'est donc	 
27	 	que le royaume de Dieu est arrivé jusqu'à vous. » On conclut	 
28	 	par la comparaison de l'un et l'autre passages que l'Esprit est	 
29	 	appelé le doigt de Dieu. Mettre ses doigts dans les oreilles d'un	 
30	 	sourd, c'est donc, par les dons du Saint-Esprit, ouvrir son âme	 
31	 	à l'obéissance.