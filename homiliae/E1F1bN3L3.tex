« Et Il vint à Nazareth, et Il leur était soumis. »	 
33	 	Maître de vertu, pouvait-Il moins faire que remplir les	 
34	 	devoirs de la piété filiale ? Et nous sommes étonnés de sa	 
35	 	déférence envers son Père quand Il se soumet à sa Mère ?	 
36	 	Ce n'est certes pas la faiblesse, mais la piété qui fait cette	 
37	 	dépendance, bien que, sortant de son antre tortueux, le	 
38	 	serpent de l'erreur lève la tête et, de ses entrailles de	 
39	 	vipère, vomisse le venin. Quand le Fils se dit envoyé,	 
40	 	l'hérétique en appelle au Père plus grand pour déclarer	 
 	--- 101 ---	 
1	 	imparfait ce fils qui peut avoir plus grand que Lui, pour	 
2	 	affirmer qu'Il a besoin d'un secours étranger, puisqu'Il est	 
3	 	envoyé. Est-ce également par besoin d'un secours humain	 
4	 	qu'Il obéissait aux ordres de sa Mère ?