Il nous faut donc veiller à ce que, si quel-	 
26	 	qu'un n'arrive pas à instruire en même temps tous les siens,	 
27	 	par une seule et même exhortation, il doit les éclairer un à	 
28	 	un, autant qu'il le peut, les former par des entretiens parti-	 
29	 	culiers. Nous devons méditer sans cesse ce qui est	 
31	 	dit aux apôtres, et à nous par les apôtres : « Vous êtes le sel	 
32	 	de la terre. » Si nous sommes sel, nous devons assaisonner	 
33	 	l'âme des fidèles. Vous qui êtes des pasteurs, songez que	 
34	 	vous faites paître le bétail de Dieu. De ce bétail il est dit à	 
 	--- 379 ---	 
1	 	Dieu par le psalmiste : « Ton bétail y habitera. » Or nous	 
2	 	voyons souvent qu'une pierre à sel est placée devant les	 
3	 	bêtes sans raison, de telle manière qu'elles lèchent la pierre	 
4	 	à sel et s'en trouvent mieux. Telle une pierre à sel entre les	 
5	 	bêtes, tel doit être le prêtre au milieu de son peuple. Car il	 
6	 	faut que le prêtre étudie ce qu'il doit dire à chacun, com-	 
7	 	ment avertir chacun : qu'au contact du prêtre comme au	 
8	 	contact du sel chacun soit pénétré de la saveur de la vie éter-	 
9	 	nelle. 