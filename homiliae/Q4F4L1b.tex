Dans cette étonnante merveille opérée par Notre-Seigneur,
	il faut remarquer les actions et les paroles,
	les actions qui ont eu lieu,
	les paroles, parce qu’elles sont des signes.
Si nous réfléchissons au sens caché de ce fait,
	nous verrons que l’aveugle représente le genre humain;
	car la cécité a été, chez le premier homme, le résultat du péché,
	et il nous a communiqué à tous, non-seulement le germe de la mort,
	mais encore celui de l’iniquité.
Puisque l’infidélité est un véritable aveuglement,
	et qu’on jouit de la vue quand on a la foi,
	le Christ, au moment de sa venue sur la terre, a-t-il trouvé un seul fidèle? 
L’Apôtre, qui était de la même nation que les Prophètes,
	a dit: «Nous avons été autrefois, par notre nature,
	les enfants de la colère comme le reste des hommes».
Si nous avons été enfants de colère,
	nous étions les enfants de la vengeance, de la peine, de la géhenne.
Comment l’étions-nous par nature,
	si ce n’est que, par le péché du premier homme,
	la corruption est devenue pour nous une seconde nature?
Si la corruption est devenue pour nous une seconde nature,
	tout homme est né aveugle, quant à son âme.
