Les hommes qui ont mis en Dieu leur espérance,
	ne doivent pas rendre le mal même pour le mal:
	et les ennemis du Christ lui rendaient le mal pour le bien;
	aussi leur annonce-t-il d’avance leur sort à venir:
	il prononce leur sentence, car il sait ce qui doit leur arriver plus tard;
	il leur prédit qu’ils mourront dans leur péché;
	puis il ajoute «Vous ne pouvez venir où je vais».
En une autre circonstance, il avait tenu à ses disciples le même langage,
	sans toutefois leur dire: «Vous mourrez dans votre péché».
Quelles paroles leur avait-il donc adressées?
Les mêmes qu’aux Juifs: «Vous ne pouvez venir où je vais».
Par là, il ne leur ôtait point l’espérance de le suivre,
	mais il les avertissait qu’ils n’iraient pas immédiatement avec lui.
Au moment où le Sauveur parlait à ses disciples,
	ils ne pouvaient pas, en effet, aller où il allait lui-même;
	mais ils devaient y parvenir plus tard;
	pour les Juifs, jamais, puisqu’il leur disait d’avance:
	«Vous mourrez dans votre péché».
