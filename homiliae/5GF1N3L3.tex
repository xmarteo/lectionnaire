Il faut noter que la lumière est rendue à l'aveugle au	 
30	 	moment où, dit le texte, Jésus approche de Jéricho. En effet	 
31	 	Jéricho signifie lune ; or dans le langage de la Bible la lune	 
32	 	donne à entendre la faiblesse de la chair : chaque mois sa	 
33	 	phase décroissante symbolise la faiblesse de notre condition	 
 	--- 123 ---	 
1	 	mortelle. Au moment donc où notre Créateur approche de	 
2	 	Jéricho, l'aveugle retrouve la lumière, car au moment où la	 
3	 	divinité a fait sienne notre chair défaillante, le genre humain	 
4	 	a retrouvé la lumière qu'il avait perdue : du fait qu'un Dieu	 
5	 	subit la misère humaine, l'homme est haussé jusqu'à la	 
6	 	condition divine. Le texte montre avec raison cet aveugle à	 
7	 	la fois assis au bord du chemin et mendiant. La Vérité a dit	 
8	 	elle-même en effet : « Je suis le chemin. »