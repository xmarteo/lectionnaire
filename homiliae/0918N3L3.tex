Il faut savoir que de même qu'un vêtement est tissé	 
14	 	entre deux pièces de bois, placées l'une en haut, l'autre en	 
15	 	bas, la charité est contenue dans les deux préceptes de	 
16	 	l'amour de Dieu et de l'amour du prochain. Car il est écrit :	 
17	 	« Tu aimeras le Seigneur ton Dieu de tout ton coeur, de toute	 
18	 	ton âme, de toutes tes forces, et tu aimeras ton prochain	 
19	 	comme toi-même. » Il faut noter que, dans l'amour du pro-	 
20	 	chain, une mesure est fixée à l'amour par ces mots : « Tu	 
21	 	aimeras ton prochain comme toi-même », mais qu'aucune	 
22	 	mesure ne limite l'amour de Dieu quand il est dit : « Tu	 
23	 	aimeras le Seigneur ton Dieu, de tout ton coeur, de toute ton	 
24	 	âme, de toutes tes forces. » En effet, en disant « de tout », ce	 
25	 	précepte ne dit pas jusqu'à quel point, mais avec quelle lar-	 
26	 	geur de coeur il faut aimer. Car celui-là aime vraiment Dieu	 
27	 	qui ne se réserve rien de lui-même. Quiconque veut avoir la	 
28	 	robe nuptiale au festin de noces doit donc observer ces deux	 
29	 	préceptes de la charité.