Voyons donc ce que le Christ a voulu mous faire entendre par ce paralytique;
	car le Sauveur a respecté, lui aussi, ce que le nombre un a de mystérieux,
	et, de tous les malades rangés autour de la piscine,
	il n’a daigné guérir que celui-là.
Dans l’âge de cet homme il a trouvé un nombre d’années qui indique une maladie:
	«Il était malade depuis trente-huit ans».
Comment ce nombre d’années indiquait-il plutôt la maladie que la santé?
C’est ce que nous allons expliquer avec un soin plus particulier.
Je désire que vous me prêtiez toute votre attention:
	le Seigneur nous aidera, moi, à vous parler convenablement,
	et vous, à me bien comprendre.
Le nombre quarante nous est signalé comme un nombre sacré,
	parce qu’en un sens, il est parfait.
Votre charité, je le suppose, ne l’ignore pas;
	et les divines Ecritures l’attestent en maints endroits.
Vous le savez, le jeûne tire sa consécration de ce nombre de jours.
En effet, Moïse a jeûné quarante jours;
	Élie a fait de même;
	et notre Seigneur et Sauveur Jésus-Christ a aussi jeûné le même espace de temps.
Moïse représentait la loi, Élie les Prophètes, et Jésus-Christ l’Évangile:
	c’est pourquoi ils apparurent tous les trois
		sur la montagne où le Sauveur se manifesta à ses disciples
		avec un visage et des vêtements tout radieux.
Dans cette apparition, Jésus se trouvait entre Moïse et Élie,
	comme si l’Évangile tirait sa force du témoignage de la loi et des Prophètes.
