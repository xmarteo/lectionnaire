Pour nous faire	 
15	 	admettre la réalité de la résurrection du Seigneur, il faut	 
16	 	encore noter ce que rapporte Luc : « Après avoir mangé avec	 
17	 	eux, il leur prescrivit de ne pas s'éloigner de Jérusalem. » Et	 
18	 	un peu plus loin : « Sous leurs yeux, il s'éleva, et une nuée	 
19	 	le reçut et le cacha à leurs regards. » Notez ces paroles,	 
20	 	remarquez-en le mystère : « Après avoir mangé avec eux, il	 
21	 	s'éleva. » Il mangea et il monta ; en mangeant, il manifestait	 
22	 	la réalité de sa chair. Marc, lui, rapporte qu'avant de mon-	 
23	 	ter au ciel, le Seigneur reprocha à ses disciples leur dureté	 
24	 	de coeur et leur manque de foi. Ne devons-nous pas voir là	 
25	 	que, si le Seigneur fit des reproches à ses disciples au	 
26	 	moment où de corps il se séparait d'eux, c'était pour que les	 
27	 	paroles qu'il leur disait alors demeurent plus profondément	 
28	 	imprimées dans leur coeur ?