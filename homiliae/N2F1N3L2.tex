Car mon Jésus à moi ne porte pas, comme ceux de	 
11	 	jadis, un nom vide et vain. En lui se trouve non seu-	 
12	 	lement l'ombre d'un grand nom, mais sa pleine vérité.	 
13	 	L'Évangéliste l'atteste, c'est du ciel que ce nom a été	 
14	 	indiqué : « Le nom, dit-il, dont l'ange l'avait appelé avant	 
15	 	sa conception ». Et remarque la profondeur de ce texte :	 
16	 	après sa naissance, Jésus est appelé par les hommes de	 
17	 	ce nom même dont l'ange l'avait appelé avant sa	 
18	 	conception. De fait, c'est le même qui est le Sauveur de	 
19	 	l'ange aussi bien que de l'homme, mais de l'homme il	 
20	 	l'est depuis l'Incarnation, tandis que de l'ange il l'est	 
21	 	« depuis l'origine de la création ».

« On l'appela du nom de	 
23	 	Jésus, ce nom dont l'ange l'avait	 
24	 	appelé. » « Toute parole sera	 
25	 	garantie par l'attestation de deux ou trois témoins »,	 
26	 	même cette Parole qui, comme dit le Prophète, s'est	 
27	 	abrégée, et qui, comme le dit plus clairement l'Évan-	 
28	 	gile, s'est faite chair.