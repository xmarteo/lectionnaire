Thomas a vu et palpé. Pourquoi	 
11	 	Jésus lui dit-il : « Parce que tu m'as vu, tu as cru » ? Mais	 
12	 	autre est ce qu'il a vu, autre ce qu'il a cru. En effet, la divi-	 
13	 	nité n'a pas pu être vue par un homme mortel. Donc, il a vu	 
14	 	l'homme et il a confessé Dieu en disant : « Mon Seigneur et	 
15	 	mon Dieu ! » C'est en voyant qu'il a cru, c'est en considé-	 
16	 	rant l'homme véritable qu'il a proclamé ce Dieu qu'il ne	 
17	 	pouvait voir.
Ce qui suit porte à la joie : « Bienheureux ceux qui,	 
19	 	sans avoir vu, ont cru. » Cette phrase nous concerne parti-	 
20	 	culièrement, nous qui possédons spirituellement celui que	 
21	 	nous n'avons pas vu corporellement. Nous sommes concer-	 
22	 	nés à condition que nos actes suivent notre foi. Celui-là croit	 
23	 	vraiment, qui met en pratique dans ses actes ce qu'il croit.