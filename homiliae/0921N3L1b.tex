Par respect, par	 
33	 	déférence à l'égard de Matthieu, les autres évangélistes n'ont	 
 	--- 171 ---	 
1	 	pas voulu lui donner son nom habituel ; ils l'ont appelé Lévi,	 
2	 	car il avait deux noms. Mais Matthieu, lui, suivant le précepte	 
3	 	de Salomon : « Le juste commence par s'accuser lui-même »	 
4	 	et cet autre : « Dis toi-même tes péchés pour être justifié »	 
5	 	se nomme Matthieu et se dit publicain, pour montrer aux	 
6	 	lecteurs que nul ne doit désespérer de son salut s'il s'est	 
7	 	converti à une vie meilleure, puisque lui-même s'est soudain	 
8	 	changé de publicain en apôtre.