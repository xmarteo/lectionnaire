Vous lisez donc à propos en Matthieu qu'il l'entoura	 
3	 	d'une haie, c'est-à-dire la fortifia du rempart de la pro-	 
4	 	tection divine, de peur qu'elle ne fût d'un abord facile	 
5	 	aux incursions des fauves spirituels. « Et il y creusa un	 
6	 	pressoir. » Comment entendre ce qu'est le pressoir ?	 
7	 	Peut-être parce que des psaumes sont intitulés : « Pour	 
8	 	les pressoirs », du fait que les mystères de la Passion du	 
9	 	Seigneur, tel un vin nouveau, y ont bouillonné avec plus	 
10	 	d'abondance, dans la chaleur de la sainte inspiration des	 
11	 	Prophètes. Aussi bien on a cru ivres ceux en qui se déver-	 
12	 	sait l'Esprit Saint (Act., II, 13). Donc Lui aussi creusa	 
13	 	un pressoir, où le fruit intérieur du raisin des âmes se	 
14	 	répandrait en un écoulement spirituel.