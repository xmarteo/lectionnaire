Le Seigneur connaissait ceux qui auraient aussi à subir la persécution;
	il connaissait ceux qui auraient aussi à être flagellés et tués à cause de lui.
Et ce n'est pas d'une autre Croix qu'il parlait,
	mais de la Passion qu'ils auraient à souffrir,
	lui-même le premier, et ses disciples après lui.
C'est pourquoi sa parole avait aussi pour but de les encourager:
	«Ne craignez pas ceux qui tuent le corps, mais qui ne peuvent tuer l'âme;
	craignez plutôt celui qui a le pouvoir de perdre en la géhenne et le corps et l'âme.»
Il les incitait à persévérer dans la confession de leur foi.
Il promettait en effet
	de confesser devant son Père ceux qui confesseraient son nom devant les hommes,
	mais de renier ceux qui le renieraient,
	et de rougir de ceux qui auraient rougi de confesser leur foi en lui.
Et malgré ces faits, certains en sont arrivés à ce point de témérité
	qu'il méprisent même les martyrs,
	blâment ceux qui se laissent mettre à mort pour confesser leur foi au Seigneur,
	ceux qui supportent tout ce que le Seigneur a prédit
	et qui s'efforcent en conséquence de suivre les traces de sa Passion,
	martyrs qu'ils sont, et témoins du Christ souffrant!
Ces détracteurs, c'est aux martyrs eux-mêmes que nous les remettons:
	lorsqu'en effet il sera demandé compte de leur sang et qu'ils recevront la gloire,
	le Christ alors couvrira de confusion
		tous ceux qui ont voulu jeter le discrédit sur leur martyre.
