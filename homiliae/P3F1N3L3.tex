Ce peu de temps paraît long, parce qu’il dure encore;
	quand il sera fini, alors nous comprendrons combien il fut court.
Que notre joie ne soit donc pas comme celle du monde dont il est dit:
	«Le monde se réjouira».
Et pourtant, dans l’enfantement de ce désir,
	ne soyons pas tristes sans joie;
	mais, comme dit l’Apôtre, «soyons joyeux par l’espérance,
	patients dans la tribulation»,
	car celle-là même qui enfante, et à qui nous sommes comparés,
	se réjouit plus de l’enfant qui va bientôt naître,
	qu’elle n’est triste de la souffrance présente.
Mais finissons ici ce discours;
	les paroles qui suivent soulèvent une question très difficile.
On ne doit pas les résumer brièvement,
	mais se réserver la possibilité de les expliquer plus commodément,
		si Dieu le veut.
