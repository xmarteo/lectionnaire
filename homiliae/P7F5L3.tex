Mais si nous avons là, au sens littéral, la teneur d'un	 
31	 	précepte vénérable qui touche au caractère religieux de	 
32	 	l'hospitalité, l'interprétation mystérieuse et spirituelle	 
33	 	nous sourit. Quand on choisit une maison, on se met en	 
34	 	quête d'un hôte digne. Voyons donc si ce ne serait pas	 
35	 	l'Église qui est désignée à notre préférence, et le Christ.	 
36	 	Est-il maison plus digne d'accueillir les prédicateurs apos-	 
37	 	toliques que la sainte Église ? qui, pour être préféré à	 
38	 	tous, a plus de titres que le Christ ? Il a coutume de laver	 
39	 	les pieds à ses hôtes, et du moment qu'Il reçoit dans sa	 
40	 	maison, Il ne souffre pas qu'on y séjourne avec des pieds	 
 	--- 252 ---	 
1	 	souillés, mais, si fangeux qu'ils soient de la vie passée,	 
2	 	Il daigne les nettoyer pour la suite du voyage. C'est donc	 
3	 	Lui seul que personne ne doit quitter, dont personne ne	 
4	 	doit changer. Il Lui est dit, à juste titre : « Seigneur, à	 
5	 	qui irons-nous ? vous avez les paroles de la vie éternelle,	 
6	 	et nous croyons »