Entre les délices du corps et celles du coeur, il y a d'or-	 
5	 	dinaire cette différence, frères très chers, que tant qu'on n'en	 
6	 	jouit pas, les délices du corps suscitent un désir profond, tan-	 
7	 	dis que, possédées, elles produisent aussitôt en celui qui les	 
8	 	savoure le dégoût par satiété. C'est le contraire pour les	 
9	 	délices spirituelles : pour qui ne les a pas, elles sont insipides,	 
10	 	pour qui les a, délectables ; celui qui y goûte en a d'autant	 
11	 	plus faim que sa faim l'y fait goûter davantage. Dans celles-	 
12	 	là, le désir est agréable et la satiété, fastidieuse ; dans celles-ci,	 
13	 	le désir est faible, mais l'expérience accroît le plaisir. Dans	 
14	 	celles-là, le désir engendre la satiété, et la satiété le dégoût ;	 
15	 	dans celles-ci, au contraire, le désir amène la satiété, et la	 
16	 	satiété le désir.