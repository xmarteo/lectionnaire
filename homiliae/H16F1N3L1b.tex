Nous lisons en premier	 
2	 	lieu la guérison de l'hydropique, en qui	 
3	 	l'enflure envahissante de la chair	 
4	 	gênait les fonctions de l'âme, éteignait	 
5	 	la flamme de l'esprit. Puis une leçon d'humilité, lors-	 
6	 	qu'en ce festin de noces est réprimé le désir d'une place	 
7	 	plus élevée : avec douceur cependant, pour que la bonté	 
8	 	de la persuasion enlève toute âpreté à l'interdiction, que	 
9	 	la raison rende efficace la persuasion, et que l'avertis-	 
10	 	sement corrige le désir. Dans son voisinage immédiat	 
11	 	vient s'insérer la bonté : la parole du Seigneur la définit	 
12	 	et distingue comme devant s'exercer envers les pauvres	 
13	 	et les faibles ; car être hospitalier pour être payé de	 
14	 	retour, c'est calcul d'avarice.