Que	 
10	 	figure le pharisien qui présume de sa fausse justice, sinon le	 
11	 	peuple juif ? Que désigne cette pécheresse qui vient pleurer	 
12	 	aux pieds du Seigneur, sinon les Gentils convertis ? Elle est	 
13	 	venue avec un vase d'albâtre, elle a versé un parfum, elle s'est	 
14	 	tenue par derrière aux pieds du Seigneur, elle a arrosé ses	 
15	 	pieds de ses larmes, elle les a essuyés de ses cheveux, et ces	 
16	 	pieds qu'elle arrosait et essuyait, elle n'a pas cessé de les cou-	 
17	 	vrir de baisers. C'est nous, oui, c'est nous qu'a représentés	 
18	 	cette femme, si, après nos fautes, nous revenons au Seigneur	 
19	 	de tout notre coeur, si nous imitons ses larmes de repentir.	 
20	 	Que représente, en effet, le parfum, sinon l'odeur d'une	 
21	 	bonne renommée ? C'est pourquoi Paul dit : « Nous sommes	 
22	 	pour Dieu la bonne odeur du Christ en tout lieu. »