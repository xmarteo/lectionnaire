« Si ta main ou ton pied te scandalise, coupe-le et jette-le	 
14	 	loin de toi. » Certes, le scandale arrive inévitablement,	 
15	 	et cependant, malheur à l'homme qui, par sa faute, fait	 
16	 	que ce qui doit inévitablement arriver dans le monde arrive	 
17	 	par lui. Donc toute affection est tranchée, toute parenté	 
18	 	rompue, de peur que les sentiments dont elles sont l'occasion	 
19	 	n'exposent chaque fidèle à donner du scandale. Quelqu'un	 
20	 	est-il aussi lié à toi que ta main, ton pied, ton oeil, est-il	 
21	 	pour toi utile, dévoué, clairvoyant et perspicace, s'il est	 
22	 	pour toi un objet de scandale, si sa conduite, en désaccord	 
23	 	avec la tienne, t'entraîne à la géhenne, mieux vaut te priver	 
24	 	de sa parenté et d'avantages temporels que de t'exposer	 
25	 	à la perdition en voulant gagner parents et amis.