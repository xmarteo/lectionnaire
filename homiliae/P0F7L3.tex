Ne trouvons pas étrange
		que le plus jeune nous soit donné comme figurant la synagogue
	et le plus vieux, l’Église;
	car bien que la synagogue soit arrivée au culte de Dieu
		avant l’Église des Gentils,
	la multitude des Gentils a précédé la synagogue
		dans la pratique des choses temporelles,
	ainsi que Paul en témoigne quand il dit:
	«Car ce n’est pas le spirituel qui est au début, mais l’animal.»
Par Pierre, le plus vieux, c’est donc l’Église des Gentils qui est figurée,
	et par Jean, le plus jeune, c’est la synagogue des Juifs.
Ils courent tous deux ensemble
	parce que, depuis le temps de leur origine jusqu’à la fin,
	la Gentilité court en même temps que la synagogue,
	dans une voie pareille et commune,
		mais non dans la communauté d’un même sentiment.
La synagogue est venue la première au tombeau,
	parce qu’elle a connu les commandements de la loi
	et entendu les prophéties de l’incarnation et de la passion du Seigneur;
	mais elle n’a pas voulu croire en un mort.
