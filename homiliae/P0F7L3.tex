Il ne faut pas nous étonner que la	 
12	 	Synagogue soit représentée par le plus jeune et l'Église par	 
13	 	le plus âgé. Car si, pour le culte de Dieu, la Synagogue a	 
14	 	devancé l'Église des nations, pour l'usage des biens de ce	 
15	 	monde, la multitude des nations a devancé la Synagogue.	 
16	 	Paul l'atteste : « Ce n'est pas l'être spirituel qui est premier,	 
17	 	mais l'être animal. » Pierre, le plus âgé, symbolise donc	 
18	 	l'Église des Gentils, et Jean, le plus jeune, la Synagogue	 
19	 	juive. Ils ont couru tous deux ensemble, parce que, de leurs	 
20	 	débuts à leur fin, Gentilité et Synagogue ont couru par un	 
21	 	chemin semblable ou commun, bien qu'avec une pensée qui	 
22	 	n'avait rien de semblable ni de commun.
La Synagogue est arrivée la première au tombeau, mais	 
24	 	n'y est pas entrée : elle a bien reçu les commandements de	 
25	 	la Loi, entendu les prophéties concernant l'incarnation et la	 
26	 	passion du Seigneur, mais elle a refusé de croire en un mort.