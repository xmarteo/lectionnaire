Aussi, quand sa grandeur divine se cacha sous la faiblesse de l’homme,
	nous ne devons pas croire qu’il renonça à sa puissance, 
	mais y voir seulement un exemple qu’il nous donna au milieu de nos douleurs.
Il ne fut livré, en effet, que lorsqu’il l'a voulu,
	il n’a été mis à mort qu’au moment où il y a consenti.
Mais parce qu’il devait s’adjoindre des membres,
	c’est-à-dire des fidèles
		qui ne posséderaient pas la même puissance que lui, puisqu’il était Dieu,
	il se cachait, il se dérobait aux poursuites des Juifs, comme pour éviter la mort,
	et ainsi donnait-il à entendre que plus tard ses membres s’uniraient à lui,
	et qu’il serait en chacun d’eux.
