Nous lisons dans	 
5	 	Isaïe : « Qui racontera sa généalogie ? » Ne pensons donc	 
6	 	pas que l'évangéliste s'oppose au prophète et que ce que	 
7	 	l'un dit inexprimable l'autre entreprenne de le raconter :	 
8	 	ici il a été question de la généalogie divine, là de l'incarnation.	 
9	 	Il commence par les réalités de sa nature charnelle afin	 
10	 	qu'à travers l'homme nous commencions à connaître Dieu.
Fils de David, fils d'Abraham. Ordre interverti, mais	 
12	 	changement nécessaire. En effet, s'il avait mis tout d'abord	 
13	 	Abraham, puis David, il lui eût fallu revenir à Abraham pour	 
14	 	suivre la trame de la généalogie.