Puis il ajouta : « Ephpheta,	 
11	 	c'est-à-dire : Ouvre-toi ; et aussitôt ses oreilles s'ouvrirent et le	 
12	 	lien de sa langue se dénoua. » Il faut noter ici que le mot	 
13	 	« Ouvre-toi » visait des oreilles closes. Quand les oreilles du	 
14	 	coeur ont été ouvertes pour l'obéissance, la conséquence natu-	 
15	 	relle est que le lien de la langue se dénoue : le bien que l'on a	 
16	 	fait soi-même, on invite les autres à le faire. Le texte ajoute	 
17	 	donc avec justesse : « Et il parlait correctement. » Celui-là en	 
18	 	effet parle correctement, qui fait d'abord, par son obéissance,	 
19	 	ce que par sa parole il exhorte à faire.