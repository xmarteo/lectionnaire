Mais les miracles de notre Seigneur	 
15	 	et Sauveur, voici comment il faut les comprendre, mes	 
16	 	frères : il faut à la fois y croire comme à des faits s'étant véri-	 
17	 	tablement produits et voir pourtant en eux des signes qui	 
18	 	nous suggèrent quelque chose. C'est que ses oeuvres sont en	 
19	 	même temps manifestation d'une puissance et expression	 
20	 	d'un mystère. Ainsi nous ignorons qui était historiquement	 
21	 	cet aveugle, mais nous savons néanmoins qui il désigne mys-	 
22	 	tiquement. L'aveugle, c'est le genre humain, qui chassé en	 
23	 	son premier père loin des joies du paradis, ignorant l'éclat	 
24	 	de la lumière d'en haut, pâtit des ténèbres auxquelles il est	 
25	 	condamné ; la présence de son Rédempteur l'éclaire pour-	 
26	 	tant, faisant que déjà par le désir il découvre les joies de la	 
27	 	lumière intérieure et s'engage sur le chemin de la vie par la	 
28	 	pratique du bien.