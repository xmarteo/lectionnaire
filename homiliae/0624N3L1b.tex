« Élisabeth mit au monde un fils. »	 
21	 	30. La naissance des saints est une	 
22	 	joie pour beaucoup, parce que c'est un	 
23	 	bien commun : car la justice est une vertu sociale. Aussi	 
24	 	à la naissance de ce juste voit-on déjà les marques de ce	 
25	 	que sera sa vie, et le charme qu'aura sa vertu est présagé	 
26	 	et signifié par l'allégresse des voisins.	 
27	 	Il est heureux que soit mentionné le temps passé par	 
28	 	le prophète au sein maternel, sans quoi la présence de Marie	 
29	 	n'eût pas été rapportée. Mais il n'est pas question du	 
30	 	temps de son enfance, car, la présence du Seigneur l'ayant	 
31	 	fortifié dès le sein de sa mère, il n'a pas connu les entraves	 
32	 	de l'enfance. Aussi ne lisons-nous dans l'Évangile rien	 
33	 	d'autre à son sujet que sa naissance et son témoignage :	 
34	 	son tressaillement au sein maternel, sa parole au désert.