Mais la foi droite de l’Église catholique condamne une telle fiction,
	et la reconnaît pour une doctrine diabolique:
	non seulement elle proclame avec assurance que c’est une erreur,
	mais elle cherche à en convaincre ceux qu’elle peut, par ses raisonnements.
Condamnons donc nous-mêmes cette erreur
	que la Sainte Église a frappée, dès le commencement, de ses anathèmes.
	Gardons-nous de penser que le Seigneur Jésus-Christ soit ce soleil
		que nous voyons se lever à l’orient et se coucher à l’occident,
	à la course duquel succède la nuit,
	dont les rayons sont obscurcis par les nuages,
	et qui, par sa révolution déterminée, passe d’un lieu dans un autre.
Non, ce n’est pas là le Christ, le Seigneur.
Le Christ n’est point ce soleil qui a été fait,
	mais il est celui par qui le soleil a été fait;
	car «par lui toutes choses ont été faites, et rien n’a été fait sans lui.»
