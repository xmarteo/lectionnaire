« Jean, dit-il, est son nom » ;	 
19	 	c'est-à-dire : ce n'est pas nous qui lui donnons un nom,	 
20	 	puisqu'il a déjà reçu de Dieu son nom. Il a son nom : nous	 
21	 	le reconnaissons, nous ne l'avons pas choisi. Les saints ont	 
22	 	ce privilège de recevoir de Dieu un nom ; ainsi Jacob est	 
23	 	appelé Israël parce qu'il a vu Dieu ; ainsi Notre Seigneur	 
24	 	a été appelé Jésus avant sa naissance ; ce n'est pas l'ange,	 
25	 	mais son Père qui Lui a imposé ce nom : « Mon fils Jésus,	 
26	 	est-il écrit, se manifestera avec ceux qui auront part à sa	 
27	 	joie, qui ont été réservés pour les quatre cents années. Et	 
28	 	voici qu'après ces années mon fils le Christ mourra et le	 
29	 	siècle se convertira » (IV Esdras, VII, 28-30). Vous le	 
30	 	voyez, les anges annoncent ce qu'ils ont entendu, non ce	 
31	 	qu'ils ont pris sur eux.	 
32	 	Ne soyez pas surpris si cette femme témoigne d'un nom	 
33	 	qu'elle n'avait pas entendu, puisque l'Esprit Saint, qui	 
34	 	l'avait confié à l'ange, le lui a révélé.