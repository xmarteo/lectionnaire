Mais celui qui	 
2	 	veut être libéré doit venir au temple, venir à Jérusalem,	 
3	 	attendre l'Oint du Seigneur, recevoir dans ses mains la	 
4	 	Parole de Dieu et comme l'étreindre dans les bras de sa foi.	 
5	 	Alors il sera libéré et ne verra point la mort, ayant vu la vie.
Vous voyez quelle abondance de grâce a répandue	 
7	 	sur tous la naissance du Seigneur, et comment la pro-	 
8	 	phétie est refusée aux incroyants (cf. I Cor., XIV, 22),	 
9	 	mais non pas aux justes. Voici qu'à son tour Siméon pro-	 
10	 	phétise que Notre Seigneur Jésus-Christ est venu pour la	 
11	 	ruine èt la résurrection d'un grand nombre, pour faire	 
12	 	entre justes et injustes le discernement des mérites et,	 
13	 	selon la valeur de nos actes, nous décerner, en juge véri-	 
14	 	dique et équitable, soit les supplices, soit les récompenses.