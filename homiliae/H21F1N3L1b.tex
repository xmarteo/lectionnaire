En Syrie et surtout en Palestine, c'est une habitude de	 
13	 	toujours mêler des paraboles à la conversation : ainsi, ce	 
14	 	que les auditeurs ne pourraient retenir d'un simple enseigne-	 
15	 	ment, ils le retiennent grâce à la comparaison et aux exemples.	 
16	 	Dans cette parabole du roi, le maître, et du serviteur qui	 
17	 	lui devait dix mille talents et par ses supplications avait	 
18	 	obtenu le pardon de son maître, le Seigneur a enseigné à	 
19	 	Pierre à pardonner lui aussi à ses compagnons d'esclavage	 
20	 	moins coupables. En effet, si ce roi et maître a remis si	 
21	 	aisément dix mille talents à son serviteur qui les lui devait,	 
22	 	combien plus les serviteurs doivent-ils remettre à leurs	 
23	 	compagnons des dettes moindres !