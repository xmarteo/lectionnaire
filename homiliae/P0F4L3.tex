On peut aussi se demander pourquoi, alors que ses dis-	 
25	 	ciples peinent sur la mer, le Seigneur se tient sur le rivage,	 
26	 	après sa résurrection, alors qu'avant sa résurrection, il a	 
27	 	devant ses disciples marché sur les flots de la mer. On	 
28	 	trouve vite la raison si l'on examine les conditions d'alors.	 
29	 	En effet, que désigne la mer, sinon le monde présent agité	 
 	--- 91 ---	 
1	 	par le mouvement des affaires et la houle d'une vie corrup-	 
2	 	tible ? Que désigne la stabilité du rivage, sinon la perpétuité	 
3	 	du repos éternel ? Alors les disciples, qui étaient encore au	 
4	 	milieu des flots de la vie mortelle, peinaient sur la mer ; notre	 
5	 	Rédempteur, lui, ayant déjà dépassé la condition d'une chair	 
6	 	corruptible, se tenait sur le rivage après sa résurrection.