« Et il y avait bien des lépreux au temps du prophète	 
7	 	Élisée, et nul d'entre eux ne fut guéri que Naaman le	 
8	 	Syrien. »	 
9	 	Il est clair que cette parole du Seigneur Sauveur nous	 
10	 	forme et nous exhorte au zèle à honorer Dieu ; elle montre	 
11	 	que nul n'est guéri et délivré de la maladie qui macule	 
12	 	sa chair s'il n'a recherché la santé avec un soin religieux :	 
13	 	car ce n'est pas aux dormeurs que les bienfaits divins	 
14	 	sont accordés, mais aux vigilants.
Nous avons dit dans un autre livre que cette veuve à qui	 
 	--- 171 ---	 
1	 	Élie fut envoyé préfigurait l'Église. Il convient que le	 
2	 	peuple vienne après l'Église. Ce peuple rassemblé d'entre	 
3	 	les étrangers, ce peuple jadis lépreux, ce peuple jadis	 
4	 	souillé avant d'être baptisé dans le fleuve mystérieux, ce	 
5	 	même peuple, après le mystère du baptême, lavé des souil-	 
6	 	lures du corps et de l'âme, commence à être non plus	 
7	 	lèpre, mais vierge sans tache et sans ride