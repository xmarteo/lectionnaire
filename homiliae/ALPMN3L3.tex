Mais voyons d'abord le plus développé.	 
26	 	« Bienheureux les pauvres en esprit, dit-Il, parce que	 
27	 	pour eux est le Royaume des cieux. » Cette béatitude a été	 
28	 	placée la première par l'un et l'autre évangéliste. Elle	 
29	 	est en effet la première selon l'ordre, et comme mère et	 
30	 	génératrice des vertus : car c'est en méprisant les biens	 
31	 	du monde qu'on méritera les éternels ; et nul ne saurait	 
32	 	obtenir la récompense du Royaume des cieux, si, cap-	 
33	 	tif de la convoitise de ce monde, il est incapable d'en	 
34	 	émerger.