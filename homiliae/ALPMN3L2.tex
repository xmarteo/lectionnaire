« Bienheureux les pauvres. »	 
15	 	S. Luc n'a noté que quatre béatitudes du Seigneur,	 
16	 	S. Matthieu huit ; mais dans les huit il y a les quatre, et	 
17	 	dans les quatre les huit. L'un s'est attaché aux quatre,	 
18	 	comme aux vertus cardinales ; l'autre a, dans huit, main-	 
19	 	tenu le nombre mystérieux : car beaucoup de psaumes	 
20	 	sont intitulés : pour l'octave ; et il vous est prescrit de	 
21	 	faire les parts pour huit, peut-être les Béatitudes (Eccl.,	 
22	 	XI, 2). De même, en effet, que l'octave est l'accom-	 
23	 	plissement de notre espérance, l'octave est aussi la somme	 
24	 	des vertus.