«Il sera	 
9	 	appelé Nazaréen.» S'il avait fait une citation précise des	 
10	 	Écritures, jamais l'évangéliste ne dirait : « Ce qui avait été	 
11	 	dit par les prophètes », mais simplement « ce qui avait été dit	 
12	 	par le prophète ». En fait, par l'emploi du pluriel, « les	 
13	 	prophètes », il montre qu'il a pris non la lettre des Écritures	 
14	 	mais leur sens. Nazaréen signifie « saint ». Le Seigneur sera	 
15	 	saint. C'est ce que rappelle toute l'Écriture.