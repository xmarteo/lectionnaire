Pourquoi l'officier royal lui demande-t-il de venir près de	 
5	 	son fils et Jésus refuse-t-il d'y aller corporellement, tandis	 
6	 	que, sans y être invité, il promet de se rendre corporelle-	 
7	 	ment auprès du serviteur du centurion ? Il ne daigne pas être	 
8	 	présent corporellement auprès du fils de l'officier royal et il	 
9	 	ne dédaigne pas d'accourir auprès du serviteur du centu-	 
10	 	rion. Qu'est-ce que cela veut dire, sinon qu'il réprime notre	 
11	 	orgueil, nous qui révérons chez les hommes, non pas leur	 
12	 	nature formée à l'image de Dieu, mais leurs dignités et leurs	 
13	 	richesses ? Attentifs à ce qui leur est extérieur, nous ne	 
14	 	voyons guère d'abord ce qui leur est intérieur : nous regar-	 
15	 	dons dans leur corps ce qui est méprisable, et négligeons de	 
16	 	considérer ce qu'ils sont. Aussi, pour montrer que les saints	 
17	 	doivent mépriser ce qui est élevé aux yeux des hommes et	 
18	 	ne pas mépriser ce que ceux-ci méprisent, notre Rédempteur	 
19	 	refusa d'aller chez le fils de l'officier royal et fut tout prêt	 
20	 	à aller auprès du serviteur du centurion.