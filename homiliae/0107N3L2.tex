A ce propos, il faut savoir que les priscillianistes, héré-	 
13	 	tiques, pensent que tout homme naît soumis à une conjonc-	 
14	 	tion d'étoiles ; et à l'appui de leur erreur ils invoquent le fait	 
15	 	qu'une étoile nouvelle s'est levée lorsque le Seigneur apparut	 
16	 	dans la chair. Pour eux, cette étoile qui apparut était son des-	 
17	 	tin. Mais pesons les mots de l'Évangile, qui dit de l'étoile :	 
18	 	« jusqu'à ce qu'elle vint s'arrêter au-dessus du lieu où était	 
19	 	l'enfant ». Ce n'est pas l'enfant qui courut vers l'étoile, mais	 
20	 	l'étoile vers l'enfant, et donc, si l'on peut parler ainsi, l'étoile	 
21	 	n'a pas été le destin de l'enfant, mais le destin de l'étoile a été	 
22	 	cet enfant qui apparut.