L’une est donc bonne, mais malheureuse,
	l’autre est meilleure et bienheureuse;
	la première a été figurée par l’apôtre Pierre,
	la seconde par l’apôtre Jean.
L’une s’écoule tout entière ici-bas,
	elle s’étendra jusqu’à la fin des temps et y trouvera son terme;
	l’autre ne recevra sa perfection qu’à la consommation des siècles,
	mais dans le siècle futur elle n’aura pas de fin.
Aussi Jésus dit-il à l’un «Suis-moi»,
	et à l’autre: «Je veux qu’il demeure ainsi jusqu’à ce que je vienne;
	que t’importe? Suis-moi».
Que veulent dire ces paroles?
À mon sens, à mon avis, elles n’ont pas d’autre signification que celle-ci:
	Suis-moi en m’imitant, en supportant comme moi les épreuves de la vie;
	pour lui, qu’il demeure jusqu’au moment
		où je viendrai mettre les hommes en possession des biens éternels.
