Ces paroles du Sauveur: «Je suis la lumière du monde»,
	semblent assez claires pour ceux qui ont des yeux
		à l’aide desquels on peut contempler cette lumière:
	ceux, au contraire, qui n’ont d’autres yeux que les yeux de leur corps,
	s’étonnent d’entendre ces paroles: «Je suis la lumière du monde»,
		sortir de la bouche de Notre-Seigneur Jésus-Christ.
Il en est, sans doute, plus d’un pour se dire à lui-même:
	Le Seigneur Jésus serait-il ce soleil,
	dont le lever et le coucher forment la mesure de nos jours?
Plusieurs hérétiques l’ont pensé:
	en effet, les Manichéens voyaient la personnification du Christ
		dans cet astre dont les rayons frappent nos regards,
	et qui, placé au centre du monde,
	sert à tous, aux hommes et aux animaux, pour se conduire.
