Il nous est facile, mes frères,
	de comprendre à quel suréminent et admirable degré le Sauveur a montré de la douceur.
Ses ennemis remarquèrent en lui une trop grande douceur, une trop grande bonté;
	car, longtemps auparavant, le Prophète avait dit de lui:
	«Armez-vous de votre glaive, ô le plus puissant des rois;
	revêtez-vous de votre gloire et de votre éclat;
	montez sur le char de la vérité, de la clémence et de la justice».
En qualité de docteur, il a apporté sur la terre la vérité;
	comme libérateur, la douceur;
	en tant que sondant les consciences, la justice.
Voilà pourquoi Isaïe avait annoncé d’avance qu’il régnerait dans l’Esprit-Saint.
Quand il parlait, la vérité se reconnaissait dans ses discours,
	et s’il ne s’élevait pas contre ses ennemis, on ne pouvait qu’admirer sa mansuétude.
En face de ces deux vertus de Jésus-Christ, de sa vérité et de sa douceur,
	ses ennemis se sentaient tourmentés par l’envie et la malignité jalouse;
mais sa troisième qualité, la justice,
	fut pour eux un véritable sujet de scandale.
