Ils ne pouvaient être	 
32	 	étrangers à la charité, puisque la Vérité marchait avec eux,	 
33	 	et ils l'invitent à être leur hôte comme on le fait pour un	 
34	 	étranger. Pourquoi disons-nous : « ils l'invitent », alors qu'il	 
 	--- 79 ---	 
1	 	est écrit : « ils le pressèrent » ? De cet exemple on peut	 
2	 	conclure qu'il ne faut pas seulement inviter les étrangers	 
3	 	comme hôtes, mais les entraîner. Les disciples mettent la	 
4	 	table, apportent les mets, et reconnaissent dans la fraction	 
5	 	du pain le Dieu qu'ils n'avaient pas reconnu quand il expli-	 
6	 	quait la sainte Écriture.
En écoutant les commandements de Dieu, ils n'ont pas	 
8	 	été éclairés ; en les pratiquant, ils ont été éclairés. Car il est	 
9	 	écrit : « Ce ne sont pas ceux qui écoutent la Loi qui sont	 
10	 	justes devant Dieu, mais ceux qui la mettent en pratique	 
11	 	seront justifiés. » Celui donc qui veut comprendre ce qu'il	 
12	 	a entendu, qu'il se hâte d'accomplir ce qu'il a déjà pu com-	 
13	 	prendre. Le Seigneur n'a pas été reconnu pendant qu'il par-	 
14	 	lait, mais a daigné se faire reconnaître pendant le repas	 
15	 	offert.