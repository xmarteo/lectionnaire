Mais comme ils ne pouvaient pas être étrangers à la charité,
	ceux que la Vérité accompagnait dans leur marche,
	ils lui offrent l’hospitalité comme à un étranger.
Mais pourquoi disons-nous, ils offrent,
	alors que le texte porte: ils le forcèrent?
Par cet exemple
		on apprend qu’on doit non seulement offrir l’hospitalité aux étrangers,
	mais qu’il faut les presser d’accepter.
Ils dressent donc la table, offrent des pains et différents mets,
	et le Dieu qu’ils n’avaient pas reconnu
		lorsqu’il leur expliquait l’Écriture,
	ils le reconnaissent à la fraction du pain.
En écoutant les préceptes de Dieu, ils n’ont pas été illuminés;
	c’est en les pratiquant, qu’ils ont reçu la lumière;
	car il est écrit:
	«Ce ne sont pas les auditeurs de la loi qui sont justes auprès de Dieu,
	mais ceux qui la pratiquent seront justifiés.»
Que celui-là donc qui veut comprendre ce qu’il a entendu
	se hâte de mettre en pratique ce qu’il a pu déjà entendre.
Voici que le Seigneur, qui n’a pas été reconnu tandis qu’il parlait,
	a daigné se faire connaître pendant qu’on lui donne à manger.
