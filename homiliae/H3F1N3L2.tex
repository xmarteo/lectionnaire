Car les justes s'indignent, mais sans indignation ; ils	 
28	 	désespèrent, mais sans désespérer ; ils s'acharnent, mais c'est	 
29	 	par amour, car même si, dans leur zèle, ils multiplient	 
30	 	au-dehors les reproches, ils gardent au-dedans la douceur de	 
31	 	la charité. Dans leur estime, ils mettent souvent au-dessus	 
32	 	d'eux ceux qu'ils corrigent et ils considèrent comme	 
 	--- 327 ---	 
1	 	meilleurs qu'eux ceux qu'ils jugent. Ce faisant, ils gardent	 
2	 	leurs inférieurs par la discipline et se gardent eux-mêmes par	 
3	 	l'humilité.