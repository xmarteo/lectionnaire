Car les délices spirituelles, en rassasiant l'âme,	 
17	 	y font grandir le désir, parce que plus on goûte leur saveur,	 
18	 	plus on connaît ce qu'il faut aimer davantage. Si on ne peut	 
19	 	les aimer quand on ne les possède pas, c'est qu'on en ignore	 
20	 	la saveur. Qui pourrait, en effet, aimer ce qu'il ignore ? Aussi	 
21	 	le psalmiste nous avertit en disant : « Goûtez et voyez comme	 
22	 	est bon le Seigneur », comme pour dire clairement : Vous ne	 
23	 	connaissez pas sa douceur, si vous ne la goûtez pas. Mais goû-	 
24	 	tez l'aliment de vie avec le palais de votre coeur, afin qu'en	 
25	 	éprouvant sa douceur, vous puissiez l'aimer. Ces délices,	 
26	 	l'homme les a perdues, lorsqu'il a péché au paradis ; il est sorti	 
27	 	de ce lieu, il a fermé sa bouche à l'éternelle douceur de cet ali-	 
28	 	ment.