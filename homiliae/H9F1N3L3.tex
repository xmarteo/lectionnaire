Le miséricordieux Rédempteur pleura la	 
4	 	ruine de la cité infidèle, avant même que cette cité connaisse	 
5	 	que sa ruine était proche. Le Seigneur dit bien en pleurant :	 
6	 	« Si tu avais connu, toi aussi… » ; sous-entendez : Tu aurais	 
7	 	pleuré, toi qui es maintenant dans la joie parce que tu	 
8	 	ignores ce qui est imminent. Ainsi poursuit-il : « En ce jour	 
9	 	qui est le tien, ce qui t'apporterait la paix. » Car tandis que	 
10	 	cette ville s'adonnait aux plaisirs des sens et ne voyait pas	 
11	 	les maux à venir, elle possédait, en ce jour qui était le sien,	 
12	 	ce qui pouvait lui apporter la paix.