Il faut également noter ceci :	 
31	 	nous lisons qu'Élie s'est élevé dans un char, pour montrer	 
32	 	qu'en homme ordinaire, il avait besoin d'une aide extérieure.	 
33	 	Ce sont les anges qui lui ont donné cette aide, parce qu'il	 
34	 	ne pouvait monter par lui-même en notre ciel, alourdi	 
 	--- 211 ---	 
1	 	qu'il était par le poids de la faiblesse naturelle. Notre	 
2	 	Rédempteur, lui, nous ne lisons pas qu'il fut élevé au ciel	 
3	 	dans un char ou avec l'aide des anges, car le Créateur de	 
4	 	toutes choses pouvait par sa propre puissance s'élever au-	 
5	 	dessus de tout. Il retournait, en effet, là où il était, et il reve-	 
6	 	nait d'un lieu où il restait, puisqu'en montant au ciel dans	 
7	 	son humanité, par sa divinité il contenait également la terre	 
8	 	et le ciel.