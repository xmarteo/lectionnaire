C'est à bon droit, certes, que,	 
8	 	lors de sa circoncision, « l'enfant né	 
9	 	pour nous » est appelé du nom de Sauveur. Car dès ce	 
10	 	moment, il commence déjà « à réaliser notre salut » « en	 
11	 	répandant pour nous son sang immaculé ». Les chrétiens	 
12	 	n'ont plus désormais à se demander pourquoi le Christ Sei-	 
13	 	gneur a voulu être circoncis. La raison de sa circoncision,	 
14	 	en effet, est la même que celle de sa naissance, la même	 
15	 	que celle de sa Passion. Rien de tout cela n'était en sa	 
16	 	faveur à lui. « Tout était en faveur des élus. » Lui, il n'a	 
17	 	pas été engendré dans le péché, il n'a pas été circoncis	 
18	 	du péché, il n'est pas mort pour son propre péché, mais	 
19	 	bien plutôt « à cause de nos fautes à nous ». « Il est appelé	 
20	 	du nom de Sauveur, comme l'ange l'a appelé avant sa	 
21	 	conception. » Oui, il l'a appelé ainsi, il ne lui a pas donné	 
22	 	ce nom. Car « ce nom lui appartient de toute éternité ».	 
23	 	C'est de sa propre naissance qu'il tient d'être Sauveur. Ce	 
24	 	nom lui est inné, il ne lui est pas donné par une créature,	 
25	 	qu'elle soit humaine ou angélique.