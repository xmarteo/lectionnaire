Mais, dira quelqu’un, comment Lazare peut-il figurer le pécheur,
	puisque le Seigneur l’aimait si tendrement?
Que celui-là écoute le Seigneur, car il nous dit:
	«Je ne suis pas venu appeler les justes, mais les pécheurs».
Si Dieu n’avait pas aimé les pécheurs,
	il ne serait pas descendu du haut du ciel sur la terre.
«Or, Jésus, entendant cela, leur dit:
	Cette maladie ne va pas à la mort, mais elle est pour la gloire de Dieu,
	afin que le Fils de Dieu soit glorifié.»
Cette glorification du Fils ne l’a pas grandi;
	c’est à nous qu’elle a profité.
Il dit donc: «Cette maladie ne va pas à la mort»,
	parce que la mort même de Lazare n’allait pas à la mort,
	mais bien plutôt au miracle qu’il devait faire
		pour amener les hommes à croire en Jésus-Christ,
	et à éviter la mort éternelle.
Remarquez comme Notre-Seigneur affirme indirectement qu’il est Dieu,
	à cause de quelques-uns qui disent que le Fils n’est pas Dieu.
