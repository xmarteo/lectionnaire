Notre orgueil a	 
21	 	donc été blâmé, car il ne sait pas estimer les hommes parce	 
22	 	qu'ils sont des hommes ; il apprécie seulement, comme nous	 
23	 	l'avons dit, ce qui est extérieur à l'homme, ne considère pas	 
24	 	sa nature, ne reconnaît pas chez lui la dignité divine. Voici	 
25	 	que le Fils de Dieu refuse d'aller auprès du fils d'un officier	 
26	 	royal et qu'il est prêt à venir sauver un serviteur. Certes, si	 
27	 	un serviteur nous demandait d'aller à lui, aussitôt notre	 
28	 	orgueil nous adresserait en secret cette réponse : « Tu n'iras	 
29	 	pas, tu te rabaisses, on oublie ta dignité, tu déchois de ton	 
30	 	rang. » Voici que, venu du ciel, quelqu'un n'a pas dédaigné	 
31	 	d'accourir sur la terre auprès d'un serviteur, et nous, êtres	 
32	 	faits de la terre, nous répugnons pourtant à être humiliés sur	 
 	--- 193 ---	 
1	 	la terre.