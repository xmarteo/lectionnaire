Alors s'accomplit la parole du prophète Jérémie qui	 
2	 	dit : « Une voix a été entendue dans Rama, bien des pleurs et	 
3	 	des cris de douleur. C'est Rachel qui pleure ses enfants. » De	 
5	 	Rachel naquit Benjamin. Or Bethléem ne se trouve point dans	 
6	 	sa tribu. Une question se pose donc : Pourquoi Rachel pleure-	 
7	 	t-elle comme siens les fils de Juda, c'est-à-dire de Bethléem ?	 
8	 	Nous répondrons rapidement : elle fut ensevelie près de	 
9	 	Bethléem à Ephrata : c'est parce que ce lieu donna l'hospitalité	 
10	 	dans son sein à sa dépouille mortelle, qu'elle a reçu ce nom de	 
11	 	mère. Autre explication : Juda et Benjamin étaient deux	 
12	 	tribus limitrophes et Hérode avait ordonné de tuer les enfants	 
13	 	non seulement de Bethléem mais de tous les environs.