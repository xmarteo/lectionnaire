Que représente, frères très chers, cet homme riche	 
22	 	qui s'habillait de pourpre et de lin fin et faisait chaque jour	 
23	 	somptueuse chère, sinon le peuple juif, qui avait extérieure-	 
24	 	ment le souci d'une belle vie et profitait des délices de la Loi	 
25	 	pour son éclat, non pour son utilité ? Et de qui Lazare cou-	 
26	 	vert d'ulcères est-il la figure, sinon du peuple des Gentils ?	 
27	 	Quand, en revenant à Dieu, il n'a pas rougi d'avouer ses	 
28	 	péchés, une plaie s'ouvrait en lui dans sa peau. Or, par la	 
29	 	plaie de la peau, l'infection venue des viscères est rejetée au	 
30	 	dehors.