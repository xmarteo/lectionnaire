S'il omet tous les autres	 
15	 	pour l'appeler fils de ceux-là, c'est parce qu'à eux seuls fut	 
16	 	faite la promesse au sujet du Christ. A Abraham, il a été	 
17	 	dit : « Toutes les nations seront bénies en ta postérité »,	 
18	 	c'est-à-dire le Christ. Et à David : « C'est un fruit de tes	 
19	 	entrailles que je mettrai sur ton trône. »
Or, de Thamar, Juda engendra Pharès et Zara. Notons-le,	 
21	 	la généalogie du Sauveur ne comporte la mention d'aucune	 
22	 	sainte femme, mais seulement de celles que blâme l'Écriture.	 
23	 	Ainsi, il veut montrer que, celui qui était venu pour les	 
24	 	pécheurs, descendant de pécheresses, effacerait les péchés de	 
25	 	tous. Aussi dans la suite, est-il parlé de Ruth, la moabite,	 
26	 	et de Bethsabée, l'épouse d'Urie.