Jésus-Christ s’est fait infirme pour nourrir des infirmes,
	pareil en cela à la poule qui nourrit ses poussins;
	c’est la comparaison qu’emploie le Sauveur lui-même.
«Combien de fois», dit-il à Jérusalem, «j’ai voulu rassembler tes enfants,
	comme une poule ramasse ses petits sous ses ailes,
	et tu ne l’as pas voulu!»
Vous savez, mes frères, comme une poule se fait petite par amour pour ses petits;
	de tous les oiseaux, elle est la seule qui se montre véritablement mère.
Nous voyons les passereaux faire leur nid sous nos yeux;
	il en est de même des hirondelles, des cigognes, des pigeons;
	mais nous ne nous apercevons qu’ils ont des petits
		qu’au moment où nous les voyons dans leurs nids.
Pour la poule, elle se fait si petite pour ses petits que,
	même lorsqu’ils en sont éloignés et même sans qu’on les voie,
	on reconnaît qu’elle est mère.
