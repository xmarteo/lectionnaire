Le passage du saint Évangile que vous venez d'en-	 
21	 	tendre, frères, est au premier abord le récit très clair d'un	 
22	 	fait, mais il nous faut brièvement chercher les mystères qu'il	 
23	 	cache. Marie-Madeleine, alors qu'il faisait encore sombre,	 
24	 	vint au tombeau. Le récit note une heure ; pour le sens	 
25	 	caché, il signifie l'état d'esprit de celle qui cherche. Marie	 
26	 	cherchait au tombeau le créateur de l'univers, qu'elle avait	 
27	 	vu mort dans la chair, et comme elle ne le trouvait pas, elle	 
28	 	crut qu'on l'avait volé. Il faisait donc encore sombre quand	 
29	 	elle vint au tombeau. Bien vite, elle courut annoncer la	 
 	--- 49 ---	 
1	 	nouvelle aux disciples. Or, ceux-là coururent plus vite que	 
2	 	les autres qui aimèrent plus que les autres, à savoir Pierre et	 
3	 	Jean.