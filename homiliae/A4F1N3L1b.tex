« Jean disait aux foules qui venaient se faire bap-	 
5	 	tiser par lui : Engeance de vipères, qui vous a appris à fuir	 
6	 	la colère qui vient ? » La colère qui vient, c'est l'assignation	 
7	 	au châtiment ultime, qu'alors le pécheur ne pourra fuir s'il	 
8	 	ne recourt maintenant aux gémissements du repentir. Il faut	 
9	 	noter que les mauvais rejetons de mauvais parents dont ils	 
10	 	imitent les actions, sont appelés engeance de vipères, parce	 
11	 	qu'en haïssant les bons et en les persécutant, en rendant à	 
12	 	certains le mal pour le bien, en cherchant tous les moyens	 
13	 	de blesser leur prochain, suivant en tout cela les chemins de	 
14	 	leurs aînés selon la chair, ils sont comme des enfants enve-	 
15	 	nimés de parents envenimés.