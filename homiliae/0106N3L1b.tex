Comme vous l'avez entendu à la lecture de l'Évangile,	 
6	 	frères, quand naquit le roi du ciel, un roi de la terre fut trou-	 
7	 	blé. Il n'est pas surprenant qu'une grandeur terrestre soit	 
8	 	confondue quand se découvre la majesté céleste. Mais il nous	 
9	 	faut chercher pourquoi, à la naissance du Rédempteur, c'est un	 
10	 	ange qui apparut aux bergers de Judée, alors qu'une étoile et	 
11	 	non un ange conduisit des mages de l'Orient jusqu'à lui, pour	 
12	 	qu'ils l'adorent. Voici, semble-t-il, le motif : aux Juifs, qui	 
13	 	usaient de leur raison, c'est un être raisonnable, un ange, qui	 
14	 	devait faire l'annonce ; les Gentils par contre, qui ne savaient	 
15	 	pas user de leur raison, sont amenés à connaître le Seigneur non	 
16	 	par une voix, mais par des signes. Car aux Juifs les prophéties	 
17	 	ont été données comme à des croyants, non comme à des	 
18	 	incroyants, et aux Gentils les signes, comme à des incroyants,	 
19	 	non comme à des croyants.