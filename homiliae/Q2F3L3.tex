« Ils font bien larges leurs phylactères et bien longues	 
26	 	leurs franges. Ils recherchent les premières places dans les festins,	 
27	 	les premières chaires dans les synagogues, ils aiment à recevoir	 
28	 	les salutations sur la place publique, et à s'entendre appeler	 
29	 	Rabbi par les gens. » Malheur à nous, misérables, héritiers	 
30	 	des vices des Pharisiens. Lorsque, par l'intermédiaire de	 
31	 	Moïse, le Seigneur eut donné les prescriptions de sa Loi,	 
32	 	il ajouta pour finir : « Tu les attacheras sur ta main et ils	 
 	--- 165 ---	 
1	 	demeureront toujours devant tes yeux. » Et voici le sens :	 
2	 	que mes préceptes soient sur ta main pour qu'ils soient	 
3	 	pratiqués dans tes actes, devant tes yeux pour que tu les	 
4	 	médites jour et nuit. Par suite d'une fausse interprétation,	 
5	 	les Pharisiens écrivaient le Décalogue de Moïse, c'est-à-dire	 
6	 	les dix commandements de la Loi sur des bandelettes de	 
7	 	parchemin, les repliaient, les attachaient sur leur front,	 
8	 	s'en faisaient pour ainsi dire une couronne sur leur tête	 
9	 	pour les porter toujours devant les yeux.