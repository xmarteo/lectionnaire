La lenteur des disciples à croire à la résurrection du	 
6	 	Seigneur ne fut pas tant le signe de leur faiblesse que, si je	 
7	 	puis dire, le gage de notre fermeté future. Comme ils dou-	 
8	 	taient, la résurrection leur a été certifiée par des preuves	 
9	 	nombreuses, et quand nous en prenons connaissance par la	 
10	 	lecture, ne sommes-nous pas fortifiés du fait de leur incré-	 
11	 	dulité même ? Marie-Madeleine, qui a cru plus vite, m'a été	 
12	 	moins utile que Thomas, qui a douté longtemps. Doutant,	 
13	 	il a touché les cicatrices des blessures du Seigneur et enlevé	 
14	 	de notre coeur la blessure du doute.