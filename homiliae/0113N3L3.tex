Les saints eux-mêmes, comprenez bien ceci, mes frères,
	les gens de bien qui appartiennent à la colombe,
	les citoyens de la sainte Jérusalem,
	les gens de bien qui font partie de l’Église,
	ceux dont l’Apôtre dit: «Le Seigneur connaîtra ceux qui sont à lui»,
	ont reçu des grâces différentes, tous n’ont pas les mêmes mérites;
	il en est qui sont plus saints et meilleurs que d’autres.
Comment donc, par exemple, si l’un est baptisé par un ministre juste et saint,
	l’autre par un ministre inférieur en mérites auprès de Dieu,
	inférieur en élévation, en continence, en sainteté de vie,
	comment tous deux cependant reçoivent ils une même et pareille grâce,
	une grâce égale en l’un et en l’autre,
	sinon parce que « c’est Celui-là qui baptise»? 
