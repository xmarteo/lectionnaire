«Vous me chercherez, et vous ne me trouverez pas,
	et, là où je suis, vous ne pouvez venir».
C’était là prédire déjà sa résurrection:
	ils n’ont pas voulu le reconnaître quand il était au milieu d’eux,
	et plus tard, lorsqu’ils virent que la multitude croyait en lui,
	ils le cherchèrent.
De grands prodiges eurent lieu,
	même au moment de la résurrection du Sauveur et de son ascension:
	alors ses disciples opérèrent des miracles éclatants,
	mais ils n’étaient que les instruments de Celui qui en avait tant fait lui-même,
	car il leur avait dit: «Vous ne pouvez rien faire sans moi».
Lorsque le boiteux qui se tenait à la porte du temple,
	se leva à la voix de Pierre, et marcha sur ses pieds,
	tous furent dans l’admiration:
	alors, le prince des Apôtres leur adressa la parole,
	et leur déclara que s’il avait guéri cet homme,
	ce n’était point en vertu de son propre pouvoir,
	mais que c’était par la puissance de Celui qu’ils avaient fait mourir.
Saisis de douleur, plusieurs lui répondirent:
	«Que ferons-nous?»
