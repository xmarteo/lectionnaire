« Et voici qu'un lépreux venait l'adorer en disant. » C'est	 
27	 	bien à propos qu'après l'exposé de la doctrine se présente	 
28	 	l'occasion d'un miracle pour que son pouvoir de faire des	 
29	 	miracles confirme auprès de ses auditeurs les propos qu'il	 
30	 	vient de tenir.	 
 	--- 155 ---	 
1	 	« Seigneur, si tu le veux, tu peux me guérir. » Celui qui solli-	 
2	 	cite sa volonté ne doute pas de son pouvoir.
Jésus, étendant la main, le toucha et dit : « Je le veux,	 
4	 	sois guéri. » Le Seigneur étend la main et, aussitôt, la lèpre	 
5	 	disparaît. En même temps, considère combien la réponse	 
6	 	est humble, sans ostentation. L'autre avait dit : « Si tu	 
7	 	veux », le maître répondit : « Je veux. » Il avait commencé :	 
8	 	« tu peux me guérir », le maître poursuit et dit : « Sois guéri. »	 
9	 	Contrairement à ce que pensent beaucoup de Latins, il ne faut	 
10	 	donc pas enchaîner et dire : « Je veux que tu sois guéri »,	 
11	 	mais couper de façon qu'il dise tout d'abord : « Je veux »,	 
12	 	et qu'ensuite il donne l'ordre : « Sois guéri ».