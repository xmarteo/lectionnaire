« Et	 
7	 	moi », dit-il, « je te dis » ; ce qui signifie : de même qu'à toi	 
8	 	mon Père a manifesté ma divinité, ainsi moi, à toi, je fais	 
9	 	connaître ton excellence : « Car tu es Pierre », c'est-à-dire :	 
10	 	bien que je sois, moi, la pierre indestructible, moi, la	 
11	 	pierre angulaire « qui, des deux, ne fais qu'un seul »,	 
12	 	bien que je sois le fondement en dehors duquel nul n'en	 
13	 	peut poser d'autre, toi aussi, cependant, tu es pierre,	 
14	 	car ma force t'affermit, en sorte que ce qui m'appartient	 
15	 	en propre par puissance te soit commun avec moi par	 
16	 	participation. « Et, sur cette pierre, je bâtirai mon Église	 
17	 	et les portes de l'enfer ne prévaudront point contre elle. »	 
18	 	Sur la solidité de ce fondement, dit-il, je construirai	 
19	 	un temple éternel et mon Église, dont la hauteur doit être	 
20	 	introduite au ciel, s'élèvera sur la fermeté de cette foi.