Notre Seigneur et Rédempteur nous annonce les	 
8	 	maux qui doivent précéder la fin du monde, afin qu'ils nous	 
9	 	troublent d'autant moins que nous les aurons mieux connus	 
10	 	par avance. Car les traits que l'on voit venir frappent moins	 
11	 	fort, et il nous est plus facile de supporter les maux du	 
12	 	monde, si nous sommes munis contre eux du bouclier de la	 
13	 	prévoyance. Voici ce que nous dit le Seigneur : « Quand	 
14	 	vous entendrez parler de guerres et de séditions, n'ayez pas	 
15	 	peur : il faut que cela arrive d'abord, mais ce ne sera pas aus-	 
16	 	sitôt la fin. » Nous devons méditer les paroles de notre	 
17	 	Rédempteur, qui nous annoncent des épreuves soit au-	 
18	 	dedans, soit au-dehors. Car les guerres supposent des enne-	 
19	 	mis, les séditions, des concitoyens. Aussi, afin de nous	 
20	 	indiquer que nous serons troublés au-dedans comme au-	 
21	 	dehors, il dit que nous souffrirons des maux différents de	 
22	 	nos ennemis et de nos frères.