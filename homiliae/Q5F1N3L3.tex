Mais ce que la	 
33	 	Vérité dit des réprouvés, les réprouvés le font voir eux-	 
34	 	mêmes par leurs oeuvres. Car le texte ajoute : « Les Juifs	 
 	--- 409 ---	 
1	 	répondirent ces mots : N'avons-nous pas raison de dire que	 
2	 	tu es un Samaritain et que tu as un démon ? »

Écoutons ce que Dieu répond, après avoir reçu une	 
4	 	telle injure : « Moi, je n'ai pas de démon, mais j'honore mon	 
5	 	Père, et vous, vous m'insultez. ». Le mot Samaritain veut	 
6	 	dire gardien, et celui-là est un vrai gardien, de qui le psal-	 
7	 	miste dit : « Si le Seigneur ne garde la cité, en vain veillent	 
8	 	les gardes » ; et à qui il est dit par Isaïe : « Garde, où en est	 
9	 	la nuit ? Garde, où en est la nuit ? ». Le Seigneur n'a pas	 
10	 	voulu répondre : Je ne suis pas un Samaritain, mais : « Moi,	 
11	 	je n'ai pas de démon ». Car deux injures lui avaient été	 
12	 	adressées. Il a repoussé l'une, accepté l'autre par son silence.