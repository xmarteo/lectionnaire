	Il est bien qu'abandonnant les Juifs pour habiter dans	 
9	 	le coeur des Gentils, le Seigneur monte au Temple : tel	 
10	 	est en effet le temple véritable, où le Seigneur est adoré	 
11	 	non pas selon la lettre, mais en esprit (Jn, IV, 24) ; c'est	 
12	 	le temple de Dieu, reposant sur l'appareil de la foi, non	 
13	 	sur des assises de pierre. Ainsi donc abandon de ceux qui	 
14	 	haïssent, élection de ceux qui allaient aimer. 2. Et s'Il	 
15	 	vient au mont des Oliviers, c'est afin de planter par la	 
16	 	vertu d'en haut les jeunes oliviers (Ps. 127, 3), dont la	 
17	 	mère est « la Jérusalem d'en haut » (Gal., IV, 26). Sur	 
18	 	cette montagne se tient le céleste jardinier : si bien que,	 
19	 	tous étant plantés dans la maison de Dieu (Ps. 91, 14),	 
20	 	chacun pourra dire pour son compte : « Pour moi, je suis	 
21	 	comme un olivier fertile dans la maison du Seigneur »	 
22	 	(Ps. 51, 10).