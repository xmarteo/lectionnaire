Vous avez entendu, frères très chers, dans la lecture de	 
5	 	l'Évangile, une leçon pour vous ; vous avez entendu dans la	 
6	 	lecture de l'Évangile ce qui nous menace nous. Voici ce que	 
7	 	dit en effet celui qui n'est pas bon par un don qu'il reçoit,	 
8	 	mais dont l'être même est bonté : « Je suis le bon pasteur ».	 
9	 	Et il indique une forme de cette bonté que nous puissions	 
10	 	imiter, en ajoutant : « Le bon pasteur donne sa vie pour ses	 
11	 	brebis ». Ce qu'il a recommandé, il l'a fait ; ce qu'il a	 
12	 	ordonné, il en a donné l'exemple. Le bon pasteur a donné	 
13	 	sa vie pour ses brebis au point de changer dans notre sacre-	 
14	 	ment son corps et son sang et de rassasier de sa chair deve-	 
15	 	nue nourriture les brebis qu'il avait rachetées.