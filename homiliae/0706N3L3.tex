Ainsi, pendant que le Seigneur	 
18	 	s'attarde sur le sommet de la montagne, s'élève soudain un	 
19	 	vent contraire qui soulève la mer. Voici les apôtres en péril,	 
20	 	et, jusqu'à la venue de Jésus, le naufrage ne cesse de demeurer	 
21	 	imminent.
A la quatrième veille de la nuit, il vint à eux en mar-	 
23	 	chant sur la mer. Les gardes et les veilles des soldats sont	 
24	 	divisées en durées de trois heures. Donc lorsque l'Évangéliste	 
25	 	dit que le Seigneur vint à eux à la quatrième veille de la	 
26	 	nuit, il montre qu'ils ont été en péril durant toute la nuit et	 
27	 	qu'il viendra à leur secours à la fin de la nuit et à la consom-	 
28	 	mation du monde.