Il est des hommes qui, aimant les biens terrestres plus	 
2	 	que les brebis, perdent la dignité de pasteurs. A leur sujet il	 
3	 	est ajouté tout de suite : « Le mercenaire, qui n'est pas le pas-	 
4	 	teur, et à qui n'appartiennent pas les brebis, voit-il venir le	 
5	 	loup, laisse les brebis et s'enfuit. » Il n'est pas appelé pas-	 
6	 	teur, mais mercenaire, celui qui ne paît pas les brebis du	 
7	 	Seigneur en les aimant d'amour, mais pour des avantages	 
8	 	temporels. Oui, mercenaire, celui qui occupe bien de fait la	 
9	 	place du pasteur, mais ne cherche pas le profit des âmes ; il	 
10	 	est avide d'avantages terrestres, heureux de la dignité de sa	 
11	 	charge, repu de profits temporels, ravi de la déférence	 
12	 	qu'ont pour lui les hommes.