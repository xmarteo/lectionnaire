Il en est qui, aimant les biens de la terre plus que leurs brebis,
	ne méritent plus le nom de pasteurs.
À leur sujet l’Évangile ajoute aussitôt:
	«Mais le mercenaire et celui qui n’est pas le pasteur,
	celui dont les brebis ne sont pas le bien propre,
	voyant venir le loup, laisse là les brebis et s’enfuit.»
On n’appelle point pasteur, mais mercenaire,
	celui qui fait paître les brebis du Seigneur
	dans l’espoir des récompenses temporelles,
	et non par le motif d’un amour profond.
Car il est mercenaire, celui qui tient la place de pasteur,
	mais ne cherche pas le bien des âmes,
	aspire après les commodités terrestres,
	se réjouit de l’honneur que lui donne sa charge,
	se nourrit des biens temporels
	et se délecte des égards que les hommes ont pour lui.
