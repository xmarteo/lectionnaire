Habaquq, lui aussi, parle de cette gloire de son	 
13	 	ascension : « Le soleil s'est élevé et la lune s'est tenue à son	 
14	 	rang. » Qui désigne-t-il sous le nom de soleil, sinon le	 
15	 	Seigneur ? Et sous le nom de lune, sinon l'Église ? Car avant	 
16	 	la montée du Seigneur dans les cieux, sa sainte Église avait	 
17	 	redouté de toute façon l'hostilité du monde ; une fois fortifiée	 
18	 	par son ascension, elle prêcha ouvertement ce qu'elle avait cru	 
19	 	en secret. Le soleil s'est donc élevé et la lune s'est tenue à son	 
20	 	rang. Oui, lorsque le Seigneur a gagné le ciel, sa sainte Église	 
21	 	a grandi en autorité dans sa prédication.