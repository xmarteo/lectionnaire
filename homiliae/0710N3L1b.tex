Le passage du saint Évangile qui a été lu est court,	 
5	 	frères très chers, mais lourd d'un grand poids de mystères.	 
6	 	Car Jésus, notre Créateur et Rédempteur, fait semblant de	 
7	 	ne pas reconnaître sa mère ; qui est pour lui sa mère, qui	 
8	 	sont ses proches, il le définit non par la parenté de la chair,	 
9	 	mais par les liens de l'esprit : « Qui est ma mère ? Qui sont	 
10	 	mes frères ? Quiconque fait la volonté de mon Père qui est	 
11	 	dans les cieux, c'est lui qui est mon frère, ma soeur, ma	 
12	 	mère ». Que veut-il nous faire comprendre par ces paroles	 
13	 	sinon qu'il rassemble en grand nombre des Gentils dociles	 
14	 	à ses commandements, et qu'il ne reconnaît pas la Judée,	 
15	 	dont il est charnellement issu ?