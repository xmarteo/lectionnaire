Écrivons en esprit les mystères si nous voulons parler ;	 
10	 	écrivons le messager du Christ « non sur des tables de	 
11	 	pierre, mais sur les tables de chair de notre coeur » (II	 
12	 	Cor., III, 3). Car parler de Jean, c'est prophétiser le Christ :	 
13	 	parlons de Jean, parlons aussi du Christ, afin que nos	 
14	 	lèvres à leur tour puissent s'ouvrir, ces lèvres qui, chez	 
15	 	un prêtre si grand, étaient, comme pour un animal sans	 
16	 	raison, bridées par le mors d'une foi hésitante.