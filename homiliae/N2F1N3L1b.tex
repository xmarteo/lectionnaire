« Grand » et admirable « mystère ! » : l'enfant	 
25	 	est circoncis et il est appelé Jésus. Que signifie le rap-	 
26	 	prochement de ces deux faits ? Car la circoncision, n'est-	 
27	 	ce pas, concerne celui qui a besoin d'être sauvé et non	 
28	 	le Sauveur, et au Sauveur il convient de circoncire plutôt	 
29	 	que d'être circoncis. Mais reconnais en lui « le Médiateur	 
30	 	entre Dieu et les hommes » : dès les débuts de sa nais-	 
31	 	sance, il unit l'humain au divin. « Il naît d'une	 
33	 	femme », mais en elle le fruit de la fécondité se forme	 
34	 	de telle façon que ne tombe pas la fleur de la virginité.	 
 	--- 109 ---	 
1	 	« Il est enveloppé de langes », mais ces langes sont	 
2	 	honorés par les louanges des anges. « Il est caché dans	 
3	 	une mangeoire », mais il est manifesté par une étoile	 
4	 	qui brille dans le ciel. Ainsi, en même temps, la cir-	 
5	 	concision prouve la vérité de l'humanité qu'il a assumée,	 
6	 	et « le nom » qui est « au-dessus de tout nom » révèle	 
7	 	« la gloire de sa majesté » : il est circoncis en tant que	 
8	 	vrai fils d'Abraham, il est appelé Jésus en tant que vrai	 
9	 	Fils de Dieu.