Marie-Madeleine, qui avait été dans la ville une péche-	 
6	 	resse, en aimant la Vérité lave de ses larmes les souillures de	 
7	 	sa faute, et la parole de la Vérité s'accomplit : « Beaucoup de	 
8	 	péchés lui ont été pardonnés parce qu'elle a beaucoup aimé. »	 
9	 	Elle que son péché avait d'abord laissée froide, brûla ensuite	 
10	 	d'un grand amour. Car, après être venue au tombeau et n'y	 
11	 	avoir pas trouvé le corps du Seigneur, elle crut qu'on l'avait	 
12	 	enlevé et l'annonça aux disciples. Ils vinrent, ils virent et ils	 
13	 	crurent qu'il en était comme la femme le leur avait dit. Le	 
14	 	texte ajoute alors à leur sujet : « Les disciples repartirent chez	 
15	 	eux », et puis : « Marie se tenait dehors près du tombeau, et	 
16	 	pleurait. »