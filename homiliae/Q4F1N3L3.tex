Le Seigneur a frappé nos sens par ce prodige,
	afin d’élever vers lui nos pensées;
	il a étalé sous nos yeux le spectacle de sa puissance,
	afin d’exciter nos âmes à la réflexion;
	il voulait que ses œuvres visibles nous fissent admirer leur invisible Auteur;
	ainsi élevés jusqu’à la hauteur de la foi, et purifiés par elle,
	nous désirerons le voir encore des yeux de notre âme,
	après avoir appris à le connaître, quoiqu’il soit invisible,
	par le spectacle présenté aux yeux de notre corps.
Ce n’est pas là, toutefois, le seul point de vue
		sous lequel nous devions envisager les miracles du Christ:
	il nous faut encore les étudier en eux-mêmes,
	et faire bien attention à ce qu’ils nous disent du Christ.
Car si nous en comprenons toute l’importance, ils ont un langage à eux:
	dès lors, en effet, que le Christ est le Verbe de Dieu,
	son action même est pour nous une véritable parole.
