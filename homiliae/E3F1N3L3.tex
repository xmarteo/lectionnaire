Et Jésus lui dit : « Garde-toi de le dire à personne. »	 
14	 	Véritablement en quoi était-il nécessaire de publier par	 
15	 	la parole ce dont il témoignait par son corps ?	 
16	 	« Mais va te montrer au prêtre et fais l'offrande que Moïse	 
17	 	a prescrite pour le leur attester. » Il l'envoie au prêtre pour	 
18	 	diverses raisons. Tout d'abord, par humilité, pour témoigner	 
19	 	aux prêtres, sa déférence. En effet, à ceux qui avaient été	 
20	 	guéris de la lèpre, la Loi prescrivait d'offrir des présents aux	 
21	 	prêtres. Ensuite, pour qu'à la vue du lépreux guéri, ils eussent	 
22	 	ou n'eussent point la foi en notre Sauveur : leur foi les	 
23	 	sauverait, leur incrédulité n'aurait pas d'excuse. En même	 
24	 	temps, il ne voulait point paraître enfreindre la Loi, ce dont	 
25	 	ils l'accusaient très souvent.