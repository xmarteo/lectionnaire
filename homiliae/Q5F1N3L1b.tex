Considérez, frères très chers, la mansuétude de Dieu.	 
5	 	Il était venu rompre le lien des péchés, et il disait : « Qui de	 
6	 	vous me convaincra de péché ? ». Il ne dédaigne pas de	 
7	 	montrer avec raison qu'il n'est pas un pécheur, lui qui pou-	 
8	 	vait par la puissance de sa divinité justifier les pécheurs. Mais	 
9	 	ce qu'il a ajouté fait trembler : « Celui qui est de Dieu écoute	 
10	 	les paroles de Dieu ; si vous n'écoutez pas, c'est parce que	 
11	 	vous n'êtes pas de Dieu ». Si en effet celui-là écoute les	 
12	 	paroles de Dieu qui est de Dieu, et si celui-là ne peut écou-	 
13	 	ter les paroles de Dieu qui n'est pas de lui, que chacun se	 
14	 	demande s'il perçoit par l'oreille du coeur les paroles de	 
15	 	Dieu, et il comprendra de qui il est. La Vérité commande de	 
16	 	désirer la patrie céleste, d'étouffer les désirs de la chair, de	 
17	 	renoncer à la gloire du monde, de ne pas désirer le bien d'au-	 
18	 	trui, de faire largesse du sien.