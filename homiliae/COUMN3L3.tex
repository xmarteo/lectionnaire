Mais il est bon de nous demander comment il nous est	 
2	 	commandé de haïr parents et proches selon la chair, alors	 
3	 	que nous avons l'ordre d'aimer même nos ennemis. Oui, la	 
4	 	Vérité dit en parlant de l'épouse : « Ce que Dieu a uni, que	 
5	 	l'homme ne le sépare pas. » Et Paul : « Maris, aimez vos	 
6	 	femmes comme le Christ a aimé l'Église. » Ainsi le disciple	 
7	 	prêche l'amour de l'épouse, pendant que le maître affirme :	 
8	 	« Celui qui ne hait pas sa femme ne peut être mon disciple. »	 
9	 	Le héraut proclame-t-il une sentence différente de celle que	 
10	 	prononce le juge ? Ou bien pouvons-nous à la fois haïr et	 
11	 	aimer ? Si nous comprenons bien le sens du précepte, nous	 
12	 	pouvons faire l'un et l'autre en discernant bien : ceux qui	 
13	 	nous sont unis par les liens de la chair, aimons-les, les	 
14	 	sachant nos proches, et en même temps, s'ils sont nos adver-	 
15	 	saires sur le chemin de Dieu, ignorons-les, les haïssant et les	 
16	 	fuyant.