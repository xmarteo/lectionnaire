C'est merveille qu'une nature corpo-	 
7	 	relle ait traversé un corps impénétrable : on ne le vit pas	 
8	 	arriver, on vit sa présence ; il fut aisé de le toucher, dif-	 
9	 	ficile de le reconnaître.

Aussi bien les disciples,	 
10	 	troublés, croyaient voir un esprit. C'est pourquoi le Sei-	 
11	 	gneur, pour nous montrer le caractère de la résurrection :	 
12	 	« Touchez, dit-il, et voyez : un esprit n'a ni chair ni os	 
13	 	comme vous voyez que je les ai. » Ce n'est donc point	 
14	 	une nature incorporelle, mais l'état de son corps ressus-	 
15	 	citė, qui lui a fait pénétrer des clôtures normalement	 
16	 	impénétrables ; car ce qui se touchc est corps, ce qui	 
17	 	se palpe est corps.

