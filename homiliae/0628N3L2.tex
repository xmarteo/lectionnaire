On raisonnera encore de la même façon
	contre ceux qui prétendent qu'il n'a souffert qu'en apparence.
Si en effet il n'a pas véritablement souffert,
	on ne peut lui assigner aucune grâce,
	puisqu'il n'aura pas souffert la Passion.
Et nous, quand nous souffrirons véritablement,
	nous le regarderons comme un séducteur lorsqu'il nous exhorte
		à subir les coups et à présenter l'autre joue,
	si lui-même en toute vérité ne l'a pas souffert le premier.
Et de même qu'il aurait trompé les témoins de sa Passion
		en paraissant être ce qu'il n'était point,
	de même il nous trompe, nous aussi,
		en nous exhortant à subir ce qu'il n'a pas subi lui-même.
Nous serons, peut-on dire encore, supérieurs au Maître,
	en souffrant et en supportant ce que ce Maître n'a ni supporté ni souffert.
Mais non ! Notre Seigneur est le seul Maître véritable,
	Fils de Dieu véritablement bon et patient,
	Verbe de Dieu le Père devenu Fils de l'Homme.
Il a lutté en effet et il a vaincu.
C'était un homme qui combattait pour ses pères;
	par son obéissance, il rachetait leur désobéissance.
Il a lié le Fort, libéré les infirmes,
	donné le salut à l'œuvre de ses mains en détruisant le péché.
Donc ceux qui prétendent qu'il s'est montré en apparence,
	qu'il n'est pas né dans la chair, qu'il ne s'est pas véritablement fait homme,
	ceux-là sont encore sous le coup de l'antique condamnation.
