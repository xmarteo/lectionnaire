La plupart tirent des significations variées de l’appellation de vigne;
	mais évidemment Isaïe, en parlant de la vigne du Seigneur des armées,
	a pensé à la maison d’Israël.
Cette vigne, quel autre que Dieu l’a plantée?
C’est donc lui qui l’a louée à des vignerons, et puis est parti en voyage;
	non que le Seigneur soit parti d’un lieu dans un autre,
	lui qui est toujours présent partout,
	mais en ce sens qu’il est plus proche de ceux qui l’aiment,
	alors qu’il s’éloigne de ceux qui l’oublient.
Le maître est resté longtemps absent,
	pour que sa réclamation ne paraisse point trop précipitée.
Aussi plus sa libéralité est indulgente,
	plus la mauvaise volonté opiniâtre est inexcusable.
