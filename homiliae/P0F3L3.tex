Et puisque notre discours en est arrivé là, consi-	 
33	 	dérons comment il se fait que selon Jean les Apôtres ont	 
34	 	cru, puisqu'ils se sont réjouis, que selon Luc ils sont	 
35	 	repris comme incrédules ; qu'ici ils ont reçu l'Esprit	 
36	 	Saint, que là il leur est prescrit de résider dans la ville	 
37	 	jusqu'à ce qu'ils soient revêtus de ce don du ciel. Il me	 
38	 	semble que l'un, en qualité d'apôtre, a touché ce qu'il	 
39	 	y a de plus grand et de plus élevé, l'autre la suite, plus	 
40	 	proche de l'humain ; l'un a suivi les détails de l'histoire,	 
 	--- 213 ---	 
1	 	l'autre a résumé. Car on ne saurait douter de celui qui	 
2	 	rend témoignage de faits auxquels il a lui-même assisté,	 
3	 	« et son témoignage est véritable » (Jn, XXI, 24) ; quant	 
4	 	à celui qui a mérité d'être évangéliste, il sied également	 
5	 	d'écarter de lui tout soupçon de négligence ou de men-	 
6	 	songe. Ainsi nous pensons que l'un et l'autre est véridique :	 
7	 	ils ne sont séparés ni par la différence des pensées ni par	 
8	 	la diversité des personnes. Car si Luc dit d'abord qu'ils	 
9	 	n'ont pas cru, plus tard cependant il montre qu'ils ont	 
10	 	cru. Si nous considérons le début, il y a opposition ;	 
11	 	pour la suite, l'accord est assuré.