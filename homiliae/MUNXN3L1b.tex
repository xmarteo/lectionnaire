Le royaume des cieux, frères très chers, est dit sem-	 
5	 	blable à des réalités terrestres, afin que du connu notre	 
6	 	esprit s'élève vers l'inconnu, qu'à partir du visible mis sous	 
7	 	nos yeux il soit entraîné vers l'invisible, et que s'échauf-	 
8	 	fant pour ainsi dire au contact de ce qu'il a appris par l'ex-	 
9	 	périence, il apprenne, lui qui sait aimer ce qu'il connaît, à	 
10	 	aimer aussi ce qu'il ne connaît pas. Le royaume des cieux	 
11	 	est comparé ici à un trésor caché dans un champ et qu'un	 
12	 	homme a trouvé ; il le cache à nouveau, et dans sa joie s'en	 
13	 	va vendre tout ce qu'il a et il achète ce champ.