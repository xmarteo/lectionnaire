Selon	 
20	 	d'autres, il s'agit de Zacharie, que Joas, roi de Juda, fit	 
21	 	tuer entre le temple et l'autel, selon le récit du livre des Rois.
Mais, observons-le, ce Zacharie n'est pas fils de Barachie,	 
23	 	mais du grand prêtre Joad. Aussi l'Écriture nous rapporte-	 
24	 	t-elle « que Joas ne se souvint pas de son père Joad et de toute	 
25	 	sa bonté pour lui ». Ainsi, nous avons le nom de Zacharie et le	 
26	 	lieu du meurtre convient aussi. Nous cherchons donc pourquoi	 
27	 	il est dit fils de Barachie et non de Joad. Barachie signifie,	 
28	 	en notre langue : béni du Seigneur, et le nom hébreu du	 
29	 	prêtre Joad signifie la sainteté. Dans l'Évangile utilisé par	 
30	 	les Nazaréens, à la place de « fils de Barachie », nous avons	 
 	--- 183 ---	 
1	 	trouvé « fils de Joad ».