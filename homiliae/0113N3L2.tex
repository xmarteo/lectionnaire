Par conséquent, qu’est-ce que la colombe lui a fait connaître,
	afin que plus tard nous ne le reconnaissions pas comme un menteur
	(ce que Dieu nous garde de penser)?
C’est évidemment cette particularité,
	savoir, que la sainteté du baptême serait attribuée à Jésus-Christ seul,
	quoique beaucoup de ministres justes ou injustes dussent le conférer.
En effet, au moment où la colombe descendait sur lui,
	on entendit une voix qui disait:
	«C’est celui-là qui baptise dans le Saint-Esprit».
Que Pierre baptise, c’est celui-là qui baptise;
	que Paul baptise, c’est celui-là qui baptise;
	que Judas baptise, c’est celui-là qui baptise.
Car si la sainteté du baptême est en proportion des mérites de ceux qui le confèrent,
	il y aura autant de baptêmes que de sortes de mérites,
	et chacun croira en avoir reçu un meilleur,
		d’autant meilleur, que le ministre en paraîtra plus méritant.
