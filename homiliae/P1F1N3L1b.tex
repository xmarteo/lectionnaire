Après la lecture de cet Évangile, une première question	 
6	 	frappe l'esprit : comment, après sa résurrection, le corps du	 
7	 	Seigneur a-t-il été un corps véritable, pour avoir pu s'intro-	 
8	 	duire, les portes étant fermées, auprès des disciples ? Il nous	 
9	 	faut savoir qu'une action de Dieu, si elle est saisie par la rai-	 
10	 	son, n'a rien de surprenant ; et la foi n'a pas de mérite, si la	 
11	 	raison humaine lui donne une preuve. Ces actes de notre	 
12	 	Rédempteur, qui ne peuvent se comprendre d'eux-mêmes,	 
13	 	sont à apprécier à partir d'autres actes faits par lui, de sorte	 
14	 	que la foi en des faits étonnants soit soutenue par des faits	 
15	 	plus étonnants encore. Ainsi le corps du Seigneur entré,	 
16	 	portes fermées, pour apparaître aux disciples, était, par sa	 
17	 	nativité, sorti du sein fermé d'une vierge pour apparaître aux	 
18	 	regards des hommes. Quoi d'étonnant si, après sa résurrec-	 
19	 	tion, alors qu'il devait vivre pour l'éternité, il est entré, portes	 
20	 	fermées, lui qui, venant pour mourir, sortit du sein d'une	 
21	 	vierge sans l'ouvrir ?