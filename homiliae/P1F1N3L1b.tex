À cette lecture de l’Évangile, une première question frappe notre esprit.
Comment, après la résurrection, le corps du Seigneur fut-il véritable,
	lui qui put, portes closes, s’introduire auprès des disciples?
Mais il nous faut savoir qu’une œuvre divine, si la raison la comprend,
	n’a plus rien de merveilleux.
D’ailleurs, la foi n’a plus de mérite
	si la raison humaine lui fournit une preuve expérimentale.
Quant à ces œuvres de notre Rédempteur
		qui d’elles-mêmes ne peuvent être comprises,
	c’est d’après une autre de ses actions qu’il faut en juger;
	de sorte qu’en face de choses étonnantes,
	notre foi trouve son appui dans des choses plus étonnantes encore.
En effet, ce corps du Seigneur
		qui s’introduisit auprès des disciples, portes closes,
	c’est celui qui, pour apparaître aux yeux humains à sa naissance,
	sortit du sein clos de la Vierge.
Qu’y a-t-il donc d’étonnant à ce qu’il s’introduise,
	portes closes, après sa résurrection,
	celui qui désormais devait vivre pour l’éternité,
	alors que, venant pour mourir,
	il était sorti, sans l’ouvrir, du sein de la Vierge.
