En expliquant ses paroles, il a bien montré qu'il	 
17	 	parlait au sens figuré : il voulait par là vous donner	 
18	 	confiance, lorsque notre faiblesse vous découvrirait le sym-	 
19	 	bolisme de ses paroles. Qui en effet m'aurait jamais cru, si	 
20	 	j'avais voulu voir dans les épines les richesses, d'autant sur-	 
21	 	tout que celles-là piquent et que celles-ci délectent ? Et	 
22	 	pourtant ce sont des épines, parce que les pensées qu'elles	 
23	 	engendrent déchirent l'âme de leurs pointes, et qu'en entraî-	 
24	 	nant au péché, elles infligent comme une sanglante blessure.	 
25	 	Ainsi que l'atteste bien dans ce passage un autre évangéliste,	 
26	 	le Seigneur ne les appelle pas richesses, mais trompeuses	 
27	 	richesses.