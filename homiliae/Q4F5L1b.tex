Cet endroit de l’Évangile
	se rapporte à l’une et à l’autre grâce dont nous allons parler.
Il a d’abord pour but de nous donner l’assurance
	que la miséricorde divine se laisse vite fléchir
		par les gémissements d’une mère veuve,
	et surtout d’une mère brisée par la maladie ou la mort de son fils unique,
	d’une veuve enfin dont le mérite et la gravité
		sont prouvés par la foule qui l’accompagne aux funérailles.
Il est destiné encore
		à nous faire voir plus qu’une simple femme dans cette veuve,
	entourée d’une grande foule de peuple,
	qui mérita d’obtenir par ses larmes
		la résurrection d’un jeune homme, son fils unique;
	parce que cette veuve est l’image de la sainte Église,
	qui, en montrant ses larmes,
	rappelle à la vie un peuple encore jeune,
	du milieu de ses funérailles et des bords du sépulcre,
	et qui reçoit défense de pleurer celui auquel était due la résurrection.
