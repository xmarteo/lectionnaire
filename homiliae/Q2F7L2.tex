il partit à l'étranger, dans un pays lointain. 
19	 	214. Qu'y a-t-il de plus éloigné que de se quitter soi-	 
20	 	même, que d'être séparé non par les espaces, mais par	 
21	 	les moeurs, de différer par les goûts, non par les pays,	 
22	 	et, les excès du monde interposant leurs flots, d'être	 
23	 	distant par la conduite ? Car quiconque se sépare du	 
24	 	Christ s'exile de la patrie, est citoyen du monde. Mais	 
25	 	nous autres « nous ne sommes pas étrangers et de pas-	 
26	 	sage, mais nous sommes citoyens du sanctuaire, et de la	 
27	 	maison de Dieu » (Éphés., II, 19) ; car « éloignés que nous	 
28	 	étions, nous avons été rapprochés dans le sang du Christ »	 
29	 	(Ib., 13). Ne soyons pas malveillants envers ceux qui	 
30	 	reviennent du pays lointain, puisque nous avons été,	 
31	 	nous aussi, en pays lointain, comme l'enseigne Isaïe ;	 
32	 	vous lisez : « Pour ceux qui résidaient au pays de l'ombre	 
33	 	mortelle, la lumière s'est levée » (Is., IX, 2). Le pays	 
34	 	lointain est donc celui de l'ombre mortelle.