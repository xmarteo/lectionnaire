Il partit à l’étranger pour une région lointaine.
	Quel pire éloignement que de s’éloigner de soi-même,
	d’être séparé de sa propre vie, non par l’espace, mais par les mœurs;
	d’en rester distant, non par des étendues de terre, mais par des passions,
	et d’être en divorce avec les Saints,
		par la barrière brûlante de la luxure mondaine.
En effet, qui se sépare du Christ est un exilé de la patrie;
	c’est un citoyen du monde.
Mais nous, nous ne sommes pas des étrangers et des gens du dehors,
	mais nous sommes les concitoyens des Saints, et les familiers de Dieu.
Car nous qui étions éloignés,
	nous avons été rapprochés par le sang du Christ.
Ne jalousons pas ceux qui reviennent d’une région lointaine,
	parce que, nous aussi, nous avons vécu dans une région lointaine,
	comme l’enseigne Isaïe, car voici ce que vous y lisez:
	Pour ceux qui demeuraient dans la région de l’ombre de la mort,
		une lumière s’est levée.
La région lointaine, c’est donc l’ombre de la mort.
