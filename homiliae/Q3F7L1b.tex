«De là, Jésus vint à la montagne».
C’était la montagne «des Oliviers», fertile en parfums et en huile.
De fait, en quel endroit, sinon sur la montagne des Oliviers,
	le Christ pouvait-il se trouver mieux pour enseigner?
L’étymologie du mot Christ, c’est l’onction,
	car le nom grec chrisma se traduit en latin par celui d’onction.
Il nous a oints, parce qu’il nous a destinés à lutter contre le démon.
Au commencement du jour, «il parut de nouveau dans le temple,
	et tout le peuple vint vers lui;
	et, s’étant assis, il les enseignait».
Et l’on ne mettait pas la main sur lui,
	parce qu’il ne jugeait pas encore à propos de souffrir.
Mais voyez quel moyen ses ennemis employèrent
	pour mettre à l’épreuve la douceur de Jésus.
