Que le Seigneur décrive ici, en	 
8	 	pleurant, la ruine de Jérusalem par les empereurs romains	 
9	 	Vespasien et Titus, aucun de ceux qui ont lu le récit de cette	 
10	 	destruction ne l'ignore. Ce sont, en effet, ces empereurs	 
11	 	romains qui sont visés par ces paroles : « Des jours viendront	 
12	 	sur toi où tes ennemis t'entoureront de retranchements, ils	 
13	 	t'investiront et te presseront de toutes parts, ils te renver-	 
14	 	seront à terre, toi et tes enfants qui sont dans tes murs. » Et	 
15	 	ce qui suit : « Ils ne laisseront pas en toi pierre sur pierre. »	 
16	 	Le déplacement même de cette cité atteste cela, puisqu'elle	 
17	 	est construite maintenant sur le lieu où le Seigneur fut	 
18	 	crucifié hors de la porte, et que la première Jérusalem,	 
19	 	comme il est dit ici, a été complètement détruite.