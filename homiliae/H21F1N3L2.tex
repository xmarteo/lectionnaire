Pour que cela soit plus	 
24	 	clair, prenons un exemple : quelqu'un d'entre vous a-t-il	 
25	 	commis un adultère, un homicide, un sacrilège, fautes de	 
26	 	plus de dix mille talents, cela lui est pardonné, à sa prière,	 
27	 	pourvu qu'il pardonne de son côté à ceux qui ont commis	 
28	 	des fautes légères ; mais si pour une offense nous sommes	 
29	 	implacables, si, pour un mot trop amer, nous entretenons	 
30	 	des discordes perpétuelles, ne nous semble-t-il pas qu'il	 
31	 	faut avec justice nous mettre en prison et que l'exemple	 
32	 	de notre conduite aboutit à nous faire refuser le pardon pour	 
33	 	nos fautes plus graves ?