il faut remarquer que le supé-	 
8	 	rieur vient à l'inférieur pour aider l'inférieur : Marie à	 
9	 	Élisabeth, le Christ à Jean ; aussi bien, plus tard, pour	 
10	 	consacrer le baptême de Jean, le Seigneur est venu à ce	 
11	 	baptême (Matth., III, 13).	 
12	 	Et tout de suite se manifestent les bienfaits de l'arrivée	 
13	 	de Marie et de la présence du Seigneur
Remarquez le choix et la précision de chaque mot.	 
17	 	Élisabeth a la première entendu la voix, mais Jean a le	 
18	 	premier ressenti la grâce : celle-là suivant l'ordre de la	 
19	 	nature a entendu, celui-ci a tressailli sous l'effet du mys-	 
20	 	tère ; elle a perçu l'arrivée de Marie, lui celle du Seigneur :	 
21	 	la femme celle de la femme, l'enfant celle de l'enfant.	 
22	 	Elles parlent grâce ; eux la réalisent au-dedans et abordent	 
23	 	le mystère de la miséricorde au profit de leurs mères ; et,	 
24	 	par un double miracle, les mères prophétisent sous l'inspi-	 
25	 	ration de leurs enfants. L'enfant a tressailli, la mère a	 
26	 	été comblée ; la mère n'a pas été comblée avant son fils,	 
27	 	mais le fils, une fois rempli de l'Esprit Saint, en a aussi	 
28	 	rempli sa mère.