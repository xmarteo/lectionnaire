Ainsi le	 
31	 	Seigneur a-t-il permis le doute de son disciple après sa résur-	 
32	 	rection, mais ne l'a pas abandonné au doute, de même	 
33	 	qu'avant sa naissance, il a voulu que Marie ait un époux, qui	 
34	 	pourtant n'est pas allé jusqu'à l'union matrimoniale. Ainsi,	 
 	--- 151 ---	 
1	 	ce disciple qui doutait et toucha, est devenu témoin de la	 
2	 	bien réelle résurrection du Seigneur, comme l'époux de sa	 
3	 	mère avait été le gardien de sa parfaite virginité.
Thomas palpa et s'écria : « Mon Seigneur et mon	 
5	 	Dieu ! » Jésus lui dit : « Parce que tu m'as vu, tu as cru. »	 
6	 	Comme l'apôtre Paul dit que la foi est la garantie des biens	 
7	 	que l'on espère, la preuve de réalités qu'on ne voit pas, il	 
8	 	est clair que la foi est la preuve de ces réalités que l'on ne	 
9	 	peut voir. Car celles qui sont visibles ne requièrent pas la	 
10	 	foi, mais la connaissance.