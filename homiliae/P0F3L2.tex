Or c'est corporellement que nous	 
18	 	ressuscitons : car « la semence est un corps de chair, d'où	 
19	 	lève un corps spirituel » (I Cor., XV, 44) ; l'un est subtil,	 
20	 	l'autre grossier, étant encore épaissi par les conditions	 
21	 	de son infirmité terrestre. 170. Comment en effet n'y	 
22	 	eût-il pas eu un corps, alors que demeuraient les marques	 
23	 	des blessures, les traces des cicatrices, que le Seigneur	 
24	 	a présentées pour être touchées ? Par là non seulement	 
25	 	Il affermit la foi, mais Il excite la dévotion : les blessures	 
26	 	reçues pour nous, Il a préféré les emporter au ciel, Il	 
27	 	n'a pas voulu les effacer, afin de montrer à Dieu le Père	 
28	 	le prix de notre libération. C'est en cet état que le Père	 
29	 	le place à sa droite, accueillant les trophées de notre	 
30	 	salut ; tels sont les témoins que la couronne de ses plaies	 
31	 	a produits pour nous.