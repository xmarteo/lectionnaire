Le Sauveur prit occasion de leur étonnement et de leurs paroles,
	pour leur adresser des paroles profondes
	et dignes d’être étudiées et discutées avec le soin le plus minutieux.
Que répond le Sauveur à ces hommes
	qui se demandaient avec étonnement comment il pouvait savoir lire
	sans avoir appris à le faire?
«Ma doctrine», leur dit-il, «ne vient pas de moi,
	mais de Celui qui m’a envoyé».
Voici le premier mystère que je rencontre dans ces paroles,
	c’est que dans ce peu de mots sortis de la bouche de Jésus,
	il semble se trouver une contradiction;
	car il ne dit pas: Cette doctrine n’est pas la mienne;
	mais il dit: «Ma doctrine ne vient pas de moi».
Si cette doctrine ne vient pas de toi, comment est-elle la tienne?
Et si elle est la tienne, comment se fait-il qu’elle ne vienne pas de toi?
Tu dis pourtant l’un et l’autre:
	«C’est ma doctrine, elle ne vient pas de moi».
