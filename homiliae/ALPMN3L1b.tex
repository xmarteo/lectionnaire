Considérez toutes choses avec soin : comment Il monte	 
12	 	avec les Apôtres et descend vers les foules. Comment en	 
13	 	effet la foule verrait-elle le Christ, sinon en bas ? Elle ne	 
14	 	le suit pas sur les hauteurs, elle ne s'élève pas aux som-	 
15	 	mets. Aussi bien, dès qu'Il descend, Il trouve des infirmes :	 
16	 	car les infirmes ne peuvent être sur les hauteurs. Matthieu	 
17	 	lui aussi (VIII, 1) nous apprend que les malades ont été	 
18	 	guéris dans la plaine : car chacun a été guéri, de façon	 
19	 	que, ses forces progressant peu à peu, il puisse gravir la	 
20	 	montagne ; aussi guérit-Il dans la plaine, c'est-à-dire	 
21	 	qu'Il soustrait au désordre, qu'Il écarte la disgrâce de	 
22	 	l'aveuglement. Il est descendu vers nos blessures, afin de	 
23	 	nous faire, par son intimité et son commerce, participer	 
24	 	à sa nature céleste.