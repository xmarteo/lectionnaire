En effet, voici la parole de notre Rédempteur:
	«Tout royaume divisé contre lui-même devient un désert.»
II est donc visible que le royaume de Judée, divisé et soumis à tant de chefs,
	touchait à son terme.
C’est aussi avec raison qu’on ne dit pas seulement sous quels princes,
	mais encore sous quels prêtres ces événements se sont passés.
Comme celui que Jean-Baptiste annonçait devait être à la fois Roi et Prêtre,
	l’Évangéliste Luc désigne le temps de sa prédication
	par la mention et des chefs temporels et des autorités sacerdotales.
