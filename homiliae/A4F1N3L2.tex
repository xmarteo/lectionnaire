Mais, puisque nous avons déjà péché,
	et que nous sommes empêtrés dans de mauvaises habitudes,
	qu’il nous dise ce que nous devons faire pour fuir la colère à venir.
Le voici: «Produisez donc de bons fruits de pénitence.»
Il faut remarquer, dans ces paroles, que l’ami de l’Époux nous enjoint de faire,
	non seulement des fruits de pénitence, mais de bons fruits de pénitence.
Car c’est une chose de produire un fruit de pénitence,
	une autre de produire un bon fruit de pénitence.
Pour bien parler de ces bons fruits de pénitence,
	il faut savoir que quiconque n’a rien commis d’illicite,
	a le droit d’user des choses licites:
	et ainsi il peut s’adonner aux œuvres de miséricorde,
	sans abandonner les biens de ce monde, s’il lui plaît d’en garder l’usage.
