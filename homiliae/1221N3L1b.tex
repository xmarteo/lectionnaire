Que remar-	 
19	 	quez-vous, frères très chers, dans tout cela ? Croyez-vous	 
20	 	que c'est par hasard que ce disciple choisi ait été absent, ait	 
21	 	appris en arrivant ensuite ce qui s'est passé, en l'apprenant	 
22	 	ait douté, en doutant ait touché, en touchant ait cru ? Non,	 
23	 	ce n'est pas par hasard, mais par un dessein divin que cela	 
24	 	s'est passé. La clémence divine a fait d'une façon mer-	 
25	 	veilleuse qu'en touchant sur son maître les blessures du	 
26	 	corps, le disciple qui doutait guérisse en nous les blessures	 
27	 	de l'incrédulité. Plus que la foi des disciples qui ont cru,	 
28	 	l'incrédulité de Thomas a été profitable à notre foi ; car, tan-	 
29	 	dis qu'il est ramené à la foi par ce toucher, notre âme à nous,	 
30	 	rejetant toute incertitude, est fortifiée dans la foi.