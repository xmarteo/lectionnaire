« Mais vous qui dites-vous que je suis ? »  
26	 	Lecteur avisé, remarque-le, dans la logique du contexte,	 
27	 	le Seigneur n'appelle pas du tout les apôtres des hommes	 
28	 	mais des dieux, car après ces mots : « Au dire des hommes,	 
29	 	qui est le Fils de l'homme ? » il ajoute : « Mais vous, qui	 
30	 	dites-vous que je suis ? » Ceux-là, parce qu'ils sont des hommes,	 
 	--- 15 ---	 
1	 	ont des opinions d'hommes ; mais vous qui êtes des dieux,	 
2	 	qui croyez-vous que je suis ? Au nom de tous les apôtres,	 
3	 	Pierre fait cette profession de foi : « Tu es le Christ, le Fils	 
4	 	du Dieu vivant. » Son Dieu, il le qualifie de vivant, pour	 
5	 	le distinguer de ces dieux qui passent pour des dieux, mais	 
6	 	qui sont des morts.