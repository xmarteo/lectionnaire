« Et	 
9	 	vous, soyez semblables à des gens qui attendent leur maître	 
10	 	à son retour de noces, afin de lui ouvrir aussitôt, lorsqu'il viendra	 
17	 	et frappera ». Il vient en accourant pour juger. Il frappe, en	 
18	 	signifiant par les peines de la maladie que le moment de la	 
19	 	mort est proche. Nous lui ouvrons aussitôt, en l'accueillant	 
20	 	avec amour. Il ne vient pas ouvrir au juge qui frappe,	 
21	 	l'homme qui tremble de quitter son corps et redoute de	 
22	 	trouver un juge en celui qu'il se rappelle avoir méprisé. Mais	 
23	 	l'homme qui est fort de son espérance et de sa conduite	 
24	 	ouvre aussitôt à celui qui frappe, car il attend son juge avec	 
25	 	joie, et quand il se rend compte que le moment de la mort	 
26	 	approche, il se réjouit de la gloire qui va le récompenser.