« Et votre âme à vous, dit-il, sera traversée d'un	 
16	 	glaive. »	 
17	 	Ni l'Écriture ni l'histoire ne nous apprend que Marie ait	 
18	 	quitté cette vie en subissant le martyre dans son corps ;	 
19	 	or, ce n'est pas l'âme, mais le corps, qu'un glaive matériel	 
20	 	peut transpercer. Ceci nous montre donc la sagesse de	 
21	 	Marie, qui n'ignore pas le mystère céleste ; car « la parole	 
22	 	de Dieu est vivante, puissante, plus aiguë que le glaive	 
23	 	le mieux aiguisé, pénétrante jusqu'à diviser l'âme et	 
24	 	l'esprit, les jointures et les moelles ; elle sonde les pensées	 
25	 	du coeur et les secrets des âmes » (Héb., IV, 12) : car tout	 
26	 	dans les âmes est à nu, à découvert devant le Fils, auquel	 
27	 	les replis de la conscience n'échappent point.