« Ils lient des fardeaux pesants, impossibles à por-	 
16	 	ter, et ils les mettent sur les épaules des hommes, mais	 
17	 	ils se refusent à les manier du bout du doigt. » Cela vise en	 
18	 	général tous les maîtres qui commandent de grandes choses	 
19	 	mais ne font pas les petites. Notons-le, ces termes, épaules et	 
20	 	doigt, fardeaux et liens qui servent à les attacher, doivent	 
21	 	être compris au sens spirituel.

« En tout, ils agissent pour se faire remarquer des	 
23	 	hommes. » Donc quiconque n'agit que pour être vu des	 
24	 	hommes est un scribe et un Pharisien.