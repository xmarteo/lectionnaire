Mais parce qu’elle vit tout à coup les taches et la laideur de son âme,
	elle courut pour être purifiée à la source de la miséricorde,
	sans rougir de paraître devant les convives.
Comme elle avait une très grande honte d’elle-même au fond de son cœur,
	elle comptait pour rien la confusion extérieure.
Qu’admirerons-nous donc, mes frères?
	Marie qui vient, ou le Seigneur qui la reçoit?
Dirai-je que le Seigneur la reçoit ou qu’il l’attire?
	Mais il vaut mieux dire qu’il l’attire et qu’il la reçoit tout ensemble,
	car c’est lui assurément qui l’attire intérieurement par sa miséricorde
	et qui l’accueille extérieurement par sa mansuétude.
