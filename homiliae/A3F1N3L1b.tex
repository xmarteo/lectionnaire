Les paroles de la lecture que voici, frères très chers,	 
5	 	nous font apprécier l'humilité de Jean. Il avait une telle	 
6	 	vertu qu'on aurait pu voir en lui le Christ ; mais il choisit	 
7	 	de rester fermement lui-même, sans se laisser entraîner vai-	 
8	 	nement par l'opinion des hommes au-dessus de lui-même.	 
9	 	Car il confessa et ne nia pas, il confessa : « Je ne suis pas, moi,	 
10	 	le Christ ». Par les mots je ne suis pas, il nia nettement être	 
11	 	ce qu'il n'était pas ; mais il ne nia pas ce qu'il était, si bien	 
12	 	qu'en disant la vérité il devenait un membre de celui dont	 
13	 	il ne s'arrogeait pas fallacieusement le nom. Refusant de	 
14	 	chercher à prendre le nom du Christ, il est devenu membre	 
15	 	du Christ, parce que le soin qu'il eut de reconnaître hum-	 
16	 	blement sa propre faiblesse lui mérita d'être élevé vraiment	 
17	 	à la hauteur du Christ.