Les paroles qu’on vient de nous lire, frères très chers,
	nous font apprécier l’humilité de Jean.
Lui, dont la vertu était si grande qu’on avait pu croire qu’il était le Christ,
	il préféra résolument demeurer dans son rôle,
	et ne pas être vainement élevé plus haut que lui-même
		dans l’opinion des hommes.
Car il le déclara et ne le nia point;
	il le proclama: «Moi, je ne suis pas le Christ.»
En disant: «Je ne le suis pas»,
	il a clairement nié être ce qu’il n’était pas;
	mais il n’a pas nié être ce qu’il était,
	afin que, parlant selon la vérité,
	il devînt membre de celui dont il ne voulait pas usurper faussement le nom.
Parce qu’il ne veut pas chercher à prendre le nom du Christ,
	il est fait membre du Christ.
Puisqu’il s’étudie à reconnaître humblement sa propre faiblesse,
	il mérite de participer véritablement à la grandeur du Christ.
