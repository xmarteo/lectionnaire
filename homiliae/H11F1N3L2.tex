Et pourquoi le Seigneur lui toucha-t-il la lan-	 
32	 	gue avec sa salive ? La salive, c'est pour nous la sagesse reçue	 
 	--- 409 ---	 
1	 	de la bouche du Rédempteur dans la parole divine. Car la sa-	 
2	 	live descend du chef dans la bouche. Quand notre langue est	 
3	 	touchée par la Sagesse en personne, elle est mise bien vite en	 
4	 	état de prêcher la parole. « Levant les yeux au ciel, il sou-	 
5	 	pira. » Ce n'est pas qu'il eût besoin de soupirer, lui qui donnait	 
6	 	ce qu'il demandait. Mais il nous apprenait à soupirer vers celui	 
7	 	qui du ciel gouverne tout, pour que nos oreilles s'ouvrent par	 
8	 	les dons de l'Esprit-Saint, et que grâce à la salive de sa bouche,	 
9	 	c'est-à-dire grâce à la science du langage divin, notre langue se	 
10	 	dénoue pour prêcher la parole.