Convertissons-	 
26	 	nous donc, et prenons garde que, pour notre perte, il	 
27	 	ne survienne entre nous quelque dispute de préséance ;	 
28	 	car si les Apôtres ont contesté, ce n'est pas une excuse	 
29	 	offerte, c'est une invitation à prendre garde. Si Pierre	 
30	 	se convertit « un jour » (Matth., XIII, 15 ; Mc, IV, 12),	 
31	 	lui qui a répondu au premier appel du Maître, qui peut	 
32	 	dire que sa propre conversion a été rapide ? Gardez-	 
33	 	vous donc de la vanité, gardez-vous du siècle ; car celui	 
34	 	qui est chargé d'affermir ses frères est celui qui a dit :	 
35	 	« Nous avons tout quitté pour vous suivre » (Lc, XVIII,	 
36	 	28).