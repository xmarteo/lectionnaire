Il	 
21	 	conversa avec eux, il leur reprocha la dureté de leur intelli-	 
22	 	gence ; il leur découvrit les mystères de la sainte Écriture qui	 
23	 	le concernaient, et pourtant, puisque dans leur coeur, pour	 
24	 	leur foi, il était encore un étranger, il feignit d'aller plus loin.	 
25	 	Nous employons fingere (« feindre ») au sens de componere	 
26	 	(« façonner »), et c'est pourquoi nous appelons figuli ceux	 
27	 	qui façonnent l'argile. La Vérité, qui est simple, n'a rien fait	 
28	 	par duplicité, mais s'est montrée à eux corporellement, telle	 
29	 	qu'elle était dans leur esprit. Il fallait les éprouver pour voir	 
30	 	si ceux qui ne l'aimaient pas encore comme Dieu pouvaient	 
31	 	du moins l'aimer comme étranger.