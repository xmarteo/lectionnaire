Porphyre et l'empereur	 
9	 	Julien dénoncent dans ce passage, soit les gaucheries d'un	 
10	 	historien mensonger, soit la sottise de ceux qui ont aussitôt	 
11	 	suivi le Sauveur, comme s'ils avaient inconsidérément suivi	 
12	 	l'appel du premier venu, alors que cet appel avait été précédé	 
13	 	de signes et de prodiges si grands et que les apôtres en ont	 
14	 	été incontestablement les témoins avant de croire. D'ailleurs,	 
15	 	le seul éclat, la majesté de sa divinité cachée, qui resplendis-	 
16	 	saient même sur sa face humaine, avaient le pouvoir d'attirer	 
17	 	ceux qui le voyaient, dès qu'ils le regardaient. Car, si l'aimant	 
18	 	et l'ambre renferment, dit-on, une force d'attraction, capable	 
19	 	d'agglomérer autour d'eux les anneaux, la paille et les fétus,	 
20	 	combien plus le Seigneur de toute créature pouvait-il attirer à	 
21	 	lui ceux qu'il voulait !