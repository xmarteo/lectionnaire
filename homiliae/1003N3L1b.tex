toute la pratique de la	 
4	 	sagesse chrétienne, bien-aimés, ne consiste ni dans l'abondance	 
5	 	des paroles, ni dans l'habileté à disputer, ni dans l'appétit de	 
6	 	louange et de gloire, mais dans la sincère et volontaire humilité	 
7	 	que le Seigneur Jésus-Christ a choisie et enseignée en guise de	 
8	 	toute force, depuis le sein de sa mère jusqu'au supplice de la croix.	 
9	 	Car un jour que ses disciples recherchaient entre eux, comme le	 
10	 	raconte l'évangéliste, « qui, parmi eux, était le plus grand dans	 
11	 	le Royaume des cieux, il appela un petit enfant, le plaça au milieu	 
12	 	d'eux et dit : En vérité, je vous le dis, si vous ne vous convertissez	 
13	 	pas et ne devenez pas comme de petits enfants, vous n'entrerez	 
14	 	pas dans le Royaume des Cieux. Qui donc se fera petit comme	 
15	 	cet enfant-là, voilà qui sera le plus grand dans le Royaume des	 
16	 	Cieux. » Le Christ aime l'enfance qu'il a d'abord vécue et dans	 
17	 	son âme et dans son corps. Le Christ aime l'enfance, maîtresse	 
18	 	d'humilité, règle d'innocence, modèle de douceur. Le Christ	 
19	 	aime l'enfance, vers elle il oriente la manière d'agir des aînés, vers	 
20	 	elle il ramène les vieillards ; il attire à son propre exemple ceux	 
21	 	qu'il élève au royaume éternel.