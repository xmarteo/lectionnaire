Voyez la clémence du Seigneur Sauveur. Il n'est pas	 
13	 	ému d'indignation, ni offensé par le crime, ni affecté par	 
14	 	l'injustice au point de délaisser la Judée ; au contraire,	 
15	 	oubliant les torts, ne songeant qu'à la clémence, tantôt	 
16	 	enseignant, tantôt délivrant, tantôt guérissant, Il cherche	 
17	 	à attendrir le coeur de ce peuple infidèle. Et il est bien	 
18	 	que S. Luc ait d'abord mentionné l'homme délivré de	 
19	 	l'esprit mauvais, puis raconté la guérison d'une femme :	 
20	 	car le Seigneur était venu soigner l'un et l'autre sexe ;	 
21	 	il fallait guérir d'abord celui qui fut créé le premier, et	 
22	 	ne pas laisser de côté celle qui avait péché par inconstance	 
23	 	d'âme plus que par perversité.