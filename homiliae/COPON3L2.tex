Cet homme qui part pour l'étran-	 
17	 	ger, n'est-ce pas notre Rédempteur, qui, dans la chair par lui	 
18	 	assumée, s'en est allé au ciel ? La terre en effet est le lieu	 
19	 	propre de la chair, qui est comme conduite à l'étranger	 
20	 	quand notre Rédempteur lui donne place dans le ciel.

Cet homme partant pour l'étranger a remis ses biens à	 
22	 	ses serviteurs, parce qu'il a accordé à ses fidèles des dons	 
23	 	spirituels. A l'un il a confié cinq talents, à un autre deux, à	 
24	 	un autre un seul. Le corps a en effet cinq sens, la vue, l'ouïe,	 
25	 	le goût, l'odorat et le toucher. Les cinq talents figurent donc	 
26	 	le don des cinq sens, c'est-à-dire la science des réalités exté-	 
27	 	rieures. Les deux talents désignent l'intelligence et l'action.	 
28	 	Le talent unique désigne seulement l'intelligence.

