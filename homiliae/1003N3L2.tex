Mais si nous voulons être capables de comprendre parfai-	 
23	 	tement comment il est possible de parvenir à une conversion	 
24	 	si admirable, et par quelle transformation il nous faut revenir à	 
25	 	l'état des enfants, laissons saint Paul nous instruire et nous dire :	 
26	 	« Ne vous montrez pas enfants en fait de jugement, mais soyez des	 
27	 	petits enfants pour la malice. » Il ne s'agit donc pas pour nous	 
28	 	de revenir aux amusements de l'enfance, ni aux maladresses des	 
29	 	débuts, mais de leur prendre une chose qui convient aussi aux	 
30	 	années de la maturité, à savoir que passent vite nos agitations	 
31	 	intérieures, que rapidement nous retrouvions la paix : ne gardons	 
32	 	aucun souvenir des offenses, n'ayons aucune avidité pour les	 
33	 	dignités, aimons nous retrouver ensemble, gardons une égalité	 
34	 	conforme à la nature. C'est un grand bien, en effet, que de ne pas	 
 	--- 283 ---	 
1	 	savoir nuire et ne pas avoir de goût pour le mal ; car faire tort et	 
2	 	rendre tort, c'est la sagesse de ce monde ; par contre, ne rendre	 
3	 	à personne le mal pour le mal, c'est l'esprit d'enfance tout	 
4	 	plein d'une égalité d'âme chrétienne.