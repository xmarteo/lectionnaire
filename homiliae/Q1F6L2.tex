Qu’il s’agisse donc de la loi, des Prophètes ou de l’Évangile,
	le nombre quarante nous est signalé comme consacré au jeûne.
Considéré dans son sens large, et pris en général,
	le jeûne consiste à s’abstenir de tout péché et de toutes les iniquités du siècle;
	oui, voilà le véritable jeûne:
	«C’est renoncer à l’impiété, aux désirs du siècle,
	et vivre dans le siècle avec tempérance, avec justice et avec piété».
Quelle est la récompense réservée à cette sorte de jeûne?
L’Apôtre nous le dit, car il ajoute ces paroles:
	«Attendant la félicité que nous espérons,
	et l’avènement glorieux du grand Dieu, de notre Sauveur, Jésus-Christ».
Dans le cours de cette vie, nous observons, en quelque sorte, l’abstinence du Carême, 
	lorsque nous nous conduisons bien
		et que nous nous abstenons du péché et des plaisirs défendus.
Mais parce que cette abstinence ne manquera pas d’être récompensée,
	«nous attendons la félicité que nous espérons,
	et l’avènement glorieux du grand Dieu, de notre Sauveur, Jésus-Christ».
Quand notre espérance aura fait place à la possession de la réalité,
	nous recevrons le denier qui doit constituer notre récompense.
D’après l’Évangile, vous vous en souvenez, je crois,
	la même rémunération est accordée
		à tous ceux qui travaillent dans la vigne du père de famille:
	il est inutile de vous rappeler tout cela,
		comme si vous étiez des personnes ignorantes et grossières.
Le denier donné aux ouvriers tire son nom du nombre dix,
	lequel, ajouté à quarante, forme celui de cinquante;
	voilà pourquoi l’observation de la Quadragésime
		exige de nous, avant Pâques, de pénibles sacrifices;
	mais après Pâques, il semble que nous devions recevoir notre récompense,
	car nous célébrons la Quinquagésime dans les transports de la joie.
Au travail salutaire des bonnes oeuvres, qui a trait au nombre quarante,
	viendra s’ajouter le denier du repos et du bonheur, qui fera le nombre cinquante.
