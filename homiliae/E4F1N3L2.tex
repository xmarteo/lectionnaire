Lui cependant dormait. Ses disciples s'approchèrent	 
13	 	de lui, l'éveillèrent disant : « Seigneur, sauve-nous. » Nous	 
14	 	trouvons la préfiguration de ce miracle dans Jonas. Les	 
15	 	autres sont épouvantés, il dort tranquillement, on l'éveille et,	 
16	 	par le pouvoir et le mystère de sa passion, il délivre ceux qui	 
17	 	l'éveillent.
Alors se levant, il apostropha les vents et la mer.	 
19	 	Ce passage nous fait comprendre que tous les éléments de la	 
20	 	création reconnaissent le Créateur. Ceux qu'il apostrophe,	 
21	 	ceux auxquels il donne des ordres, reconnaissent Celui qui leur	 
22	 	commande, non pas selon l'erreur des hérétiques qui pensent	 
23	 	que tout est doué d'une âme, mais à cause de la toute-puis-	 
24	 	sance du Créateur. Ce qui à nos yeux est insensible est sensible	 
25	 	à Lui.