Or ce mort était porté au tombeau, dans un cercueil,	 
13	 	par les quatre éléments de la matière ; mais il avait l'espé-	 
14	 	rance de la résurrection, puisqu'il était porté sur le bois	 
15	 	(celui-ci, il est vrai, ne nous a pas servi tout d'abord,	 
16	 	mais, une fois que Jésus l'eut touché, il commença à nous	 
17	 	procurer la vie) : c'était un signe que le salut se répandrait	 
18	 	sur le peuple par le gibet de la Croix. Ayant donc entendu	 
19	 	la parole de Dieu, les lugubres porteurs de ce deuil s'ar-	 
20	 	rêtèrent : ils entraînaient le corps humain dans le courant	 
21	 	mortel de sa nature matérielle. N'est-ce pas cela et ne	 
22	 	sommes-nous pas étendus sans vie comme dans un cer-	 
23	 	cueil, instrument des derniers devoirs, lorsque le feu d'une	 
24	 	convoitise sans mesure nous consume, ou que l'humeur	 
25	 	froide nous envahit, ou qu'une certaine indolence habi-	 
26	 	tuelle du corps émousse la vigueur de l'âme, ou que notre	 
27	 	esprit, vide de la pure lumière, repaît notre intelligence	 
28	 	de brouillards épais ? Tels sont les porteurs pour nos	 
29	 	obsèques.	 
 