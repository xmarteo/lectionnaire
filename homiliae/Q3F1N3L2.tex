Oui, les foules, qui paraissaient moins instruites,
	admiraient toujours les miracles du Seigneur;
	ceux-là, au contraire, ou bien les niaient,
	ou, s’ils ne pouvaient les nier,
		s’efforçaient de les dénaturer par une explication défavorable,
	attribuant ces œuvres, non pas à la puissance divine,
		mais à l’esprit immonde.
D’autres encore, pour le tenter, lui demandaient un prodige dans le ciel,
	ou bien manifestaient le désir qu’à la manière d’Élie,
	il fît descendre le feu du ciel;
	ou bien, comme l’avait fait Samuel,
	qu’en été, on entendît gronder le tonnerre, qu’on vît briller les éclairs,
	et qu’on reçût des torrents d’eau;
	comme s’ils n’eussent pas pu encore calomnier ces faits
	et dire qu’ils étaient produits par des causes inconnues
		et par diverses perturbations de l’air.
Mais toi, qui dénatures ce que tu vois de tes yeux,
	ce que tu touches de tes mains, ce dont tu sens l’utilité,
	que ferais-tu des prodiges qui te viendraient du ciel?
Tu répondrais certainement que les magiciens en Égypte
	ont fait aussi de nombreux prodiges venus du ciel.
