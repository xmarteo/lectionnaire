Les commande-	 
16	 	ments du Seigneur sont, en effet, multiples et un : multiples	 
17	 	par la diversité des oeuvres, un dans la racine de l'amour. Le	 
18	 	Seigneur insinue lui-même comment il faut pratiquer cet	 
19	 	amour lorsque, dans son Écriture, à de nombreuses reprises,	 
20	 	il ordonne qu'on aime ses amis en lui et ses ennemis à cause	 
21	 	de lui. Il possède la vraie charité, celui qui aime son ami en	 
22	 	Dieu et son ennemi pour Dieu. Car il est des gens qui	 
23	 	aiment leur prochain, mais d'une affection qui vient de la	 
24	 	parenté et de la chair. Les saintes Écritures ne s'opposent	 
25	 	certes pas à cet amour ; mais autre chose est ce qui est	 
26	 	accordé spontanément à la nature, autre chose ce que l'on	 
27	 	doit aux commandements du Seigneur par un amour qui	 
28	 	obéit.