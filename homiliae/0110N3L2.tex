Or ceux-ci lui dirent : « A Bethléem de Judée. » Ici une	 
15	 	erreur des copistes. A notre avis, comme nous le lisons dans	 
16	 	le texte hébreu, l'évangéliste a écrit à l'origine, « de Juda »,	 
17	 	et non « de Judée ». Y a-t-il en effet à l'étranger une autre	 
18	 	Bethléem dont il aurait voulu la distinguer en mettant ici	 
19	 	« de Judée » ? Mais s'il est écrit : « de Juda », c'est parce	 
20	 	qu'il y a une autre Bethléem en Galilée. Lis le livre de Josué,	 
21	 	fils de Nun. Enfin, dans la citation empruntée à la prophétie	 
22	 	de Michée, il y a : « Et toi Bethléem, terre de Juda ».