Mais comme la foi de ces hommes qui	 
22	 	regardaient ce corps visible hésitait, il leur montra aussitôt	 
23	 	ses mains et son côté ; il leur donna à toucher cette chair qu'il	 
24	 	avait fait entrer, les portes étant fermées. En cela, il fit	 
25	 	paraître deux choses étonnantes et pour la raison humaine	 
26	 	toutes contraires, en montrant son corps après la résurrec-	 
27	 	tion : incorruptible, et néanmoins palpable. Car ce qui est	 
28	 	palpé est nécessairement sujet à la corruption, et l'on ne peut	 
29	 	palper ce qui est incorruptible. Mais, d'une manière mer-	 
30	 	veilleuse et qui dépasse la raison, notre Rédempteur a	 
 	--- 139 ---	 
1	 	présenté, après sa résurrection, un corps à la fois incorrup-	 
2	 	tible et palpable : ainsi, en le montrant incorruptible, il nous	 
3	 	invite à la récompense ; en le montrant palpable, il affermit	 
4	 	notre foi. Il s'est donc montré incorruptible et palpable pour	 
5	 	faire constater que son corps, après la résurrection, était de	 
6	 	même nature, mais dans un état de gloire autre.