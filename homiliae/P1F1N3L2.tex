Mais comme, en présence de ce corps qu’on pouvait voir,
	la foi des spectateurs était hésitante,
	il leur montra aussitôt ses mains et son côté,
	et leur donna à palper cette chair qu’il avait introduite portes closes.
En quoi il montrait deux merveilles
	qui, pour la raison humaine, semblent fort opposées,
	quand, après sa résurrection,
		il montra un corps incorruptible et cependant palpable;
	car il est nécessaire que ce qui est palpable se corrompe
	et il est impossible de palper ce qui est incorruptible.
Cependant, d’une manière merveilleuse et qui dépasse l’entendement,
	notre Rédempteur, après sa résurrection,
	nous montra un corps à la fois palpable et incorruptible.
En nous le montrant incorruptible il nous invitait à la récompense,
	et en nous le présentant palpable il affermissait notre foi.
Si donc il se montra, après sa résurrection,
	à la fois incorruptible et palpable,
	c’était pour nous manifester qu’après la résurrection,
	ce corps était le même quant à sa nature, mais autre à raison de sa gloire.
