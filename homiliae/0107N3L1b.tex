Ayant appris la naissance de notre roi, Hérode en vient	 
33	 	à machiner une ruse pour ne pas être privé de sa royauté	 
34	 	terrestre. Il demande qu'on lui rapporte où se trouve l'en-	 
 	--- 249 ---	 
1	 	fant ; il feint de vouloir l'adorer, afin de le faire disparaître	 
2	 	s'il arrive à le trouver. Mais que peut la malice humaine	 
3	 	contre le dessein de la divinité ? N'est-il pas écrit : « Il n'est	 
4	 	pas de sagesse, il n'est pas de prudence, il n'est pas de dessein	 
5	 	en face du Seigneur » ? L'étoile qui était apparue aux mages	 
6	 	les conduit ; ils découvrent le roi nouveau-né, lui offrent des	 
7	 	présents, sont avertis durant leur sommeil de ne pas revenir	 
8	 	auprès d'Hérode. Et il arrive ainsi que ce Jésus qu'il cherche,	 
9	 	Hérode ne peut le trouver. Ce personnage n'est-il pas la	 
10	 	figure des hypocrites qui, feignant de chercher, ne méritent	 
11	 	jamais de trouver le Seigneur ?