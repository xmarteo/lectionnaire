Jésus prit avec lui Pierre et Jacques et son frère Jean
	puis, ayant avec eux gravi, à l’écart, une haute montagne,
	il leur manifesta l’éclat de sa gloire.
Certes, ils avaient reconnu en lui la majesté de Dieu,
	mais ils ignoraient la puissance du corps qui couvrait la divinité.
Et c’est pourquoi Jésus avait promis en termes formels et précis
	que certains de ses disciples présents ne goûteraient pas la mort,
	avant de voir le Fils de l’homme venant en son règne,
	c’est-à-dire dans la splendeur royale
		qui appartient spirituellement
		à la nature de l’homme élevé à l’union hypostatique,
	et qu’il voulut montrer aux yeux de ces trois hommes.
Quant à la vision ineffable et inaccessible de la divinité elle-même,
	vision réservée pour la vie éternelle aux cœurs purs,
	ils ne pouvaient aucunement, encore revêtus de leur chair,
	la regarder et la voir.
