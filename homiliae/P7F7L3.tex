Et c'est à bon droit qu'Il com-	 
36	 	mence le samedi, pour montrer qu'Il est le Créateur, fai-	 
37	 	sant entrer les oeuvres dans la trame des oeuvres, conti-	 
38	 	nuant l'ouvrage qu'Il avait jadis commencé Lui-même.	 
39	 	Tel l'ouvrier qui s'apprête à réparer une maison : il com-	 
 	--- 175 ---	 
1	 	mence, non par les fondations mais par les toits, à démo-	 
2	 	lir le délabré ; donc il met la main tout d'abord au point	 
3	 	par où il avait autrefois terminé. 59. Puis Il commence	 
4	 	par le moindre pour en venir au plus considérable. Déli-	 
5	 	vrer du démon, même des hommes le peuvent - par la	 
6	 	parole de Dieu, il est vrai ; - commander aux morts de	 
7	 	ressusciter n'appartient qu'à la puissance de Dieu.
Peut-être aussi, figurée par cette femme, belle-mère	 
15	 	de Simon et d'André, était-ce notre chair qui souffrait des	 
16	 	fièvres variées des péchés et brûlait des transports déme-	 
17	 	surés des diverses convoitises. La fièvre d'aimer n'est pas	 
18	 	moindre, dirai-je, que celle qui échauffe. Cette fièvre-là	 
19	 	brûle l'âme, l'autre le corps. Car notre fièvre, c'est la dé-	 
20	 	bauche, notre fièvre, c'est l'avarice, notre fièvre, c'est le luxe, notre fièvre, c'est l'ambition, notre fièvre, c'est la colère.