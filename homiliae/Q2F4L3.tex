Le Christ considérait la tendresse d’une mère
	qui voulait consoler sa vieillesse par la récompense de ses fils,
	et qui, bien qu’angoissée de désirs irréalisables,
	supportait l’absence de ses enfants très chers.
Considérez aussi qu’elle est femme, c’est-à-dire appartenant au sexe faible,
	que le Seigneur n’avait pas encore fortifié par sa propre passion.
Considérez, dis-je, qu’elle est fille d’Ève
	et qu’elle succombe à la convoitise immodérée qui, de cette première femme,
	s’est transfusée à toute sa descendance.
Le Seigneur ne l’avait pas encore rachetée de son propre sang,
	il n’avait pas encore lavé dans son sang
		cette ambition coupable d’un honneur immodéré,
	enracinée dans le cœur de tous.
Cette femme avait donc failli en conséquence d’une erreur héréditaire.
