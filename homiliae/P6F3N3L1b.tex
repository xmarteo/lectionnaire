Il nous faut examiner le sens de ces paroles de Marc :	 
4	 	« Il s'assit à la droite de Dieu », et le sens de celles-ci,	 
5	 	d'Étienne : « Je vois les cieux ouverts et le Fils de l'homme	 
6	 	debout à la droite de Dieu. » Pourquoi Marc atteste-t-il	 
7	 	qu'il le voit assis alors qu'Étienne atteste qu'il le voit debout.	 
8	 	Vous le savez, frères, être assis est le propre du juge, être	 
9	 	debout convient à celui qui combat ou qui assiste.