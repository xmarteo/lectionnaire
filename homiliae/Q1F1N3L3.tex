Mais il nous faut savoir que la tentation s'exerce selon	 
28	 	trois degrés : la suggestion, la complaisance, le consente-	 
29	 	ment. Quand nous sommes tentés, nous, nous glissons bien	 
 	--- 351 ---	 
1	 	souvent dans la complaisance ou même le consentement,	 
2	 	parce que nés du péché de la chair nous portons en nous-	 
3	 	mêmes la cause des combats à soutenir. Mais le Dieu qui,	 
4	 	prenant chair dans le sein de la Vierge, est venu sans péché	 
5	 	dans le monde n'avait à souffrir en lui-même aucune dis-	 
6	 	sension. Il a donc pu être tenté par suggestion, mais la com-	 
7	 	plaisance dans le péché n'a pas mordu son âme. Ainsi toute	 
8	 	la tentation diabolique s'est exercée au-dehors, non au-	 
9	 	dedans.