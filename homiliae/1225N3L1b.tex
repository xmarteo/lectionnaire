Puisque nous allons, par une largesse du Seigneur, célé-	 
6	 	brer trois fois la messe aujourd'hui, nous ne pouvons par-	 
7	 	ler longuement du passage de l'Évangile qui a été lu. Cette	 
8	 	nativité de notre Rédempteur nous oblige pourtant à dire	 
9	 	quelques mots, si brefs soient-ils. Pourquoi ce recensement	 
10	 	du monde au moment où le Seigneur va naître ? N'est-ce	 
11	 	pas, comme il apparaît clairement, parce que venait dans la	 
12	 	chair celui qui recenserait ses élus dans l'éternité ? Des	 
13	 	réprouvés, le prophète dit par contre : « Qu'ils soient effa-	 
14	 	cés du livre des vivants, et ne soient pas inscrits avec les jus-	 
15	 	tes. » C'est aussi fort à propos que le Rédempteur naît à	 
16	 	Bethléem, car Bethléem veut dire maison du pain. Il dit en	 
17	 	effet lui-même : « Je suis le pain vivant descendu du ciel. »	 
18	 	Le lieu où naîtrait le Seigneur avait été appelé d'avance mai-	 
19	 	son du pain, parce que devait s'y rendre visible, dans la	 
20	 	matérialité de la chair, celui qui réconforterait l'âme des élus	 
21	 	par un rassasiement intérieur. Il ne naît pas dans la maison	 
22	 	de ses parents, mais en chemin. Il voulait assurément mon-	 
23	 	trer que par l'humanité qu'il avait assumée il naissait en	 
24	 	quelque sorte hors de chez lui.