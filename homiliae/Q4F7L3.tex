Il est donc la lumière qui a fait la lumière que nous voyons.
	Aimons cette divine lumière, désirons-en l’intelligence,
		ayons soif de cette lumière,
	afin que nous puissions sous sa conduite arriver un jour jusqu’à elle,
	et que nous vivions en elle de manière à ne jamais mourir complètement.
C’est en parlant de cette lumière,
	qu’autrefois et longtemps avant qu’elle paraisse,
	le Prophète a chanté dans un Psaume:
	«En vous est une source de vie,
		et dans votre lumière, nous verrons la lumière.»
Remarquez ce que proclame à l’avance au sujet de cette lumière
	l’antique parole d’un des plus saints serviteurs de Dieu:
	«Vous sauverez, Seigneur, les hommes et les animaux,
	puisque vous avez, ô Dieu, multiplié votre miséricorde.»
