Quoi de plus doux, de plus bienveillant que le Seigneur ?	 
4	 	Les Pharisiens le mettent à l'épreuve, leurs pièges se brisent.	 
5	 	Selon les termes du psalmiste : « Leurs coups ont été comme	 
6	 	des flèches d'enfants », et cependant, par respect pour le	 
7	 	sacerdoce, pour la dignité de ce titre, il exhorte le peuple à	 
8	 	leur rester soumis, en considération, non de leur conduite,	 
9	 	mais de leur enseignement. Dans cette phrase : « Les scribes	 
10	 	et les Pharisiens sont assis sur la chaire de Moïse », par chaire il	 
11	 	désigne la doctrine de la Loi. Nous devons donc prendre	 
12	 	également dans le sens de doctrine l'expression du psaume :	 
13	 	« Il ne s'est point assis sur la chaire de pestilence » et « Il ren-	 
14	 	versa les chaires des vendeurs de colombes. »