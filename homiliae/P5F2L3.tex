En effet si un homme aussi saint,
	et qui était absorbé par les soucis de son royaume,
	louait le Seigneur sept fois par jour
	et lui offrait toujours les sacrifices du matin et du soir,
	que devons-nous faire,
	nous qui devons prier d’autant plus que nous chutons souvent,
	à cause de la fragilité de la chair et de l’esprit;
	nous, qui, lassés de la route
		et fatigués par le cours de ce temps et les détours du chemin,
	ne pouvons nous passer de ce pain reconstituant
		qui fortifie le cœur de l’homme?
Le Seigneur nous enseigne qu’il faut veiller non seulement pendant la nuit,
	mais presque à tout instant.
En effet, il vient le soir, et à la seconde, et à la troisième veille,
	et il frappe souvent à notre porte.
Bienheureux donc les serviteurs
	que le Seigneur, lorsqu’il viendra, trouvera veillant.
