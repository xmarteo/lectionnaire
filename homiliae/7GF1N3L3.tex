Pour instruire son	 
26	 	peuple, comme une vigne à cultiver, le Seigneur n'a jamais	 
27	 	cessé d'envoyer des ouvriers : en formant les moeurs de son	 
28	 	peuple, d'abord par les Pères, ensuite par les docteurs de la	 
29	 	loi et les prophètes, et à la fin par les apôtres, il a comme	 
30	 	travaillé à la culture de sa vigne au moyen d'ouvriers.	 
 	--- 427 ---	 
1	 	D'ailleurs tout homme qui sous une forme quelconque et	 
2	 	dans une mesure quelconque a bien oeuvré, avec une foi	 
3	 	droite, a été un ouvrier de cette vigne. Ouvrier le matin, à	 
4	 	la troisième heure, la sixième et la neuvième, tel est désigné	 
5	 	l'antique peuple juif, qui dans ses élus, depuis le début du	 
6	 	monde, s'est efforcé d'honorer Dieu avec une foi droite, ne	 
7	 	cessant de travailler à cultiver sa vigne. A la onzième heure	 
8	 	sont appelés les Gentils, à qui il est dit : « Pourquoi restez-	 
9	 	vous ici tout le jour sans rien faire ? ».