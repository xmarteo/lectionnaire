La lecture du saint Évangile, frères très chers, nous	 
6	 	avertit de prendre bien garde qu'ayant reçu plus que les	 
7	 	autres en ce monde, on le voit bien, nous n'en soyons jugés	 
8	 	plus sévèrement par l'auteur du monde. Plus s'accroissent	 
9	 	les dons, plus s'allongent les comptes à rendre. Chacun doit	 
10	 	donc être d'autant plus humble du fait de sa charge qu'il se	 
11	 	voit plus obligé de rendre compte. Voici qu'un homme par-	 
12	 	tant pour l'étranger appelle ses serviteurs et leur distribue	 
13	 	des talents à faire valoir. Longtemps après il revient pour	 
14	 	établir les comptes, récompense ceux qui ont bien travaillé	 
15	 	pour le bénéfice apporté, mais condamne le serviteur trop	 
16	 	indolent devant la tâche.