« Et moi je vous dis : Aimez vos ennemis, faites du	 
26	 	bien à ceux qui vous haïssent. » Beaucoup, mesurant les pré-	 
27	 	ceptes de Dieu à leur propre faiblesse et non aux forces des	 
28	 	saints, pensent que ces préceptes sont inapplicables. Bien	 
29	 	assez pour nos vertus, disent-ils, de ne pas haïr nos ennemis,	 
30	 	les aimer est un précepte qui dépasse les possibilités de la	 
 	--- 127 ---	 
1	 	nature humaine. Sachons donc que le Christ ne prescrit pas	 
2	 	l'impossible, mais ce qui est parfait : cela, David l'a fait à	 
3	 	l'égard de Saül et d'Absalon. De même le martyr Étienne pria	 
4	 	pour ses ennemis qui le lapidaient. Paul souhaite d'être	 
5	 	anathème pour ses persécuteurs. Tout cela, le Christ l'a ensei-	 
6	 	gné et fait, lui qui disait : « Père, pardonne-leur ; ils ne	 
7	 	savent pas ce qu'ils font. »