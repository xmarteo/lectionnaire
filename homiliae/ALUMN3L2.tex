Alors qu'il proposait à ceux qui le suivaient des comman-	 
17	 	dements nouveaux, il disait : « Si quelqu'un ne renonce pas	 
18	 	à tout ce qu'il possède, il ne peut pas être mon disciple. »	 
19	 	Comme s'il disait ouvertement : Vous qui désiriez le bien	 
20	 	d'autrui dans votre vie ancienne, dans le zèle d'une vie nou-	 
21	 	velle, donnez aussi le vôtre. Écoutons ce qu'il dit dans cette	 
22	 	lecture : « Celui qui veut venir à ma suite, qu'il se renie lui-	 
23	 	même. » Là, il nous dit de renier nos biens, ici, de nous	 
24	 	renier nous-mêmes. Il n'est peut-être pas pénible pour	 
25	 	l'homme de quitter ce qui lui appartient, mais il lui est très	 
26	 	pénible de se quitter lui-même. C'est peu de chose que de	 
27	 	renier ce que l'on a, c'est vraiment beaucoup de renier ce	 
28	 	que l'on est.