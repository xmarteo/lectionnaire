Considérons maintenant qui étaient ceux qui, leur	 
4	 	égarement dévoilé, furent chassés du paradis et liés dans	 
5	 	une forteresse. Vous voyez comment, expulsés par la	 
6	 	mort, la Vie les a rappelés. Aussi lisons-nous selon Mat-	 
7	 	thieu qu'il y avait ânesse et ânon ; de la sorte, comme	 
8	 	dans les deux humains l'un et l'autre sexe avait été	 
9	 	expulsé, dans les deux animaux l'un et l'autre sexe est	 
10	 	rappelé. D'une part donc l'ânesse figurait Ève, mère	 
11	 	d'erreur ; d'autre part son petit représentait l'ensemble	 
12	 	du peuple des Gentils ; aussi est-ce le petit de l'ânesse	 
13	 	qui sert de monture. 5. Et réellement « personne ne l'a	 
14	 	monté », car personne avant le Christ n'avait appelé à	 
15	 	l'Église les peuples des nations ; aussi bien avez-vous lu	 
16	 	en Marc : « Que nul homme encore n'a monté. » (Mc, XI,	 
17	 	2).