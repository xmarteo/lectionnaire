Notons qu'elles remarquent un ange assis à droite : que	 
6	 	signifie cela ? Que désigne la gauche sinon la vie présente,	 
7	 	et que désigne la droite sinon la vie éternelle ? Il est écrit :	 
8	 	« Son bras gauche est sous ma tête et sa droite m'étreindra. »	 
9	 	Comme notre Rédempteur était maintenant passé au-delà de	 
10	 	la corruption de la vie présente, l'ange venu annoncer sa vie	 
11	 	éternelle, assis à droite, était bien à sa place. Cet ange appa-	 
12	 	rut enveloppé d'une robe blanche, car il annonçait les joies	 
13	 	de notre fête ; la blancheur de son vêtement présageait la	 
14	 	splendeur de notre solennité. Mais faut-il dire de la nôtre ou	 
15	 	de la sienne ? A dire vrai, c'est à la fois la sienne et la nôtre :	 
16	 	la résurrection de notre Rédempteur a bien été notre fête,	 
17	 	puisqu'elle nous a ramenés à l'immortalité, mais elle est aussi	 
18	 	la fête des anges, puisqu'en nous ramenant dans le ciel elle	 
19	 	a complété leur nombre.