Nous ne sommes pas le sel de la terre si nous ne péné-	 
10	 	trons pas de cette saveur le coeur de nos auditeurs. Cette	 
11	 	saveur, celui-là la communique vraiment à son prochain, qui	 
12	 	ne lui refuse pas l'annonce de la parole.

Mais nous ne prêcherons vraiment aux autres leur	 
14	 	devoir que si nous illustrons notre parole par des actes.
Il n'est pas de manquement	 
4	 	à Dieu plus grave, je pense, frères très chers, que celui des	 
5	 	prêtres : c'est que Dieu voit ceux qu'il a établis pour redres-	 
6	 	ser les autres donner eux-mêmes l'exemple de la déprava-	 
7	 	tion, c'est que nous péchons nous-mêmes, nous qui aurions	 
8	 	dû empêcher le péché. Nous ne cher-	 
17	 	chons nullement le bien des âmes, nous sommes chaque jour	 
18	 	occupés de nos intérêts, nous convoitons les biens de la	 
19	 	terre, nous cherchons de toute notre âme à capter la gloire	 
20	 	humaine. Du fait même que nous sommes placés à la tête	 
21	 	des autres, que nous avons une plus grande liberté d'action,	 
22	 	nous faisons du service de bénédiction que nous avons reçu	 
23	 	une raison de nous élever ; nous n'avons cure des intérêts de	 
24	 	Dieu, nous nous occupons des affaires de la terre ; nous	 
25	 	avons reçu d'être placés au lieu saint, et nous nous empê-	 
26	 	trons dans les activités terrestres.