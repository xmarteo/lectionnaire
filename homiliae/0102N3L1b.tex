« Jérusalem, Jérusalem qui tues les prophètes ! »
Jérusalem ! Ce ne sont point les pierres et les édifices de la	 
6	 	cité qu'il appelle ainsi, mais ses habitants. Et il la plaint	 
7	 	avec une affection paternelle. De même, nous lisons ailleurs	 
8	 	qu'il pleura à sa vue. Ces mots : « Combien de fois ai-je voulu	 
9	 	rassembler tes enfants ! » témoignent que tous les prophètes du	 
10	 	passé avaient été envoyés par lui. Cette comparaison avec la	 
11	 	poule qui rassemble ses poussins sous ses ailes, nous la lisons	 
12	 	dans le cantique du Deutéronome : « Comme l'aigle protège	 
13	 	son nid et veille sur ses petits, déployant ses ailes, il les a	 
14	 	pris et portés sur ses plumes. »