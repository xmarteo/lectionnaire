car je vous le dis, leurs anges voient sans cesse	 
8	 	la face de mon Père qui est dans les cieux. » Il l'avait dit plus	 
9	 	haut, dans sa comparaison avec la main, le pied et l'oeil,	 
10	 	il faut rompre les parentés et les relations qui peuvent être	 
11	 	occasion de scandale. Aussi la prescription suivante adoucit-	 
12	 	elle la sévérité de sa condamnation : « Prenez garde à ne	 
13	 	pas mépriser un de ces tout-petits. » Je prescris la sévérité,	 
14	 	dit-il, mais en enseignant en même temps la clémence. Autant	 
15	 	qu'il est en vous, ne méprisez pas, mais, tout en faisant	 
16	 	votre salut, cherchez aussi leur guérison. Toutefois, si vous	 
17	 	les voyez persévérer dans le péché, esclaves du vice, mieux	 
18	 	vaut vous sauver seuls que périr plusieurs ensembles. « Parce	 
19	 	que leurs anges dans les cieux voient toujours la face du	 
20	 	Père. » Si grande est la dignité des âmes que chacune, dès	 
21	 	sa naissance, a un ange préposé à sa garde. Aussi lisons-nous	 
22	 	dans l'Apocalypse de Jean : Écris ceci à l'ange d'Éphèse,	 
23	 	de Thyatire, à l'ange de Philadelphie et aux anges des quatre	 
24	 	autres Églises ; et l'Apôtre a prescrit aussi aux femmes de	 
25	 	se voiler la tête dans les églises à cause des anges.