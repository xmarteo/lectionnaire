Car le Christ n’est pas seulement chef, il est aussi corps:
	et pour être dans sa perfection,
	il faut qu’il soit tête et corps tout ensemble.
Ce que sont ses membres, il l’est donc lui-même;
	mais ce qu’il est, ses membres ne le sont pas de prime-abord.
Si ses membres n’étaient pas un autre lui-même, dirait-il:
	«Saul, pourquoi me persécuter?»
Car ce n’était pas lui en personne que Saul persécutait sur la terre:
	c’étaient ses membres, c’est-à-dire ses fidèles;
	néanmoins, il ne les appelle ni ses saints, ni ses serviteurs,
	ni enfin, d’une manière plus honorable, ses frères;
	en parlant d’eux, il dit: Moi,
	ou, en d’autres termes mes membres, dont je suis le chef.
