Il fait remarquer que les mots flatteurs et la douceur	 
10	 	feinte doivent être jugés au fruit des actes, en sorte que	 
11	 	nous n'attendions pas de quelqu'un qu'il soit tel qu'il	 
12	 	se dépeint en paroles, mais tel qu'il se comporte en	 
13	 	actes, parce que beaucoup d'hommes ont la rage du loup	 
14	 	cachée sous un vêtement de brebis. Ainsi, comme les	 
15	 	épines ne produisent pas de raisin, ni les ronces de figue,	 
16	 	comme les arbres mauvais ne portent pas de fruits	 
17	 	comestibles, il nous apprend que chez de tels hommes non	 
18	 	plus il n'y a pas place pour la réalisation d'une bonne	 
19	 	action et que, de ce fait, c'est à ses fruits qu'il faut recon-	 
20	 	naître un chacun. On ne gagne pas, en effet, le Royaume	 
21	 	des cieux par le seul office des mots, et ce n'est pas celui	 
22	 	qui aura dit : Seigneur, Seigneur qui en héritera avec	 
23	 	lui.