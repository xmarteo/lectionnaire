Que désigne le	 
23	 	figuier, sinon la nature humaine ? Que signifie la femme	 
24	 	courbée, sinon cette même nature humaine ? Elle a été bien	 
25	 	plantée comme le figuier, et bien créée comme la femme ;	 
26	 	mais tombée librement dans le péché, elle ne conserve ni le	 
27	 	fruit de ses actes, ni sa ferme rectitude. Se précipitant volon-	 
28	 	tairement dans le péché, refusant de porter du fruit par son	 
29	 	obéissance, elle a perdu cette ferme rectitude. Créée à	 
30	 	l'image de Dieu, elle n'a pas gardé sa dignité, elle a dédai-	 
31	 	gné de rester ce qu'elle avait été plantée ou créée. Le maître	 
32	 	de la vigne est venu trois fois vers le figuier, parce qu'il a	 
 	--- 257 ---	 
1	 	cherché la nature humaine avant la Loi, sous la Loi, sous la	 
2	 	grâce, attendant, avertissant, visitant.