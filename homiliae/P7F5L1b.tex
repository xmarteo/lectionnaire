Quel doit être celui qui annonce le Royaume de Dieu,	 
39	 	les préceptes de l'Évangile l'indiquent : sans bâton, sans	 
 	--- 251 ---	 
1	 	besace, sans chaussure, sans pain, sans argent, c'est-à-	 
2	 	dire ne recherchant pas l'aide des ressources de ce monde,	 
3	 	abandonné à la foi, et comptant que moins il recherchera	 
4	 	les biens temporels, plus ils pourront lui échoir. On peut,	 
5	 	si on le veut, entendre tout cela au sens suivant : ce pas-	 
6	 	sage aurait pour but de former un état d'âme tout spiri-	 
7	 	tuel, qui semble avoir dépouillé le corps comme un vète-	 
8	 	ment, non seulement en renonçant au pouvoir et en mépri-	 
9	 	sant les richesses, mais en écartant même les attraits de	 
10	 	la chair. 66. Il leur est fait, avant tout, une recommanda-	 
11	 	tion générale de paix et de constance : ils apporteront la	 
12	 	paix, garderont la constance, observeront les règles du	 
13	 	droit de l'hospitalité : il ne convient pas au prédicateur	 
14	 	du Royaume des cieux, affirme-t-Il, de courir de maison	 
15	 	en maison, et de modifier les lois inviolables de l'hospita-	 
16	 	lité.