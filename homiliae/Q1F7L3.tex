Mais cette intelligence sublime, bien-aimés, objet	 
2	 	de félicitations, avait à être instruite du mystère de la	 
3	 	nature inférieure du Christ, de peur que la foi de l'Apôtre,	 
4	 	élevée jusqu'à la gloire de confesser la Divinité, ne jugeât	 
5	 	inconvenante et indigne du Dieu impassible notre fai-	 
6	 	blesse qu'il avait assumée, et ne crût la nature humaine	 
7	 	déjà glorifiée à ce point en lui qu'elle ne pût être ni	 
8	 	affectée par le supplice ni dissoute par la mort. Et c'est	 
9	 	pourquoi, comme le Seigneur disait qu'il lui fallait « aller	 
10	 	à Jérusalem et souffrir beaucoup de la part des anciens,	 
11	 	des scribes et des princes des prêtres, et être mis à mort,	 
12	 	et ressusciter le troisième jour », saint Pierre, éclairé par	 
13	 	la lumière d'en haut et encore tout embrasé de la brûlante	 
14	 	ardeur avec laquelle il avait confessé le Fils de Dieu,	 
15	 	repoussa avec un dégoût spontané et, pensait-il, religieux	 
16	 	la perspective de moqueries ignominieuses, et d'une	 
17	 	mort déshonorante et cruelle ; Jésus le reprit alors par	 
18	 	une amicale réprimande et l'anima du désir de partager	 
19	 	sa passion avec lui.