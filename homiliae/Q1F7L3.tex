Or mes biens-aimés, cette hauteur d’intelligence que Jésus avait louée,
	il fallait l’instruire du mystère de la substance inférieure.
La foi de l’apôtre, qui avait été élevée
		jusqu’à la gloire de confesser la divinité dans le Christ,
	ne devait pas juger inconvenante et indigne du Dieu impassible
		l’adoption de notre faiblesse,
	ni, non plus, penser que, dans le Christ,
	l’humaine nature avait été glorifiée
	jusqu’à ne pouvoir ni être atteinte par le supplice
	ni être détruite par la mort.
C’est pourquoi, le Seigneur ayant dit qu’il lui fallait aller à Jérusalem,
	y souffrir beaucoup de la part des anciens,
		des scribes et des princes des prêtres,
	y être tué et y ressusciter le troisième jour;
	et le bienheureux Pierre qui, sous l’illumination de la charité d’en-haut,
	était tout fervent de sa très enthousiaste confession du Fils de Dieu,
	ayant repoussé, avec un dégoût qu’il croyait religieux et bien permis,
	l’outrage des moqueries et la honte d’une mort très cruelle,
	Jésus le reprit avec une bienveillante sévérité
		et l’excita au désir de partager sa passion.
