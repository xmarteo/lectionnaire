Lui, connaissant les secrets	 
15	 	intimes de leurs pensées, - car il n'est rien que Dieu ne	 
16	 	scrute des choses qui sont cachées au coeur de l'homme -,	 
17	 	se fit apporter un denier et demanda de qui était l'in-	 
18	 	scription et l'effigie. Les Pharisiens répondirent qu'elles	 
19	 	étaient de César. Il leur dit qu'il fallait rendre à César	 
20	 	ce qui est à César et à Dieu ce qui est à Dieu.

Ô réponse pleinement miraculeuse et évidence	 
22	 	absolue de la parole céleste ! Tout y est dosé entre le	 
23	 	mépris du siècle et l'outrage d'une offense à César, en	 
 	--- 155 ---	 
1	 	sorte qu'en prononçant qu'il fallait restituer à César ce	 
2	 	qui lui appartenait il délivrait les esprits consacrés à Dieu	 
3	 	de tout souci et obligation d'ordre humain.