« Et toi, enfant, on t'appellera prophète du Très-	 
29	 	Haut. »	 
30	 	Il est bien que, dans cette prophétie sur le Seigneur, il	 
31	 	adresse la parole à son prophète pour montrer qu'il y a	 
32	 	là encore un bienfait du Seigneur : faute de quoi, dans cette	 
33	 	énumération des bienfaits généraux, il eût semblé, comme	 
34	 	un ingrat, taire ceux qu'il avait reçus, qu'il reconnaissait	 
35	 	dans son fils. Mais quelques-uns jugeront peut-être dérai-	 
36	 	sonnable et extravagant d'adresser la parole à un enfant	 
37	 	de huit jours. Pourtant, à la réflexion, nous comprenons	 
38	 	parfaitement qu'il pouvait, une fois né, entendre la voix	 
39	 	de son père, ayant entendu le salut de Marie avant de	 
40	 	naître.