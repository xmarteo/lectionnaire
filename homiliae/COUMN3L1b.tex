Si nous considérons, frères très chers, quels sont les	 
6	 	biens qui nous sont promis dans le ciel et quelle est leur	 
7	 	grandeur, tous ceux que l'on possède sur la terre perdent	 
8	 	pour nous leur valeur. Comparés au bonheur du ciel, les	 
9	 	biens de ce monde sont, en effet, un fardeau, non un sou-	 
10	 	tien. Comparée à la vie éternelle, la vie temporelle doit être	 
11	 	appelée mort plutôt que vie. L'usure quotidienne de notre	 
12	 	corps corruptible est-elle, en effet, autre chose qu'une	 
13	 	longue mort ? Quelle langue peut dire, quelle intelligence	 
14	 	peut comprendre la grandeur des joies de la cité céleste :	 
15	 	prendre place dans les choeurs des anges, avec ces esprits	 
16	 	bienheureux se tenir en présence de la gloire du Créateur,	 
17	 	contempler le visage de Dieu face à face, voir la lumière sans	 
18	 	limite, être libéré de la crainte de la mort, jouir du don	 
19	 	d'une éternelle incorruptibilité ?