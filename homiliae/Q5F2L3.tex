Ils se voyaient souillés d’un crime énorme d’impiété,
	car ils avaient mis à mort celui qu’ils auraient dû respecter et adorer:
	et leur crime leur semblait impossible à expier.
C’était là une grande faute:
	à la considérer dans sa laideur, il y avait de quoi tomber dans le désespoir;
	mais le désespoir leur était défendu,
	puisque, sur la croix, le Seigneur Jésus a bien voulu prier pour eux,
	et qu’il avait dit:	«Père, pardonne-leur, car ils ne savent ce qu’ils font.»
Parmi un grand nombre d’hommes qui devaient le méconnaître toujours,
	il en apercevait quelques-uns, destinés à lui appartenir;
	il demandait leur pardon au moment même où ils l’insultaient:
	et ce qu’il considérait alors, ce n’était pas la mort qu’ils lui donnaient,
	c’était la mort qu’il endurait pour eux.
