Nous essuyons	 
8	 	avec nos cheveux les pieds du Seigneur, quand, ayant	 
9	 	pour les saints la compassion de l'amour, nous les aidons	 
10	 	aussi de notre superflu. La compassion de l'âme doit être	 
11	 	telle qu'une main généreuse montre sa peine. Car il baigne	 
12	 	de ses larmes les pieds du Rédempteur, mais sans les essuyer	 
13	 	de ses cheveux, celui qui compatit à la douleur de son pro-	 
14	 	chain et pourtant ne lui montre pas sa compassion en don-	 
15	 	nant de son superflu. Il pleure et n'essuie pas, celui qui lui	 
16	 	prodigue des paroles de compassion, mais, faute de lui don-	 
17	 	ner ce qui lui manque, ne met pas fin à l'acuité de sa peine.	 
18	 	La femme couvre de baisers les pieds qu'elle essuie. Nous le	 
19	 	faisons pleinement, si nous aimons avec dévouement ceux	 
20	 	que nous aidons de nos largesses, en sorte que le besoin où	 
21	 	se trouve le prochain ne nous soit pas pesant, que son indi-	 
22	 	gence même ne nous soit pas à charge et que la charité ne	 
23	 	s'engourdisse pas dans l'âme, tandis que la main donne le	 
24	 	nécessaire.