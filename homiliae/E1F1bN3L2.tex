« Pourquoi me cherchiez-vous? »	 
18	 	Il y a dans le Christ deux filiations : l'une est de son	 
19	 	Père, l'autre de sa Mère. La première, par son Père, est	 
20	 	toute divine, tandis que par sa Mère Il s'est abaissé à	 
21	 	nos labeurs et à nos usages. Dès lors tout ce qui, dans ses	 
22	 	actes, dépasse la nature, l'âge, la coutume, ne doit pas	 
23	 	être attribué aux facultés humaines, mais rapporté aux	 
24	 	énergies divines.	 
25	 	Ailleurs sa Mère le pousse à un acte mystérieux (Jn,	 
26	 	II, 3) ; ici cette Mère est reprise de réclamer encore qu'il	 
27	 	agisse en homme. Mais, comme ici on le montre âgé de	 
28	 	douze ans, comme là on nous apprend qu'Il a des dis-	 
29	 	ciples, vous voyez que cette Mère a été renseignée sur son	 
30	 	Fils au point de réclamer de sa maturité un mystère,	 
31	 	elle que déconcertait chez l'enfant ce prodige.