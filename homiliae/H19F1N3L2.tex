Le royaume des cieux est donc l'as-	 
8	 	semblée des justes, car le Seigneur règne en eux comme dans	 
9	 	les cieux, puisque, soupirant après les biens d'en haut, ils ne	 
10	 	désirent plus rien sur la terre. Disons donc : « Le royaume	 
11	 	des cieux est semblable à un roi qui célébrait les noces de	 
12	 	son fils. »
Déjà votre charité comprend qui est le roi, père d'un	 
14	 	fils qui est roi. C'est celui à qui le psalmiste dit : « Dieu,	 
15	 	donne ton jugement au roi et ta justice au fils du roi. » Le	 
16	 	texte ajoute : « Qui célébrait les noces de son fils. » Dieu le	 
17	 	Père a célébré les noces de Dieu son Fils, quand il l'a uni à	 
18	 	la nature humaine dans le sein de la Vierge, quand il a voulu	 
19	 	que Celui qui est Dieu avant les siècles se fasse homme à la	 
20	 	fin des siècles.