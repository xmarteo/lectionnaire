Mais si quelqu’un est tombé dans la fornication, ou pire, l’adultère,
	il doit d’autant plus se priver de ce qui est permis,
	qu’il se souvient d’avoir commis des actions défendues.
En effet, les fruits des bonnes œuvres ne doivent pas être pareils
	en celui qui a peu péché, et en celui qui a péché beaucoup;
	ou bien en celui qui n’a jamais commis de crimes,
	celui qui en a commis quelques-uns, et celui qui en a commis beaucoup.
Ces paroles: «Produisez de bons fruits de pénitence»,
	sont donc un appel à la conscience de chacun,
	l’invitant à acquérir par la pénitence
		un trésor de mérites d’autant plus grand,
	qu’il s’est causé de plus grands dommages par le péché.
