Mais si l'on est	 
28	 	tombé dans une faute de fornication, ou peut-être, ce qui est	 
29	 	plus grave, d'adultère, on doit se priver du licite dans la	 
30	 	mesure où l'on se rappelle avoir commis l'illicite. Le fruit	 
31	 	de l'oeuvre de bienfaisance ne doit pas être égal en celui qui	 
32	 	a moins failli et celui qui a failli davantage, ou en celui qui	 
33	 	n'a pas fait de chute grave et celui qui en a fait plusieurs. Par	 
34	 	cette directive : « Faites de dignes fruits de repentir », cha-	 
 	--- 459 ---	 
1	 	cun est intimement pressé de chercher d'autant plus par le	 
2	 	repentir le mérite des oeuvres de bienfaisance qu'il a fait plus	 
3	 	de tort par sa faute.