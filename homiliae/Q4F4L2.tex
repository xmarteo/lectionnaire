Le Sauveur est donc venu, et qu’a-t-il fait?
	Une chose toute mystérieuse et bien digne de remarque.
Il cracha à terre et fit de la boue avec sa salive,
	car le Verbe s’est fait chair,
	et il en frotta les yeux de l’aveugle.
Les yeux de cet homme étaient couverts de boue, et il ne voyait pas encore.
Le Sauveur l’envoya à la piscine qui porte le nom de Siloé.
L’Évangéliste a bien voulu nous indiquer le nom de cette piscine,
	et nous dire qu’il signifie «l’Envoyé».
Vous savez qui a été envoyé;
	s’il ne l’avait pas été, nul d’entre nous n’eût été délivré du péché.
L’aveugle lava donc ses yeux dans cette piscine dont le nom signifie l’Envoyé,
	et il fut baptisé dans le Christ.
Si, en un certain sens, Jésus baptisa en lui-même l’aveugle-né
		au moment où il lui rendait la vue,
	quand il frotta ses yeux avec de la boue, il le fit, sans doute, catéchumène.
