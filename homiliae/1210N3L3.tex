Quand Dieu dit au serpent : « J'établirai une inimitié entre	 
7	 	toi et la femme », qui d'autre te semble-t-il avoir prédit ?	 
8	 	Et si tu hésites encore, pensant que ce n'est pas de Marie	 
9	 	qu'il a parlé, écoute la suite : « Elle t'écrasera la tête. »	 
10	 	A qui cette victoire fut-elle réservée, sinon à Marie ? Aucun	 
11	 	doute, c'est elle qui a écrasé la tête venimeuse, elle qui a	 
12	 	réduit à néant les suggestions de toutes sortes venant du	 
13	 	Mauvais, tant celles des attraits de la chair que celles de	 
14	 	l'orgueil de l'esprit.

Et qui d'autre Salomon cherchait-il quand il disait :	 
16	 	« La femme forte, qui la trouvera ? » Il connaissait bien,	 
17	 	cet homme sage, la faiblesse de ce sexe, son corps fragile,	 
18	 	son esprit inconstant. Pourtant il avait lu que Dieu avait	 
19	 	promis que celui qui avait vaincu par la femme serait à	 
20	 	son tour vaincu par la femme, et il voyait que c'était chose	 
21	 	bien convenable. Aussi, rempli d'une véhémente admiration,	 
22	 	il s'écriait : « La femme forte, qui la trouvera ? », ce qui	 
23	 	était dire : si, de la main d'une femme, dépend notre salut	 
24	 	à tous, la restauration de l'innocence et la victoire sur	 
25	 	l'ennemi, il faut absolument la prévoir forte, celle qui serait	 
26	 	capable d'un si grand ouvrage.