Il faut	 
14	 	encore noter là que l'homme cache le trésor découvert	 
15	 	pour le préserver, car il ne saurait garder la ferveur de son	 
16	 	désir du ciel contre les esprits du mal s'il ne le cache pas	 
17	 	aux louanges humaines. Dans la vie présente, en effet, nous	 
18	 	sommes pour ainsi dire en chemin, allant vers la patrie. Or	 
19	 	les esprits du mal, tels des brigands, nous bloquent la	 
20	 	route. C'est vouloir être détroussé que porter ostensible-	 
21	 	ment un trésor sur la route. Je ne dis pas cela pour que	 
22	 	notre prochain ne voie pas nos actions, puisqu'il est écrit :	 
23	 	« Qu'ils voient vos bonnes actions et glorifient votre	 
24	 	Père », mais pour que nous ne recherchions pas dans ces	 
25	 	actions des louanges extérieures. Que l'action soit	 
26	 	publique, mais que l'intention demeure cachée : donnons	 
27	 	ainsi par notre bonne action un exemple au prochain, et	 
28	 	pourtant par l'intention de ne plaire qu'à Dieu seul sou-	 
29	 	haitons toujours le secret.