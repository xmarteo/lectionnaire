Si tu ne peux marcher par le sentier élevé de la virginité,
	suis au moins Dieu par le chemin très sûr de l'humilité.
Si quelqu'un, fût-il vierge, s'écartait de la rectitude de ce chemin,
	je l'affirme en toute vérité,
	même lui «ne suivrait pas l'Agneau partout où il va».
L'humble qui est souillé suit bien l'Agneau,
	l'homme vierge orgueilleux le suit aussi,
	mais aucun des deux ne le «suit partout où il va».
Ni le premier ne peut s'élever à la pureté de l'Agneau sans tache,
	ni l'autre ne daigne s'abaisser à la douceur avec laquelle il s'est tu,
	non pas devant celui qui le tondait, mais devant celui qui le tuait.
Pourtant le pécheur, en le suivant dans l'humilité,
	a choisi un parti plus salutaire que l'orgueilleux qui le suit dans la virginité.
En effet, l'humble pénitence du premier purifiera sa souillure,
	tandis que l'orgueil du second souillera même sa continence.
