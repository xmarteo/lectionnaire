Puisque, par la largesse de Dieu, vous êtes déjà	 
8	 	entrés dans la salle de noces, c'est-à-dire la sainte Église,	 
9	 	veillez avec soin, frères, à ce que le roi, en y entrant, ne	 
10	 	trouve quelque chose à reprocher dans le vêtement de votre	 
11	 	âme. Il faut considérer avec un coeur plein de crainte ce qui	 
12	 	est ajouté aussitôt : « Le roi entra pour voir les convives et	 
13	 	il aperçut là un homme qui n'avait pas la robe nuptiale. »	 
14	 	Que signifie à votre avis, mes très chers frères, cette robe	 
15	 	nuptiale ? Si nous disons que c'est le baptême ou la foi, qui	 
16	 	donc est allé à ces noces sans le baptême ou la foi ? Si l'on	 
17	 	se trouve au-dehors, c'est du fait que l'on n'a pas encore cru.	 
18	 	Que devons-nous donc entendre par cette robe nuptiale,	 
19	 	sinon la charité ? Car il entre dans la salle de noces, mais	 
20	 	sans la robe nuptiale, celui qui, entré dans la sainte Église, a	 
21	 	la foi, mais non la charité. C'est à juste titre que la charité	 
22	 	est appelée robe nuptiale, puisque notre Créateur l'a revê-	 
23	 	tue, quand il vint aux noces où il s'est uni à l'Église.