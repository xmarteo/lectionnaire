Il est venu	 
5	 	avant la Loi, car il a fait que chacun connaisse, par son intel-	 
6	 	ligence naturelle, comment il devait agir envers le prochain :	 
7	 	comme envers lui-même. Il est venu sous la Loi, car il a	 
8	 	enseigné par ses commandements. Après la Loi, il est venu	 
9	 	par la grâce, car il a montré la présence de sa bonté.	 
10	 	Pourtant, il se plaint de n'avoir pas trouvé de fruit pendant	 
11	 	ces trois ans, car il est des âmes dépravées que la loi	 
12	 	naturelle qui leur est insufflée ne corrige pas, que les com-	 
13	 	mandements ne forment pas, que les merveilles de son incar-	 
14	 	nation ne convertissent pas. Que symbolise le vigneron,	 
15	 	sinon l'ordre des pasteurs ? Étant à la tête de l'Église, ils	 
16	 	prennent soin de la vigne du Seigneur.