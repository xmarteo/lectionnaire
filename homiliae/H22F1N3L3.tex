Si en effet	 
4	 	rien de ce qui est à César n'est demeuré entre nos mains,	 
5	 	nous ne serons pas liés par l'engagement de lui rendre ce	 
6	 	qui lui appartient ; mais si plutôt nous veillons aux	 
7	 	affaires qui sont les siennes, si nous disposons du	 
8	 	droit de notre puissance en nous prêtant comme des	 
9	 	tenanciers à la gestion d'un patrimoine qui n'est pas à	 
10	 	nous, ce n'est pas une injustice à déplorer de restituer à	 
11	 	César ce qui est à César et d'avoir à rendre à Dieu ce	 
12	 	qui lui revient, le corps, l'âme, la volonté. C'est	 
13	 	Dieu en effet qui produit et accroît ces biens que nous	 
14	 	tenons et, par conséquent, il n'y a que justice à res-	 
15	 	tituer tout ce que l'on est à celui auquel on se rappelle	 
16	 	devoir son origine et son développement.