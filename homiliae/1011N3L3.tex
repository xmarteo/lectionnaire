Heureuse est Marie: ni l'humilité ne lui fait défaut, ni la virginité.
Virginité unique, certes,
	celle que la fécondité n'a pas altérée mais honorée,
	et en même temps humilité sans égale,
	celle que la virginité féconde n'a pas détruite mais portée à son sommet;
	et de plus, fécondité sans nulle autre pareille,
	celle qu'accompagnent à la fois la virginité et l'humilité.
Les trois sont admirables, incomparables, uniques.
L'étonnant serait que tu n'hésites pas, en les considérant,
	à décider laquelle est la plus digne de ton admiration,
	ce qui est le plus stupéfiant:
	la fécondité de la vierge ou l'intégrité de la mère,
	la sublimité de l'enfant ou l'humilité jointe à une telle sublimité.
Sauf, sans aucun doute, qu'à chaque chose prise séparément,
	il faut préférer les trois ensemble,
	et qu'il est incomparablement plus éminent et plus heureux
		de les avoir reçues toutes les trois que seulement l'une d'entre elles.
Si nous lisons et voyons que «Dieu est admirable dans ses saints»,
	quoi d'étonnant qu'il se soit montré plus admirable en sa mère?
Ainsi donc, vous, les femmes mariées,
	vénérez l'intégrité de la chair dans une chair corruptible;
	et vous, vierges consacrées, admirez la fécondité dans une vierge;
	et vous, tous les hommes, imitez l'humilité de la Mère de Dieu.
