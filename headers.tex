%%%%%%%%%%%%%%% INDICES %%%%%%%%%%%%%%%

\usepackage{imakeidx}

\indexsetup{level=\section*,toclevel=section,noclearpage,othercode=\footnotesize\thispagestyle{empty}}
\makeindex[name=L,title=Index des livres bibliques, columns=2,columnseprule,options=-s custom]
\makeindex[name=F,title=Index des jours liturgiques, columns=2,columnseprule, options=-s custom]

\newcommand{\fullbook}{DEFAULTBOOK}
\newcommand{\bookorder}{ZZ}
\newcommand{\booklookup}[4]{%
  \IfSubStr{#2}{#1}{ \renewcommand{\fullbook}{#3}\renewcommand{\bookorder}{#4} }{}%
}%
\newcommand{\customreadingindex}[3]{%
	\renewcommand{\fullbook}{DEFAULTBOOK}%
	\renewcommand{\bookorder}{ZZ}
	\booklookup{#1}{Gn}{Livre de la Genèse}{AA}
	\booklookup{#1}{Ex}{Livre de l’Exode}{AB}
	\booklookup{#1}{Lv}{Livre du Lévitique}{AC}
	\booklookup{#1}{Nb}{Livre des Nombres}{AD}
	\booklookup{#1}{Dt}{Livre du Deutéronome}{AE}
	\booklookup{#1}{Jos}{Livre de Josué}{AF}
	\booklookup{#1}{Jg}{Livre des Juges}{AG}
	\booklookup{#1}{Rt}{Livre de Ruth}{AH}
	\booklookup{#1}{1 S}{Premier livre de Samuel}{AJ}
	\booklookup{#1}{2 S}{Deuxième livre de Samuel}{AK}
	\booklookup{#1}{1 R}{Premier livre des Rois}{AL}
	\booklookup{#1}{2 R}{Deuxième livre des Rois}{AM}
	\booklookup{#1}{1 Ch}{Premier livre des Chroniques}{AN}
	\booklookup{#1}{2 Ch}{Deuxième livre des Chroniques}{AP}
	\booklookup{#1}{Esd}{Livre d’Esdras}{AQ}
	\booklookup{#1}{Ne}{Livre de Néhémie}{AR}
	\booklookup{#1}{Tb}{Livre de Tobie}{AS}
	\booklookup{#1}{Jdt}{Livre de Judith}{AT}
	\booklookup{#1}{Est}{Livre d’Esther}{AU}
	\booklookup{#1}{1 M}{Premier Livre des Martyrs d’Israël}{AV}
	\booklookup{#1}{2 M}{Deuxième Livre des Martyrs d’Israël}{AX}
	\booklookup{#1}{Jb}{Livre de Job}{AY}
	\booklookup{#1}{Pr}{Livre des Proverbes}{AZ}
	\booklookup{#1}{Qo}{L'ecclésiaste}{BA}
	\booklookup{#1}{Ct}{Cantique des cantiques}{BB}
	\booklookup{#1}{Sg}{Livre de la Sagesse}{BC}
	\booklookup{#1}{Si}{Livre de Ben Sira le Sage}{BD}
	\booklookup{#1}{Is}{Livre d’Isaïe}{BE}
	\booklookup{#1}{Jr}{Livre de Jérémie}{BF}
	\booklookup{#1}{Lam}{Livre des lamentations de Jérémie}{BG}
	\booklookup{#1}{Ba}{Livre de Baruch}{BH}
	\booklookup{#1}{Ep Jr}{Lettre de Jérémie}{BJ}
	\booklookup{#1}{Ez}{Livre d’Ézékiel}{BK}
	\booklookup{#1}{Dn}{Livre de Daniel}{BL}
	\booklookup{#1}{Os}{Livre d’Osée}{BM}
	\booklookup{#1}{Jl}{Livre de Joël}{BN}
	\booklookup{#1}{Am}{Livre d’Amos}{BP}
	\booklookup{#1}{Abd}{Livre d'Abdias}{BQ}
	\booklookup{#1}{Jon}{Livre de Jonas}{BR}
	\booklookup{#1}{Mi}{Livre de Michée}{BS}
	\booklookup{#1}{Na}{Livre de Nahum}{BT}
	\booklookup{#1}{Ha}{Livre d’Habaquc}{BU}
	\booklookup{#1}{So}{Livre de Sophonie}{BV}
	\booklookup{#1}{Ag}{Livre d’Aggée}{BX}
	\booklookup{#1}{Za}{Livre de Zacharie}{BY}
	\booklookup{#1}{Ml}{Livre de Malachie}{BZ}
	\booklookup{#1}{Mt}{Évangile de Jésus-Christ selon saint Matthieu}{CA}
	\booklookup{#1}{Mc}{Évangile de Jésus-Christ selon saint Marc}{CB}
	\booklookup{#1}{Lc}{Évangile de Jésus-Christ selon saint Luc}{CC}
	\booklookup{#1}{Jn}{Évangile de Jésus-Christ selon saint Jean}{CD}
	\booklookup{#1}{Ac}{Livre des Actes des Apôtres}{CE}
	\booklookup{#1}{Rm}{Lettre de saint Paul Apôtre aux Romains}{DA}
	\booklookup{#1}{1 Co}{Première lettre de saint Paul Apôtre aux Corinthiens}{DB}
	\booklookup{#1}{2 Co}{Deuxième lettre de saint Paul Apôtre aux Corinthiens}{DC}
	\booklookup{#1}{Ga}{Lettre de saint Paul Apôtre aux Galates}{DD}
	\booklookup{#1}{Ep}{Lettre de saint Paul Apôtre aux Ephésiens}{DE}
	\booklookup{#1}{Ph}{Lettre de saint Paul Apôtre aux Philippiens}{DF}
	\booklookup{#1}{Col}{Lettre de saint Paul Apôtre aux Colossiens}{DG}
	\booklookup{#1}{1 Th}{Première lettre de saint Paul Apôtre aux Thessaloniciens}{DH}
	\booklookup{#1}{2 Th}{Deuxième lettre de saint Paul Apôtre aux Thessaloniciens}{DJ}
	\booklookup{#1}{1 Tm}{Première lettre de saint Paul Apôtre à Timothée}{DK}
	\booklookup{#1}{2 Tm}{Deuxième lettre de saint Paul Apôtre à Timothée}{DL}
	\booklookup{#1}{Tt}{Lettre de saint Paul Apôtre à Tite}{DM}
	\booklookup{#1}{Phm}{Lettre de saint Paul Apôtre à Philémon}{DN}
	\booklookup{#1}{He}{Lettre aux Hébreux}{DP}
	\booklookup{#1}{Jc}{Lettre de saint Jacques Apôtre}{DQ}
	\booklookup{#1}{1 P}{Première lettre de saint Pierre Apôtre}{DR}
	\booklookup{#1}{2 P}{Deuxième lettre de saint Pierre Apôtre}{DS}
	\booklookup{#1}{1 Jn}{Première lettre de saint Jean}{DT}
	\booklookup{#1}{2 Jn}{Deuxième lettre de saint Jean}{DU}
	\booklookup{#1}{3 Jn}{Troisième lettre de saint Jean}{DV}
	\booklookup{#1}{Jd}{Lettre de saint Jude}{DX}
	\booklookup{#1}{Ap}{Livre de l’Apocalypse}{DZ}
	\index[L]{\bookorder@\fullbook!#2}
	\ifblank{#3}{}{\index[L]{\bookorder@\fullbook!#3}}
}%


%%%%%%%%%%%%%%% STANDARD PACKAGES %%%%%%%%%%%%%%%

%% Smaller format for the convenience of not redoing all the gregorio size tweaks, to be printed on KDP standard 8.25in x 11in or larger.
%\usepackage[paperwidth=165mm, paperheight=220mm]{geometry}
%% Normal scale
%\usepackage[paperwidth=8.25in, paperheight=11in]{geometry}
%% Slightly smaller format
\usepackage[paperwidth=180mm, paperheight=240mm]{geometry}

\usepackage{fontspec}
\usepackage[nolocalmarks]{polyglossia}
\usepackage[table]{xcolor}
\usepackage{fancyhdr}
\usepackage{titlesec}
\usepackage{setspace}
\usepackage{expl3}
\usepackage{hyperref}
\usepackage{refcount}
\usepackage{needspace}
\usepackage{etoolbox}
\usepackage{enumitem}

%%%%%%%%%%%%%%% HYPHENATION AND TYPOGRAPHICAL CONVENTIONS %%%%%%%%%%%%%%

\setdefaultlanguage[variant=french, frenchitemlabels=false]{french}

%%%%%%%%%%%%%%% GEOMETRY %%%%%%%%%%%%%%%

%% Margin targets on KDP standard 8.25in x 11in = 209.6mm x 279.4mm : inner = 25mm, outer = 12mm, top = 12mm, bottom = 12mm
%% Calculated margins for 165x220: inner = 22.9, outer=top=bottom=11.4 (with some margin of error)
\geometry{
inner=25mm,
outer=12mm,
top=18mm,
bottom=15mm,
headsep=6mm,
}

%% General scale of all graphical elements.
%% Values different from 1 are largely untested.
%% Used in those commands (e.g. everything FontSpec) that use a scale parameter.
\newcommand{\customscale}{1}

%% Provide the command \fpevalc as a copy of the code-level \fp_eval:n.
%% \fpevalc allows to evaluate floating point calculation for scaled parameters, e.g. \setSomeStretchFactor{\fpevalc{0,6 * \customscale}}
\ExplSyntaxOn
\cs_new_eq:NN \fpevalc \fp_eval:n
\ExplSyntaxOff

%% No indentation of paragraphs. This is extremely important for both hymns, readings, gregorio scores, etc. to be displayed properly.
\setlength{\parindent}{0mm}

%% We want to allow large inter-words space 
%% to avoid overfull boxes in two-columns rubrics.
%\sloppy %% this is commented out for books that are mainly one-column to help detect badly formatted lines

%%%%%%%%%%%%%%% GREGORIO CONFIG %%%%%%%%%%%%%%%

\usepackage[autocompile]{gregoriotex}

%% text above lines shall be of color gregoriocolor
\grechangestyle{abovelinestext}{\color{gregoriocolor}\footnotesize}
%% fine-tuning of space beween the staff and the text above lines
\newcommand{\altraise}{-1.4mm} %% default is -0.1cm
\grechangedim{abovelinestextraise}{\altraise}{scalable}
\grechangedim{abovelinestextheight}{6mm}{scalable}

%% fine-tuning of space between the staff and the lyrics
\newcommand{\textraise}{2.8ex} %% default is 3.48471 ex
\grechangedim{spacelinestext}{\textraise}{scalable}

%% fine-tuning of space between the initial and the annotations
\newcommand{\annraise}{0mm} %% default is -0.2mm
\grechangedim{annotationraise}{\annraise}{scalable}

%% \officepartannotation converts a letter (IHARPT) into the office part to be printed as annotation,
%% storing the result into \result.

\newcommand{\result}{}
\newcommand{\lookup}[3]{%
  \IfSubStr{#2}{#1}{ \renewcommand{\result}{#3} }{}%
}%
\newcommand{\officepartannotation}[1]{%
  \renewcommand{\result}{#1}%
  \lookup{#1}{T}{}%
  \lookup{#1}{H}{Hymn.}%
  \lookup{#1}{A}{Ant.}%
  \lookup{#1}{P}{}%
  \lookup{#1}{R}{Resp.}%
  \lookup{#1}{I}{Invit.}%
  \result%
}%

%% header capture setup for the mode
\newcommand{\defaultannotationshift}{-2mm}
\newcommand{\modeannotation}[1]{\greannotation{\hspace{\defaultannotationshift}#1}}
\gresetheadercapture{mode}{modeannotation}{string}

%% outputs a score with no label, indexing, initials or annotations
%% for 
\newcommand{\unindexedscore}[1]{
  \gresetinitiallines{0}
  %% the use of a directory called "gabc" is linked
  %% to the management of gabc files by the website: do not change 
  %% without also changing the website static files structure
  \gregorioscore{\subfix{gabc/#1}}
  \gresetinitiallines{1}
}

%% outputs a score with label, indexing, and annotations. no initials if [n] is passed
\makeatletter
\newcommand{\gscore}[5][y]{
  %% #1 (passed as option) : y = initial, n = no initial
  %% #2 : name of the score file, should be a code, e.g. Q4F4A3 or 1225N1R1
  %% #3 : office-part among the values: T, H, A, P, R, I (toni communes, hy., ant., psalmus, resp., invit.)
  %% #4 : if applicable, a number between 1 and 9 (rank of the ant./resp.) - else: empty
  %% #5 : the indexed name of the piece
  
  %% this prevents page breaks between the phantom section and its label, and the actual score.
  \needspace{4\baselineskip} 
  \protected@edef\@currentlabelname{#5}
  \phantomsection
  \label{#2}
  %% we add the office part, and number of that ant. or resp. in the current office, if applicable
  %% todo : the negative hspace is here because somehow the initial and annotation (first line only) are misaligned by 1mm _with this initial font size_.
  %% this should probably be fixed in a more elegant way.
  \greannotation[c]{\hspace{-1.4mm}\hspace{\defaultannotationshift}\officepartannotation{#3}#4}
  %% if #5 (indexed name) is blank, nothing is indexed.
  %% this is for pieces that are repetitions of another piece (antiphons after psalms)
  \ifblank{#5}{}{\index[#3]{#5}}
  %% if optional arg #1 has been passed as 'n', set no initial
  \ifx n#1\gresetinitiallines{0}\fi
  %% the use of a directory called "gabc" is linked
  %% to the management of gabc files by the website: do not change 
  %% without also changing the website static files structure
  \gregorioscore{\subfix{gabc/#2}}
  %% if optional arg #1 has been passed as 'n', unset no initial
  \ifx n#1\gresetinitiallines{1}\fi
  \vspace{1mm}
}
\makeatother

\newcommand{\chantedscripturereading}[1]{
	\gregorioscore{scriptura/#1}
}

%%%%%%%%%%%%%%% FONTS %%%%%%%%%%%%%%%

%%%%%%%%%%%%%%% Main font
\setmainfont[Ligatures=TeX, Scale=\customscale]{Charis SIL}
%\setmainfont[Ligatures=TeX, Scale=\customscale]{TeXGyreBonum-Regular}
\setstretch{\fpevalc{1.05 * \customscale}}

%%%%%%%%%%%%%%% Score initials
%% \initialsize resizes the initials, with one argument (size in points)
\newcommand{\initialsize}[1]{
    \grechangestyle{initial}{\fontspec{Zallman Caps}\fontsize{#1}{#1}\selectfont}
}
%% default initial size is 32 points
\newcommand{\defaultinitialsize}{28}
\initialsize{\defaultinitialsize}

%% spacing before and after initials to kern the Zallman Caps.
%% this should be changed if we move away from Zallman Caps.
\newcommand{\initialspace}[2]{
  \grechangedim{afterinitialshift}{#2}{scalable}
  \grechangedim{beforeinitialshift}{#1}{scalable}
}
%% default space before and after initials is 0mm to the left and 2mm to the right.
\newcommand{\defaultinitialspace}{\initialspace{0mm}{-\defaultannotationshift}}
\defaultinitialspace{}

%%%%%%%%%%%%%%% Score annotations
\grechangestyle{annotation}{\small}

%%%%%%%%%%%%%%% GRAPHICAL ELEMENTS %%%%%%%%%%%%%%%

%% V/, R/, A/ and + signs for in-line use (\vv \rr \aa \cc)
\newcommand{\specialcharhsep}{3mm} % space after invoking R/ or V/ or A/ outside rubrics
\newcommand{\vv}{\textcolor{gregoriocolor}{\fontspec[Scale=\customscale]{Charis SIL}℣.\hspace{\specialcharhsep}}}
\newcommand{\rr}{\textcolor{gregoriocolor}{\fontspec[Scale=\customscale]{Charis SIL}℟.\hspace{\specialcharhsep}}}
\renewcommand{\aa}{\textcolor{gregoriocolor}{\fontspec[Scale=\customscale]{Charis SIL}\Abar.\hspace{\specialcharhsep}}}
\newcommand{\cc}{\textcolor{gregoriocolor}{\fontspec[Scale=\customscale]{FreeSerif}\symbol{"2720}~}}
%% Same special characters, for in-score use (<sp>V/ R/ A/ +</sp>)
\gresetspecial{V/}{\textcolor{gregoriocolor}{\fontspec[Scale=\customscale]{Charis SIL}℣.~}}
\gresetspecial{R/}{\textcolor{gregoriocolor}{\fontspec[Scale=\customscale]{Charis SIL}℟.~}}
\gresetspecial{A/}{\textcolor{gregoriocolor}{\fontspec[Scale=\customscale]{Charis SIL}\Abar.~}}
\gresetspecial{+}{{\fontspec[Scale=\customscale]{FreeSerif}†~}}
\gresetspecial{*}{\gresixstar}
%% Same special characters, for use in rubrics (no space, and no red command since it will be reddified with the rest)
\newcommand{\vvrub}{{\fontspec[Scale=\customscale]{Charis SIL}℣.~}}
\newcommand{\rrrub}{{\fontspec[Scale=\customscale]{Charis SIL}℟.~}}
\newcommand{\aarub}{{\fontspec[Scale=\customscale]{Charis SIL}\Abar.~}}

%% the asterisk as found in the mediants of text-only psalms
\newcommand{\psstar}{\GreSpecial{*}}
\newcommand{\pscross}{\GreSpecial{+}}
%% also, most psalms do not call those but use † and * - todo

%% Roman Numerals
\usepackage{modroman}
\newcommand{\Rnum}[1]{\nbRoman{#1}}
\newcommand{\rnum}[1]{\nbshortroman{#1}}

%% Macro to print versicles
\newcommand{\versiculus}[2]{\par\vv #1 \\ \rr #2\par}

\newcommand{\versiculustpall}[2]
	{\versiculus{#1 \rubric{(T.P.} Allelúja. \rubric{)}}{#2 \rubric{(T.P.} Allelúja. \rubric{)}}}

%% Macro to print rubrics
\newcommand{\rubric}[1]{\textcolor{gregoriocolor}{\emph{#1}}}

%% Macro to print the name of a score in normal characters inside a \rubric
\newcommand{\normaltext}[1]{{\normalfont\normalcolor #1}}
\newcommand{\scorename}[1]{\normaltext{\nameref{M-#1}}}

%% Macro to print a full reference to a responsory
%% #1 is the R/ number in the feast, #2 is the R/ code, #3 is an optional additional text, like "sine Gloria Patri".
\newcommand{\respref}[3]{\rubric{%
\rrrub #1 \scorename{#2}, pag.\ \pageref{M-#2}%
\if\relax\detokenize{#3}\relax%
.%
\else%
, #3.%
\fi%
}}

\newcommand{\resprefcumgp}[2]
	{\respref{#1}{#2}{sed cum \normaltext{Glória Patri} in fine}}
	
\newcommand{\resprefsinegp}[2]
	{\respref{#1}{#2}{sine \normaltext{Glória Patri}}}

%% Macro to print the common rubric that signals the Te Deum
\newcommand{\tedeumrubric}{\rubric{Lectione ultima peracta Hymnus \normaltext{Te Deum} cantatur.}}

%%% Macro to print a reading
\newcommand{\reading}[7]{
	%% #1 : reading code (unique), normally unused except for consistency checks
	%% #2 : source file name (no extension), which should match the reading code except when the reading is a duplicate
	%% #3 : the reading incipit
	%% #4 : the rubric (normally indicating the book from which the reading is)
	%% #5 : the index book, if an index entry must be created
	%% #6 : the index chapter, if applicable
	%% #7 : a second index chapter, if applicable (no reading can be from two different books: for gospel incipits, they are separate)
\needspace{8\baselineskip} % break page if less than 8 lines are available
\phantomsection
\label{#1}
\ifblank{#5}{}{
	\customreadingindex{#5}{#6}{#7}
	%\index[L]{#5!#6}
}
\ifblank{#3}{
	\vspace{5mm}
}{
	\begin{center}
	#3\par
	\end{center}
}
\ifblank{#4}{}{
	\rubric{#4}\par
}
% this next part makes the tab character active, I think by making it lowercase ~ and defining it as such (because \def<tab>{something} does not work)
\begingroup
  \begingroup
    \lccode`\~=9\relax
  \lowercase{%
  \endgroup
    \def~%
  }{\hspace{8mm}}%
  \catcode9=\active
  \obeylines\input{scriptura/#2.tex}%
\endgroup
}

%%%%%%%%%%%%%%% COLUMN MANAGEMENT %%%%%%%%%%%%%%%

\usepackage{parcolumns}
\setlength\columnseprule{0.4pt}

\newcommand{\twocolrubric}[2]{
	\begin{parcolumns}[rulebetween]{2}
	\colchunk{%
      \rubric{#1}
	}
	\colchunk{%
	  \rubric{#2} 
    }
	\end{parcolumns}
}

\usepackage{paracol} % worse than parcolumns in terms of space management, but can do facing pages
\twosided[pm]

%%%%%%%%%%%%%%% HEADER STYLES %%%%%%%%%%%%%%%

\pagestyle{fancy}
\fancyhead{}
\fancyfoot{}
\renewcommand{\headrulewidth}{0pt}
\setlength{\headheight}{20pt}
\fancyhead[RO]{\small\rightmark\hspace{1cm}\thepage}
\fancyhead[LE]{\small\thepage\hspace{1cm}\leftmark}

% this command is called every time the left and right header texts are set (e.g. by calling \feast)
% the hyphenpenalty override is needed only on older versions of gregorio which do not reset it correctly after typesetting the score.
% see https://tex.stackexchange.com/questions/581013/lualatex-hyphenation-issue-in-fancyhdr-with-gregoriotex-and-multicols-latin-te
\newcommand{\setheaders}[2]{
	\renewcommand{\rightmark}{\hyphenpenalty=50{\sc#2}}
	\renewcommand{\leftmark}{\hyphenpenalty=50{\sc#1}}
}
\setheaders{}{}

%%%%%%%%%%%%%%% TITLE STYLES %%%%%%%%%%%%%%%

%% Titles are centered and small-caps
\titleformat{\chapter}[block]{\Large\filcenter\sc}{}{}{}
\titleformat{\section}[block]{\large\filcenter\sc}{}{}{}
\titleformat{\subsection}[block]{\filcenter\sc}{}{}{}
\setcounter{secnumdepth}{0}
%% Fine-tuning of space around titles
\titlespacing*{\paragraph}{0pt}{1ex}{.6ex}

%% Typesets all titles throughout the NR except Nocturn titles and a few special titles.
%% Using a continuation is necessary because there are 11 arguments.
%% Only \feast, \nocturn, \intermediatetitle and \smalltitle should ever be used.
\newcommand{\feast}[6]{
  %% #1: feast code, e.g. 1225 or A1F1
  %% #2: feast title
  %% #3: left header title
  %% #4: right header title
  %% #5: title level
    %% title level 1 : full page width, a few major feasts + titles of temporale, sanctorale, etc.
	%% title level 2 : all feasts, sundays and major ferias
	%% title level 3 : ferias
  %% #6: incipit date (goes above feast title)
  %% cont'd #1: 1954 rank
  %% cont'd #2: 1960 rank
  %% cont'd #3: name of the feast as it shows up in the index
  %% cont'd #4: 1945 feast-wide rubrics
  %% cont'd #5: 1960 feast-wide rubrics

  %% needspace: should be barely more than the vertical space for the titles, rubrics excluded.
  %% this is to ensure that the page does not get cut after the title or the phantomsection.
  \needspace{15\baselineskip}
  %% phantomsection is to allow the label to attach to the title and not the previous counter object.
  \phantomsection
  \label{#1}
  \begin{center}
  %% we typeset a line for the date if the date is not blank
  \ifblank{#6}{}{
    {\large #6}\\%
  }%
  %% the actual title
  \ifx 1#5{\setstretch{1.2}\sc\huge #2\par}\fi
  \ifx 2#5{\Large #2\par}\fi
  \ifx 3#5{\large #2\par}\fi
  \end{center}
  %% If this is a level 1 title, empty pagestyle
  \ifx 1#5\thispagestyle{empty}\fi
  
  %% we define the header titles manually
  \setheaders{#3}{#4}
  
  %% moving on to a continuation macro to unpack the last 5 arguments
  \feastcontinued
}
\newcommand{\feastcontinued}[5]
{
  %% we name the last 5 arguments
  \def\oldrank{#1}%
  \def\newrank{#2}%
  \def\indexfeastname{#3}%
  \def\oldrubric{#4}%
  \def\newrubric{#5}%
  %% we index the feast if the indexing name is given
  \ifblank{#3}{}{
	\index[F]{\indexfeastname}
  }
  %% we typeset a two-column rank & rubrics block if one rank is filled in
  \ifblank{#1}{}{%
    \begin{parcolumns}[rulebetween]{2}
	\colchunk{%
      {\centering\oldrank\par}%
	  \ifblank{#4}{}{\rubric{\oldrubric}}
	}
	\colchunk{%
	  {\centering\newrank\par}%
	  \ifblank{#5}{}{\rubric{\newrubric}}
    }
	\end{parcolumns}
	\vspace{2mm}
  }
}

\newcommand{\smalltitle}[1]{
  \par{\centering\textbf{#1}\par}
}

\newcommand{\intermediatetitle}[1]{
  \begin{center}
  {\large #1}
  \end{center}
 }

\newcommand{\nocturn}[1]{
  \needspace{8\baselineskip}
  \intermediatetitle{In \Rnum{#1} Nocturno}
}

%% For parts of the book that are typeset with bilingual facing pages, this allows to skip to the next left-hand page at the beginning of such a part.
\newcommand*\cleartoleftpage{%
  \clearpage
  \ifodd\value{page}\hbox{}\newpage\fi
}

%% command to wrap printindex and set the headers for indices
\newcommand{\cprintindex}[2]{
	\setheaders{Index}{#2}
	\pagestyle{fancy}
	\thispagestyle{empty}
	\printindex[#1]
}

