Joseph, âgé de dix-sept ans, faisait paître le petit bétail avec ses frères. Le jeune homme accompagnait les fils de Bilha et les fils de Zilpa, femmes de son père. Il fit part à leur père de la mauvaise réputation de ses frères.

03 Israël, c’est-à-dire Jacob, aimait Joseph plus que tous ses autres enfants, parce qu’il était le fils de sa vieillesse, et il lui fit faire une tunique de grand prix.

04 En voyant qu’il leur préférait Joseph, ses autres fils se mirent à détester celui-ci, et ils ne pouvaient plus lui parler sans hostilité.

05 Joseph eut un songe et le raconta à ses frères qui l’en détestèrent d’autant plus.

06 « Écoutez donc, leur dit-il, le songe que j’ai eu.

07 Nous étions en train de lier des gerbes au milieu des champs, et voici que ma gerbe se dressa et resta debout. Alors vos gerbes l’ont entourée et se sont prosternées devant ma gerbe. »

08 Ses frères lui répliquèrent : « Voudrais-tu donc régner sur nous ? nous dominer ? » Ils le détestèrent encore plus, à cause de ses songes et de ses paroles.

09 Il eut encore un autre songe et le raconta à ses frères. Il leur dit : « Écoutez, j’ai encore eu un songe : voici que le soleil, la lune et onze étoiles se prosternaient devant moi. »

10 Il le raconta également à son père qui le réprimanda et lui dit : « Qu’est-ce que c’est que ce songe que tu as eu ? Nous faudra-t-il venir, moi, ta mère et tes frères, nous prosterner jusqu’à terre devant toi ? »