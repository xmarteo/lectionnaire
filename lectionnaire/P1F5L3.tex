Par les mains des Apôtres,
	beaucoup de signes et de prodiges s’accomplissaient dans le peuple.
Tous les croyants, d’un même cœur, se tenaient sous le portique de Salomon.
Personne d’autre n’osait se joindre à eux;
	cependant tout le peuple faisait leur éloge;
	de plus en plus, des foules d’hommes et de femmes,
	en devenant croyants, s’attachaient au Seigneur.
On allait jusqu’à sortir les malades sur les places,
	en les mettant sur des civières et des brancards:
	ainsi, au passage de Pierre, son ombre couvrirait l’un ou l’autre.
La foule accourait aussi des villes voisines de Jérusalem,
	en amenant des gens malades ou tourmentés par des esprits impurs.
	Et tous étaient guéris.
