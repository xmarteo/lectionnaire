En ces jours-là, Pierre se leva au milieu des frères
	qui étaient réunis au nombre d’environ cent vingt personnes, et il déclara:
	«Frères, il fallait que l’Écriture s’accomplisse.
En effet, par la bouche de David, l’Esprit Saint avait d’avance parlé de Judas,
	qui en est venu à servir de guide aux gens qui ont arrêté Jésus:
	ce Judas était l’un de nous et avait reçu sa part de notre ministère;
	puis, avec le salaire de l’injustice, il acheta un domaine;
	il tomba la tête la première, son ventre éclata,
	et toutes ses entrailles se répandirent.
Tous les habitants de Jérusalem en furent informés,
	si bien que ce domaine fut appelé dans leur propre dialecte Hakeldama,
	c’est-à-dire Domaine-du-Sang.
Car il est écrit au livre des Psaumes:
	Que son domaine devienne un désert, et que personne n’y habite,
	et encore: Qu’un autre prenne sa charge.
Or, il y a des hommes qui nous ont accompagnés
	durant tout le temps où le Seigneur Jésus a vécu parmi nous,
	depuis le commencement, lors du baptême donné par Jean,
	jusqu’au jour où il fut enlevé d’auprès de nous.
Il faut donc que l’un d’entre eux devienne, avec nous,
	témoin de sa résurrection.»
On en présenta deux:
	Joseph appelé Barsabbas, puis surnommé Justus, et Matthias.
Ensuite, on fit cette prière: «Toi, Seigneur, qui connais tous les cœurs,
	désigne lequel des deux tu as choisi
		pour qu’il prenne, dans le ministère apostolique,
	la place que Judas a désertée
		en allant à la place qui est désormais la sienne.»
On tira au sort entre eux, et le sort tomba sur Matthias,
	qui fut donc associé par suffrage aux onze Apôtres.
