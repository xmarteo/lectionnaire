En tant que coopérateurs de Dieu,
	nous vous exhortons encore
		à ne pas laisser sans effet la grâce reçue de lui.
Car il dit dans l’Écriture:
	Au moment favorable je t’ai exaucé, au jour du salut je t’ai secouru.
Le voici maintenant le moment favorable, le voici maintenant le jour du salut.
Pour que notre ministère ne soit pas exposé à la critique,
	nous veillons à ne choquer personne en rien.
Au contraire, en tout, nous nous recommandons nous-mêmes
		comme des ministres de Dieu:
	par beaucoup d’endurance, dans les détresses, les difficultés,
	les angoisses, les coups, la prison, les émeutes,
	les fatigues, le manque de sommeil et de nourriture,
	par la chasteté, la connaissance, la patience et la bonté,
		la sainteté de l’esprit et la sincérité de l’amour,
	par une parole de vérité, par une puissance qui vient de Dieu;
	nous nous présentons avec les armes de la justice
		pour l’attaque et la défense,
	dans la gloire et le mépris, dans la mauvaise et la bonne réputation.
On nous traite d’imposteurs, et nous disons la vérité;
	on nous prend pour des inconnus, et nous sommes très connus;
	on nous croit mourants, et nous sommes bien vivants;
	on nous punit, et nous ne sommes pas mis à mort;
	on nous croit tristes, et nous sommes toujours joyeux;
	pauvres, et nous faisons tant de riches;
	démunis de tout, et nous possédons tout.
