C’est pourquoi, bien-aimés, en attendant cela,
	faites tout pour qu’on vous trouve sans tache ni défaut, dans la paix.
Et dites-vous bien que la longue patience de notre Seigneur, c’est votre salut,
	comme vous l’a écrit également Paul, notre frère bien-aimé,
	avec la sagesse qui lui a été donnée.
C’est ce qu’il dit encore dans toutes les lettres où il traite de ces sujets;
	on y trouve des textes difficiles à comprendre,
	que torturent des gens sans instruction et sans solidité,
	comme ils le font pour le reste des Écritures:
	cela les mène à leur propre perdition.
Quant à vous, bien-aimés, vous voilà prévenus; prenez garde:
	ne vous laissez pas entraîner dans l’égarement des gens dévoyés,
	et n’abandonnez pas l’attitude de fermeté qui est la vôtre.
Mais continuez à grandir
	dans la grâce et la connaissance de notre Seigneur et Sauveur, Jésus Christ.
À lui la gloire, dès maintenant et jusqu’au jour de l’éternité. Amen.
