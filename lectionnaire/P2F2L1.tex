Alors quelques membres du groupe des pharisiens qui étaient devenus croyants
	intervinrent pour dire qu’il fallait circoncire les païens
	et leur ordonner d’observer la loi de Moïse.
Les Apôtres et les Anciens se réunirent pour examiner cette affaire.
Comme cela provoquait une intense discussion, Pierre se leva et leur dit:
	«Frères, vous savez bien comment Dieu, dans les premiers temps,
	a manifesté son choix parmi vous:
	c’est par ma bouche que les païens ont entendu la parole de l’Évangile
	et sont venus à la foi.
Dieu, qui connaît les cœurs, leur a rendu témoignage
	en leur donnant l’Esprit Saint tout comme à nous;
	sans faire aucune distinction entre eux et nous,
	il a purifié leurs cœurs par la foi.
Maintenant, pourquoi donc mettez-vous Dieu à l’épreuve
	en plaçant sur la nuque des disciples
		un joug que nos pères et nous-mêmes n’avons pas eu la force de porter?
Oui, nous le croyons,
	c’est par la grâce du Seigneur Jésus que nous sommes sauvés,
	de la même manière qu’eux.»
Toute la multitude garda le silence,
	puis on écouta Barnabé et Paul exposer tous les signes et les prodiges
	que Dieu avait accomplis grâce à eux parmi les nations.
