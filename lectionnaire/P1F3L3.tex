Hommes d’Israël, écoutez les paroles que voici.
	Il s’agit de Jésus le Nazaréen, homme que Dieu a accrédité auprès de vous
	en accomplissant par lui des miracles,
	des prodiges et des signes au milieu de vous,
	comme vous le savez vous-mêmes.
Cet homme, livré selon le dessein bien arrêté et la prescience de Dieu,
	vous l’avez supprimé en le clouant sur le bois par la main des impies.
Mais Dieu l’a ressuscité en le délivrant des douleurs de la mort,
	car il n’était pas possible qu’elle le retienne en son pouvoir.
En effet, c’est de lui que parle David dans le psaume:
	Je voyais le Seigneur devant moi sans relâche:
	il est à ma droite, je suis inébranlable.
	C’est pourquoi mon cœur est en fête, et ma langue exulte de joie;
	ma chair elle-même reposera dans l’espérance:
	tu ne peux m’abandonner au séjour des morts
	ni laisser ton fidèle voir la corruption.
	Tu m’as appris des chemins de vie,
	tu me rempliras d’allégresse par ta présence.
Frères, il est permis de vous dire avec assurance,
	au sujet du patriarche David,
	qu’il est mort, qu’il a été enseveli,
	et que son tombeau est encore aujourd’hui chez nous.
Comme il était prophète, il savait que Dieu lui avait juré
	de faire asseoir sur son trône un homme issu de lui.
Il a vu d’avance la résurrection du Christ, dont il a parlé ainsi:
	Il n’a pas été abandonné à la mort,
	et sa chair n’a pas vu la corruption.
