À notre arrivée à Rome,
	Paul reçut l’autorisation d’habiter en ville avec le soldat qui le gardait.
Trois jours après, il fit appeler les notables des Juifs.
Quand ils arrivèrent, il leur dit:
	«Frères, moi qui n’ai rien fait contre notre peuple
		et les coutumes reçues de nos pères,
	je suis prisonnier depuis Jérusalem où j’ai été livré aux mains des Romains.
Après m’avoir interrogé, ceux-ci voulaient me relâcher,
	puisque, dans mon cas, il n’y avait aucun motif de condamnation à mort.
Mais, devant l’opposition des Juifs,
	j’ai été obligé de faire appel à l’empereur,
	sans vouloir pour autant accuser ma nation.
C’est donc pour ce motif que j’ai demandé à vous voir et à vous parler,
	car c’est à cause de l’espérance d’Israël que je porte ces chaînes.»
