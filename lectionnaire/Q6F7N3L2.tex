Voilà pourquoi il est le médiateur d’une alliance nouvelle,
		d’un testament nouveau:
	puisque sa mort a permis le rachat
		des transgressions commises sous le premier Testament,
	ceux qui sont appelés peuvent recevoir l’héritage éternel jadis promis.
Or, quand il y a testament,
	il est nécessaire que soit constatée la mort de son auteur.
Car un testament ne vaut qu’après la mort,
	il est sans effet tant que son auteur est en vie.
C’est pourquoi le premier Testament lui-même n’a pas été inauguré
	sans que soit utilisé du sang.
