Paul et ceux qui l’accompagnaient s’embarquèrent à Paphos
	et arrivèrent à Pergé en Pamphylie.
	Mais Jean-Marc les abandonna pour s’en retourner à Jérusalem.
Quant à eux, ils poursuivirent leur voyage au-delà de Pergé
	et arrivèrent à Antioche de Pisidie.
Le jour du sabbat, ils entrèrent à la synagogue et prirent place.
Après la lecture de la Loi et des Prophètes,
	les chefs de la synagogue leur envoyèrent dire:
	«Frères, si vous avez une parole d’exhortation pour le peuple, parlez.»
Paul se leva, fit un signe de la main et dit:
	«Israélites, et vous aussi qui craignez Dieu, écoutez:
	Le Dieu de ce peuple, le Dieu d’Israël a choisi nos pères;
	il a fait grandir son peuple pendant le séjour en Égypte
	et il l’en a fait sortir à bras étendu.
Pendant une quarantaine d’années, il les a supportés au désert
	et, après avoir exterminé tour à tour sept nations au pays de Canaan,
	il a partagé pour eux ce pays en héritage.
Tout cela dura environ quatre cent cinquante ans.
	Ensuite, il leur a donné des juges, jusqu’au prophète Samuel.»
