% !TEX TS-program = lualatex
% !TEX encoding = UTF-8

\documentclass[twoside, french]{book}

%%%%%%%%%%%%%%% INDICES %%%%%%%%%%%%%%%

\usepackage{imakeidx}

\indexsetup{level=\section*,toclevel=section,noclearpage,othercode=\footnotesize\thispagestyle{empty}}
\makeindex[name=L,title=Index des livres bibliques, columns=2,columnseprule,options=-s custom]
\makeindex[name=F,title=Index des jours liturgiques, columns=2,columnseprule, options=-s custom]

\newcommand{\fullbook}{DEFAULTBOOK}
\newcommand{\bookorder}{ZZ}
\newcommand{\booklookup}[4]{%
  \IfSubStr{#2}{#1}{ \renewcommand{\fullbook}{#3}\renewcommand{\bookorder}{#4} }{}%
}%
\newcommand{\customreadingindex}[3]{%
	\renewcommand{\fullbook}{DEFAULTBOOK}%
	\renewcommand{\bookorder}{ZZ}
	\booklookup{#1}{Gn}{Livre de la Genèse}{AA}
	\booklookup{#1}{Ex}{Livre de l'Exode}{AB}
	\booklookup{#1}{Lv}{Livre du Lévitique}{AC}
	\booklookup{#1}{Nb}{Livre des Nombres}{AD}
	\booklookup{#1}{Dt}{Livre du Deutéronome}{AE}
	\booklookup{#1}{Jos}{Livre de Josué}{AF}
	\booklookup{#1}{Jg}{Livre des Juges}{AG}
	\booklookup{#1}{Rt}{Livre de Ruth}{AH}
	\booklookup{#1}{1 S}{Premier livre de Samuel}{AJ}
	\booklookup{#1}{2 S}{Deuxième livre de Samuel}{AK}
	\booklookup{#1}{1 R}{Premier livre des Rois}{AL}
	\booklookup{#1}{2 R}{Deuxième livre des Rois}{AM}
	\booklookup{#1}{1 Ch}{Premier livre des Chroniques}{AN}
	\booklookup{#1}{2 Ch}{Deuxième livre des Chroniques}{AP}
	\booklookup{#1}{Esd}{Livre d'Esdras}{AQ}
	\booklookup{#1}{Ne}{Livre de Néhémie}{AR}
	\booklookup{#1}{Tb}{Livre de Tobie}{AS}
	\booklookup{#1}{Jd}{Livre de Judith}{AT}
	\booklookup{#1}{Est}{Livre d'Esther}{AU}
	\booklookup{#1}{1 M}{Premier Livre des Martyrs d'Israël}{AV}
	\booklookup{#1}{2 M}{Deuxième Livre des Martyrs d'Israël}{AX}
	\booklookup{#1}{Jb}{Livre de Job}{AY}
	\booklookup{#1}{Pr}{Livre des Proverbes}{AZ}
	\booklookup{#1}{Qo}{L'ecclésiaste}{BA}
	\booklookup{#1}{Ct}{Cantique des cantiques}{BB}
	\booklookup{#1}{Sg}{Livre de la Sagesse}{BC}
	\booklookup{#1}{Si}{Livre de Ben Sira le Sage}{BD}
	\booklookup{#1}{Is}{Livre d'Isaïe}{BE}
	\booklookup{#1}{Jr}{Livre de Jérémie}{BF}
	\booklookup{#1}{Lam}{Livre des lamentations de Jérémie}{BG}
	\booklookup{#1}{Ba}{Livre de Baruch}{BH}
	\booklookup{#1}{Ep Jr}{Lettre de Jérémie}{BJ}
	\booklookup{#1}{Ez}{Livre d'Ezekiel}{BK}
	\booklookup{#1}{Dn}{Livre de Daniel}{BL}
	\booklookup{#1}{Os}{Livre d'Osée}{BM}
	\booklookup{#1}{Jl}{Livre de Joël}{BN}
	\booklookup{#1}{Am}{Livre d'Amos}{BP}
	\booklookup{#1}{Abd}{Livre d'Abdias}{BQ}
	\booklookup{#1}{Jon}{Livre de Jonas}{BR}
	\booklookup{#1}{Mi}{Livre de Michée}{BS}
	\booklookup{#1}{Na}{Livre de Nahum}{BT}
	\booklookup{#1}{Ha}{Livre d'Habaquc}{BU}
	\booklookup{#1}{So}{Livre de Sophonie}{BV}
	\booklookup{#1}{Ag}{Livre d'Aggée}{BX}
	\booklookup{#1}{Za}{Livre de Zacharie}{BY}
	\booklookup{#1}{Ml}{Livre de Malachie}{BZ}
	\booklookup{#1}{Mt}{Évangile de Jésus-Christ selon saint Matthieu}{CA}
	\booklookup{#1}{Mc}{Évangile de Jésus-Christ selon saint Marc}{CB}
	\booklookup{#1}{Lc}{Évangile de Jésus-Christ selon saint Luc}{CC}
	\booklookup{#1}{Jn}{Évangile de Jésus-Christ selon saint Jean}{CD}
	\booklookup{#1}{Ac}{Livre des Actes des Apôtres}{CE}
	\booklookup{#1}{Rm}{Lettre de saint Paul Apôtre aux Romains}{DA}
	\booklookup{#1}{1 Co}{Première lettre de saint Paul Apôtre aux Corinthiens}{DB}
	\booklookup{#1}{2 Co}{Deuxième lettre de saint Paul Apôtre aux Corinthiens}{DC}
	\booklookup{#1}{Ga}{Lettre de saint Paul Apôtre aux Galates}{DD}
	\booklookup{#1}{Ep}{Lettre de saint Paul Apôtre aux Ephésiens}{DE}
	\booklookup{#1}{Ph}{Lettre de saint Paul Apôtre aux Philippiens}{DF}
	\booklookup{#1}{Col}{Lettre de saint Paul Apôtre aux Colossiens}{DG}
	\booklookup{#1}{1 Th}{Première lettre de saint Paul Apôtre aux Thessaloniciens}{DH}
	\booklookup{#1}{2 Th}{Deuxième lettre de saint Paul Apôtre aux Thessaloniciens}{DJ}
	\booklookup{#1}{1 Tm}{Première lettre de saint Paul Apôtre à Timothée}{DK}
	\booklookup{#1}{2 Tm}{Deuxième lettre de saint Paul Apôtre à Timothée}{DL}
	\booklookup{#1}{Tt}{Lettre de saint Paul Apôtre à Tite}{DM}
	\booklookup{#1}{Phm}{Lettre de saint Paul Apôtre à Philémon}{DN}
	\booklookup{#1}{He}{Lettre aux Hébreux}{DP}
	\booklookup{#1}{Jc}{Lettre de saint Jacques Apôtre}{DQ}
	\booklookup{#1}{1 P}{Première lettre de saint Pierre Apôtre}{DR}
	\booklookup{#1}{2 P}{Deuxième lettre de saint Pierre Apôtre}{DS}
	\booklookup{#1}{1 Jn}{Première lettre de saint Jean}{DT}
	\booklookup{#1}{2 Jn}{Deuxième lettre de saint Jean}{DU}
	\booklookup{#1}{3 Jn}{Troisième lettre de saint Jean}{DV}
	\booklookup{#1}{Jd}{Lettre de saint Jude}{DX}
	\booklookup{#1}{Ap}{Livre de l'Apocalypse}{DZ}
	\index[L]{\bookorder@\fullbook!#2}
	\ifblank{#3}{}{\index[L]{\bookorder@\fullbook!#3}}
}%


%%%%%%%%%%%%%%% STANDARD PACKAGES %%%%%%%%%%%%%%%

%% Smaller format for the convenience of not redoing all the gregorio size tweaks, to be printed on A4 or larger.
\usepackage[paperwidth=160mm, paperheight=226.3mm]{geometry}

\usepackage{fontspec}
\usepackage[nolocalmarks]{polyglossia}
\usepackage[table]{xcolor}
\usepackage{fancyhdr}
\usepackage{titlesec}
\usepackage{setspace}
\usepackage{expl3}
\usepackage{hyperref}
\usepackage{refcount}
\usepackage{needspace}
\usepackage{etoolbox}
\usepackage{enumitem}

%%%%%%%%%%%%%%% HYPHENATION AND TYPOGRAPHICAL CONVENTIONS %%%%%%%%%%%%%%

\setdefaultlanguage[variant=french, frenchitemlabels=false]{french}

%%%%%%%%%%%%%%% GEOMETRY %%%%%%%%%%%%%%%

%% Keeping in mind those are for A5 paper, to scale
\geometry{
inner=15mm,
outer=10mm,
top=12mm,
bottom=10mm,
headsep=3mm,
}

%% General scale of all graphical elements.
%% Values different from 1 are largely untested.
%% Used in those commands (e.g. everything FontSpec) that use a scale parameter.
\newcommand{\customscale}{1}

%% Provide the command \fpevalc as a copy of the code-level \fp_eval:n.
%% \fpevalc allows to evaluate floating point calculation for scaled parameters, e.g. \setSomeStretchFactor{\fpevalc{0,6 * \customscale}}
\ExplSyntaxOn
\cs_new_eq:NN \fpevalc \fp_eval:n
\ExplSyntaxOff

%% No indentation of paragraphs. This is extremely important for both hymns, readings, gregorio scores, etc. to be displayed properly.
\setlength{\parindent}{0mm}

%% We want to allow large inter-words space 
%% to avoid overfull boxes in two-columns rubrics.
%\sloppy %% this is commented out for books that are mainly one-column to help detect badly formatted lines

%%%%%%%%%%%%%%% GREGORIO CONFIG %%%%%%%%%%%%%%%

\usepackage[autocompile]{gregoriotex}

%% text above lines shall be of color gregoriocolor
\grechangestyle{abovelinestext}{\color{gregoriocolor}\footnotesize}
%% fine-tuning of space beween the staff and the text above lines
\newcommand{\altraise}{-1.4mm} %% default is -0.1cm
\grechangedim{abovelinestextraise}{\altraise}{scalable}
\grechangedim{abovelinestextheight}{6mm}{scalable}

%% fine-tuning of space between the staff and the lyrics
\newcommand{\textraise}{2.8ex} %% default is 3.48471 ex
\grechangedim{spacelinestext}{\textraise}{scalable}

%% fine-tuning of space between the initial and the annotations
\newcommand{\annraise}{0mm} %% default is -0.2mm
\grechangedim{annotationraise}{\annraise}{scalable}

%% \officepartannotation converts a letter (IHARPT) into the office part to be printed as annotation,
%% storing the result into \result.

\newcommand{\result}{}
\newcommand{\lookup}[3]{%
  \IfSubStr{#2}{#1}{ \renewcommand{\result}{#3} }{}%
}%
\newcommand{\officepartannotation}[1]{%
  \renewcommand{\result}{#1}%
  \lookup{#1}{T}{}%
  \lookup{#1}{H}{Hymn.}%
  \lookup{#1}{A}{Ant.}%
  \lookup{#1}{P}{}%
  \lookup{#1}{R}{Resp.}%
  \lookup{#1}{I}{Invit.}%
  \result%
}%

%% header capture setup for the mode
\newcommand{\defaultannotationshift}{-2mm}
\newcommand{\modeannotation}[1]{\greannotation{\hspace{\defaultannotationshift}#1}}
\gresetheadercapture{mode}{modeannotation}{string}

%% outputs a score with no label, indexing, initials or annotations
%% for 
\newcommand{\unindexedscore}[1]{
  \gresetinitiallines{0}
  %% the use of a directory called "gabc" is linked
  %% to the management of gabc files by the website: do not change 
  %% without also changing the website static files structure
  \gregorioscore{\subfix{gabc/#1}}
  \gresetinitiallines{1}
}

%% outputs a score with label, indexing, and annotations. no initials if [n] is passed
\makeatletter
\newcommand{\gscore}[5][y]{
  %% #1 (passed as option) : y = initial, n = no initial
  %% #2 : name of the score file, should be a code, e.g. Q4F4A3 or 1225N1R1
  %% #3 : office-part among the values: T, H, A, P, R, I (toni communes, hy., ant., psalmus, resp., invit.)
  %% #4 : if applicable, a number between 1 and 9 (rank of the ant./resp.) - else: empty
  %% #5 : the indexed name of the piece
  
  %% this prevents page breaks between the phantom section and its label, and the actual score.
  \needspace{4\baselineskip} 
  \protected@edef\@currentlabelname{#5}
  \phantomsection
  \label{#2}
  %% we add the office part, and number of that ant. or resp. in the current office, if applicable
  %% todo : the negative hspace is here because somehow the initial and annotation (first line only) are misaligned by 1mm _with this initial font size_.
  %% this should probably be fixed in a more elegant way.
  \greannotation[c]{\hspace{-1.4mm}\hspace{\defaultannotationshift}\officepartannotation{#3}#4}
  %% if #5 (indexed name) is blank, nothing is indexed.
  %% this is for pieces that are repetitions of another piece (antiphons after psalms)
  \ifblank{#5}{}{\index[#3]{#5}}
  %% if optional arg #1 has been passed as 'n', set no initial
  \ifx n#1\gresetinitiallines{0}\fi
  %% the use of a directory called "gabc" is linked
  %% to the management of gabc files by the website: do not change 
  %% without also changing the website static files structure
  \gregorioscore{\subfix{gabc/#2}}
  %% if optional arg #1 has been passed as 'n', unset no initial
  \ifx n#1\gresetinitiallines{1}\fi
  \vspace{1mm}
}
\makeatother

%%%%%%%%%%%%%%% FONTS %%%%%%%%%%%%%%%

%%%%%%%%%%%%%%% Main font
\setmainfont[Ligatures=TeX, Scale=\customscale]{Charis SIL}
%\setmainfont[Ligatures=TeX, Scale=\customscale]{TeXGyreBonum-Regular}
\setstretch{\fpevalc{1.05 * \customscale}}

%%%%%%%%%%%%%%% Score initials
%% \initialsize resizes the initials, with one argument (size in points)
\newcommand{\initialsize}[1]{
    \grechangestyle{initial}{\fontspec{Zallman Caps}\fontsize{#1}{#1}\selectfont}
}
%% default initial size is 32 points
\newcommand{\defaultinitialsize}{28}
\initialsize{\defaultinitialsize}

%% spacing before and after initials to kern the Zallman Caps.
%% this should be changed if we move away from Zallman Caps.
\newcommand{\initialspace}[2]{
  \grechangedim{afterinitialshift}{#2}{scalable}
  \grechangedim{beforeinitialshift}{#1}{scalable}
}
%% default space before and after initials is 0mm to the left and 2mm to the right.
\newcommand{\defaultinitialspace}{\initialspace{0mm}{-\defaultannotationshift}}
\defaultinitialspace{}

%%%%%%%%%%%%%%% Score annotations
\grechangestyle{annotation}{\small}

%%%%%%%%%%%%%%% GRAPHICAL ELEMENTS %%%%%%%%%%%%%%%

%% V/, R/, A/ and + signs for in-line use (\vv \rr \aa \cc)
\newcommand{\specialcharhsep}{3mm} % space after invoking R/ or V/ or A/ outside rubrics
\newcommand{\vv}{\textcolor{gregoriocolor}{\fontspec[Scale=\customscale]{Charis SIL}℣.\hspace{\specialcharhsep}}}
\newcommand{\rr}{\textcolor{gregoriocolor}{\fontspec[Scale=\customscale]{Charis SIL}℟.\hspace{\specialcharhsep}}}
\renewcommand{\aa}{\textcolor{gregoriocolor}{\fontspec[Scale=\customscale]{Charis SIL}\Abar.\hspace{\specialcharhsep}}}
\newcommand{\cc}{\textcolor{gregoriocolor}{\fontspec[Scale=\customscale]{FreeSerif}\symbol{"2720}~}}
%% Same special characters, for in-score use (<sp>V/ R/ A/ +</sp>)
\gresetspecial{V/}{\textcolor{gregoriocolor}{\fontspec[Scale=\customscale]{Charis SIL}℣.~}}
\gresetspecial{R/}{\textcolor{gregoriocolor}{\fontspec[Scale=\customscale]{Charis SIL}℟.~}}
\gresetspecial{A/}{\textcolor{gregoriocolor}{\fontspec[Scale=\customscale]{Charis SIL}\Abar.~}}
\gresetspecial{+}{{\fontspec[Scale=\customscale]{FreeSerif}†~}}
\gresetspecial{*}{\gresixstar}
%% Same special characters, for use in rubrics (no space, and no red command since it will be reddified with the rest)
\newcommand{\vvrub}{{\fontspec[Scale=\customscale]{Charis SIL}℣.~}}
\newcommand{\rrrub}{{\fontspec[Scale=\customscale]{Charis SIL}℟.~}}
\newcommand{\aarub}{{\fontspec[Scale=\customscale]{Charis SIL}\Abar.~}}

%% the asterisk as found in the mediants of text-only psalms
\newcommand{\psstar}{\GreSpecial{*}}
\newcommand{\pscross}{\GreSpecial{+}}
%% also, most psalms do not call those but use † and * - todo

%% Roman Numerals
\usepackage{modroman}
\newcommand{\Rnum}[1]{\nbRoman{#1}}
\newcommand{\rnum}[1]{\nbshortroman{#1}}

%% Macro to print versicles
\newcommand{\versiculus}[2]{\par\vv #1 \\ \rr #2\par}

\newcommand{\versiculustpall}[2]
	{\versiculus{#1 \rubric{(T.P.} Allelúja. \rubric{)}}{#2 \rubric{(T.P.} Allelúja. \rubric{)}}}

%% Macro to print rubrics
\newcommand{\rubric}[1]{\textcolor{gregoriocolor}{\emph{#1}}}

%% Macro to print the name of a score in normal characters inside a \rubric
\newcommand{\normaltext}[1]{{\normalfont\normalcolor #1}}
\newcommand{\scorename}[1]{\normaltext{\nameref{M-#1}}}

%% Macro to print a full reference to a responsory
%% #1 is the R/ number in the feast, #2 is the R/ code, #3 is an optional additional text, like "sine Gloria Patri".
\newcommand{\respref}[3]{\rubric{%
\rrrub #1 \scorename{#2}, pag.\ \pageref{M-#2}%
\if\relax\detokenize{#3}\relax%
.%
\else%
, #3.%
\fi%
}}

\newcommand{\resprefcumgp}[2]
	{\respref{#1}{#2}{sed cum \normaltext{Glória Patri} in fine}}
	
\newcommand{\resprefsinegp}[2]
	{\respref{#1}{#2}{sine \normaltext{Glória Patri}}}

%% Macro to print the common rubric that signals the Te Deum
\newcommand{\tedeumrubric}{\rubric{Lectione ultima peracta Hymnus \normaltext{Te Deum} cantatur.}}

%%% Macro to print a reading
\newcommand{\reading}[7]{
	%% #1 : reading code (unique), normally unused except for consistency checks
	%% #2 : source file name (no extension), which should match the reading code except when the reading is a duplicate
	%% #3 : the reading incipit
	%% #4 : the rubric (normally indicating the book from which the reading is)
	%% #5 : the index book, if an index entry must be created
	%% #6 : the index chapter, if applicable
	%% #7 : a second index chapter, if applicable (no reading can be from two different books: for gospel incipits, they are separate)
\needspace{8\baselineskip} % break page if less than 8 lines are available
\ifblank{#5}{}{
	\customreadingindex{#5}{#6}{#7}
	%\index[L]{#5!#6}
}
\begin{center}
#3\par
\end{center}
\rubric{#4}\par
% this next part makes the tab character active, I think by making it lowercase ~ and defining it as such (because \def<tab>{something} does not work)
\begingroup
  \begingroup
    \lccode`\~=9\relax
  \lowercase{%
  \endgroup
    \def~%
  }{\hspace{1cm}}%
  \catcode9=\active
  \obeylines\input{scriptura/#2.tex}%
\endgroup
}

%%%%%%%%%%%%%%% COLUMN MANAGEMENT %%%%%%%%%%%%%%%

\usepackage{parcolumns}
\setlength\columnseprule{0.4pt}

\newcommand{\twocolrubric}[2]{
	\begin{parcolumns}[rulebetween]{2}
	\colchunk{%
      \rubric{#1}
	}
	\colchunk{%
	  \rubric{#2} 
    }
	\end{parcolumns}
}

%%%%%%%%%%%%%%% HEADER STYLES %%%%%%%%%%%%%%%

\pagestyle{fancy}
\fancyhead{}
\fancyfoot{}
\renewcommand{\headrulewidth}{0pt}
\setlength{\headheight}{20pt}
\fancyhead[RO]{\small\rightmark\hspace{1cm}\thepage}
\fancyhead[LE]{\small\thepage\hspace{1cm}\leftmark}

% this command is called every time the left and right header texts are set (e.g. by calling \feast)
% the hyphenpenalty override is needed only on older versions of gregorio which do not reset it correctly after typesetting the score.
% see https://tex.stackexchange.com/questions/581013/lualatex-hyphenation-issue-in-fancyhdr-with-gregoriotex-and-multicols-latin-te
\newcommand{\setheaders}[2]{
	\renewcommand{\rightmark}{\hyphenpenalty=50{\sc#2}}
	\renewcommand{\leftmark}{\hyphenpenalty=50{\sc#1}}
}
\setheaders{}{}

%%%%%%%%%%%%%%% TITLE STYLES %%%%%%%%%%%%%%%

%% Titles are centered and small-caps
\titleformat{\chapter}[block]{\Large\filcenter\sc}{}{}{}
\titleformat{\section}[block]{\large\filcenter\sc}{}{}{}
\titleformat{\subsection}[block]{\filcenter\sc}{}{}{}
\setcounter{secnumdepth}{0}
%% Fine-tuning of space around titles
\titlespacing*{\paragraph}{0pt}{1ex}{.6ex}

%% Typesets all titles throughout the NR except Nocturn titles and a few special titles.
%% Using a continuation is necessary because there are 11 arguments.
%% Only \feast, \nocturn, \intermediatetitle and \smalltitle should ever be used.
\newcommand{\feast}[6]{
  %% #1: feast code, e.g. 1225 or A1F1
  %% #2: feast title
  %% #3: left header title
  %% #4: right header title
  %% #5: title level
    %% title level 1 : full page width, a few major feasts + titles of temporale, sanctorale, etc.
	%% title level 2 : all feasts, sundays and major ferias
	%% title level 3 : ferias
  %% #6: incipit date (goes above feast title)
  %% cont'd #1: 1954 rank
  %% cont'd #2: 1960 rank
  %% cont'd #3: name of the feast as it shows up in the index
  %% cont'd #4: 1945 feast-wide rubrics
  %% cont'd #5: 1960 feast-wide rubrics

  %% needspace: should be barely more than the vertical space for the titles, rubrics excluded.
  %% this is to ensure that the page does not get cut after the title or the phantomsection.
  \needspace{8\baselineskip}
  %% phantomsection is to allow the label to attach to the title and not the previous counter object.
  \phantomsection
  \label{#1}
  \begin{center}
  %% we typeset a line for the date if the date is not blank
  \ifblank{#6}{}{
    {\large #6}\\%
  }%
  %% the actual title
  \ifx 1#5{\setstretch{1.2}\sc\huge #2\par}\fi
  \ifx 2#5{\Large #2\par}\fi
  \ifx 3#5{\large #2\par}\fi
  \end{center}
  %% If this is a level 1 title, empty pagestyle
  \ifx 1#5\thispagestyle{empty}\fi
  
  %% we define the header titles manually
  \setheaders{#3}{#4}
  
  %% moving on to a continuation macro to unpack the last 5 arguments
  \feastcontinued
}
\newcommand{\feastcontinued}[5]
{
  %% we name the last 5 arguments
  \def\oldrank{#1}%
  \def\newrank{#2}%
  \def\indexfeastname{#3}%
  \def\oldrubric{#4}%
  \def\newrubric{#5}%
  %% we index the feast if the indexing name is given
  \ifblank{#3}{}{
	\index[F]{\indexfeastname}
  }
  %% we typeset a two-column rank & rubrics block if one rank is filled in
  \ifblank{#1}{}{%
    \begin{parcolumns}[rulebetween]{2}
	\colchunk{%
      {\centering\oldrank\par}%
	  \ifblank{#4}{}{\rubric{\oldrubric}}
	}
	\colchunk{%
	  {\centering\newrank\par}%
	  \ifblank{#5}{}{\rubric{\newrubric}}
    }
	\end{parcolumns}
	\vspace{2mm}
  }
}

\newcommand{\smalltitle}[1]{
  \par{\centering\textbf{#1}\par}
}

\newcommand{\intermediatetitle}[1]{
  \begin{center}
  {\large #1}
  \end{center}
 }

\newcommand{\nocturn}[1]{
  \needspace{8\baselineskip}
  \intermediatetitle{In \Rnum{#1} Nocturno}
}

%% command to wrap printindex and set the headers for indices
\newcommand{\cprintindex}[2]{
	\setheaders{Index}{#2}
	\pagestyle{fancy}
	\thispagestyle{empty}
	\printindex[#1]
}



\begin{document}
\pagenumbering{gobble}

\null \newpage \null \newpage \null \newpage \null \newpage

\begin{titlepage}
\begin{center}
\null\vspace{2cm}
{\Large Nocturnale Romanum}

\vspace{5mm}

{\large Tome III}

\vspace{5cm}

{\Huge Lectionnaire nocturne}

\vspace{1cm}

{\Large pour l'Office romain}

\vfill

Traduction officielle liturgique

\vspace{5mm}

MMXXIII

\end{center}
\end{titlepage}

\setlength{\parindent}{6mm}

\intermediatetitle{Présentation générale}
\vspace{2cm}

Ce livre contient les lectures scripturaires chantées à l'heure de Matines selon le rite romain,
au premier nocturne des fêtes, à l'unique nocturne des féries et des fêtes mineures,
aux premier et troisième nocturnes des Offices des Ténèbres, et aux trois nocturnes de l'Office des défunts.

Ces lectures sont données en français, selon la traduction AELF, la seule traduction de l'intégralité de la Bible approuvée pour l'usage liturgique.
En effet, ce livre a été spécialement concu pour la célébration chorale de l'heure de Matines en présence de laïcs,
qui ne disposent pas d'un bréviaire bilingue, et pour le bénéfice desquels le chant des lectures en français est préférable.

Ce livre ne contient que les rubriques les plus essentielles à son emploi,
et suppose que le sacristain ou le cérémoniaire chargé de la mise en place des livres de chœur dispose d'un \emph{ordo} qui lui indique quelles lectures employer.

Le texte latin est donné en plus du français, par exception, pour les Matines de Noël, les Ténèbres, et l'Office des défunts, à cause des spécificités de ces occasions.

Pour le bénéfice des communautés, qui, par indult, emploient des rubriques antérieures à 1960,
on trouvera à leur emplacement certaines fêtes, vigiles et octaves, supprimées, déplacées ou modifiées entre 1950 et 1960.
Pour la même raison, il n'est pas fait mention, le dimanche, du regroupement de la troisième lecture avec la deuxième, pour faire place à l'homélie,
selon la réduction des Matines du dimanche opérée en 1960:
chaque communauté pourra effectuer ou non ce regroupement, selon les rubriques qu'elle emploie.

Les lectures sont présentées découpées en phrases et en fragments de phrase, un par ligne;
en effet, les spécificités de la langue française empêchent d'insérer la formule cadentielle convenable absolument à chaque signe de ponctuation;
à l'inverse, certaines phrases très longues sans ponctuation interne ont été découpées, suggérant d'insérer à tel ou tel emplacement une formule cadentielle
même en l'absence de ponctuation.

Les pages qui suivent donnent des indications de base sur le chant des lectures en français.

En l'absence d'autres règles que celles de l'intelligibilité et du respect du texte, il va de soi que les tons proposés ici, et le découpage des phrases, sont purement indicatifs.

\vfill

\begin{center}{\sc Benedicat Dominus eos, qui hæc legunt:\\
Pro eis, qui hæc fecerunt, orate ad Dominum}
\end{center}

\newpage

\intermediatetitle{Chant des lectures en français}
\vspace{2cm}

Les tons des lectures comprennent quatre formules cadentielles:
\begin{itemize}
\item une pour la virgule, qui peut être aussi employée au milieu d'une phrase, là où une virgule pourrait se trouver, même si elle est absente;
\item une pour le point-virgule, les deux-points, les points de suspension, le point d'exclamation, même s'il peut être opportun de l'omettre si ces signes sont trop rapprochés les uns des autres;
\item une pour le point d'interrogation, qui, contrairement aux autres, modifie la note sur laquelle se fait la récitation pendant tout le membre de phrase qui précède l'interrogation;
\item une pour le point final de la phrase.
\end{itemize}

Sauf pour les tons ornés particuliers donnés en leur lieu, il n'y a pas de formule d'attaque, le chant commence sur la note de récitation.

Chaque lecture peut comporter jusqu'à deux niveaux de retrait.
Si une ligne est alignée à gauche, il faut terminer la précédente par la cadence de point final.
Si une ligne est au premier niveau de retrait, il faut terminer la précédente par une cadence de virgule ou de point-virgule, selon la ponctuation.
Si une ligne est au deuxième niveau de retrait, il faut maintenir le \emph{recto tono} à la fin de la précédente, en marquant une très légère pause;
éventuellement, le lecteur pourra choisir d'y employer la cadence de virgule.
Dans tous les cas, si une ligne se termine par un point d'interrogation, il faut modifier sur toute la ligne
(ou, en cas d'erreur, dès qu'on s'en rend compte) la note de récitation, conformément à la cadence appropriée.

Il arrive qu'une ligne se termine par un point, mais que la suivante commence en retrait:
c'est que les deux phrases sont courtes et liées entre elles, de telle sorte que, même si le texte officiel écrit un point, il aurait pu utiliser le point-virgule;
on a ainsi disposé le texte pour éviter de faire revenir trop souvent la cadence de point final, plus rare dans la Vulgate que dans la traduction officielle.
Il faut alors employer entre ces deux phrases la cadence de point-virgule.

Il arrive qu'une ligne se terminant par un point d'interrogation soit au deuxième niveau de retrait:
il faut alors finir la précédente \emph{recto tono}, et baisser d'un demi-ton la note de récitation sur cette ligne, et finir par la cadence d'interrogation, en combinant les règles précédentes.

Il arrive très souvent qu'une ponctuation survienne au sein d'une ligne, par exemple lors d'une énumération:
il faut alors maintenir le \emph{recto tono} en marquant une très légère pause.
Ceci est vrai également quand plusieurs questions courtes se suivent
--- cf. Isaïe 6: 8, vendredi de la première semaine de l'Avent: \emph{«Qui enverrai-je? qui sera notre messager?»} ---
dans ce cas, il faut ignorer la présence du premier point d'interrogation, mais, conformément à ce qui a été dit ci-dessus, baisser la note de récitation dès le début de la ligne.

La dernière note de la cadence tombe toujours sur la dernière syllabe forte du dernier mot de la ligne, qui est le plus souvent la dernière syllabe,
sauf si le mot se termine par un E muet, qu'on pourra utilement vocaliser légèrement si cela permet d'améliorer l'intelligibilité du texte, selon l'acoustique du lieu;
dans ce cas, la dernière syllabe forte est celle qui précède le E muet. Si le dernier mot de la ligne se termine par un groupe vocalique qui peut être prononcé
en diérèse ou en synérèse, on privilégiera la synérèse: ainsi le mot \emph{interrogation} comme dans les exemples ci-dessous.

Les autres notes d'une cadence s'organisent toujours par rapport à la finale, sans répétitions. Si un E muet au sein d'une cadence est suivi d'une consonne,
il compte comme une syllabe dans la cadence, et est vocalisé, un peu plus brièvement qu'une syllabe ordinaire: ainsi \emph{je ne vous félicite pas}, dans un exemple ci-dessous.
Lors du chant en français, il importe de ne pas allonger les notes de la cadence autres que la finale, comme on le ferait en latin,
mais de rester aussi près que possible du rythme de la lecture à voix haute, tout en tenant compte de l'acoustique qui peut forcer à ralentir l'ensemble.

On donne ci-dessous trois tons possibles, en rappelant les tons du \emph{Jube domne}, de la bénédiction, et du \emph{Tu autem},
qui sont donnés aussi en leur lieu au \emph{Nocturnale Romanum}. Il est aussi toujours possible de chanter \emph{recto tono}.


\vspace{1cm}
\smalltitle{Ton simple}
\vspace{4mm}
Quand on emploie ce ton, on peut employer la cadence de point-virgule pour certaines virgules, afin de donner de la variété aux longs passages \emph{recto tono}.

\unindexedscore{ORLectA}

\pagebreak
\smalltitle{Ton solennel}
\unindexedscore{ORLectB}

\vspace{1cm}
\smalltitle{Autre ton ad libitum, plus ancien}
\unindexedscore{ORLectC}

\vspace{1cm}
\smalltitle{Office des défunts et Office des Ténèbres}
\vspace{4mm}
À l'Office des défunts, et aux deuxième et troisième nocturnes des Offices des Ténèbres, on emploie le ton simple,
mais on termine chaque lecture avec la cadence de conclusion ci-dessous, même si elle se termine par un point d'interrogation.

\unindexedscore{ORLectD}

\setlength{\parindent}{0mm}


\cleardoublepage
\pagenumbering{arabic}
\begin{titlepage}
\begin{center}
\null

\vspace{8cm}

{\Huge Lectionnaire nocturne}

\vspace{1cm}

{\Large pour l'Office romain}

\vfill

\end{center}
\end{titlepage}

\feast{PT}
        {Propre du Temps}{Propre du Temps}{Propre du Temps}{1}{}
        {}{}{}{}{}
\addcontentsline{toc}{chapter}{Propre du Temps}
\feast{A1F1}
        {Premier dimanche de l’Avent}{Propre du Temps}{Première semaine de l’Avent}{2}{}
        {}{}{Avent!A@Première semaine}{}{}
\reading{A1F1N1L1}{A1F1N1L1}{Incipit liber Isaíæ Prophétæ}{Is. 1: 1-3}{Is}{1}{}
\reading{A1F1N1L2}{A1F1N1L2}{}{Is. 1: 4-6}{}{}{}
\reading{A1F1N1L3}{A1F1N1L3}{}{Is. 1: 7-9}{}{}{}
\feast{A1F2}
        {Lundi}{Propre du Temps}{Première semaine de l’Avent}{3}{}
        {}{}{}{}{}
\reading{A1F2L1}{A1F2L1}{De Isaía Prophéta}{Is. 1: 16-18}{Is}{1}{}
\reading{A1F2L2}{A1F2L2}{}{Is. 1: 19-23}{}{}{}
\reading{A1F2L3}{A1F2L3}{}{Is. 1: 24-28}{}{}{}
\feast{A1F3}
        {Mardi}{Propre du Temps}{Première semaine de l’Avent}{3}{}
        {}{}{}{}{}
\reading{A1F3L1}{A1F3L1}{De Isaía Prophéta}{Is. 2: 1-3}{Is}{2}{}
\reading{A1F3L2}{A1F3L2}{}{Is. 2: 4-6}{}{}{}
\reading{A1F3L3}{A1F3L3}{}{Is. 2: 7-9}{}{}{}
\feast{A1F4}
        {Mercredi}{Propre du Temps}{Première semaine de l’Avent}{3}{}
        {}{}{}{}{}
\reading{A1F4L1}{A1F4L1}{De Isaía Prophéta}{Is. 3: 1-4}{Is}{3}{}
\reading{A1F4L2}{A1F4L2}{}{Is. 3: 5-7}{}{}{}
\reading{A1F4L3}{A1F4L3}{}{Is. 3: 8-11}{}{}{}
\feast{A1F5}
        {Jeudi}{Propre du Temps}{Première semaine de l’Avent}{3}{}
        {}{}{}{}{}
\reading{A1F5L1}{A1F5L1}{De Isaía Prophéta}{Is. 4: 1-3}{Is}{4}{}
\reading{A1F5L2}{A1F5L2}{}{Is. 5: 1-4}{Is}{5}{}
\reading{A1F5L3}{A1F5L3}{}{Is. 5: 5-7}{}{}{}
\feast{A1F6}
        {Vendredi}{Propre du Temps}{Première semaine de l’Avent}{3}{}
        {}{}{}{}{}
\reading{A1F6L1}{A1F6L1}{De Isaía Prophéta}{Is. 6: 1-3}{Is}{6}{}
\reading{A1F6L2}{A1F6L2}{}{Is. 6: 4-7}{}{}{}
\reading{A1F6L3}{A1F6L3}{}{Is. 6: 8-10}{}{}{}
\feast{A1F7}
        {Samedi}{Propre du Temps}{Première semaine de l’Avent}{3}{}
        {}{}{}{}{}
\reading{A1F7L1}{A1F7L1}{De Isaía Prophéta}{Is. 7: 1-3}{Is}{7}{}
\reading{A1F7L2}{A1F7L2}{}{Is. 7: 4-6}{}{}{}
\reading{A1F7L3}{A1F7L3}{}{Is. 7: 10-15}{}{}{}
\feast{A2F1}
        {Deuxième dimanche de l’Avent}{Propre du Temps}{Deuxième semaine de l’Avent}{2}{}
        {}{}{Avent!B@Deuxième semaine}{}{}
\reading{A2F1N1L1}{A2F1N1L1}{De Isaía Prophéta}{Is. 11: 1-4}{Is}{11}{}
\reading{A2F1N1L2}{A2F1N1L2}{}{Is. 11: 4-7}{}{}{}
\reading{A2F1N1L3}{A2F1N1L3}{}{Is. 11: 8-10}{}{}{}
\feast{A2F2}
        {Lundi}{Propre du Temps}{Deuxième semaine de l’Avent}{3}{}
        {}{}{}{}{}
\reading{A2F2L1}{A2F2L1}{De Isaía Prophéta}{Is. 13: 1-4}{Is}{13}{}
\reading{A2F2L2}{A2F2L2}{}{Is. 13: 4-8}{}{}{}
\reading{A2F2L3}{A2F2L3}{}{Is. 13: 9-11}{}{}{}
\feast{A2F3}
        {Mardi}{Propre du Temps}{Deuxième semaine de l’Avent}{3}{}
        {}{}{}{}{}
\reading{A2F3L1}{A2F3L1}{De Isaía Prophéta}{Is. 14: 1-2}{Is}{14}{}
\reading{A2F3L2}{A2F3L2}{}{Is. 14: 3-6}{}{}{}
\reading{A2F3L3}{A2F3L3}{}{Is. 14: 12-15}{}{}{}
\feast{A2F4}
        {Mercredi}{Propre du Temps}{Deuxième semaine de l’Avent}{3}{}
        {}{}{}{}{}
\reading{A2F4L1}{A2F4L1}{De Isaía Prophéta}{Is. 16: 1-4}{Is}{16}{}
\reading{A2F4L2}{A2F4L2}{}{Is. 16: 4-6}{}{}{}
\reading{A2F4L3}{A2F4L3}{}{Is. 16: 7-8}{}{}{}
\feast{A2F5}
        {Jeudi}{Propre du Temps}{Deuxième semaine de l’Avent}{3}{}
        {}{}{}{}{}
\reading{A2F5L1}{A2F5L1}{De Isaía Prophéta}{Is. 19: 1-2}{Is}{19}{}
\reading{A2F5L2}{A2F5L2}{}{Is. 19: 3-6}{}{}{}
\reading{A2F5L3}{A2F5L3}{}{Is. 19: 11-13}{}{}{}
\feast{A2F6}
        {Vendredi}{Propre du Temps}{Deuxième semaine de l’Avent}{3}{}
        {}{}{}{}{}
\reading{A2F6L1}{A2F6L1}{De Isaía Prophéta}{Is. 24: 1-3}{Is}{24}{}
\reading{A2F6L2}{A2F6L2}{}{Is. 24: 4-6}{}{}{}
\reading{A2F6L3}{A2F6L3}{}{Is. 24: 7-16}{}{}{}
\feast{A2F7}
        {Samedi}{Propre du Temps}{Deuxième semaine de l’Avent}{3}{}
        {}{}{}{}{}
\reading{A2F7L1}{A2F7L1}{De Isaía Prophéta}{Is. 25: 1-4}{Is}{25}{}
\reading{A2F7L2}{A2F7L2}{}{Is. 25: 4-7}{}{}{}
\reading{A2F7L3}{A2F7L3}{}{Is. 25: 8-12}{}{}{}
\feast{A3F1}
        {Troisième dimanche de l’Avent}{Propre du Temps}{Troisième semaine de l’Avent}{2}{}
        {}{}{Avent!C@Troisième semaine}{}{}
\rubric{Les lectures scripturaires de ce dimanche, du lundi et du mardi qui suivent, si elles sont empêchées, sont transférées aux jours des Quatre-Temps où l'office serait à trois nocturnes sans lectures scripturaires propres.}
\reading{A3F1N1L1}{A3F1N1L1}{De Isaía Prophéta}{Is. 26: 1-6}{Is}{26}{}
\reading{A3F1N1L2}{A3F1N1L2}{}{Is. 26: 7-10}{}{}{}
\reading{A3F1N1L3}{A3F1N1L3}{}{Is. 26: 11-14}{}{}{}
\feast{A3F2}
        {Lundi}{Propre du Temps}{Troisième semaine de l’Avent}{3}{}
        {}{}{}{}{}
\reading{A3F2L1}{A3F2L1}{De Isaía Prophéta}{Is. 28: 1-3}{Is}{28}{}
\reading{A3F2L2}{A3F2L2}{}{Is. 28: 4-7}{}{}{}
\reading{A3F2L3}{A3F2L3}{}{Is. 28: 16-18}{}{}{}
\feast{A3F3}
        {Mardi}{Propre du Temps}{Troisième semaine de l’Avent}{3}{}
        {}{}{}{}{}
\reading{A3F3L1}{A3F3L1}{De Isaía Prophéta}{Is. 30: 18-20}{Is}{30}{}
\reading{A3F3L2}{A3F3L2}{}{Is. 30: 22-25}{}{}{}
\reading{A3F3L3}{A3F3L3}{}{Is. 30: 26-28}{}{}{}
\feast{A3F4}
        {Mercredi, vendredi et samedi des Quatre-Temps de l’Avent}{Propre du Temps}{Troisième semaine de l’Avent}{3}{}
        {}{}{Quatre-Temps!A@de l’Avent}{}{}
\rubric{Lectures à l'Homéliaire, sauf si l'office est à trois nocturnes; alors, en l'absence de lectures scripturaires propres, on emploie les éventuelles lectures empêchées des jours précédents ou des jours suivants, tout en en conservant l'ordre.}
\feast{A3F5}
        {Jeudi}{Propre du Temps}{Troisième semaine de l’Avent}{3}{}
        {}{}{}{}{}
\reading{A3F5L1}{A3F5L1}{De Isaía Prophéta}{Is. 33: 1-2}{Is}{33}{}
\reading{A3F5L2}{A3F5L2}{}{Is. 33: 3-6}{}{}{}
\reading{A3F5L3}{A3F5L3}{}{Is. 33: 14-17}{}{}{}
\feast{A4F1}
        {Quatrième dimanche de l’Avent}{Propre du Temps}{Quatrième semaine de l’Avent}{2}{}
        {}{}{Avent!D@Quatrième semaine}{}{}
\reading{A4F1N1L1}{A4F1N1L1}{De Isaía Prophéta}{Is. 35: 1-7}{Is}{35}{}
\reading{A4F1N1L2}{A4F1N1L2}{}{Is. 35: 7-10}{}{}{}
\reading{A4F1N1L3}{A4F1N1L3}{}{Is. 41: 1-4}{Is}{41}{}
\feast{A4F2}
        {Lundi}{Propre du Temps}{Quatrième semaine de l’Avent}{3}{}
        {}{}{}{}{}
\reading{A4F2L1}{A4F2L1}{De Isaía Prophéta}{Is. 41: 8-10}{Is}{41}{}
\reading{A4F2L2}{A4F2L2}{}{Is. 41: 11-13}{}{}{}
\reading{A4F2L3}{A4F2L3}{}{Is. 41: 14-16}{}{}{}
\feast{A4F3}
        {Mardi}{Propre du Temps}{Quatrième semaine de l’Avent}{3}{}
        {}{}{}{}{}
\reading{A4F3L1}{A4F3L1}{De Isaía Prophéta}{Is. 42: 1-4}{Is}{42}{}
\reading{A4F3L2}{A4F3L2}{}{Is. 42: 5-7}{}{}{}
\reading{A4F3L3}{A4F3L3}{}{Is. 42: 10-13}{}{}{}
\feast{A4F4}
        {Mercredi}{Propre du Temps}{Quatrième semaine de l’Avent}{3}{}
        {}{}{}{}{}
\reading{A4F4L1}{A4F4L1}{De Isaía Prophéta}{Is. 51: 1-3}{Is}{51}{}
\reading{A4F4L2}{A4F4L2}{}{Is. 51: 4-6}{}{}{}
\reading{A4F4L3}{A4F4L3}{}{Is. 51: 7-8}{}{}{}
\feast{A4F5}
        {Jeudi}{Propre du Temps}{Quatrième semaine de l’Avent}{3}{}
        {}{}{}{}{}
\reading{A4F5L1}{A4F5L1}{De Isaía Prophéta}{Is. 64: 1-4}{Is}{64}{}
\reading{A4F5L2}{A4F5L2}{}{Is. 64: 5-7}{}{}{}
\reading{A4F5L3}{A4F5L3}{}{Is. 64: 8-11}{}{}{}
\feast{A4F6}
        {Vendredi}{Propre du Temps}{Quatrième semaine de l’Avent}{3}{}
        {}{}{}{}{}
\reading{A4F6L1}{A4F6L1}{De Isaía Prophéta}{Is. 66: 5-8}{Is}{66}{}
\reading{A4F6L2}{A4F6L2}{}{Is. 66: 9-12}{}{}{}
\reading{A4F6L3}{A4F6L3}{}{Is. 66: 13-16}{}{}{}
\feast{1224}
        {Vigile de la Nativité du Seigneur}{Propre du Temps}{Vigile de la Nativité du Seigneur}{3}{24 décembre}
        {}{}{Notre-Seigneur Jésus-Christ!Nativité!Vigile}{}{}
\rubric{Lectures à l'Homéliaire.}
\cleartoleftpage{}
\feast{1225}
        {Nativité du Seigneur}{Propre du Temps}{Nativité du Seigneur}{1}{25 décembre}
        {}{}{Notre-Seigneur Jésus-Christ!Nativité}{}{}
\rubric{Les trois lectures qui suivent se chantent avec leurs bénédictions comme indiqué, mais sans titre.}
\unindexedscore{1225N1L1lata}
\chantedscripturereading{1225N1L1lat}
\vfill
\reading{1225N1L1}{1225N1L1}{}{Is. 8: 23; 9: 1-5}{Is}{8}{9}
\vfill
\unindexedscore{1225N1Lb}
\cleartoleftpage{}
\unindexedscore{1225N1L2lata}
\chantedscripturereading{1225N1L2lat}
\vfill
\reading{1225N1L2}{1225N1L2}{}{Is. 40: 1-8}{Is}{40}{}
\vfill
\unindexedscore{1225N1Lb}
\cleartoleftpage{}
\unindexedscore{1225N1L3lata}
\chantedscripturereading{1225N1L3lat}
\vfill
\reading{1225N1L3}{1225N1L3}{}{Is. 52: 1-6}{Is}{52}{}
\vfill
\unindexedscore{1225N1Lb}
\pagebreak
\feast{1226}
        {Saint Étienne, protomartyr}{Propre du Temps}{Saint Étienne, protomartyr}{2}{26 décembre}
        {}{}{Etienne@Étienne, protomartyr}{}{}
\reading{1226N1L1}{1226N1L1}{De Actibus Apostolórum}{Ac. 6: 1-6}{Ac}{6}{}
\reading{1226N1L2}{1226N1L2}{}{Ac. 6: 7-10; 7: 54}{Ac}{7}{}
\reading{1226N1L3}{1226N1L3}{}{Ac. 7: 55-60}{}{}{}
\feast{1227}
        {Saint Jean, apôtre et évangéliste}{Propre du Temps}{Saint Jean, apôtre et évangéliste}{2}{27 décembre}
        {}{}{Jean, apôtre et évangéliste}{}{}
\reading{1227N1L1}{P6F1N1L1}{Incipit Epístola prima beáti Joánnis Apóstoli}{1 Jn. 1: 1-5}{1 Jn}{1}{}
\reading{1227N1L2}{P6F1N1L2}{}{1 Jn. 1: 6-10}{}{}{}
\reading{1227N1L3}{P6F1N1L3}{}{1 Jn. 2: 1-6}{1 Jn}{2}{}
\feast{1228}
        {Saints Innocents}{Propre du Temps}{Les Saints Innocents}{2}{28 décembre}
        {}{}{Innocents}{}{}
\reading{1228N1L1}{1228N1L1}{De Jeremía Prophéta}{Jr 31: 15-17}{Jr}{31}{}
\reading{1228N1L2}{1228N1L2}{}{Jr 31: 18-20}{}{}{}
\reading{1228N1L3}{1228N1L3}{}{Jr 31: 21-23}{}{}{}
\feast{N1F1}
        {Dimanche dans l’Octave de la Nativité}{Propre du Temps}{Dimanche dans l’Octave de la Nativité}{2}{}
        {}{}{Notre-Seigneur Jésus-Christ!Nativité!Dimanche dans l’Octave}{}{}
\rubric{Lectures scripturaires du jour calendaire, ci-après.}
\feast{1229}
        {Saint Thomas, évêque et martyr}{Propre du Temps}{Saint Thomas, évêque et martyr}{2}{29 décembre}
        {}{}{Thomas, évêque et martyr}{}{}
\rubric{Si le commencement de l'épître aux Romains est empêché, il est transféré au premier jour où on lit l'Écriture courante.}
\reading{1229N1L1}{1229N1L1}{Incipit Epístola beáti Pauli Apóstoli ad Romános}{Rm. 1: 1-7}{Rm}{1}{}
\reading{1229N1L2}{1229N1L2}{}{Rm. 1: 8-12}{}{}{}
\reading{1229N1L3}{1229N1L3}{}{Rm. 1: 13-19}{}{}{}
\feast{1230}
        {Sixième jour dans l’Octave de la Nativité}{Propre du Temps}{Pendant l’Octave de la Nativité}{3}{30 décembre}
        {}{}{}{}{}
\reading{1230N1L1}{1230N1L1}{De Epístola ad Romános}{Rm. 2: 1-4}{Rm}{2}{}
\reading{1230N1L2}{1230N1L2}{}{Rm. 2: 5-8}{}{}{}
\reading{1230N1L3}{1230N1L3}{}{Rm. 2: 9-13}{}{}{}
\feast{1231}
        {Saint Sylvestre, pape et confesseur}{Propre du Temps}{Saint Sylvestre, pape et confesseur}{2}{31 décembre}
        {}{}{Sylvestre pape}{}{}
\reading{1231N1L1}{1231N1L1}{De Epístola ad Romános}{Rm. 3: 19-22}{Rm}{3}{}
\reading{1231N1L2}{1231N1L2}{}{Rm. 3: 23-26}{}{}{}
\reading{1231N1L3}{1231N1L3}{}{Rm. 3: 27-31}{}{}{}
\feast{0101}
        {Circoncision du Seigneur et Octave de la Nativité}{Propre du Temps}{Circoncision du Seigneur}{2}{1\ier janvier}
        {}{}{Notre-Seigneur Jésus-Christ!Circoncision}{}{}
\reading{0101N1L1}{0101N1L1}{De Epístola ad Romános}{Rm. 4: 1-8}{Rm}{4}{}
\reading{0101N1L2}{0101N1L2}{}{Rm. 4: 9-12}{}{}{}
\reading{0101N1L3}{0101N1L3}{}{Rm. 4: 13-17}{}{}{}
\feast{N2F1}
        {Fête du Saint Nom de Jésus}{Propre du Temps}{Saint Nom de Jésus}{2}{Dimanche entre la Circoncision et l’Épiphanie}
        {}{}{Notre-Seigneur Jésus-Christ!Saint Nom}{}{}
\reading{N2F1N1L1}{N2F1N1L1}{De Actibus Apostolórum}{Ac. 3: 1-8}{Ac}{3}{}
\reading{N2F1N1L2}{N2F1N1L2}{}{Ac. 3: 9-16}{}{}{}
\reading{N2F1N1L3}{N2F1N1L3}{}{Ac. 4: 5-12}{Ac}{4}{}
\feast{0102}
        {Octave de Saint Étienne}{Propre du Temps}{Octave de Saint Étienne}{2}{2 janvier}
        {}{}{Etienne@Étienne, protomartyr!Octave}{}{}
\reading{0102N1L1}{0102N1L1}{De Epístola ad Romános}{Rm. 5: 1-5}{Rm}{5}{}
\reading{0102N1L2}{0102N1L2}{}{Rm. 5: 6-9}{}{}{}
\reading{0102N1L3}{0102N1L3}{}{Rm. 5: 10-12}{}{}{}
\feast{0103}
        {Octave de Saint Jean}{Propre du Temps}{Octave de Saint Jean}{2}{3 janvier}
        {}{}{Jean, apôtre et évangéliste!Octave}{}{}
\reading{0103N1L1}{0103N1L1}{De Epístola ad Romános}{Rm. 6: 1-5}{Rm}{6}{}
\reading{0103N1L2}{0103N1L2}{}{Rm. 6: 6-11}{}{}{}
\reading{0103N1L3}{0103N1L3}{}{Rm. 6: 12-18}{}{}{}
\feast{0104}
        {Octave des Saints Innocents}{Propre du Temps}{Octave des Saints Innocents}{2}{4 janvier}
        {}{}{Innocents!Octave}{}{}
\reading{0104N1L1}{0104N1L1}{De Epístola ad Romános}{Rm. 7: 1-3}{Rm}{7}{}
\reading{0104N1L2}{0104N1L2}{}{Rm. 7: 4-6}{}{}{}
\reading{0104N1L3}{0104N1L3}{}{Rm. 7: 7-9}{}{}{}
\feast{0105}
        {Vigile de l’Épiphanie du Seigneur}{Propre du Temps}{Vigile de l’Épiphanie du Seigneur}{2}{5 janvier}
        {}{}{Notre-Seigneur Jésus-Christ!Épiphanie!Vigile}{}{}
\reading{0105N1L1}{0105N1L1}{De Epístola ad Romános}{Rm. 8: 1-4}{Rm}{8}{}
\reading{0105N1L2}{0105N1L2}{}{Rm. 8: 5-9}{}{}{}
\reading{0105N1L3}{0105N1L3}{}{Rm. 8: 9-11}{}{}{}
\feast{0106b}
        {}{Propre du Temps}{6 janvier}{3}{6 janvier, de la férie}
        {}{}{}{}{}
\reading{0106bN1L1}{0106bN1L1}{De Epístola ad Romános}{Rm. 8: 12-16}{Rm}{8}{}
\reading{0106bN1L2}{0106bN1L2}{}{Rm. 8: 17-20}{}{}{}
\reading{0106bN1L3}{0106bN1L3}{}{Rm. 8: 21-25}{}{}{}
\feast{0106}
        {Épiphanie du Seigneur}{Propre du Temps}{Épiphanie du Seigneur}{1}{6 janvier}
        {}{}{Notre-Seigneur Jésus-Christ!Épiphanie}{}{}
\reading{0106N1L1}{0106N1L1}{De Isaía Prophéta}{Is. 55: 1-4}{Is}{55}{}
\reading{0106N1L2}{0106N1L2}{}{Is. 60: 1-6}{Is}{60}{}
\reading{0106N1L3}{0106N1L3}{}{Is. 61: 10-11; 62: 1}{Is}{61}{62}
\feast{E1F1}
        {Fête de la Sainte Famille}{Propre du Temps}{Sainte Famille}{2}{Dimanche dans l’Octave de l’Épiphanie}
        {}{}{Notre-Seigneur Jésus-Christ!Sainte Famille}{}{}
\reading{E1F1N1L1}{E1F1N1L1}{De Epístola beáti Pauli Apóstoli ad Colossénses}{Col. 3: 12-16}{Col}{3}{}
\reading{E1F1N1L2}{E1F1N1L2}{}{Col. 3: 17-21}{}{}{}
\reading{E1F1N1L3}{E1F1N1L3}{}{Col. 3: 22-25; 4: 1-2}{Col}{4}{}
\rubric{Le premier jour sans lectures scripturaires propres qui suit la fête de la Sainte Famille,
on commence la première épître aux Corinthiens, telle que donnée ci-dessous au premier dimanche après l'Épiphanie.}
\feast{0107}
        {Deuxième jour dans l’Octave de l’Épiphanie}{Propre du Temps}{Pendant l’Octave de l’Épiphanie}{3}{7 janvier}
        {}{}{}{}{}
\reading{0107N1L1}{0107N1L1}{De Epístola ad Romános}{Rm. 9: 1-5}{Rm}{9}{}
\reading{0107N1L2}{0107N1L2}{}{Rm. 9: 6-10}{}{}{}
\reading{0107N1L3}{0107N1L3}{}{Rm. 9: 11-16}{}{}{}
\feast{0108}
        {Troisième jour dans l’Octave de l’Épiphanie}{Propre du Temps}{Pendant l’Octave de l’Épiphanie}{3}{8 janvier}
        {}{}{}{}{}
\reading{0108N1L1}{0108N1L1}{De Epístola ad Romános}{Rm. 12: 1-3}{Rm}{12}{}
\reading{0108N1L2}{0108N1L2}{}{Rm. 12: 4-8}{}{}{}
\reading{0108N1L3}{0108N1L3}{}{Rm. 12: 9-16}{}{}{}
\feast{0109}
        {Quatrième jour dans l’Octave de l’Épiphanie}{Propre du Temps}{Pendant l’Octave de l’Épiphanie}{3}{9 janvier}
        {}{}{}{}{}
\reading{0109N1L1}{0109N1L1}{De Epístola ad Romános}{Rm. 13: 1-4}{Rm}{13}{}
\reading{0109N1L2}{0109N1L2}{}{Rm. 13: 4-7}{}{}{}
\reading{0109N1L3}{0109N1L3}{}{Rm. 13: 8-10}{}{}{}
\feast{0110}
        {Cinquième jour dans l’Octave de l’Épiphanie}{Propre du Temps}{Pendant l’Octave de l’Épiphanie}{3}{10 janvier}
        {}{}{}{}{}
\reading{0110N1L1}{0110N1L1}{De Epístola ad Romános}{Rm. 14: 1-4}{Rm}{14}{}
\reading{0110N1L2}{0110N1L2}{}{Rm. 14: 5-8}{}{}{}
\reading{0110N1L3}{0110N1L3}{}{Rm. 14: 9-13}{}{}{}
\feast{0111}
        {Sixième jour dans l’Octave de l’Épiphanie}{Propre du Temps}{Pendant l’Octave de l’Épiphanie}{3}{11 janvier}
        {}{}{}{}{}
\reading{0111N1L1}{0111N1L1}{De Epístola ad Romános}{Rm. 15: 1-4}{Rm}{15}{}
\reading{0111N1L2}{0111N1L2}{}{Rm. 15: 5-11}{}{}{}
\reading{0111N1L3}{0111N1L3}{}{Rm. 15: 12-16}{}{}{}
\feast{0112}
        {Septième jour dans l’Octave de l’Épiphanie}{Propre du Temps}{Pendant l’Octave de l’Épiphanie}{3}{12 janvier}
        {}{}{}{}{}
\reading{0112N1L1}{0112N1L1}{De Epístola ad Romános}{Rm. 16: 1-5}{Rm}{16}{}
\reading{0112N1L2}{0112N1L2}{}{Rm. 16: 17-19}{}{}{}
\reading{0112N1L3}{0112N1L3}{}{Rm. 16: 21-24}{}{}{}
\feast{0113}
        {Octave de l’Épiphanie et Baptême du Seigneur}{Propre du Temps}{Octave de l’Épiphanie}{2}{13 janvier}
        {}{}{Notre-Seigneur Jésus-Christ!Épiphanie!Octave, Baptême du Seigneur}{}{}
\feast{E1F1b}
        {Premier dimanche après l’Épiphanie, dans l’Octave}{Propre du Temps}{Première semaine après l’Épiphanie}{2}{}
        {}{}{}{}{}
\reading{E1F1bN1L1}{E1F1bN1L1}{Incipit Epístola prima beáti Pauli Apóstoli ad Corínthios}{1 Co. 1: 1-3}{1 Co}{1}{}
\reading{E1F1bN1L2}{E1F1bN1L2}{}{1 Co. 1: 4-9}{}{}{}
\reading{E1F1bN1L3}{E1F1bN1L3}{}{1 Co. 1: 10-13}{}{}{}
\feast{E1F2}
        {Lundi}{Propre du Temps}{Première semaine après l’Épiphanie}{3}{}
        {}{}{}{}{}
\reading{E1F2L1}{E1F2L1}{De Epístola prima ad Corínthios}{1 Co. 2: 1-5}{1 Co}{2}{}
\reading{E1F2L2}{E1F2L2}{}{1 Co. 2: 6-9}{}{}{}
\reading{E1F2L3}{E1F2L3}{}{1 Co. 2: 10-13}{}{}{}
\feast{E1F3}
        {Mardi}{Propre du Temps}{Première semaine après l’Épiphanie}{3}{}
        {}{}{}{}{}
\reading{E1F3L1}{E1F3L1}{De Epístola prima ad Corínthios}{1 Co. 5: 1-5}{1 Co}{5}{}
\reading{E1F3L2}{E1F3L2}{}{1 Co. 5: 6-8}{}{}{}
\reading{E1F3L3}{E1F3L3}{}{1 Co. 5: 9-11}{}{}{}
\feast{E1F4}
        {Mercredi}{Propre du Temps}{Première semaine après l’Épiphanie}{3}{}
        {}{}{}{}{}
\reading{E1F4L1}{E1F4L1}{De Epístola prima ad Corínthios}{1 Co. 6: 1-6}{1 Co}{6}{}
\reading{E1F4L2}{E1F4L2}{}{1 Co. 6: 7-11}{}{}{}
\reading{E1F4L3}{E1F4L3}{}{1 Co. 6: 12-18}{}{}{}
\feast{E1F5}
        {Jeudi}{Propre du Temps}{Première semaine après l’Épiphanie}{3}{}
        {}{}{}{}{}
\reading{E1F5L1}{E1F5L1}{De Epístola prima ad Corínthios}{1 Co. 7: 1-4}{1 Co}{7}{}
\reading{E1F5L2}{E1F5L2}{}{1 Co. 7: 5-9}{}{}{}
\reading{E1F5L3}{E1F5L3}{}{1 Co. 7: 10-14}{}{}{}
\feast{E1F6}
        {Vendredi}{Propre du Temps}{Première semaine après l’Épiphanie}{3}{}
        {}{}{}{}{}
\reading{E1F6L1}{E1F6L1}{De Epístola prima ad Corínthios}{1 Co. 13: 1-3}{1 Co}{13}{}
\reading{E1F6L2}{E1F6L2}{}{1 Co. 13: 4-10}{}{}{}
\reading{E1F6L3}{E1F6L3}{}{1 Co. 13: 11-13}{}{}{}
\feast{E1F7}
        {Samedi}{Propre du Temps}{Première semaine après l’Épiphanie}{3}{}
        {}{}{}{}{}
\rubric{Si ce samedi, ou l'un des autres samedis après l'Épiphanie, est suivi du dimanche de la Septuagésime, et que le dimanche après l’Épiphanie ainsi empêché n’est pas transféré aux derniers dimanches après la Pentecôte, selon les rubriques, alors on emploie ce samedi le commencement de l’épître paulinienne du dimanche empêché.}
\reading{E1F7L1}{E1F7L1}{De Epístola prima ad Corínthios}{1 Co. 16: 1-4}{1 Co}{16}{}
\reading{E1F7L2}{E1F7L2}{}{1 Co. 16: 5-9}{}{}{}
\reading{E1F7L3}{E1F7L3}{}{1 Co. 16: 10-14}{}{}{}
\feast{E2F1}
        {Deuxième dimanche après l’Épiphanie}{Propre du Temps}{Deuxième semaine après l’Épiphanie}{2}{}
        {}{}{}{}{}
\reading{E2F1N1L1}{E2F1N1L1}{Incipit Epístola secúnda beáti Pauli Apóstoli ad Corínthios}{2 Co. 1: 1-5}{2 Co}{1}{}
\reading{E2F1N1L2}{E2F1N1L2}{}{2 Co. 1: 6-7}{}{}{}
\reading{E2F1N1L3}{E2F1N1L3}{}{2 Co. 1: 8-11}{}{}{}
\feast{E2F2}
        {Lundi}{Propre du Temps}{Deuxième semaine après l’Épiphanie}{3}{}
        {}{}{}{}{}
\reading{E2F2L1}{E2F2L1}{De Epístola secúnda ad Corínthios}{2 Co. 3: 1-3}{2 Co}{3}{}
\reading{E2F2L2}{E2F2L2}{}{2 Co. 3: 4-8}{}{}{}
\reading{E2F2L3}{E2F2L3}{}{2 Co. 3: 9-14}{}{}{}
\feast{E2F3}
        {Mardi}{Propre du Temps}{Deuxième semaine après l’Épiphanie}{3}{}
        {}{}{}{}{}
\reading{E2F3L1}{E2F3L1}{De Epístola secúnda ad Corínthios}{2 Co. 5: 1-4}{2 Co}{5}{}
\reading{E2F3L2}{E2F3L2}{}{2 Co. 5: 6-10}{}{}{}
\reading{E2F3L3}{E2F3L3}{}{2 Co. 5: 11-15}{}{}{}
\feast{E2F4}
        {Mercredi}{Propre du Temps}{Deuxième semaine après l’Épiphanie}{3}{}
        {}{}{}{}{}
\reading{E2F4L1}{E2F4L1}{De Epístola secúnda ad Corínthios}{2 Co. 7: 1-3}{2 Co}{7}{}
\reading{E2F4L2}{E2F4L2}{}{2 Co. 7: 4-7}{}{}{}
\reading{E2F4L3}{E2F4L3}{}{2 Co. 7: 8-10}{}{}{}
\feast{E2F5}
        {Jeudi}{Propre du Temps}{Deuxième semaine après l’Épiphanie}{3}{}
        {}{}{}{}{}
\reading{E2F5L1}{E2F5L1}{De Epístola secúnda ad Corínthios}{2 Co. 10: 1-3}{2 Co}{10}{}
\reading{E2F5L2}{E2F5L2}{}{2 Co. 10: 4-7}{}{}{}
\reading{E2F5L3}{E2F5L3}{}{2 Co. 10: 8-12}{}{}{}
\feast{E2F6}
        {Vendredi}{Propre du Temps}{Deuxième semaine après l’Épiphanie}{3}{}
        {}{}{}{}{}
\reading{E2F6L1}{E2F6L1}{De Epístola secúnda ad Corínthios}{2 Co. 12: 1-4}{2 Co}{12}{}
\reading{E2F6L2}{E2F6L2}{}{2 Co. 12: 5-9}{}{}{}
\reading{E2F6L3}{E2F6L3}{}{2 Co. 12: 9-11}{}{}{}
\feast{E2F7}
        {Samedi}{Propre du Temps}{Deuxième semaine après l’Épiphanie}{3}{}
        {}{}{}{}{}
\reading{E2F7L1}{E2F7L1}{De Epístola secúnda ad Corínthios}{2 Co. 13: 1-4}{2 Co}{13}{}
\reading{E2F7L2}{E2F7L2}{}{2 Co. 13: 5-9}{}{}{}
\reading{E2F7L3}{E2F7L3}{}{2 Co. 13: 10-13}{}{}{}
\feast{E3F1}
        {Troisième dimanche après l’Épiphanie}{Propre du Temps}{Troisième semaine après l’Épiphanie}{2}{}
        {}{}{}{}{}
\reading{E3F1N1L1}{E3F1N1L1}{Incipit Epístola beáti Pauli Apóstoli ad Gálatas}{Ga. 1: 1-5}{Ga}{1}{}
\reading{E3F1N1L2}{E3F1N1L2}{}{Ga. 1: 6-10}{}{}{}
\reading{E3F1N1L3}{E3F1N1L3}{}{Ga. 1: 11-14}{}{}{}
\feast{E3F2}
        {Lundi}{Propre du Temps}{Troisième semaine après l’Épiphanie}{3}{}
        {}{}{}{}{}
\reading{E3F2L1}{E3F2L1}{De Epístola ad Gálatas}{Ga. 3: 1-6}{Ga}{3}{}
\reading{E3F2L2}{E3F2L2}{}{Ga. 3: 7-10}{}{}{}
\reading{E3F2L3}{E3F2L3}{}{Ga. 3: 11-14}{}{}{}
\feast{E3F3}
        {Mardi}{Propre du Temps}{Troisième semaine après l’Épiphanie}{3}{}
        {}{}{}{}{}
\reading{E3F3L1}{E3F3L1}{De Epístola ad Gálatas}{Ga. 5: 1-5}{Ga}{5}{}
\reading{E3F3L2}{E3F3L2}{}{Ga. 5: 6-10}{}{}{}
\reading{E3F3L3}{E3F3L3}{}{Ga. 5: 11-17}{}{}{}
\feast{E3F4}
        {Mercredi}{Propre du Temps}{Troisième semaine après l’Épiphanie}{3}{}
        {}{}{}{}{}
\reading{E3F4L1}{E3F4L1}{Incipit Epístola beáti Pauli Apóstoli ad Ephésios}{Ep. 1: 1-4}{Ep}{1}{}
\reading{E3F4L2}{E3F4L2}{}{Ep. 1: 5-10}{}{}{}
\reading{E3F4L3}{E3F4L3}{}{Ep. 1: 11-14}{}{}{}
\feast{E3F5}
        {Jeudi}{Propre du Temps}{Troisième semaine après l’Épiphanie}{3}{}
        {}{}{}{}{}
\reading{E3F5L1}{E3F5L1}{De Epístola ad Ephésios}{Ep. 4: 1-6}{Ep}{4}{}
\reading{E3F5L2}{E3F5L2}{}{Ep. 4: 7-10}{}{}{}
\reading{E3F5L3}{E3F5L3}{}{Ep. 4: 11-15}{}{}{}
\feast{E3F6}
        {Vendredi}{Propre du Temps}{Troisième semaine après l’Épiphanie}{3}{}
        {}{}{}{}{}
\reading{E3F6L1}{E3F6L1}{De Epístola ad Ephésios}{Ep. 5: 1-4}{Ep}{5}{}
\reading{E3F6L2}{E3F6L2}{}{Ep. 5: 5-8}{}{}{}
\reading{E3F6L3}{E3F6L3}{}{Ep. 5: 9-14}{}{}{}
\feast{E3F7}
        {Samedi}{Propre du Temps}{Troisième semaine après l’Épiphanie}{3}{}
        {}{}{}{}{}
\reading{E3F7L1}{E3F7L1}{De Epístola ad Ephésios}{Ep. 6: 1-4}{Ep}{6}{}
\reading{E3F7L2}{E3F7L2}{}{Ep. 6: 5-9}{}{}{}
\reading{E3F7L3}{E3F7L3}{}{Ep. 6: 10-13}{}{}{}
\feast{E4F1}
        {Quatrième dimanche après l’Épiphanie}{Propre du Temps}{Quatrième semaine après l’Épiphanie}{2}{}
        {}{}{}{}{}
\reading{E4F1N1L1}{E4F1N1L1}{Incipit Epístola beáti Pauli Apóstoli ad Philippénses}{Ph. 1: 1-7}{Ph}{1}{}
\reading{E4F1N1L2}{E4F1N1L2}{}{Ph. 1: 8-14}{}{}{}
\reading{E4F1N1L3}{E4F1N1L3}{}{Ph. 1: 15-18}{}{}{}
\feast{E4F2}
        {Lundi}{Propre du Temps}{Quatrième semaine après l’Épiphanie}{3}{}
        {}{}{}{}{}
\reading{E4F2L1}{E4F2L1}{De Epístola ad Philippénses}{Ph. 4: 1-3}{Ph}{4}{}
\reading{E4F2L2}{E4F2L2}{}{Ph. 4: 4-7}{}{}{}
\reading{E4F2L3}{E4F2L3}{}{Ph. 4: 8-10}{}{}{}
\feast{E4F3}
        {Mardi}{Propre du Temps}{Quatrième semaine après l’Épiphanie}{3}{}
        {}{}{}{}{}
\reading{E4F3L1}{E4F3L1}{Incipit Epístola beáti Pauli Apóstoli ad Colossénses}{Col. 1: 1-8}{Col}{1}{}
\reading{E4F3L2}{E4F3L2}{}{Col. 1: 9-12}{}{}{}
\reading{E4F3L3}{E4F3L3}{}{Col. 1: 13-18}{}{}{}
\feast{E4F4}
        {Mercredi}{Propre du Temps}{Quatrième semaine après l’Épiphanie}{3}{}
        {}{}{}{}{}
\reading{E4F4L1}{E4F4L1}{De Epístola ad Colossénses}{Col. 3: 12-15}{Col}{3}{}
\reading{E4F4L2}{E4F4L2}{}{Col. 3: 16-21}{}{}{}
\reading{E4F4L3}{E1F1N1L3}{}{Col. 3: 22-25; 4: 1-2}{Col}{4}{}
\feast{E4F5}
        {Jeudi}{Propre du Temps}{Quatrième semaine après l’Épiphanie}{3}{}
        {}{}{}{}{}
\reading{E4F5L1}{E4F5L1}{Incipit Epístola prima beáti Pauli Apóstoli ad Thessalonicénses}{1 Th. 1: 1-5}{1 Th}{1}{}
\reading{E4F5L2}{E4F5L2}{}{1 Th. 1: 6-10}{}{}{}
\reading{E4F5L3}{E4F5L3}{}{1 Th. 2: 1-6}{1 Th}{2}{}
\feast{E4F6}
        {Vendredi}{Propre du Temps}{Quatrième semaine après l’Épiphanie}{3}{}
        {}{}{}{}{}
\reading{E4F6L1}{E4F6L1}{De Epístola prima ad Thessalonicénses}{1 Th. 4: 1-5}{1 Th}{4}{}
\reading{E4F6L2}{E4F6L2}{}{1 Th. 4: 6-8}{}{}{}
\reading{E4F6L3}{E4F6L3}{}{1 Th. 4: 9-12}{}{}{}
\feast{E4F7}
        {Samedi}{Propre du Temps}{Quatrième semaine après l’Épiphanie}{3}{}
        {}{}{}{}{}
\reading{E4F7L1}{E4F7L1}{Incipit Epístola secúnda beáti Pauli Apóstoli ad Thessalonicénses}{2 Th. 1: 1-5}{2 Th}{1}{}
\reading{E4F7L2}{E4F7L2}{}{2 Th. 1: 6-12}{}{}{}
\reading{E4F7L3}{E4F7L3}{}{2 Th. 2: 1-4}{2 Th}{2}{}
\feast{E5F1}
        {Cinquième dimanche après l’Épiphanie}{Propre du Temps}{Cinquième semaine après l’Épiphanie}{2}{}
        {}{}{}{}{}
\reading{E5F1N1L1}{E5F1N1L1}{Incipit Epístola prima beáti Pauli Apóstoli ad Timótheum}{1 Tm. 1: 1-4}{1 Tm}{1}{}
\reading{E5F1N1L2}{E5F1N1L2}{}{1 Tm. 1: 5-11}{}{}{}
\reading{E5F1N1L3}{E5F1N1L3}{}{1 Tm. 1: 12-16}{}{}{}
\feast{E5F2}
        {Lundi}{Propre du Temps}{Cinquième semaine après l’Épiphanie}{3}{}
        {}{}{}{}{}
\reading{E5F2L1}{E5F2L1}{De Epístola prima ad Timótheum}{1 Tm. 3: 1-7}{1 Tm}{3}{}
\reading{E5F2L2}{E5F2L2}{}{1 Tm. 3: 8-13}{}{}{}
\reading{E5F2L3}{E5F2L3}{}{1 Tm. 3: 14-16; 4: 1}{1 Tm}{4}{}
\feast{E5F3}
        {Mardi}{Propre du Temps}{Cinquième semaine après l’Épiphanie}{3}{}
        {}{}{}{}{}
\reading{E5F3L1}{E5F3L1}{Incipit Epístola secúnda beáti Pauli Apóstoli ad Timótheum}{2 Tm. 1: 1-5}{2 Tm}{1}{}
\reading{E5F3L2}{E5F3L2}{}{2 Tm. 1: 6-9}{}{}{}
\reading{E5F3L3}{E5F3L3}{}{2 Tm. 1: 10-13}{}{}{}
\feast{E5F4}
        {Mercredi}{Propre du Temps}{Cinquième semaine après l’Épiphanie}{3}{}
        {}{}{}{}{}
\reading{E5F4L1}{E5F4L1}{De Epístola secúnda ad Timótheum}{2 Tm. 3: 1-5}{2 Tm}{3}{}
\reading{E5F4L2}{E5F4L2}{}{2 Tm. 3: 6-9}{}{}{}
\reading{E5F4L3}{E5F4L3}{}{2 Tm. 3: 10-13}{}{}{}
\feast{E5F5}
        {Jeudi}{Propre du Temps}{Cinquième semaine après l’Épiphanie}{3}{}
        {}{}{}{}{}
\reading{E5F5L1}{E5F5L1}{Incipit Epístola beáti Paul Apóstoli ad Titum}{Tt. 1: 1-4}{Tt}{1}{}
\reading{E5F5L2}{E5F5L2}{}{Tt. 1: 5-9}{}{}{}
\reading{E5F5L3}{E5F5L3}{}{Tt. 1: 10-15}{}{}{}
\feast{E5F6}
        {Vendredi}{Propre du Temps}{Cinquième semaine après l’Épiphanie}{3}{}
        {}{}{}{}{}
\reading{E5F6L1}{E5F6L1}{De Epístola ad Titum}{Tt. 2: 15; 3: 1-2}{Tt}{2}{3}
\reading{E5F6L2}{E5F6L2}{}{Tt. 3: 3-7}{}{}{}
\reading{E5F6L3}{E5F6L3}{}{Tt. 3: 8-11}{}{}{}
\feast{E5F7}
        {Samedi}{Propre du Temps}{Cinquième semaine après l’Épiphanie}{3}{}
        {}{}{}{}{}
\reading{E5F7L1}{E5F7L1}{Incipit Epístola beáti Pauli Apóstoli ad Philémonem}{Phm. 1: 1-6}{Phm}{1}{}
\reading{E5F7L2}{E5F7L2}{}{Phm. 1: 7-11}{}{}{}
\reading{E5F7L3}{E5F7L3}{}{Phm. 1: 12-19}{}{}{}
\feast{E6F1}
        {Sixième dimanche après l’Épiphanie}{Propre du Temps}{Sixième semaine après l’Épiphanie}{2}{}
        {}{}{}{}{}
\reading{E6F1N1L1}{E6F1N1L1}{Incipit Epístola beáti Pauli Apóstoli ad Hebrǽos}{He. 1: 1-4}{He}{1}{}
\reading{E6F1N1L2}{E6F1N1L2}{}{He. 1: 5-9}{}{}{}
\reading{E6F1N1L3}{E6F1N1L3}{}{He. 1: 10-14}{}{}{}
\feast{E6F2}
        {Lundi}{Propre du Temps}{Sixième semaine après l’Épiphanie}{3}{}
        {}{}{}{}{}
\reading{E6F2L1}{E6F2L1}{De Epístola ad Hebrǽos}{He. 3: 1-4}{He}{3}{}
\reading{E6F2L2}{E6F2L2}{}{He. 3: 5-8}{}{}{}
\reading{E6F2L3}{E6F2L3}{}{He. 3: 12-16}{}{}{}
\feast{E6F3}
        {Mardi}{Propre du Temps}{Sixième semaine après l’Épiphanie}{3}{}
        {}{}{}{}{}
\reading{E6F3L1}{E6F3L1}{De Epístola ad Hebrǽos}{He. 4: 1-3}{He}{4}{}
\reading{E6F3L2}{E6F3L2}{}{He. 4: 4-7}{}{}{}
\reading{E6F3L3}{E6F3L3}{}{He. 4: 8-12}{}{}{}
\feast{E6F4}
        {Mercredi}{Propre du Temps}{Sixième semaine après l’Épiphanie}{3}{}
        {}{}{}{}{}
\reading{E6F4L1}{E6F4L1}{De Epístola ad Hebrǽos}{He. 6: 1-3}{He}{6}{}
\reading{E6F4L2}{E6F4L2}{}{He. 6: 4-6}{}{}{}
\reading{E6F4L3}{E6F4L3}{}{He. 6: 7-10}{}{}{}
\feast{E6F5}
        {Jeudi}{Propre du Temps}{Sixième semaine après l’Épiphanie}{3}{}
        {}{}{}{}{}
\reading{E6F5L1}{E6F5L1}{De Epístola ad Hebrǽos}{He. 7: 1-3}{He}{7}{}
\reading{E6F5L2}{E6F5L2}{}{He. 7: 4-6}{}{}{}
\reading{E6F5L3}{E6F5L3}{}{He. 7: 7-12}{}{}{}
\feast{E6F6}
        {Vendredi}{Propre du Temps}{Sixième semaine après l’Épiphanie}{3}{}
        {}{}{}{}{}
\reading{E6F6L1}{E6F6L1}{De Epístola ad Hebrǽos}{He. 11: 1-4}{He}{11}{}
\reading{E6F6L2}{E6F6L2}{}{He. 11: 5-7}{}{}{}
\reading{E6F6L3}{E6F6L3}{}{He. 11: 8-10}{}{}{}
\feast{E6F7}
        {Samedi}{Propre du Temps}{Sixième semaine après l’Épiphanie}{3}{}
        {}{}{}{}{}
\reading{E6F7L1}{E6F7L1}{De Epístola ad Hebrǽos}{He. 13: 1-4}{He}{13}{}
\reading{E6F7L2}{E6F7L2}{}{He. 13: 5-8}{}{}{}
\reading{E6F7L3}{E6F7L3}{}{He. 13: 9-12}{}{}{}
\feast{7GF1}
        {Dimanche de la Septuagésime}{Propre du Temps}{Semaine de la Septuagésime}{2}{}
        {}{}{Septuagésime}{}{}
\rubric{Si ces lectures sont empêchées, elles sont employées au premier jour de la semaine où on lit l'Écriture courante. Il est fait de même pour les dimanches de la Sexagésime et de la Quinquagésime.}
\reading{7GF1N1L1}{7GF1N1L1}{Incipit liber Génesis}{Gn. 1: 1-8}{Gn}{1}{}
\reading{7GF1N1L2}{7GF1N1L2}{}{Gn. 1: 9-19}{}{}{}
\reading{7GF1N1L3}{7GF1N1L3}{}{Gn. 1: 20-26}{}{}{}
\feast{7GF2}
        {Lundi}{Propre du Temps}{Semaine de la Septuagésime}{3}{}
        {}{}{}{}{}
\reading{7GF2L1}{7GF2L1}{De libro Génesis}{Gn. 1: 27-31}{Gn}{1}{}
\reading{7GF2L2}{7GF2L2}{}{Gn. 2: 1-6}{Gn}{2}{}
\reading{7GF2L3}{7GF2L3}{}{Gn. 2: 6-10}{}{}{}
\feast{7GF3}
        {Mardi}{Propre du Temps}{Semaine de la Septuagésime}{3}{}
        {}{}{}{}{}
\reading{7GF3L1}{7GF3L1}{De libro Génesis}{Gn. 2: 15-18}{Gn}{2}{}
\reading{7GF3L2}{7GF3L2}{}{Gn. 2: 19-20}{}{}{}
\reading{7GF3L3}{7GF3L3}{}{Gn. 2: 21-24}{}{}{}
\feast{7GF4}
        {Mercredi}{Propre du Temps}{Semaine de la Septuagésime}{3}{}
        {}{}{}{}{}
\reading{7GF4L1}{7GF4L1}{De libro Génesis}{Gn. 3: 1-7}{Gn}{3}{}
\reading{7GF4L2}{7GF4L2}{}{Gn. 3: 7-13}{}{}{}
\reading{7GF4L3}{7GF4L3}{}{Gn. 3: 14-20}{}{}{}
\feast{7GF5}
        {Jeudi}{Propre du Temps}{Semaine de la Septuagésime}{3}{}
        {}{}{}{}{}
\reading{7GF5L1}{7GF5L1}{De libro Génesis}{Gn. 4: 1-7}{Gn}{4}{}
\reading{7GF5L2}{7GF5L2}{}{Gn. 4: 8-12}{}{}{}
\reading{7GF5L3}{7GF5L3}{}{Gn. 4: 13-16}{}{}{}
\feast{7GF6}
        {Vendredi}{Propre du Temps}{Semaine de la Septuagésime}{3}{}
        {}{}{}{}{}
\reading{7GF6L1}{7GF6L1}{De libro Génesis}{Gn. 4: 17-22}{Gn}{4}{}
\reading{7GF6L2}{7GF6L2}{}{Gn. 4: 23-26}{}{}{}
\reading{7GF6L3}{7GF6L3}{}{Gn. 5: 1-5}{Gn}{5}{}
\feast{7GF7}
        {Samedi}{Propre du Temps}{Semaine de la Septuagésime}{3}{}
        {}{}{}{}{}
\reading{7GF7L1}{7GF7L1}{De libro Génesis}{Gn. 5: 5-21}{Gn}{5}{}
\reading{7GF7L2}{7GF7L2}{}{Gn. 5: 22-27}{}{}{}
\reading{7GF7L3}{7GF7L3}{}{Gn. 5: 28-31}{}{}{}
\feast{6GF1}
        {Dimanche de la Sexagésime}{Propre du Temps}{Semaine de la Sexagésime}{2}{}
        {}{}{Sexagésime}{}{}
\reading{6GF1N1L1}{6GF1N1L1}{De libro Génesis}{Gn. 5: 31; 6: 1-4}{Gn}{5}{6}
\reading{6GF1N1L2}{6GF1N1L2}{}{Gn. 6: 5-8}{}{}{}
\reading{6GF1N1L3}{6GF1N1L3}{}{Gn. 6: 9-15}{}{}{}
\feast{6GF2}
        {Lundi}{Propre du Temps}{Semaine de la Sexagésime}{3}{}
        {}{}{}{}{}
\reading{6GF2L1}{6GF2L1}{De libro Génesis}{Gn. 7: 1-4}{Gn}{7}{}
\reading{6GF2L2}{6GF2L2}{}{Gn. 7: 5, 10-12}{}{}{}
\reading{6GF2L3}{6GF2L3}{}{Gn. 7: 13-14, 17}{}{}{}
\feast{6GF3}
        {Mardi}{Propre du Temps}{Semaine de la Sexagésime}{3}{}
        {}{}{}{}{}
\reading{6GF3L1}{6GF3L1}{De libro Génesis}{Gn. 8: 1-4}{Gn}{8}{}
\reading{6GF3L2}{6GF3L2}{}{Gn. 8: 5-9}{}{}{}
\reading{6GF3L3}{6GF3L3}{}{Gn. 8: 10-13}{}{}{}
\feast{6GF4}
        {Mercredi}{Propre du Temps}{Semaine de la Sexagésime}{3}{}
        {}{}{}{}{}
\reading{6GF4L1}{6GF4L1}{De libro Génesis}{Gn. 8: 15-19}{Gn}{8}{}
\reading{6GF4L2}{6GF4L2}{}{Gn. 8: 20-22}{}{}{}
\reading{6GF4L3}{6GF4L3}{}{Gn. 9: 1-6}{Gn}{9}{}
\feast{6GF5}
        {Jeudi}{Propre du Temps}{Semaine de la Sexagésime}{3}{}
        {}{}{}{}{}
\reading{6GF5L1}{6GF5L1}{De libro Génesis}{Gn. 9: 12-15}{Gn}{9}{}
\reading{6GF5L2}{6GF5L2}{}{Gn. 9: 20-23}{}{}{}
\reading{6GF5L3}{6GF5L3}{}{Gn. 9: 24-29}{}{}{}
\feast{6GF6}
        {Vendredi}{Propre du Temps}{Semaine de la Sexagésime}{3}{}
        {}{}{}{}{}
\reading{6GF6L1}{6GF6L1}{De libro Génesis}{Gn. 10: 1-6}{Gn}{10}{}
\reading{6GF6L2}{6GF6L2}{}{Gn. 11: 1-4}{Gn}{11}{}
\reading{6GF6L3}{6GF6L3}{}{Gn. 11: 5-8}{}{}{}
\feast{6GF7}
        {Samedi}{Propre du Temps}{Semaine de la Sexagésime}{3}{}
        {}{}{}{}{}
\reading{6GF7L1}{6GF7L1}{De libro Génesis}{Gn. 11: 10-15}{Gn}{11}{}
\reading{6GF7L2}{6GF7L2}{}{Gn. 11: 16-23}{}{}{}
\reading{6GF7L3}{6GF7L3}{}{Gn. 11: 24-30}{}{}{}
\feast{5GF1}
        {Dimanche de la Quinquagésime}{Propre du Temps}{Semaine de la Quinquagésime}{2}{}
        {}{}{Quinquagésime}{}{}
\reading{5GF1N1L1}{5GF1N1L1}{De libro Génesis}{Gn. 12: 1-6}{Gn}{12}{}
\reading{5GF1N1L2}{5GF1N1L2}{}{Gn. 12: 7-13}{}{}{}
\reading{5GF1N1L3}{5GF1N1L3}{}{Gn. 12: 14-19}{}{}{}
\feast{5GF2}
        {Lundi}{Propre du Temps}{Semaine de la Quinquagésime}{3}{}
        {}{}{}{}{}
\rubric{Si les lectures du lundi ou du mardi avant les Cendres sont empêchées, et qu’une fête à trois nocturnes tombant le jeudi, le vendredi ou le samedi après les Cendres n'a pas de lectures scripturaires propres, on emploie les lectures empêchées. Si les lectures du lundi et du mardi sont empêchées, et qu'il n'est pas possible d’employer les deux ultérieurement, on privilégie celles du mardi.}
\reading{5GF2L1}{5GF2L1}{De libro Génesis}{Gn. 13: 1-6}{Gn}{13}{}
\reading{5GF2L2}{5GF2L2}{}{Gn. 13: 7-11}{}{}{}
\reading{5GF2L3}{5GF2L3}{}{Gn. 13: 11-16}{}{}{}
\feast{5GF3}
        {Mardi}{Propre du Temps}{Semaine de la Quinquagésime}{3}{}
        {}{}{}{}{}
\reading{5GF3L1}{5GF3L1}{De libro Génesis}{Gn. 14: 8-12}{Gn}{14}{}
\reading{5GF3L2}{5GF3L2}{}{Gn. 14: 13-16}{}{}{}
\reading{5GF3L3}{5GF3L3}{}{Gn. 14: 17-20}{}{}{}
\feast{5GF4}
        {Mercredi des Cendres et féries de Carême}{Propre du Temps}{Semaine de la Quinquagésime}{2}{}
        {}{}{Carême!0@Mercredi des Cendres}{}{}
\rubric{À compter du mercredi des Cendres, jusqu’au lundi de la semaine sainte inclusivement, les lectures des féries sont à l'Homéliaire.}
\feast{Q1F1}
        {Premier dimanche de Carême}{Propre du Temps}{Première semaine de Carême}{2}{}
        {}{}{Carême!A@Première semaine}{}{}
\reading{Q1F1N1L1}{Q1F1N1L1}{De Epístola secúnda beáti Pauli Apóstoli ad Corínthios}{2 Co. 6: 1-10}{2 Co}{6}{}
\reading{Q1F1N1L2}{Q1F1N1L2}{}{2 Co. 6: 11-16}{}{}{}
\reading{Q1F1N1L3}{Q1F1N1L3}{}{2 Co. 7: 4-9}{2 Co}{7}{}
\feast{Q2F1}
        {Deuxième dimanche de Carême}{Propre du Temps}{Deuxième semaine de Carême}{2}{}
        {}{}{Carême!B@Deuxième semaine}{}{}
\reading{Q2F1N1L1}{Q2F1N1L1}{De libro Génesis}{Gn. 27: 1-10}{Gn}{27}{}
\reading{Q2F1N1L2}{Q2F1N1L2}{}{Gn. 27: 11-20}{}{}{}
\reading{Q2F1N1L3}{Q2F1N1L3}{}{Gn. 27: 21-29}{}{}{}
\feast{Q3F1}
        {Troisième dimanche de Carême}{Propre du Temps}{Troisième semaine de Carême}{2}{}
        {}{}{Carême!C@Troisième semaine}{}{}
\reading{Q3F1N1L1}{Q3F1N1L1}{De libro Génesis}{Gn. 37: 2-10}{Gn}{37}{}
\reading{Q3F1N1L2}{Q3F1N1L2}{}{Gn. 37: 11-20}{}{}{}
\reading{Q3F1N1L3}{Q3F1N1L3}{}{Gn. 37: 21-28}{}{}{}
\feast{Q4F1}
        {Quatrième dimanche de Carême}{Propre du Temps}{Quatrième semaine de Carême}{2}{}
        {}{}{Carême!D@Quatrième semaine}{}{}
\reading{Q4F1N1L1}{Q4F1N1L1}{De libro Éxodi}{Ex. 3: 1-6}{Ex}{3}{}
\reading{Q4F1N1L2}{Q4F1N1L2}{}{Ex. 3: 7-10}{}{}{}
\reading{Q4F1N1L3}{Q4F1N1L3}{}{Ex. 3: 11-15}{}{}{}
\feast{Q5F1}
        {Dimanche de la Passion}{Propre du Temps}{Semaine de la Passion}{2}{}
        {}{}{Carême!E@Passion}{}{}
\reading{Q5F1N1L1}{Q5F1N1L1}{Incipit liber Jeremíæ Prophétæ}{Jr. 1: 1-6}{Jr}{1}{}
\reading{Q5F1N1L2}{Q5F1N1L2}{}{Jr. 1: 7-13}{}{}{}
\reading{Q5F1N1L3}{Q5F1N1L3}{}{Jr. 1: 14-19}{}{}{}
\feast{Q6F1}
        {Dimanche des Rameaux}{Propre du Temps}{Semaine Sainte}{2}{}
        {}{}{Carême!E@Rameaux}{}{}
\reading{Q6F1N1L1}{Q6F1N1L1}{De Jeremía Prophéta}{Jr. 2: 12-17}{Jr}{2}{}
\reading{Q6F1N1L2}{Q6F1N1L2}{}{Jr. 2: 18-22}{}{}{}
\reading{Q6F1N1L3}{Q6F1N1L3}{}{Jr. 2: 29-32}{}{}{}
\feast{Q6F3}
        {Mardi de la Semaine Sainte}{Propre du Temps}{Semaine Sainte}{3}{}
        {}{}{}{}{}
\reading{Q6F3L1}{Q6F3L1}{De Jeremía Prophéta}{Jr. 11: 15-20}{Jr}{11}{}
\reading{Q6F3L2}{Q6F3L2}{}{Jr. 12: 1-4}{Jr}{12}{}
\reading{Q6F3L3}{Q6F3L3}{}{Jr. 12: 7-11}{}{}{}
\feast{Q6F4}
        {Mercredi de la Semaine Sainte}{Propre du Temps}{Semaine Sainte}{3}{}
        {}{}{}{}{}
\reading{Q6F4L1}{Q6F4L1}{De Jeremía Prophéta}{Jr. 17: 13-18}{Jr}{17}{}
\reading{Q6F4L2}{Q6F4L2}{}{Jr. 18: 13-18}{Jr}{18}{}
\reading{Q6F4L3}{Q6F4L3}{}{Jr. 18: 19-23}{}{}{}

\cleartoleftpage{}
\feast{Q6F5}
        {Jeudi Saint}{Propre du Temps}{Jeudi Saint}{2}{}
        {}{}{Triduum Pascal!A@Jeudi Saint}{}{}
\rubric{Les Jeudi, Vendredi et Samedi Saints,
	les lectures sont chantées sans absolution, ni bénédiction, ni conclusion,
	sauf la conclusion spéciale des Lamentations, notée à son emplacement.}
\intermediatetitle{Premier nocturne}
\chantedscripturereading{Q6F5N1L1lat}
\vfill
\reading{Q6F5N1L1}{Q6F5N1L1}{Incipit Lamentátio Jeremíæ Prophétæ}{Lam. 1: 1-5}{Lam}{1}{}
\cleartoleftpage{}
\chantedscripturereading{Q6F5N1L2lat}
\vfill
\reading{Q6F5N1L2}{Q6F5N1L2}{}{Lam. 1: 6-9}{}{}{}
\vfill
\cleartoleftpage{}
\chantedscripturereading{Q6F5N1L3lat}
\vfill
\reading{Q6F5N1L3}{Q6F5N1L3}{}{Lam. 1: 10-14}{}{}{}
\vfill
\cleartoleftpage{}

\begin{paracol}[1]*{2}
\intermediatetitle{Troisième nocturne}
\switchcolumn
\nocturn{3}
\switchcolumn*
\reading{Q6F5N3L1}{Q6F5N3L1}{De Epístola prima beáti Pauli Apóstoli ad Corínthios}{1 Co. 11: 17-22}{1 Co}{11}{}
\switchcolumn
\reading{Q6F5N3L1lat}{Q6F5N3L1lat}{De Epístola prima beáti Pauli Apóstoli ad Corínthios}{\null}{}{}{}
\switchcolumn*
\reading{Q6F5N3L2}{Q6F5N3L2}{}{1 Co. 11: 23-26}{}{}{}
\switchcolumn
\reading{Q6F5N3L2lat}{Q6F5N3L2lat}{}{\null}{}{}{}
\switchcolumn*
\reading{Q6F5N3L3}{Q6F5N3L3}{}{1 Co. 11: 27-34}{}{}{}
\switchcolumn
\reading{Q6F5N3L3lat}{Q6F5N3L3lat}{}{\null}{}{}{}
\end{paracol}

\cleartoleftpage{}
\reading{Q6F5N3L3b}{Q6F5N3L3b}{}{}{}{}{}
\feast{Q6F6}
        {Vendredi Saint}{Propre du Temps}{Vendredi Saint}{2}{}
        {}{}{Triduum Pascal!B@Vendredi Saint}{}{}
\intermediatetitle{Premier nocturne}
\chantedscripturereading{Q6F6N1L1lat}
\vfill
\reading{Q6F6N1L1}{Q6F6N1L1}{De Lamentatióne Jeremíæ Prophétæ}{Lam. 2: 8-11}{Lam}{2}{}
\vfill
\cleartoleftpage{}
\chantedscripturereading{Q6F6N1L2lat}
\pagebreak\null\vfill
\reading{Q6F6N1L2}{Q6F6N1L2}{}{Lam. 2: 12-15}{}{}{}
\vfill
\cleartoleftpage{}
\chantedscripturereading{Q6F6N1L3lat}
\pagebreak\null\vfill
\reading{Q6F6N1L3}{Q6F6N1L3}{}{Lam. 3: 1-9}{Lam}{3}{}
\vfill
\cleartoleftpage{}

\begin{paracol}[1]*{2}
\intermediatetitle{Troisième nocturne}
\switchcolumn
\nocturn{3}
\switchcolumn*
\reading{Q6F6N3L1}{Q6F6N3L1}{De Epístola beáti Pauli Apóstoli ad Hebrǽos}{He. 4: 11-15}{He}{4}{}
\switchcolumn
\reading{Q6F6N3L1lat}{Q6F6N3L1lat}{De Epístola beáti Pauli Apóstoli ad Hebrǽos}{\null}{}{}{}
\switchcolumn*
\reading{Q6F6N3L2}{Q6F6N3L2}{}{He. 4: 16; 5: 1-3}{He}{4}{5}
\switchcolumn
\reading{Q6F6N3L2lat}{Q6F6N3L2lat}{}{\null}{}{}{}
\switchcolumn*
\reading{Q6F6N3L3}{Q6F6N3L3}{}{He. 5: 4-10}{}{}{}
\switchcolumn
\reading{Q6F6N3L3lat}{Q6F6N3L3lat}{}{\null}{}{}{}
\end{paracol}

\cleartoleftpage{}
\reading{Q6F6N3L3b}{Q6F6N3L3b}{}{}{}{}{}
\feast{Q6F7}
        {Samedi Saint}{Propre du Temps}{Samedi Saint}{2}{}
        {}{}{Triduum Pascal!C@Samedi Saint}{}{}
\intermediatetitle{Premier nocturne}
\chantedscripturereading{Q6F7N1L1lat}
\vfill
\reading{Q6F7N1L1}{Q6F7N1L1}{De Lamentatióne Jeremíæ Prophétæ}{Lam. 3: 22-30}{Lam}{3}{}
\vfill
\cleartoleftpage{}
\chantedscripturereading{Q6F7N1L2lat}
\pagebreak\null\vfill
\reading{Q6F7N1L2}{Q6F7N1L2}{}{Lam. 4: 1-6}{Lam}{4}{}
\vfill
\cleartoleftpage{}
\chantedscripturereading{Q6F7N1L3lat}
\vfill
\reading{Q6F7N1L3}{Q6F7N1L3}{Incipit Orátio Jeremíæ Prophétæ}{Lam. 5: 1-11}{Lam}{5}{}
\vfill
\null\par\vspace{1cm}
\rubric{Un autre ton ad libitum est donné page suivante pour cette lecture.}
\vspace{1cm}
\cleartoleftpage{}
\chantedscripturereading{Q6F7N1L3latb}
\cleartoleftpage{}

\begin{paracol}[1]*{2}
\intermediatetitle{Troisième nocturne}
\switchcolumn
\nocturn{3}
\switchcolumn*
\reading{Q6F7N3L1}{Q6F7N3L1}{De Epístola beáti Pauli Apóstoli ad Hebrǽos}{He. 9: 11-14}{He}{9}{}
\switchcolumn
\reading{Q6F7N3L1lat}{Q6F7N3L1lat}{De Epístola beáti Pauli Apóstoli ad Hebrǽos}{\null}{}{}{}
\switchcolumn*
\reading{Q6F7N3L2}{Q6F7N3L2}{}{He. 9: 15-18}{}{}{}
\switchcolumn
\reading{Q6F7N3L2lat}{Q6F7N3L2lat}{}{\null}{}{}{}
\switchcolumn*
\reading{Q6F7N3L3}{Q6F7N3L3}{}{He. 9: 19-22}{}{}{}
\switchcolumn
\reading{Q6F7N3L3lat}{Q6F7N3L3lat}{}{\null}{}{}{}
\end{paracol}

\feast{P0F1}
        {Résurrection du Seigneur}{Propre du Temps}{Résurrection du Seigneur}{1}{}
        {}{}{Notre-Seigneur Jésus-Christ!Résurrection}{}{}
\rubric{À Pâques et pendant son Octave, Matines à un nocturne avec lectures à l'Homéliaire.}
\feast{P1F1}
        {Premier dimanche après Pâques}{Propre du Temps}{Première semaine après Pâques}{2}{}
        {}{}{Temps Pascal!A@Première semaine}{}{}
\reading{P1F1N1L1}{P1F1N1L1}{De Epístola beáti Pauli Apóstoli ad Colossénses}{Col. 3: 1-7}{Col}{3}{}
\reading{P1F1N1L2}{P1F1N1L2}{}{Col. 3: 8-13}{}{}{}
\reading{P1F1N1L3}{P1F1N1L3}{}{Col. 3: 14-17}{}{}{}
\feast{P1F2}
        {Lundi}{Propre du Temps}{Première semaine après Pâques}{3}{}
        {}{}{}{}{}
\reading{P1F2L1}{P1F2L1}{Incipit liber Actuum Apostolórum}{Ac. 1: 1-8}{Ac}{1}{}
\reading{P1F2L2}{P1F2L2}{}{Ac. 1: 9-14}{}{}{}
\reading{P1F2L3}{P1F2L3}{}{Ac. 1: 15-26}{}{}{}
\feast{P1F3}
        {Mardi}{Propre du Temps}{Première semaine après Pâques}{3}{}
        {}{}{}{}{}
\reading{P1F3L1}{P1F3L1}{De Actibus Apostolórum}{Ac. 2: 1-8}{Ac}{2}{}
\reading{P1F3L2}{P1F3L2}{}{Ac. 2: 14-21}{}{}{}
\reading{P1F3L3}{P1F3L3}{}{Ac. 2: 22-27}{}{}{}
\feast{P1F4}
        {Mercredi}{Propre du Temps}{Première semaine après Pâques}{3}{}
        {}{}{}{}{}
\reading{P1F4L1}{P1F4L1}{De Actibus Apostolórum}{Ac. 3: 1-6}{Ac}{3}{}
\reading{P1F4L2}{P1F4L2}{}{Ac. 3: 7-11}{}{}{}
\reading{P1F4L3}{P1F4L3}{}{Ac. 3: 12-16}{}{}{}
\feast{P1F5}
        {Jeudi}{Propre du Temps}{Première semaine après Pâques}{3}{}
        {}{}{}{}{}
\reading{P1F5L1}{P1F5L1}{De Actibus Apostolórum}{Ac. 5: 1-6}{Ac}{5}{}
\reading{P1F5L2}{P1F5L2}{}{Ac. 5: 7-11}{}{}{}
\reading{P1F5L3}{P1F5L3}{}{Ac. 5: 12-16}{}{}{}
\feast{P1F6}
        {Vendredi}{Propre du Temps}{Première semaine après Pâques}{3}{}
        {}{}{}{}{}
\reading{P1F6L1}{P1F6L1}{De Actibus Apostolórum}{Ac. 8: 9-13}{Ac}{8}{}
\reading{P1F6L2}{P1F6L2}{}{Ac. 8: 14-19}{}{}{}
\reading{P1F6L3}{P1F6L3}{}{Ac. 8: 19-24}{}{}{}
\feast{P1F7}
        {Samedi}{Propre du Temps}{Première semaine après Pâques}{3}{}
        {}{}{}{}{}
\reading{P1F7L1}{P1F7L1}{De Actibus Apostolórum}{Ac. 10: 1-8}{Ac}{10}{}
\reading{P1F7L2}{P1F7L2}{}{Ac. 10: 9-17}{}{}{}
\reading{P1F7L3}{P1F7L3}{}{Ac. 10: 34-41}{}{}{}
\feast{P2F1}
        {Deuxième dimanche après Pâques}{Propre du Temps}{Deuxième semaine après Pâques}{2}{}
        {}{}{Temps Pascal!B@Deuxième semaine}{}{}
\reading{P2F1N1L1}{P2F1N1L1}{De Actibus Apostolórum}{Ac. 13: 13-20}{Ac}{13}{}
\reading{P2F1N1L2}{P2F1N1L2}{}{Ac. 13: 21-25}{}{}{}
\reading{P2F1N1L3}{P2F1N1L3}{}{Ac. 13: 26-33}{}{}{}
\feast{P2F2}
        {Lundi}{Propre du Temps}{Deuxième semaine après Pâques}{3}{}
        {}{}{}{}{}
\reading{P2F2L1}{P2F2L1}{De Actibus Apostolórum}{Ac. 15: 5-12}{Ac}{15}{}
\reading{P2F2L2}{P2F2L2}{}{Ac. 15: 13-21}{}{}{}
\reading{P2F2L3}{P2F2L3}{}{Ac. 15: 22-29}{}{}{}
\feast{P2F3}
        {Mardi}{Propre du Temps}{Deuxième semaine après Pâques}{3}{}
        {}{}{}{}{}
\reading{P2F3L1}{P2F3L1}{De Actibus Apostolórum}{Ac. 17: 22-27}{Ac}{17}{}
\reading{P2F3L2}{P2F3L2}{}{Ac. 17: 28-33}{}{}{}
\reading{P2F3L3}{P2F3L3}{}{Ac. 17: 34; 18: 1-4}{Ac}{18}{}
\feast{P2F4}
        {Mercredi}{Propre du Temps}{Deuxième semaine après Pâques}{3}{}
        {}{}{}{}{}
\reading{P2F4L1}{P2F4L1}{De Actibus Apostolórum}{Ac. 20: 17-24}{Ac}{20}{}
\reading{P2F4L2}{P2F4L2}{}{Ac. 20: 25-31}{}{}{}
\reading{P2F4L3}{P2F4L3}{}{Ac. 20: 32-38}{}{}{}
\feast{P2F5}
        {Jeudi}{Propre du Temps}{Deuxième semaine après Pâques}{3}{}
        {}{}{}{}{}
\reading{P2F5L1}{P2F5L1}{De Actibus Apostolórum}{Ac. 24: 10-16}{Ac}{24}{}
\reading{P2F5L2}{P2F5L2}{}{Ac. 24: 17-21}{}{}{}
\reading{P2F5L3}{P2F5L3}{}{Ac. 24: 22-27}{}{}{}
\feast{P2F6}
        {Vendredi}{Propre du Temps}{Deuxième semaine après Pâques}{3}{}
        {}{}{}{}{}
\reading{P2F6L1}{P2F6L1}{De Actibus Apostolórum}{Ac. 25: 1-5}{Ac}{25}{}
\reading{P2F6L2}{P2F6L2}{}{Ac. 25: 6-8}{}{}{}
\reading{P2F6L3}{P2F6L3}{}{Ac. 25: 9-12}{}{}{}
\feast{P2F7}
        {Samedi}{Propre du Temps}{Deuxième semaine après Pâques}{3}{}
        {}{}{}{}{}
\reading{P2F7L1}{P2F7L1}{De Actibus Apostolórum}{Ac. 28: 16-20}{Ac}{28}{}
\reading{P2F7L2}{P2F7L2}{}{Ac. 28: 21-24}{}{}{}
\reading{P2F7L3}{P2F7L3}{}{Ac. 28: 25-31}{}{}{}
\feast{P3F1}
        {Troisième dimanche après Pâques}{Propre du Temps}{Troisième semaine après Pâques}{2}{}
        {}{}{Temps Pascal!C@Troisième semaine}{}{}
\reading{P3F1N1L1}{P3F1N1L1}{Incipit liber Apocalýpsis beáti Joánnis Apóstoli}{Ap. 1: 1-6}{Ap}{1}{}
\reading{P3F1N1L2}{P3F1N1L2}{}{Ap. 1: 7-11}{}{}{}
\reading{P3F1N1L3}{P3F1N1L3}{}{Ap. 1: 12-19}{}{}{}
\feast{P3F2}
        {Lundi}{Propre du Temps}{Troisième semaine après Pâques}{3}{}
        {}{}{}{}{}
\reading{P3F2L1}{P3F2L1}{De libro Apocalýpsis beáti Joánnis Apóstoli}{Ap. 2: 1-7}{Ap}{2}{}
\reading{P3F2L2}{P3F2L2}{}{Ap. 2: 8-11}{}{}{}
\reading{P3F2L3}{P3F2L3}{}{Ap. 2: 12-17}{}{}{}
\feast{P3F3}
        {Mardi}{Propre du Temps}{Troisième semaine après Pâques}{3}{}
        {}{}{}{}{}
\reading{P3F3L1}{P3F3L1}{De libro Apocalýpsis beáti Joánnis Apóstoli}{Ap. 4: 1-5}{Ap}{4}{}
\reading{P3F3L2}{P3F3L2}{}{Ap. 4: 6-8}{}{}{}
\reading{P3F3L3}{P3F3L3}{}{Ap. 4: 9-11}{}{}{}
\feast{P3F4}
        {Mercredi}{Propre du Temps}{Troisième semaine après Pâques}{3}{}
        {}{}{}{}{}
\reading{P3F4L1}{P3F4L1}{De libro Apocalýpsis beáti Joánnis Apóstoli}{Ap. 5: 1-7}{Ap}{5}{}
\reading{P3F4L2}{P3F4L2}{}{Ap. 5: 8-10}{}{}{}
\reading{P3F4L3}{P3F4L3}{}{Ap. 5: 11-14}{}{}{}
\feast{P3F5}
        {Jeudi}{Propre du Temps}{Troisième semaine après Pâques}{3}{}
        {}{}{}{}{}
\reading{P3F5L1}{P3F5L1}{De libro Apocalýpsis beáti Joánnis Apóstoli}{Ap. 15: 1-4}{Ap}{15}{}
\reading{P3F5L2}{P3F5L2}{}{Ap. 15: 5-8}{}{}{}
\reading{P3F5L3}{P3F5L3}{}{Ap. 16: 1-6}{Ap}{16}{}
\feast{P3F6}
        {Vendredi}{Propre du Temps}{Troisième semaine après Pâques}{3}{}
        {}{}{}{}{}
\reading{P3F6L1}{P3F6L1}{De libro Apocalýpsis beáti Joánnis Apóstoli}{Ap. 19: 1-5}{Ap}{19}{}
\reading{P3F6L2}{P3F6L2}{}{Ap. 19: 6-10}{}{}{}
\reading{P3F6L3}{P3F6L3}{}{Ap. 19: 11-16}{}{}{}
\feast{P3F7}
        {Samedi}{Propre du Temps}{Troisième semaine après Pâques}{3}{}
        {}{}{}{}{}
\reading{P3F7L1}{P3F7L1}{De libro Apocalýpsis beáti Joánnis Apóstoli}{Ap. 22: 1-7}{Ap}{22}{}
\reading{P3F7L2}{P3F7L2}{}{Ap. 22: 8-12}{}{}{}
\reading{P3F7L3}{P3F7L3}{}{Ap. 22: 13-21}{}{}{}
\feast{P4F1}
        {Quatrième dimanche après Pâques}{Propre du Temps}{Quatrième semaine après Pâques}{2}{}
        {}{}{Temps Pascal!D@Quatrième semaine}{}{}
\rubric{Si ce dimanche tombe le 2 mai, et qu'on a célébré la veille la fête des saints Philippe et Jacques, apôtres,
	on emploie les lectures du lundi suivant, 3 mai, empêchées par la fête de l'Invention de la Sainte Croix.}
\reading{P4F1N1L1}{P4F1N1L1}{Incipit Epístola cathólica beáti Jacóbi Apóstoli}{Jc. 1: 1-6}{Jc}{1}{}
\reading{P4F1N1L2}{P4F1N1L2}{}{Jc. 1: 6-11}{}{}{}
\reading{P4F1N1L3}{P4F1N1L3}{}{Jc. 1: 12-16}{}{}{}
\feast{P4F2}
        {Lundi}{Propre du Temps}{Quatrième semaine après Pâques}{3}{}
        {}{}{}{}{}
\reading{P4F2L1}{P4F2L1}{De Epístola beáti Jacóbi Apóstoli}{Jc. 1: 17-20}{Jc}{1}{}
\reading{P4F2L2}{P4F2L2}{}{Jc. 1: 21-24}{}{}{}
\reading{P4F2L3}{P4F2L3}{}{Jc. 1: 25-27}{}{}{}
\feast{P4F3}
        {Mardi}{Propre du Temps}{Quatrième semaine après Pâques}{3}{}
        {}{}{}{}{}
\reading{P4F3L1}{P4F3L1}{De Epístola beáti Jacóbi Apóstoli}{Jc. 2: 1-4}{Jc}{2}{}
\reading{P4F3L2}{P4F3L2}{}{Jc. 2: 5-9}{}{}{}
\reading{P4F3L3}{P4F3L3}{}{Jc. 2: 10-13}{}{}{}
\feast{P4F4}
        {Mercredi}{Propre du Temps}{Quatrième semaine après Pâques}{3}{}
        {}{}{}{}{}
\reading{P4F4L1}{P4F4L1}{De Epístola beáti Jacóbi Apóstoli}{Jc. 2: 14-17}{Jc}{2}{}
\reading{P4F4L2}{P4F4L2}{}{Jc. 2: 18-22}{}{}{}
\reading{P4F4L3}{P4F4L3}{}{Jc. 2: 23-26}{}{}{}
\feast{P4F5}
        {Jeudi}{Propre du Temps}{Quatrième semaine après Pâques}{3}{}
        {}{}{}{}{}
\reading{P4F5L1}{P4F5L1}{De Epístola beáti Jacóbi Apóstoli}{Jc. 3: 1-3}{Jc}{3}{}
\reading{P4F5L2}{P4F5L2}{}{Jc. 3: 4-6}{}{}{}
\reading{P4F5L3}{P4F5L3}{}{Jc. 3: 6-10}{}{}{}
\feast{P4F6}
        {Vendredi}{Propre du Temps}{Quatrième semaine après Pâques}{3}{}
        {}{}{}{}{}
\reading{P4F6L1}{P4F6L1}{De Epístola beáti Jacóbi Apóstoli}{Jc. 4: 1-4}{Jc}{4}{}
\reading{P4F6L2}{P4F6L2}{}{Jc. 4: 5-10}{}{}{}
\reading{P4F6L3}{P4F6L3}{}{Jc. 4: 11-15}{}{}{}
\feast{P4F7}
        {Samedi}{Propre du Temps}{Quatrième semaine après Pâques}{3}{}
        {}{}{}{}{}
\reading{P4F7L1}{P4F7L1}{De Epístola beáti Jacóbi Apóstoli}{Jc. 5: 1-6}{Jc}{5}{}
\reading{P4F7L2}{P4F7L2}{}{Jc. 5: 7-11}{}{}{}
\reading{P4F7L3}{P4F7L3}{}{Jc. 5: 12-16}{}{}{}
\feast{P5F1}
        {Cinquième dimanche après Pâques}{Propre du Temps}{Cinquième semaine après Pâques}{2}{}
        {}{}{Temps Pascal!E@Cinquième semaine}{}{}
\reading{P5F1N1L1}{P5F1N1L1}{Incipit Epístola prima beáti Petri Apóstoli}{1 P. 1: 1-5}{1 P}{1}{}
\reading{P5F1N1L2}{P5F1N1L2}{}{1 P. 1: 6-12}{}{}{}
\reading{P5F1N1L3}{P5F1N1L3}{}{1 P. 1: 13-21}{}{}{}
\feast{P5F2}
        {Lundi des Rogations}{Propre du Temps}{Cinquième semaine après Pâques}{2}{}
        {}{}{Rogations!A@Lundi}{}{}
\rubric{Lectures à l'Homéliaire, sauf si une fête à trois nocturnes est célébrée ce jour-là.
	Dans ce cas, et si on doit y employer l'Écriture courante,
	on emploie les lectures éventuellement empêchées au cours de la semaine;
	en priorité, celles de la veille dimanche, puis celles du vendredi,
	puis celles du mardi, puis celles du samedi, tout en en conservant l'ordre
	entre le lundi des Rogations et le mercredi des Rogations, vigile de l'Ascension.}
\feast{P5F3}
        {Mardi des Rogations}{Propre du Temps}{Cinquième semaine après Pâques}{2}{}
        {}{}{Rogations!B@Mardi}{}{}
\reading{P5F3L1}{P5F3L1}{De Epístola prima beáti Petri Apóstoli}{1 P. 4: 1-7}{1 P}{4}{}
\reading{P5F3L2}{P5F3L2}{}{1 P. 4: 7-11}{}{}{}
\reading{P5F3L3}{P5F3L3}{}{1 P. 4: 12-17}{}{}{}
\feast{P5F4}
        {Mercredi des Rogations, vigile de l’Ascension}{Propre du Temps}{Cinquième semaine après Pâques}{2}{}
        {}{}{Notre-Seigneur Jésus-Christ!Ascension!Vigile}{}{}
\rubric{La rubrique du lundi des Rogations s'applique à l'identique au mercredi.}
\feast{P5F5}
        {Ascension du Seigneur}{Propre du Temps}{Ascension du Seigneur}{1}{}
        {}{}{Notre-Seigneur Jésus-Christ!Ascension}{}{}
\reading{P5F5N1L1}{P5F5N1L1}{Incipit liber Actuum Apostolórum}{Ac. 1: 1-5}{Ac}{1}{}
\reading{P5F5N1L2}{P5F5N1L2}{}{Ac. 1: 6-9}{}{}{}
\reading{P5F5N1L3}{P5F5N1L3}{}{Ac. 1: 10-14}{}{}{}
\rubric{Pendant l'Octave de l'Ascension, si on doit employer les lectures de l'Écriture courante, que ce soit pour les jours dans l'Octave ou pour les fêtes, on emploie celles qui suivent pour chaque jour.}
\feast{P5F6}
        {Vendredi dans l’Octave de l’Ascension}{Propre du Temps}{Pendant l’Octave de l’Ascension}{3}{}
        {}{}{}{}{}
\reading{P5F6N1L1}{P5F6N1L1}{Incipit Epístola secúnda beáti Petri Apóstoli}{2 P. 1: 1-4}{2 P}{1}{}
\reading{P5F6N1L2}{P5F6N1L2}{}{2 P. 1: 5-9}{}{}{}
\reading{P5F6N1L3}{P5F6N1L3}{}{2 P. 1: 10-15}{}{}{}
\feast{P5F7}
        {Samedi dans l’Octave de l’Ascension}{Propre du Temps}{Pendant l’Octave de l’Ascension}{3}{}
        {}{}{}{}{}
\reading{P5F7N1L1}{P5F7N1L1}{De Epístola secúnda beáti Petri Apóstoli}{2 P. 3: 1-7}{2 P}{3}{}
\reading{P5F7N1L2}{P5F7N1L2}{}{2 P. 3: 8-13}{}{}{}
\reading{P5F7N1L3}{P5F7N1L3}{}{2 P. 3: 14-18}{}{}{}
\feast{P6F1}
        {Dimanche dans l’Octave de l’Ascension}{Propre du Temps}{Pendant l’Octave de l’Ascension}{2}{}
        {}{}{Notre-Seigneur Jésus-Christ!Ascension!Dimanche dans l’Octave}{}{}
\reading{P6F1N1L1}{P6F1N1L1}{Incipit Epístola prima beáti Joánnis Apóstoli}{1 Jn. 1: 1-5}{1 Jn}{1}{}
\reading{P6F1N1L2}{P6F1N1L2}{}{1 Jn. 1: 6-10}{}{}{}
\reading{P6F1N1L3}{P6F1N1L3}{}{1 Jn. 2: 1-6}{1 Jn}{2}{}
\feast{P6F2}
        {Lundi dans l’Octave de l’Ascension}{Propre du Temps}{Pendant l’Octave de l’Ascension}{3}{}
        {}{}{}{}{}
\reading{P6F2N1L1}{P6F2N1L1}{De Epístola prima beáti Joánnis Apóstoli}{1 Jn. 3: 1-6}{1 Jn}{3}{}
\reading{P6F2N1L2}{P6F2N1L2}{}{1 Jn. 3: 7-12}{}{}{}
\reading{P6F2N1L3}{P6F2N1L3}{}{1 Jn. 3: 13-18}{}{}{}
\feast{P6F3}
        {Mardi dans l’Octave de l’Ascension}{Propre du Temps}{Pendant l’Octave de l’Ascension}{3}{}
        {}{}{}{}{}
\reading{P6F3N1L1}{P6F3N1L1}{De Epístola prima beáti Joánnis Apóstoli}{1 Jn. 4: 1-6}{1 Jn}{4}{}
\reading{P6F3N1L2}{P6F3N1L2}{}{1 Jn. 4: 7-14}{}{}{}
\reading{P6F3N1L3}{P6F3N1L3}{}{1 Jn. 4: 15-21}{}{}{}
\feast{P6F4}
        {Mercredi dans l’Octave de l’Ascension}{Propre du Temps}{Pendant l’Octave de l’Ascension}{3}{}
        {}{}{}{}{}
\reading{P6F4N1L1}{P6F4N1L1}{Incipit Epístola secúnda beáti Joánnis Apóstoli}{2 Jn. 1: 1-5}{2 Jn}{1}{}
\reading{P6F4N1L2}{P6F4N1L2}{}{2 Jn. 1: 6-9}{}{}{}
\reading{P6F4N1L3}{P6F4N1L3}{}{2 Jn. 1: 10-13}{}{}{}
\feast{P6F5}
        {Octave de l’Ascension}{Propre du Temps}{Pendant l’Octave de l’Ascension}{2}{}
        {}{}{Notre-Seigneur Jésus-Christ!Ascension!Octave}{}{}
\reading{P6F5N1L1}{P6F5N1L1}{De Epístola beáti Pauli Apóstoli ad Ephésios}{Ep. 4: 1-8}{Ep}{4}{}
\reading{P6F5N1L2}{P6F5N1L2}{}{Ep. 4: 9-14}{}{}{}
\reading{P6F5N1L3}{P6F5N1L3}{}{Ep. 4: 15-21}{}{}{}
\feast{P6F6}
        {Vendredi après l’Octave de l’Ascension}{Propre du Temps}{Après l’Octave de l’Ascension}{3}{}
        {}{}{}{}{}
\reading{P6F6N1L1}{P6F6N1L1}{Incipit Epístola tértia beáti Joánnis Apóstoli}{3 Jn. 1: 1-4}{3 Jn}{1}{}
\reading{P6F6N1L2}{P6F6N1L2}{}{3 Jn. 1: 5-10}{}{}{}
\reading{P6F6N1L3}{P6F6N1L3}{}{3 Jn. 1: 11-15}{}{}{}
\feast{P6F7}
        {Vigile de la Pentecôte}{Propre du Temps}{Vigile de la Pentecôte}{3}{}
        {}{}{Pentecôte!Vigile}{}{}
\reading{P6F7N1L1}{P6F7N1L1}{Incipit Epístola cathólica beáti Judæ Apóstoli}{Jd. 1: 1-4}{Jd}{1}{}
\reading{P6F7N1L2}{P6F7N1L2}{}{Jd. 1: 5-8}{}{}{}
\reading{P6F7N1L3}{P6F7N1L3}{}{Jd. 1: 9-13}{}{}{}
\feast{P7F1}
        {Pentecôte}{Propre du Temps}{Pentecôte}{1}{}
        {}{}{Pentecôte}{}{}
\rubric{À la Pentecôte et pendant son Octave, Matines à un nocturne avec lectures à l'Homéliaire.}
\feast{H1F1}
        {Fête de la Sainte Trinité}{Propre du Temps}{Fête de la Sainte Trinité}{2}{Premier dimanche après la Pentecôte}
        {}{}{Trinité}{}{}
\reading{H1F1N1L1}{H1F1N1L1}{De Isaía Prophéta}{Is. 6: 1-4}{Is}{6}{}
\reading{H1F1N1L2}{H1F1N1L2}{}{Is. 6: 5-8}{}{}{}
\reading{H1F1N1L3}{H1F1N1L3}{}{Is. 6: 9-12}{}{}{}
\feast{H1F2}
        {Lundi}{Propre du Temps}{Première semaine après la Pentecôte}{3}{}
        {}{}{}{}{}
\rubric{Si les lectures suivantes sont empêchées, elles sont transférées au mardi ou au mercredi,
	si on y emploie l'Écriture courante; sinon, elles sont omises.}
\reading{H1F2L1}{H1F2L1}{Incipit liber primus Regum}{1 S. 1: 1-3}{1 S}{1}{}
\reading{H1F2L2}{H1F2L2}{}{1 S. 1: 4-8}{}{}{}
\reading{H1F2L3}{H1F2L3}{}{1 S. 1: 9-11}{}{}{}
\feast{H1F3}
        {Mardi}{Propre du Temps}{Première semaine après la Pentecôte}{3}{}
        {}{}{}{}{}
\reading{H1F3L1}{H1F3L1}{De libro primo Regum}{1 S. 1: 12-18}{1 S}{1}{}
\reading{H1F3L2}{H1F3L2}{}{1 S. 1: 18-22}{}{}{}
\reading{H1F3L3}{H1F3L3}{}{1 S. 1: 23-28}{}{}{}
\feast{H1F4}
        {Mercredi}{Propre du Temps}{Première semaine après la Pentecôte}{3}{}
        {}{}{}{}{}
\reading{H1F4L1}{H1F4L1}{De libro primo Regum}{1 S. 2: 12-14}{1 S}{2}{}
\reading{H1F4L2}{H1F4L2}{}{1 S. 2: 15-17}{}{}{}
\reading{H1F4L3}{H1F4L3}{}{1 S. 2: 18-21}{}{}{}
\feast{H1F5}
        {Fête du Très Saint Sacrement}{Propre du Temps}{Fête du Très Saint Sacrement}{2}{Jeudi après la Trinité}
        {}{}{Notre-Seigneur Jésus-Christ!Saint-Sacrement}{}{}
\reading{H1F5N1L1}{H1F5N1L1}{De Epístola prima beáti Pauli Apóstoli ad Corínthios}{1 Co. 11: 20-22}{1 Co}{11}{}
\reading{H1F5N1L2}{H1F5N1L2}{}{1 Co. 11: 23-26}{}{}{}
\reading{H1F5N1L3}{H1F5N1L3}{}{1 Co. 11: 27-32}{}{}{}
\feast{H1F6}
        {Vendredi dans l’Octave du Très Saint Sacrement}{Propre du Temps}{Pendant l’Octave du Très Saint Sacrement}{3}{}
        {}{}{}{}{}
\reading{H1F6N1L1}{H1F6N1L1}{De libro primo Regum}{1 S. 2: 27-29}{1 S}{2}{}
\reading{H1F6N1L2}{H1F6N1L2}{}{1 S. 2: 30-33}{}{}{}
\reading{H1F6N1L3}{H1F6N1L3}{}{1 S. 2: 34-36}{}{}{}
\feast{H1F7}
        {Samedi dans l’Octave du Très Saint Sacrement}{Propre du Temps}{Pendant l’Octave du Très Saint Sacrement}{3}{}
        {}{}{}{}{}
\reading{H1F7N1L1}{H1F7N1L1}{De libro primo Regum}{1 S. 3: 1-7}{1 S}{3}{}
\reading{H1F7N1L2}{H1F7N1L2}{}{1 S. 3: 8-12}{}{}{}
\reading{H1F7N1L3}{H1F7N1L3}{}{1 S. 3: 15-20}{}{}{}
\feast{H2F1}
        {Dimanche dans l’Octave du Très Saint Sacrement}{Propre du Temps}{Pendant l’Octave du Très Saint Sacrement}{2}{}
        {}{}{}{}{}
\reading{H2F1N1L1}{H2F1N1L1}{De libro primo Regum}{1 S. 4: 1-3}{1 S}{4}{}
\reading{H2F1N1L2}{H2F1N1L2}{}{1 S. 4: 4-6}{}{}{}
\reading{H2F1N1L3}{H2F1N1L3}{}{1 S. 4: 7-11}{}{}{}
\feast{H2F2}
        {Lundi dans l’Octave du Très Saint Sacrement}{Propre du Temps}{Pendant l’Octave du Très Saint Sacrement}{3}{}
        {}{}{}{}{}
\reading{H2F2N1L1}{H2F2N1L1}{De libro primo Regum}{1 S. 5: 1-5}{1 S}{5}{}
\reading{H2F2N1L2}{H2F2N1L2}{}{1 S. 5: 6-8}{}{}{}
\reading{H2F2N1L3}{H2F2N1L3}{}{1 S. 5: 8-11}{}{}{}
\feast{H2F3}
        {Mardi dans l’Octave du Très Saint Sacrement}{Propre du Temps}{Pendant l’Octave du Très Saint Sacrement}{3}{}
        {}{}{}{}{}
\reading{H2F3N1L1}{H2F3N1L1}{De libro primo Regum}{1 S. 6: 1-3}{1 S}{6}{}
\reading{H2F3N1L2}{H2F3N1L2}{}{1 S. 6: 6-10}{}{}{}
\reading{H2F3N1L3}{H2F3N1L3}{}{1 S. 6: 12-15}{}{}{}
\feast{H2F4}
        {Mercredi dans l’Octave du Très Saint Sacrement}{Propre du Temps}{Pendant l’Octave du Très Saint Sacrement}{3}{}
        {}{}{}{}{}
\reading{H2F4N1L1}{H2F4N1L1}{De libro primo Regum}{1 S. 6: 19-21; 7: 1}{1 S}{6}{7}
\reading{H2F4N1L2}{H2F4N1L2}{}{1 S. 7: 2-4}{}{}{}
\reading{H2F4N1L3}{H2F4N1L3}{}{1 S. 7: 5-8}{}{}{}
\feast{H2F5}
        {Octave du Très Saint Sacrement}{Propre du Temps}{Pendant l’Octave du Très Saint Sacrement}{3}{}
        {}{}{}{}{}
\reading{H2F5N1L1}{H2F5N1L1}{De libro primo Regum}{1 S. 8: 4-6}{1 S}{8}{}
\reading{H2F5N1L2}{H2F5N1L2}{}{1 S. 8: 7-9}{}{}{}
\reading{H2F5N1L3}{H2F5N1L3}{}{1 S. 8: 10-14}{}{}{}
\feast{H2F6}
        {Fête du Sacré Cœur de Jésus}{Propre du Temps}{Fête du Sacré Cœur de Jésus}{2}{Vendredi après l’Octave du Saint Sacrement}
        {}{}{Notre-Seigneur Jésus-Christ!Cœur sacré}{}{}
\reading{H2F6N1L1}{H2F6N1L1}{De Jerémia Prophéta}{Jr. 24: 5-7}{Jr}{24}{}
\reading{H2F6N1L2}{H2F6N1L2}{}{Jr. 30: 18-19, 21-24}{Jr}{30}{}
\reading{H2F6N1L3}{H2F6N1L3}{}{Jr. 31: 1-3, 31-33}{Jr}{31}{}
\feast{H2F7}
        {Samedi dans l’Octave du Sacré Cœur de Jésus}{Propre du Temps}{Pendant l’Octave du Sacré Cœur de Jésus}{3}{}
        {}{}{}{}{}
\reading{H2F7N1L1}{H2F7N1L1}{De libro primo Regum}{1 S. 9: 1-4}{1 S}{9}{}
\reading{H2F7N1L2}{H2F7N1L2}{}{1 S. 9: 5-8}{}{}{}
\reading{H2F7N1L3}{H2F7N1L3}{}{1 S. 9: 14-17}{}{}{}
\feast{H3F1}
        {Dimanche dans l’Octave du Sacré Cœur de Jésus}{Propre du Temps}{Pendant l’Octave du Sacré Cœur de Jésus}{2}{}
        {}{}{}{}{}
\reading{H3F1N1L1}{H3F1N1L1}{De libro primo Regum}{1 S. 9: 18-21}{1 S}{9}{}
\reading{H3F1N1L2}{H3F1N1L2}{}{1 S. 9: 22-25}{}{}{}
\reading{H3F1N1L3}{H3F1N1L3}{}{1 S. 9: 26-27; 10: 1}{1 S}{10}{}
\feast{H3F2}
        {Lundi dans l’Octave du Sacré Cœur de Jésus}{Propre du Temps}{Pendant l’Octave du Sacré Cœur de Jésus}{3}{}
        {}{}{}{}{}
\reading{H3F2N1L1}{H3F2N1L1}{De libro primo Regum}{1 S. 10: 17-19}{1 S}{10}{}
\reading{H3F2N1L2}{H3F2N1L2}{}{1 S. 10: 20-24}{}{}{}
\reading{H3F2N1L3}{H3F2N1L3}{}{1 S. 10: 25-27}{}{}{}
\feast{H3F3}
        {Mardi dans l’Octave du Sacré Cœur de Jésus}{Propre du Temps}{Pendant l’Octave du Sacré Cœur de Jésus}{3}{}
        {}{}{}{}{}
\reading{H3F3N1L1}{H3F3N1L1}{De libro primo Regum}{1 S. 12: 1-5}{1 S}{12}{}
\reading{H3F3N1L2}{H3F3N1L2}{}{1 S. 12: 6-9}{}{}{}
\reading{H3F3N1L3}{H3F3N1L3}{}{1 S. 12: 10-14}{}{}{}
\feast{H3F4}
        {Mercredi dans l’Octave du Sacré Cœur de Jésus}{Propre du Temps}{Pendant l’Octave du Sacré Cœur de Jésus}{3}{}
        {}{}{}{}{}
\reading{H3F4N1L1}{H3F4N1L1}{De libro primo Regum}{1 S. 13: 1-4}{1 S}{13}{}
\reading{H3F4N1L2}{H3F4N1L2}{}{1 S. 13: 5-8}{}{}{}
\reading{H3F4N1L3}{H3F4N1L3}{}{1 S. 13: 9-14}{}{}{}
\feast{H3F5}
        {Jeudi dans l’Octave du Sacré Cœur de Jésus}{Propre du Temps}{Pendant l’Octave du Sacré Cœur de Jésus}{3}{}
        {}{}{}{}{}
\reading{H3F5N1L1}{H3F5N1L1}{De libro primo Regum}{1 S. 14: 6-11}{1 S}{14}{}
\reading{H3F5N1L2}{H3F5N1L2}{}{1 S. 14: 12-15}{}{}{}
\reading{H3F5N1L3}{H3F5N1L3}{}{1 S. 14: 16-20}{}{}{}
\feast{H3F6}
        {Octave du Sacré Cœur de Jésus}{Propre du Temps}{Pendant l’Octave du Sacré Cœur de Jésus}{3}{}
        {}{}{}{}{}
\reading{H3F6N1L1}{H3F6N1L1}{De libro primo Regum}{1 S. 15: 1-3}{1 S}{15}{}
\reading{H3F6N1L2}{H3F6N1L2}{}{1 S. 15: 4-8}{}{}{}
\reading{H3F6N1L3}{H3F6N1L3}{}{1 S. 15: 9-11}{}{}{}
\feast{H3F7}
        {Sabbato post Octavam Ss. Cordis Jesu}{Propre du Temps}{Troisième semaine après la Pentecôte}{3}{}
        {}{}{}{}{}
\reading{H3F7L1}{H3F7L1}{De libro primo Regum}{1 S. 16: 1-3}{1 S}{16}{}
\reading{H3F7L2}{H3F7L2}{}{1 S. 16: 4-7}{}{}{}
\reading{H3F7L3}{H3F7L3}{}{1 S. 16: 8-11}{}{}{}
\feast{H4F1}
        {Quatrième dimanche après la Pentecôte}{Propre du Temps}{Quatrième semaine après la Pentecôte}{2}{}
        {}{}{}{}{}
\reading{H4F1N1L1}{H4F1N1L1}{De libro primo Regum}{1 S. 17: 1-7}{1 S}{17}{}
\reading{H4F1N1L2}{H4F1N1L2}{}{1 S. 17: 8-11}{}{}{}
\reading{H4F1N1L3}{H4F1N1L3}{}{1 S. 17: 12-16}{}{}{}
\feast{H4F2}
        {Lundi}{Propre du Temps}{Quatrième semaine après la Pentecôte}{3}{}
        {}{}{}{}{}
\reading{H4F2L1}{H4F2L1}{De libro primo Regum}{1 S. 17: 25-26}{1 S}{17}{}
\reading{H4F2L2}{H4F2L2}{}{1 S. 17: 31-33}{}{}{}
\reading{H4F2L3}{H4F2L3}{}{1 S. 17: 34-36}{}{}{}
\feast{H4F3}
        {Mardi}{Propre du Temps}{Quatrième semaine après la Pentecôte}{3}{}
        {}{}{}{}{}
\reading{H4F3L1}{H4F3L1}{De libro primo Regum}{1 S. 17: 38-40}{1 S}{17}{}
\reading{H4F3L2}{H4F3L2}{}{1 S. 17: 41-46}{}{}{}
\reading{H4F3L3}{H4F3L3}{}{1 S. 17: 48-51}{}{}{}
\feast{H4F4}
        {Mercredi}{Propre du Temps}{Quatrième semaine après la Pentecôte}{3}{}
        {}{}{}{}{}
\reading{H4F4L1}{H4F4L1}{De libro primo Regum}{1 S. 18: 6-8}{1 S}{18}{}
\reading{H4F4L2}{H4F4L2}{}{1 S. 18: 9-13}{}{}{}
\reading{H4F4L3}{H4F4L3}{}{1 S. 18: 14-17}{}{}{}
\feast{H4F5}
        {Jeudi}{Propre du Temps}{Quatrième semaine après la Pentecôte}{3}{}
        {}{}{}{}{}
\reading{H4F5L1}{H4F5L1}{De libro primo Regum}{1 S. 19: 1-3}{1 S}{19}{}
\reading{H4F5L2}{H4F5L2}{}{1 S. 19: 4-6}{}{}{}
\reading{H4F5L3}{H4F5L3}{}{1 S. 19: 8-10}{}{}{}
\feast{H4F6}
        {Vendredi}{Propre du Temps}{Quatrième semaine après la Pentecôte}{3}{}
        {}{}{}{}{}
\reading{H4F6L1}{H4F6L1}{De libro primo Regum}{1 S. 20: 1-2}{1 S}{20}{}
\reading{H4F6L2}{H4F6L2}{}{1 S. 20: 3-4}{}{}{}
\reading{H4F6L3}{H4F6L3}{}{1 S. 20: 5-7}{}{}{}
\feast{H4F7}
        {Samedi}{Propre du Temps}{Quatrième semaine après la Pentecôte}{3}{}
        {}{}{}{}{}
\reading{H4F7L1}{H4F7L1}{De libro primo Regum}{1 S. 21: 2-4}{1 S}{21}{}
\reading{H4F7L2}{H4F7L2}{}{1 S. 21: 5-7}{}{}{}
\reading{H4F7L3}{H4F7L3}{}{1 S. 21: 7-9}{}{}{}
\feast{H5F1}
        {Cinquième dimanche après la Pentecôte}{Propre du Temps}{Cinquième semaine après la Pentecôte}{2}{}
        {}{}{}{}{}
\reading{H5F1N1L1}{H5F1N1L1}{Incipit liber secúndus Regum}{2 S. 1: 1-4}{2 S}{1}{}
\reading{H5F1N1L2}{H5F1N1L2}{}{2 S. 1: 5-10}{}{}{}
\reading{H5F1N1L3}{H5F1N1L3}{}{2 S. 1: 11-15}{}{}{}
\feast{H5F2}
        {Lundi}{Propre du Temps}{Cinquième semaine après la Pentecôte}{3}{}
        {}{}{}{}{}
\reading{H5F2L1}{H5F2L1}{De libro secúndo Regum}{2 S. 2: 1-4}{2 S}{2}{}
\reading{H5F2L2}{H5F2L2}{}{2 S. 2: 4-7}{}{}{}
\reading{H5F2L3}{H5F2L3}{}{2 S. 2: 8-11}{}{}{}
\feast{H5F3}
        {Mardi}{Propre du Temps}{Cinquième semaine après la Pentecôte}{3}{}
        {}{}{}{}{}
\reading{H5F3L1}{H5F3L1}{De libro secúndo Regum}{2 S. 3: 6-10}{2 S}{3}{}
\reading{H5F3L2}{H5F3L2}{}{2 S. 3: 12-16}{}{}{}
\reading{H5F3L3}{H5F3L3}{}{2 S. 3: 17-21}{}{}{}
\feast{H5F4}
        {Mercredi}{Propre du Temps}{Cinquième semaine après la Pentecôte}{3}{}
        {}{}{}{}{}
\reading{H5F4L1}{H5F4L1}{De libro secúndo Regum}{2 S. 4: 5-8}{2 S}{4}{}
\reading{H5F4L2}{H5F4L2}{}{2 S. 4: 9-12}{}{}{}
\reading{H5F4L3}{H5F4L3}{}{2 S. 5: 1-7}{2 S}{5}{}
\feast{H5F5}
        {Jeudi}{Propre du Temps}{Cinquième semaine après la Pentecôte}{3}{}
        {}{}{}{}{}
\reading{H5F5L1}{H5F5L1}{De libro secúndo Regum}{2 S. 6: 1-3}{2 S}{6}{}
\reading{H5F5L2}{H5F5L2}{}{2 S. 6: 4-7}{}{}{}
\reading{H5F5L3}{H5F5L3}{}{2 S. 6: 8-12}{}{}{}
\feast{H5F6}
        {Vendredi}{Propre du Temps}{Cinquième semaine après la Pentecôte}{3}{}
        {}{}{}{}{}
\reading{H5F6L1}{H5F6L1}{De libro secúndo Regum}{2 S. 7: 4-6}{2 S}{7}{}
\reading{H5F6L2}{H5F6L2}{}{2 S. 7: 7-11}{}{}{}
\reading{H5F6L3}{H5F6L3}{}{2 S. 7: 12-17}{}{}{}
\feast{H5F7}
        {Samedi}{Propre du Temps}{Cinquième semaine après la Pentecôte}{3}{}
        {}{}{}{}{}
\reading{H5F7L1}{H5F7L1}{De libro secúndo Regum}{2 S. 11: 1-4}{2 S}{11}{}
\reading{H5F7L2}{H5F7L2}{}{2 S. 11: 5-11}{}{}{}
\reading{H5F7L3}{H5F7L3}{}{2 S. 11: 12-17}{}{}{}
\feast{H6F1}
        {Sixième dimanche après la Pentecôte}{Propre du Temps}{Sixième semaine après la Pentecôte}{2}{}
        {}{}{}{}{}
\reading{H6F1N1L1}{H6F1N1L1}{De libro secúndo Regum}{2 S. 12: 1-4}{2 S}{12}{}
\reading{H6F1N1L2}{H6F1N1L2}{}{2 S. 12: 5-9}{}{}{}
\reading{H6F1N1L3}{H6F1N1L3}{}{2 S. 12: 10-16}{}{}{}
\feast{H6F2}
        {Lundi}{Propre du Temps}{Sixième semaine après la Pentecôte}{3}{}
        {}{}{}{}{}
\reading{H6F2L1}{H6F2L1}{De libro secúndo Regum}{2 S. 13: 22-25}{2 S}{13}{}
\reading{H6F2L2}{H6F2L2}{}{2 S. 13: 26-29}{}{}{}
\reading{H6F2L3}{H6F2L3}{}{2 S. 13: 30-34}{}{}{}
\feast{H6F3}
        {Mardi}{Propre du Temps}{Sixième semaine après la Pentecôte}{3}{}
        {}{}{}{}{}
\reading{H6F3L1}{H6F3L1}{De libro secúndo Regum}{2 S. 14: 4-7}{2 S}{14}{}
\reading{H6F3L2}{H6F3L2}{}{2 S. 14: 10-14}{}{}{}
\reading{H6F3L3}{H6F3L3}{}{2 S. 14: 19-21}{}{}{}
\feast{H6F4}
        {Mercredi}{Propre du Temps}{Sixième semaine après la Pentecôte}{3}{}
        {}{}{}{}{}
\reading{H6F4L1}{H6F4L1}{De libro secúndo Regum}{2 S. 15: 1-3}{2 S}{15}{}
\reading{H6F4L2}{H6F4L2}{}{2 S. 15: 4-6}{}{}{}
\reading{H6F4L3}{H6F4L3}{}{2 S. 15: 7-10}{}{}{}
\feast{H6F5}
        {Jeudi}{Propre du Temps}{Sixième semaine après la Pentecôte}{3}{}
        {}{}{}{}{}
\reading{H6F5L1}{H6F5L1}{De libro secúndo Regum}{2 S. 15: 13-15}{2 S}{15}{}
\reading{H6F5L2}{H6F5L2}{}{2 S. 15: 16-18}{}{}{}
\reading{H6F5L3}{H6F5L3}{}{2 S. 15: 19-20}{}{}{}
\feast{H6F6}
        {Vendredi}{Propre du Temps}{Sixième semaine après la Pentecôte}{3}{}
        {}{}{}{}{}
\reading{H6F6L1}{H6F6L1}{De libro secúndo Regum}{2 S. 16: 5-8}{2 S}{16}{}
\reading{H6F6L2}{H6F6L2}{}{2 S. 16: 9-10}{}{}{}
\reading{H6F6L3}{H6F6L3}{}{2 S. 16: 11-12}{}{}{}
\feast{H6F7}
        {Samedi}{Propre du Temps}{Sixième semaine après la Pentecôte}{3}{}
        {}{}{}{}{}
\reading{H6F7L1}{H6F7L1}{De libro secúndo Regum}{2 S. 18: 6-8}{2 S}{18}{}
\reading{H6F7L2}{H6F7L2}{}{2 S. 18: 9-12}{}{}{}
\reading{H6F7L3}{H6F7L3}{}{2 S. 18: 14-17}{}{}{}
\feast{H7F1}
        {Septième dimanche après la Pentecôte}{Propre du Temps}{Septième semaine après la Pentecôte}{2}{}
        {}{}{}{}{}
\reading{H7F1N1L1}{H7F1N1L1}{Incipit liber tértius Regum}{1 R. 1: 1-4}{1 R}{1}{}
\reading{H7F1N1L2}{H7F1N1L2}{}{1 R. 1: 5-8}{}{}{}
\reading{H7F1N1L3}{H7F1N1L3}{}{1 R. 1: 11-15}{}{}{}
\feast{H7F2}
        {Lundi}{Propre du Temps}{Septième semaine après la Pentecôte}{3}{}
        {}{}{}{}{}
\reading{H7F2L1}{H7F2L1}{De libro tértio Regum}{1 R. 1: 28-31}{1 R}{1}{}
\reading{H7F2L2}{H7F2L2}{}{1 R. 1: 32-35}{}{}{}
\reading{H7F2L3}{H7F2L3}{}{1 R. 1: 38-40}{}{}{}
\feast{H7F3}
        {Mardi}{Propre du Temps}{Septième semaine après la Pentecôte}{3}{}
        {}{}{}{}{}
\reading{H7F3L1}{H7F3L1}{De libro tértio Regum}{1 R. 2: 1-4}{1 R}{2}{}
\reading{H7F3L2}{H7F3L2}{}{1 R. 2: 5-6}{}{}{}
\reading{H7F3L3}{H7F3L3}{}{1 R. 2: 7-9}{}{}{}
\feast{H7F4}
        {Mercredi}{Propre du Temps}{Septième semaine après la Pentecôte}{3}{}
        {}{}{}{}{}
\reading{H7F4L1}{H7F4L1}{De libro tértio Regum}{1 R. 3: 5-6}{1 R}{3}{}
\reading{H7F4L2}{H7F4L2}{}{1 R. 3: 7-9}{}{}{}
\reading{H7F4L3}{H7F4L3}{}{1 R. 3: 10-13}{}{}{}
\feast{H7F5}
        {Jeudi}{Propre du Temps}{Septième semaine après la Pentecôte}{3}{}
        {}{}{}{}{}
\reading{H7F5L1}{H7F5L1}{De libro tértio Regum}{1 R. 5: 1-4}{1 R}{5}{}
\reading{H7F5L2}{H7F5L2}{}{1 R. 5: 5-9}{}{}{}
\reading{H7F5L3}{H7F5L3}{}{1 R. 5: 10-14}{}{}{}
\feast{H7F6}
        {Vendredi}{Propre du Temps}{Septième semaine après la Pentecôte}{3}{}
        {}{}{}{}{}
\reading{H7F6L1}{H7F6L1}{De libro tértio Regum}{1 R. 5: 15-18}{1 R}{5}{}
\reading{H7F6L2}{H7F6L2}{}{1 R. 5: 19-20}{}{}{}
\reading{H7F6L3}{H7F6L3}{}{1 R. 5: 21-23}{}{}{}
\feast{H7F7}
        {Samedi}{Propre du Temps}{Septième semaine après la Pentecôte}{3}{}
        {}{}{}{}{}
\reading{H7F7L1}{H7F7L1}{De libro tértio Regum}{1 R. 7: 51; 8: 1-2}{1 R}{7}{8}
\reading{H7F7L2}{H7F7L2}{}{1 R. 8: 3-7}{}{}{}
\reading{H7F7L3}{H7F7L3}{}{1 R. 8: 9-12}{}{}{}
\feast{H8F1}
        {Huitième dimanche après la Pentecôte}{Propre du Temps}{Huitième semaine après la Pentecôte}{2}{}
        {}{}{}{}{}
\reading{H8F1N1L1}{H8F1N1L1}{De libro tértio Regum}{1 R. 9: 1-5}{1 R}{9}{}
\reading{H8F1N1L2}{H8F1N1L2}{}{1 R. 9: 6-9}{}{}{}
\reading{H8F1N1L3}{H8F1N1L3}{}{1 R. 9: 10-14}{}{}{}
\feast{H8F2}
        {Lundi}{Propre du Temps}{Huitième semaine après la Pentecôte}{3}{}
        {}{}{}{}{}
\reading{H8F2L1}{H8F2L1}{De libro tértio Regum}{1 R. 10: 1-3}{1 R}{10}{}
\reading{H8F2L2}{H8F2L2}{}{1 R. 10: 4-7}{}{}{}
\reading{H8F2L3}{H8F2L3}{}{1 R. 10: 8-11}{}{}{}
\feast{H8F3}
        {Mardi}{Propre du Temps}{Huitième semaine après la Pentecôte}{3}{}
        {}{}{}{}{}
\reading{H8F3L1}{H8F3L1}{De libro tértio Regum}{1 R. 11: 1-4}{1 R}{11}{}
\reading{H8F3L2}{H8F3L2}{}{1 R. 11: 5-8}{}{}{}
\reading{H8F3L3}{H8F3L3}{}{1 R. 11: 9-12}{}{}{}
\feast{H8F4}
        {Mercredi}{Propre du Temps}{Huitième semaine après la Pentecôte}{3}{}
        {}{}{}{}{}
\reading{H8F4L1}{H8F4L1}{De libro tértio Regum}{1 R. 11: 26-28}{1 R}{11}{}
\reading{H8F4L2}{H8F4L2}{}{1 R. 11: 29-31}{}{}{}
\reading{H8F4L3}{H8F4L3}{}{1 R. 11: 40-43}{}{}{}
\feast{H8F5}
        {Jeudi}{Propre du Temps}{Huitième semaine après la Pentecôte}{3}{}
        {}{}{}{}{}
\reading{H8F5L1}{H8F5L1}{De libro tértio Regum}{1 R. 12: 1-5}{1 R}{12}{}
\reading{H8F5L2}{H8F5L2}{}{1 R. 12: 6-8}{}{}{}
\reading{H8F5L3}{H8F5L3}{}{1 R. 12: 13-16}{}{}{}
\feast{H8F6}
        {Vendredi}{Propre du Temps}{Huitième semaine après la Pentecôte}{3}{}
        {}{}{}{}{}
\reading{H8F6L1}{H8F6L1}{De libro tértio Regum}{1 R. 14: 5-6}{1 R}{14}{}
\reading{H8F6L2}{H8F6L2}{}{1 R. 14: 7-9}{}{}{}
\reading{H8F6L3}{H8F6L3}{}{1 R. 14: 10-12}{}{}{}
\feast{H8F7}
        {Samedi}{Propre du Temps}{Huitième semaine après la Pentecôte}{3}{}
        {}{}{}{}{}
\reading{H8F7L1}{H8F7L1}{De libro tértio Regum}{1 R. 18: 21-22}{1 R}{18}{}
\reading{H8F7L2}{H8F7L2}{}{1 R. 18: 23-24}{}{}{}
\reading{H8F7L3}{H8F7L3}{}{1 R. 18: 25-27}{}{}{}
\feast{H9F1}
        {Neuvième dimanche après la Pentecôte}{Propre du Temps}{Neuvième semaine après la Pentecôte}{2}{}
        {}{}{}{}{}
\reading{H9F1N1L1}{H9F1N1L1}{Incipit liber quartus Regum}{2 R. 1: 1-4}{2 R}{1}{}
\reading{H9F1N1L2}{H9F1N1L2}{}{2 R. 1: 4-6}{}{}{}
\reading{H9F1N1L3}{H9F1N1L3}{}{2 R. 1: 7-10}{}{}{}
\feast{H9F2}
        {Lundi}{Propre du Temps}{Neuvième semaine après la Pentecôte}{3}{}
        {}{}{}{}{}
\reading{H9F2L1}{H9F2L1}{De libro quarto Regum}{2 R. 2: 5-7}{2 R}{2}{}
\reading{H9F2L2}{H9F2L2}{}{2 R. 2: 8-10}{}{}{}
\reading{H9F2L3}{H9F2L3}{}{2 R. 2: 11-13}{}{}{}
\feast{H9F3}
        {Mardi}{Propre du Temps}{Neuvième semaine après la Pentecôte}{3}{}
        {}{}{}{}{}
\reading{H9F3L1}{H9F3L1}{De libro quarto Regum}{2 R. 3: 6-9}{2 R}{3}{}
\reading{H9F3L2}{H9F3L2}{}{2 R. 3: 10-13}{}{}{}
\reading{H9F3L3}{H9F3L3}{}{2 R. 3: 13-18}{}{}{}
\feast{H9F4}
        {Mercredi}{Propre du Temps}{Neuvième semaine après la Pentecôte}{3}{}
        {}{}{}{}{}
\reading{H9F4L1}{H9F4L1}{De libro quarto Regum}{2 R. 4: 1-4}{2 R}{4}{}
\reading{H9F4L2}{H9F4L2}{}{2 R. 4: 5-10}{}{}{}
\reading{H9F4L3}{H9F4L3}{}{2 R. 4: 11-17}{}{}{}
\feast{H9F5}
        {Jeudi}{Propre du Temps}{Neuvième semaine après la Pentecôte}{3}{}
        {}{}{}{}{}
\reading{H9F5L1}{H9F5L1}{De libro quarto Regum}{2 R. 6: 24-27}{2 R}{6}{}
\reading{H9F5L2}{H9F5L2}{}{2 R. 6: 27-32}{}{}{}
\reading{H9F5L3}{H9F5L3}{}{2 R. 6: 32-33; 7: 1}{2 R}{7}{}
\feast{H9F6}
        {Vendredi}{Propre du Temps}{Neuvième semaine après la Pentecôte}{3}{}
        {}{}{}{}{}
\reading{H9F6L1}{H9F6L1}{De libro quarto Regum}{2 R. 8: 1-3}{2 R}{8}{}
\reading{H9F6L2}{H9F6L2}{}{2 R. 8: 4-6}{}{}{}
\reading{H9F6L3}{H9F6L3}{}{2 R. 8: 7-10}{}{}{}
\feast{H9F7}
        {Samedi}{Propre du Temps}{Neuvième semaine après la Pentecôte}{3}{}
        {}{}{}{}{}
\reading{H9F7L1}{H9F7L1}{De libro quarto Regum}{2 R. 9: 1-5}{2 R}{9}{}
\reading{H9F7L2}{H9F7L2}{}{2 R. 9: 6-10}{}{}{}
\reading{H9F7L3}{H9F7L3}{}{2 R. 9: 11-13}{}{}{}
\feast{H10F1}
        {Dixième dimanche après la Pentecôte}{Propre du Temps}{Dixième semaine après la Pentecôte}{2}{}
        {}{}{}{}{}
\reading{H10F1N1L1}{H10F1N1L1}{De libro quarto Regum}{2 R. 9: 29-34}{2 R}{9}{}
\reading{H10F1N1L2}{H10F1N1L2}{}{2 R. 9: 35-37; 10: 1-3}{2 R}{10}{}
\reading{H10F1N1L3}{H10F1N1L3}{}{2 R. 10: 4-7}{}{}{}
\feast{H10F2}
        {Lundi}{Propre du Temps}{Dixième semaine après la Pentecôte}{3}{}
        {}{}{}{}{}
\reading{H10F2L1}{H10F2L1}{De libro quarto Regum}{2 R. 11: 1-3}{2 R}{11}{}
\reading{H10F2L2}{H10F2L2}{}{2 R. 11: 4-7}{}{}{}
\reading{H10F2L3}{H10F2L3}{}{2 R. 11: 9-12}{}{}{}
\feast{H10F3}
        {Mardi}{Propre du Temps}{Dixième semaine après la Pentecôte}{3}{}
        {}{}{}{}{}
\reading{H10F3L1}{H10F3L1}{De libro quarto Regum}{2 R. 12: 1-4}{2 R}{12}{}
\reading{H10F3L2}{H10F3L2}{}{2 R. 12: 5-6}{}{}{}
\reading{H10F3L3}{H10F3L3}{}{2 R. 12: 7-9}{}{}{}
\feast{H10F4}
        {Mercredi}{Propre du Temps}{Dixième semaine après la Pentecôte}{3}{}
        {}{}{}{}{}
\reading{H10F4L1}{H10F4L1}{De libro quarto Regum}{2 R. 13: 14-17}{2 R}{13}{}
\reading{H10F4L2}{H10F4L2}{}{2 R. 13: 18-20}{}{}{}
\reading{H10F4L3}{H10F4L3}{}{2 R. 13: 21, 24-25}{}{}{}
\feast{H10F5}
        {Jeudi}{Propre du Temps}{Dixième semaine après la Pentecôte}{3}{}
        {}{}{}{}{}
\reading{H10F5L1}{H10F5L1}{De libro quarto Regum}{2 R. 17: 6-9}{2 R}{17}{}
\reading{H10F5L2}{H10F5L2}{}{2 R. 17: 13-15}{}{}{}
\reading{H10F5L3}{H10F5L3}{}{2 R. 17: 18-21}{}{}{}
\feast{H10F6}
        {Vendredi}{Propre du Temps}{Dixième semaine après la Pentecôte}{3}{}
        {}{}{}{}{}
\reading{H10F6L1}{H10F6L1}{De libro quarto Regum}{2 R. 17: 21-23}{2 R}{17}{}
\reading{H10F6L2}{H10F6L2}{}{2 R. 17: 24-25}{}{}{}
\reading{H10F6L3}{H10F6L3}{}{2 R. 17: 26-27}{}{}{}
\feast{H10F7}
        {Samedi}{Propre du Temps}{Dixième semaine après la Pentecôte}{3}{}
        {}{}{}{}{}
\reading{H10F7L1}{H10F7L1}{De libro quarto Regum}{2 R. 18: 1-5}{2 R}{18}{}
\reading{H10F7L2}{H10F7L2}{}{2 R. 18: 5-8}{}{}{}
\reading{H10F7L3}{H10F7L3}{}{2 R. 18: 9-12}{}{}{}
\feast{H11F1}
        {Onzième dimanche après la Pentecôte}{Propre du Temps}{Onzième semaine après la Pentecôte}{2}{}
        {}{}{}{}{}
\reading{H11F1N1L1}{H11F1N1L1}{De libro quarto Regum}{2 R. 20: 1-3}{2 R}{20}{}
\reading{H11F1N1L2}{H11F1N1L2}{}{2 R. 20: 4-7}{}{}{}
\reading{H11F1N1L3}{H11F1N1L3}{}{2 R. 20: 8-11}{}{}{}
\feast{H11F2}
        {Lundi}{Propre du Temps}{Onzième semaine après la Pentecôte}{3}{}
        {}{}{}{}{}
\reading{H11F2L1}{H11F2L1}{De libro quarto Regum}{2 R. 22: 1-5}{2 R}{22}{}
\reading{H11F2L2}{H11F2L2}{}{2 R. 22: 8-10}{}{}{}
\reading{H11F2L3}{H11F2L3}{}{2 R. 22: 10-13}{}{}{}
\feast{H11F3}
        {Mardi}{Propre du Temps}{Onzième semaine après la Pentecôte}{3}{}
        {}{}{}{}{}
\reading{H11F3L1}{H11F3L1}{De libro quarto Regum}{2 R. 23: 2-3}{2 R}{23}{}
\reading{H11F3L2}{H11F3L2}{}{2 R. 23: 4-5}{}{}{}
\reading{H11F3L3}{H11F3L3}{}{2 R. 23: 6-8}{}{}{}
\feast{H11F4}
        {Mercredi}{Propre du Temps}{Onzième semaine après la Pentecôte}{3}{}
        {}{}{}{}{}
\reading{H11F4L1}{H11F4L1}{De libro quarto Regum}{2 R. 23: 24-26}{2 R}{23}{}
\reading{H11F4L2}{H11F4L2}{}{2 R. 23: 27-30}{}{}{}
\reading{H11F4L3}{H11F4L3}{}{2 R. 23: 30-34}{}{}{}
\feast{H11F5}
        {Jeudi}{Propre du Temps}{Onzième semaine après la Pentecôte}{3}{}
        {}{}{}{}{}
\reading{H11F5L1}{H11F5L1}{De libro quarto Regum}{2 R. 23: 36-37; 24: 1}{2 R}{23}{24}
\reading{H11F5L2}{H11F5L2}{}{2 R. 24: 2-4}{}{}{}
\reading{H11F5L3}{H11F5L3}{}{2 R. 24: 5-7}{}{}{}
\feast{H11F6}
        {Vendredi}{Propre du Temps}{Onzième semaine après la Pentecôte}{3}{}
        {}{}{}{}{}
\reading{H11F6L1}{H11F6L1}{De libro quarto Regum}{2 R. 24: 8-11}{2 R}{24}{}
\reading{H11F6L2}{H11F6L2}{}{2 R. 24: 12-14}{}{}{}
\reading{H11F6L3}{H11F6L3}{}{2 R. 24: 15-17}{}{}{}
\feast{H11F7}
        {Samedi}{Propre du Temps}{Onzième semaine après la Pentecôte}{3}{}
        {}{}{}{}{}
\reading{H11F7L1}{H11F7L1}{De libro quarto Regum}{2 R. 24: 18-20; 25: 1-3}{2 R}{24}{25}
\reading{H11F7L2}{H11F7L2}{}{2 R. 25: 4-7}{}{}{}
\reading{H11F7L3}{H11F7L3}{}{2 R. 25: 8-13}{}{}{}
\rubric{Selon les rubriques que l'on suit, le premier dimanche d'août peut être celui qui tombe entre le 1\ier et le 7 de ce mois, ou bien celui qui tombe entre le 29 juillet et le 4 août; et il en est ainsi pour les mois suivants, jusqu'à novembre inclus.}
\feast{08H1F1}
        {Premier dimanche d’août}{Propre du Temps}{Première semaine d’août}{2}{}
        {}{}{}{}{}
\reading{08H1F1N1L1}{08H1F1N1L1}{Incípiunt Parábolæ Salomónis}{Pr. 1: 1-6}{Pr}{1}{}
\reading{08H1F1N1L2}{08H1F1N1L2}{}{Pr. 1: 7-14}{}{}{}
\reading{08H1F1N1L3}{08H1F1N1L3}{}{Pr. 1: 15-19}{}{}{}
\feast{08H1F2}
        {Lundi}{Propre du Temps}{Première semaine d’août}{3}{}
        {}{}{}{}{}
\reading{08H1F2L1}{08H1F2L1}{De Parábolis Salomónis}{Pr. 3: 1-6}{Pr}{3}{}
\reading{08H1F2L2}{08H1F2L2}{}{Pr. 3: 7-10}{}{}{}
\reading{08H1F2L3}{08H1F2L3}{}{Pr. 3: 11-15}{}{}{}
\feast{08H1F3}
        {Mardi}{Propre du Temps}{Première semaine d’août}{3}{}
        {}{}{}{}{}
\reading{08H1F3L1}{08H1F3L1}{De Parábolis Salomónis}{Pr. 5: 1-6}{Pr}{5}{}
\reading{08H1F3L2}{08H1F3L2}{}{Pr. 5: 7-13}{}{}{}
\reading{08H1F3L3}{08H1F3L3}{}{Pr. 5: 20-23}{}{}{}
\feast{08H1F4}
        {Mercredi}{Propre du Temps}{Première semaine d’août}{3}{}
        {}{}{}{}{}
\reading{08H1F4L1}{08H1F4L1}{De Parábolis Salomónis}{Pr. 8: 1-6}{Pr}{8}{}
\reading{08H1F4L2}{08H1F4L2}{}{Pr. 8: 7-11}{}{}{}
\reading{08H1F4L3}{08H1F4L3}{}{Pr. 8: 12-17}{}{}{}
\feast{08H1F5}
        {Jeudi}{Propre du Temps}{Première semaine d’août}{3}{}
        {}{}{}{}{}
\reading{08H1F5L1}{08H1F5L1}{De Parábolis Salomónis}{Pr. 10: 1-5}{Pr}{10}{}
\reading{08H1F5L2}{08H1F5L2}{}{Pr. 10: 6-10}{}{}{}
\reading{08H1F5L3}{08H1F5L3}{}{Pr. 10: 11-16}{}{}{}
\feast{08H1F6}
        {Vendredi}{Propre du Temps}{Première semaine d’août}{3}{}
        {}{}{}{}{}
\reading{08H1F6L1}{08H1F6L1}{De Parábolis Salomónis}{Pr. 14: 1-5}{Pr}{14}{}
\reading{08H1F6L2}{08H1F6L2}{}{Pr. 14: 6-11}{}{}{}
\reading{08H1F6L3}{08H1F6L3}{}{Pr. 14: 12-16}{}{}{}
\feast{08H1F7}
        {Samedi}{Propre du Temps}{Première semaine d’août}{3}{}
        {}{}{}{}{}
\reading{08H1F7L1}{08H1F7L1}{De Parábolis Salomónis}{Pr. 16: 1-5}{Pr}{16}{}
\reading{08H1F7L2}{08H1F7L2}{}{Pr. 16: 5-9}{}{}{}
\reading{08H1F7L3}{08H1F7L3}{}{Pr. 16: 10-15}{}{}{}
\feast{08H2F1}
        {Deuxième dimanche d’août}{Propre du Temps}{Deuxième semaine d’août}{2}{}
        {}{}{}{}{}
\reading{08H2F1N1L1}{08H2F1N1L1}{Incipit liber Ecclesiástes}{Qo. 1: 1-7}{Qo}{1}{}
\reading{08H2F1N1L2}{08H2F1N1L2}{}{Qo. 1: 8-11}{}{}{}
\reading{08H2F1N1L3}{08H2F1N1L3}{}{Qo. 1: 12-17}{}{}{}
\feast{08H2F2}
        {Lundi}{Propre du Temps}{Deuxième semaine d’août}{3}{}
        {}{}{}{}{}
\reading{08H2F2L1}{08H2F2L1}{De libro Ecclesiástæ}{Qo. 2: 1-4}{Qo}{2}{}
\reading{08H2F2L2}{08H2F2L2}{}{Qo. 2: 7-9}{}{}{}
\reading{08H2F2L3}{08H2F2L3}{}{Qo. 2: 10-11}{}{}{}
\feast{08H2F3}
        {Mardi}{Propre du Temps}{Deuxième semaine d’août}{3}{}
        {}{}{}{}{}
\reading{08H2F3L1}{08H2F3L1}{De libro Ecclesiástæ}{Qo. 3: 1-8}{Qo}{3}{}
\reading{08H2F3L2}{08H2F3L2}{}{Qo. 3: 9-13}{}{}{}
\reading{08H2F3L3}{08H2F3L3}{}{Qo. 3: 14-17}{}{}{}
\feast{08H2F4}
        {Mercredi}{Propre du Temps}{Deuxième semaine d’août}{3}{}
        {}{}{}{}{}
\reading{08H2F4L1}{08H2F4L1}{De libro Ecclesiástæ}{Qo. 4: 1-4}{Qo}{4}{}
\reading{08H2F4L2}{08H2F4L2}{}{Qo. 4: 5-8}{}{}{}
\reading{08H2F4L3}{08H2F4L3}{}{Qo. 4: 9-13}{}{}{}
\feast{08H2F5}
        {Jeudi}{Propre du Temps}{Deuxième semaine d’août}{3}{}
        {}{}{}{}{}
\reading{08H2F5L1}{08H2F5L1}{De libro Ecclesiástæ}{Qo. 5: 1-4}{Qo}{5}{}
\reading{08H2F5L2}{08H2F5L2}{}{Qo. 5: 5-8}{}{}{}
\reading{08H2F5L3}{08H2F5L3}{}{Qo. 5: 9-13}{}{}{}
\feast{08H2F6}
        {Vendredi}{Propre du Temps}{Deuxième semaine d’août}{3}{}
        {}{}{}{}{}
\reading{08H2F6L1}{08H2F6L1}{De libro Ecclesiástæ}{Qo. 6: 1-2}{Qo}{6}{}
\reading{08H2F6L2}{08H2F6L2}{}{Qo. 6: 3-6}{}{}{}
\reading{08H2F6L3}{08H2F6L3}{}{Qo. 6: 6-9}{}{}{}
\feast{08H2F7}
        {Samedi}{Propre du Temps}{Deuxième semaine d’août}{3}{}
        {}{}{}{}{}
\reading{08H2F7L1}{08H2F7L1}{De libro Ecclesiástæ}{Qo. 6: 12; 7: 1-2}{Qo}{6}{7}
\reading{08H2F7L2}{08H2F7L2}{}{Qo. 7: 3-8}{}{}{}
\reading{08H2F7L3}{08H2F7L3}{}{Qo. 7: 10-13}{}{}{}
\feast{08H3F1}
        {Troisième dimanche d’août}{Propre du Temps}{Troisième semaine d’août}{2}{}
        {}{}{}{}{}
\reading{08H3F1N1L1}{08H3F1N1L1}{Incipit liber Sapiéntiæ}{Sg. 1: 1-4}{Sg}{1}{}
\reading{08H3F1N1L2}{08H3F1N1L2}{}{Sg. 1: 5-8}{}{}{}
\reading{08H3F1N1L3}{08H3F1N1L3}{}{Sg. 1: 9-11}{}{}{}
\feast{08H3F2}
        {Lundi}{Propre du Temps}{Troisième semaine d’août}{3}{}
        {}{}{}{}{}
\reading{08H3F2L1}{08H3F2L1}{De libro Sapiéntiæ}{Sg. 3: 1-6}{Sg}{3}{}
\reading{08H3F2L2}{08H3F2L2}{}{Sg. 3: 7-11}{}{}{}
\reading{08H3F2L3}{08H3F2L3}{}{Sg. 5: 15-20}{Sg}{5}{}
\feast{08H3F3}
        {Mardi}{Propre du Temps}{Troisième semaine d’août}{3}{}
        {}{}{}{}{}
\reading{08H3F3L1}{08H3F3L1}{De libro Sapiéntiæ}{Sg. 6: 1-4}{Sg}{6}{}
\reading{08H3F3L2}{08H3F3L2}{}{Sg. 6: 5-8}{}{}{}
\reading{08H3F3L3}{08H3F3L3}{}{Sg. 6: 9-12}{}{}{}
\feast{08H3F4}
        {Mercredi}{Propre du Temps}{Troisième semaine d’août}{3}{}
        {}{}{}{}{}
\reading{08H3F4L1}{08H3F4L1}{De libro Sapiéntiæ}{Sg. 7: 1-6}{Sg}{7}{}
\reading{08H3F4L2}{08H3F4L2}{}{Sg. 7: 7-10}{}{}{}
\reading{08H3F4L3}{08H3F4L3}{}{Sg. 7: 11-14}{}{}{}
\feast{08H3F5}
        {Jeudi}{Propre du Temps}{Troisième semaine d’août}{3}{}
        {}{}{}{}{}
\reading{08H3F5L1}{08H3F5L1}{De libro Sapiéntiæ}{Sg. 9: 13-19}{Sg}{9}{}
\reading{08H3F5L2}{08H3F5L2}{}{Sg. 10: 1-5}{Sg}{10}{}
\reading{08H3F5L3}{08H3F5L3}{}{Sg. 10: 6-9}{}{}{}
\feast{08H3F6}
        {Vendredi}{Propre du Temps}{Troisième semaine d’août}{3}{}
        {}{}{}{}{}
\reading{08H3F6L1}{08H3F6L1}{De libro Sapiéntiæ}{Sg. 13: 1-3}{Sg}{13}{}
\reading{08H3F6L2}{08H3F6L2}{}{Sg. 13: 4-7}{}{}{}
\reading{08H3F6L3}{08H3F6L3}{}{Sg. 13: 8-10}{}{}{}
\feast{08H3F7}
        {Samedi}{Propre du Temps}{Troisième semaine d’août}{3}{}
        {}{}{}{}{}
\reading{08H3F7L1}{08H3F7L1}{De libro Sapiéntiæ}{Sg. 15: 1-3}{Sg}{15}{}
\reading{08H3F7L2}{08H3F7L2}{}{Sg. 15: 4-6}{}{}{}
\reading{08H3F7L3}{08H3F7L3}{}{Sg. 15: 7-8}{}{}{}
\feast{08H4F1}
        {Quatrième dimanche d’août}{Propre du Temps}{Quatrième semaine d’août}{2}{}
        {}{}{}{}{}
\reading{08H4F1N1L1}{08H4F1N1L1}{Incipit liber Ecclesiástici}{Si. 1: 1-5}{Si}{1}{}
\reading{08H4F1N1L2}{08H4F1N1L2}{}{Si. 1: 6-10}{}{}{}
\reading{08H4F1N1L3}{08H4F1N1L3}{}{Si. 1: 11-14}{}{}{}
\feast{08H4F2}
        {Lundi}{Propre du Temps}{Quatrième semaine d’août}{3}{}
        {}{}{}{}{}
\reading{08H4F2L1}{08H4F2L1}{De libro Ecclesiástici}{Si. 1: 18-20}{Si}{1}{}
\reading{08H4F2L2}{08H4F2L2}{}{Si. 1: 21-26}{}{}{}
\reading{08H4F2L3}{08H4F2L3}{}{Si. 1: 27-30}{}{}{}
\feast{08H4F3}
        {Mardi}{Propre du Temps}{Quatrième semaine d’août}{3}{}
        {}{}{}{}{}
\reading{08H4F3L1}{08H4F3L1}{De libro Ecclesiástici}{Si. 2: 1-3}{Si}{2}{}
\reading{08H4F3L2}{08H4F3L2}{}{Si. 2: 4-6}{}{}{}
\reading{08H4F3L3}{08H4F3L3}{}{Si. 2: 7-12}{}{}{}
\feast{08H4F4}
        {Mercredi}{Propre du Temps}{Quatrième semaine d’août}{3}{}
        {}{}{}{}{}
\reading{08H4F4L1}{08H4F4L1}{De libro Ecclesiástici}{Si. 3: 1-3}{Si}{3}{}
\reading{08H4F4L2}{08H4F4L2}{}{Si. 3: 4-7}{}{}{}
\reading{08H4F4L3}{08H4F4L3}{}{Si. 3: 8-11}{}{}{}
\feast{08H4F5}
        {Jeudi}{Propre du Temps}{Quatrième semaine d’août}{3}{}
        {}{}{}{}{}
\reading{08H4F5L1}{08H4F5L1}{De libro Ecclesiástici}{Si. 3: 21-25}{Si}{3}{}
\reading{08H4F5L2}{08H4F5L2}{}{Si. 3: 26-28}{}{}{}
\reading{08H4F5L3}{08H4F5L3}{}{Si. 3: 29-31}{}{}{}
\feast{08H4F6}
        {Vendredi}{Propre du Temps}{Quatrième semaine d’août}{3}{}
        {}{}{}{}{}
\reading{08H4F6L1}{08H4F6L1}{De libro Ecclesiástici}{Si. 4: 1-4}{Si}{4}{}
\reading{08H4F6L2}{08H4F6L2}{}{Si. 4: 5-7}{}{}{}
\reading{08H4F6L3}{08H4F6L3}{}{Si. 4: 8-10}{}{}{}
\feast{08H4F7}
        {Samedi}{Propre du Temps}{Quatrième semaine d’août}{3}{}
        {}{}{}{}{}
\reading{08H4F7L1}{08H4F7L1}{De libro Ecclesiástici}{Si. 4: 20-23}{Si}{4}{}
\reading{08H4F7L2}{08H4F7L2}{}{Si. 4: 24-27}{}{}{}
\reading{08H4F7L3}{08H4F7L3}{}{Si. 4: 28-31}{}{}{}
\feast{08H5F1}
        {Cinquième dimanche d’août}{Propre du Temps}{Cinquième semaine d’août}{2}{}
        {}{}{}{}{}
\reading{08H5F1N1L1}{08H5F1N1L1}{De libro Ecclesiástici}{Si. 5: 1-5}{Si}{5}{}
\reading{08H5F1N1L2}{08H5F1N1L2}{}{Si. 5: 6-9}{}{}{}
\reading{08H5F1N1L3}{08H5F1N1L3}{}{Si. 5: 10-14}{}{}{}
\feast{08H5F2}
        {Lundi}{Propre du Temps}{Cinquième semaine d’août}{3}{}
        {}{}{}{}{}
\reading{08H5F2L1}{08H5F2L1}{De libro Ecclesiástici}{Si. 7: 1-5}{Si}{7}{}
\reading{08H5F2L2}{08H5F2L2}{}{Si. 7: 6-9}{}{}{}
\reading{08H5F2L3}{08H5F2L3}{}{Si. 7: 10-14}{}{}{}
\feast{08H5F3}
        {Mardi}{Propre du Temps}{Cinquième semaine d’août}{3}{}
        {}{}{}{}{}
\reading{08H5F3L1}{08H5F3L1}{De libro Ecclesiástici}{Si. 10: 1-5}{Si}{10}{}
\reading{08H5F3L2}{08H5F3L2}{}{Si. 10: 6-9}{}{}{}
\reading{08H5F3L3}{08H5F3L3}{}{Si. 10: 10-13}{}{}{}
\feast{08H5F4}
        {Mercredi}{Propre du Temps}{Cinquième semaine d’août}{3}{}
        {}{}{}{}{}
\reading{08H5F4L1}{08H5F4L1}{De libro Ecclesiástici}{Si. 13: 1-5}{Si}{13}{}
\reading{08H5F4L2}{08H5F4L2}{}{Si. 13: 7-12}{}{}{}
\reading{08H5F4L3}{08H5F4L3}{}{Si. 13: 13-18}{}{}{}
\feast{08H5F5}
        {Jeudi}{Propre du Temps}{Cinquième semaine d’août}{3}{}
        {}{}{}{}{}
\reading{08H5F5L1}{08H5F5L1}{De libro Ecclesiástici}{Si. 14: 1-5}{Si}{14}{}
\reading{08H5F5L2}{08H5F5L2}{}{Si. 14: 6-10}{}{}{}
\reading{08H5F5L3}{08H5F5L3}{}{Si. 14: 11-16}{}{}{}
\feast{08H5F6}
        {Vendredi}{Propre du Temps}{Cinquième semaine d’août}{3}{}
        {}{}{}{}{}
\reading{08H5F6L1}{08H5F6L1}{De libro Ecclesiástici}{Si. 21: 1-4}{Si}{21}{}
\reading{08H5F6L2}{08H5F6L2}{}{Si. 21: 5-9}{}{}{}
\reading{08H5F6L3}{08H5F6L3}{}{Si. 21: 10-13}{}{}{}
\feast{08H5F7}
        {Samedi}{Propre du Temps}{Cinquième semaine d’août}{3}{}
        {}{}{}{}{}
\reading{08H5F7L1}{08H5F7L1}{De libro Ecclesiástici}{Si. 32: 1-3}{Si}{32}{}
\reading{08H5F7L2}{08H5F7L2}{}{Si. 32: 4-7}{}{}{}
\reading{08H5F7L3}{08H5F7L3}{}{Si. 32: 8-13}{}{}{}
\feast{09H1F1}
        {Premier dimanche de septembre}{Propre du Temps}{Première semaine de septembre}{2}{}
        {}{}{}{}{}
\reading{09H1F1N1L1}{09H1F1N1L1}{Incipit liber Job}{Jb. 1: 1-3}{Jb}{1}{}
\reading{09H1F1N1L2}{09H1F1N1L2}{}{Jb. 1: 4-5}{}{}{}
\reading{09H1F1N1L3}{09H1F1N1L3}{}{Jb. 1: 6-11}{}{}{}
\feast{09H1F2}
        {Lundi}{Propre du Temps}{Première semaine de septembre}{3}{}
        {}{}{}{}{}
\reading{09H1F2L1}{09H1F2L1}{De libro Job}{Jb. 1: 13-16}{Jb}{1}{}
\reading{09H1F2L2}{09H1F2L2}{}{Jb. 1: 17-19}{}{}{}
\reading{09H1F2L3}{09H1F2L3}{}{Jb. 1: 20-22}{}{}{}
\feast{09H1F3}
        {Mardi}{Propre du Temps}{Première semaine de septembre}{3}{}
        {}{}{}{}{}
\reading{09H1F3L1}{09H1F3L1}{De libro Job}{Jb. 2: 1-5}{Jb}{2}{}
\reading{09H1F3L2}{09H1F3L2}{}{Jb. 2: 6-10}{}{}{}
\reading{09H1F3L3}{09H1F3L3}{}{Jb. 2: 11-13}{}{}{}
\feast{09H1F4}
        {Mercredi}{Propre du Temps}{Première semaine de septembre}{3}{}
        {}{}{}{}{}
\reading{09H1F4L1}{09H1F4L1}{De libro Job}{Jb. 3: 1-5}{Jb}{3}{}
\reading{09H1F4L2}{09H1F4L2}{}{Jb. 3: 6-10}{}{}{}
\reading{09H1F4L3}{09H1F4L3}{}{Jb. 3: 11-16}{}{}{}
\feast{09H1F5}
        {Jeudi}{Propre du Temps}{Première semaine de septembre}{3}{}
        {}{}{}{}{}
\reading{09H1F5L1}{09H1F5L1}{De libro Job}{Jb. 4: 1-6}{Jb}{4}{}
\reading{09H1F5L2}{09H1F5L2}{}{Jb. 4: 7-11}{}{}{}
\reading{09H1F5L3}{09H1F5L3}{}{Jb. 4: 12-18}{}{}{}
\feast{09H1F6}
        {Vendredi}{Propre du Temps}{Première semaine de septembre}{3}{}
        {}{}{}{}{}
\reading{09H1F6L1}{09H1F6L1}{De libro Job}{Jb. 6: 1-4}{Jb}{6}{}
\reading{09H1F6L2}{09H1F6L2}{}{Jb. 6: 5-7}{}{}{}
\reading{09H1F6L3}{09H1F6L3}{}{Jb. 6: 8-13}{}{}{}
\feast{09H1F7}
        {Samedi}{Propre du Temps}{Première semaine de septembre}{3}{}
        {}{}{}{}{}
\reading{09H1F7L1}{09H1F7L1}{De libro Job}{Jb. 7: 1-4}{Jb}{7}{}
\reading{09H1F7L2}{09H1F7L2}{}{Jb. 7: 5-8}{}{}{}
\reading{09H1F7L3}{09H1F7L3}{}{Jb. 7: 9-12}{}{}{}
\feast{09H2F1}
        {Deuxième dimanche de septembre}{Propre du Temps}{Deuxième semaine de septembre}{2}{}
        {}{}{}{}{}
\reading{09H2F1N1L1}{09H2F1N1L1}{De libro Job}{Jb. 9: 1-5}{Jb}{9}{}
\reading{09H2F1N1L2}{09H2F1N1L2}{}{Jb. 9: 6-10}{}{}{}
\reading{09H2F1N1L3}{09H2F1N1L3}{}{Jb. 9: 11-17}{}{}{}
\feast{09H2F2}
        {Lundi}{Propre du Temps}{Deuxième semaine de septembre}{3}{}
        {}{}{}{}{}
\reading{09H2F2L1}{09H2F2L1}{De libro Job}{Jb. 27: 1-5}{Jb}{27}{}
\reading{09H2F2L2}{09H2F2L2}{}{Jb. 27: 6-10}{}{}{}
\reading{09H2F2L3}{09H2F2L3}{}{Jb. 27: 11-15}{}{}{}
\feast{09H2F3}
        {Mardi}{Propre du Temps}{Deuxième semaine de septembre}{3}{}
        {}{}{}{}{}
\reading{09H2F3L1}{09H2F3L1}{De libro Job}{Jb. 28: 12-16}{Jb}{28}{}
\reading{09H2F3L2}{09H2F3L2}{}{Jb. 28: 17-22}{}{}{}
\reading{09H2F3L3}{09H2F3L3}{}{Jb. 28: 23-28}{}{}{}
\feast{09H2F4}
        {Mercredi}{Propre du Temps}{Deuxième semaine de septembre}{3}{}
        {}{}{}{}{}
\reading{09H2F4L1}{09H2F4L1}{De libro Job}{Jb. 31: 1-6}{Jb}{31}{}
\reading{09H2F4L2}{09H2F4L2}{}{Jb. 31: 7-12}{}{}{}
\reading{09H2F4L3}{09H2F4L3}{}{Jb. 31: 13-18}{}{}{}
\feast{09H2F5}
        {Jeudi}{Propre du Temps}{Deuxième semaine de septembre}{3}{}
        {}{}{}{}{}
\reading{09H2F5L1}{09H2F5L1}{De libro Job}{Jb. 38: 1-7}{Jb}{38}{}
\reading{09H2F5L2}{09H2F5L2}{}{Jb. 38: 8-13}{}{}{}
\reading{09H2F5L3}{09H2F5L3}{}{Jb. 38: 14-20}{}{}{}
\feast{09H2F6}
        {Vendredi}{Propre du Temps}{Deuxième semaine de septembre}{3}{}
        {}{}{}{}{}
\reading{09H2F6L1}{09H2F6L1}{De libro Job}{Jb. 40: 6-10}{Jb}{40}{}
\reading{09H2F6L2}{09H2F6L2}{}{Jb. 40: 11-16}{}{}{}
\reading{09H2F6L3}{09H2F6L3}{}{Jb. 42: 1-6}{Jb}{42}{}
\feast{09H2F7}
        {Samedi}{Propre du Temps}{Deuxième semaine de septembre}{3}{}
        {}{}{}{}{}
\reading{09H2F7L1}{09H2F7L1}{De libro Job}{Jb. 42: 7-8}{Jb}{42}{}
\reading{09H2F7L2}{09H2F7L2}{}{Jb. 42: 9-11}{}{}{}
\reading{09H2F7L3}{09H2F7L3}{}{Jb. 42: 12-17}{}{}{}
\feast{09H3F1}
        {Troisième dimanche de septembre}{Propre du Temps}{Troisième semaine de septembre}{2}{}
        {}{}{}{}{}
\reading{09H3F1N1L1}{09H3F1N1L1}{Incipit liber Tobíæ}{Tb. 1: 1-4}{Tb}{1}{}
\reading{09H3F1N1L2}{09H3F1N1L2}{}{Tb. 1: 5-6, 8-9}{}{}{}
\reading{09H3F1N1L3}{09H3F1N1L3}{}{Tb. 1: 10-14}{}{}{}
\feast{09H3F2}
        {Lundi}{Propre du Temps}{Troisième semaine de septembre}{3}{}
        {}{}{}{}{}
\reading{09H3F2L1}{09H3F2L1}{De libro Tobíæ}{Tb. 2: 1-4}{Tb}{2}{}
\reading{09H3F2L2}{09H3F2L2}{}{Tb. 2: 8-10}{}{}{}
\reading{09H3F2L3}{09H3F2L3}{}{Tb. 2: 10}{}{}{}
\feast{09H3F3}
        {Mardi}{Propre du Temps}{Troisième semaine de septembre}{3}{}
        {}{}{}{}{}
\reading{09H3F3L1}{09H3F3L1}{De libro Tobíæ}{Tb. 2: 11-13}{Tb}{2}{}
\reading{09H3F3L2}{09H3F3L2}{}{Tb. 2: 14; 3: 1-3}{Tb}{3}{}
\reading{09H3F3L3}{09H3F3L3}{}{Tb. 3: 4-6}{}{}{}
\feast{09H3F4}
        {Mercredi, vendredi et samedi des Quatre-Temps de septembre}{Propre du Temps}{Troisième semaine de septembre}{3}{}
        {}{}{Quatre-Temps!de septembre}{}{}
\rubric{Lectures à l'Homéliaire, sauf si l'office est à trois nocturnes; alors, en l'absence de lectures scripturaires propres, on emploie les éventuelles lectures empêchées des jours précédents ou des jours suivants, tout en en conservant l'ordre, en donnant la priorité à celles du dimanche.}
\feast{09H3F5}
        {Jeudi}{Propre du Temps}{Troisième semaine de septembre}{3}{}
        {}{}{}{}{}
\reading{09H3F5L1}{09H3F5L1}{De libro Tobíæ}{Tb. 12: 1-4}{Tb}{12}{}
\reading{09H3F5L2}{09H3F5L2}{}{Tb. 12: 5-6, 8-10}{}{}{}
\reading{09H3F5L3}{09H3F5L3}{}{Tb. 12: 11-17}{}{}{}
\feast{09H3F6}
        {Vendredi des Quatre-Temps de septembre}{Propre du Temps}{Troisième semaine de septembre}{3}{}
        {}{}{}{}{}
\feast{09H3F7}
        {Samedi des Quatre-Temps de septembre}{Propre du Temps}{Troisième semaine de septembre}{3}{}
        {}{}{}{}{}
\feast{09H4F1}
        {Quatrième dimanche de septembre}{Propre du Temps}{Quatrième semaine de septembre}{2}{}
        {}{}{}{}{}
\reading{09H4F1N1L1}{09H4F1N1L1}{Incipit liber Judith}{Jdt. 1: 1-4}{Jdt}{1}{}
\reading{09H4F1N1L2}{09H4F1N1L2}{}{Jdt. 1: 1, 5-10}{}{}{}
\reading{09H4F1N1L3}{09H4F1N1L3}{}{Jdt. 1: 11-12; 2: 1-2}{Jdt}{2}{}
\feast{09H4F2}
        {Lundi}{Propre du Temps}{Quatrième semaine de septembre}{3}{}
        {}{}{}{}{}
\reading{09H4F2L1}{09H4F2L1}{De libro Judith}{Jdt. 4: 1-2, 4-5}{Jdt}{4}{}
\reading{09H4F2L2}{09H4F2L2}{}{Jdt. 4: 6-9}{}{}{}
\reading{09H4F2L3}{09H4F2L3}{}{Jdt. 4: 10-14}{}{}{}
\feast{09H4F3}
        {Mardi}{Propre du Temps}{Quatrième semaine de septembre}{3}{}
        {}{}{}{}{}
\reading{09H4F3L1}{09H4F3L1}{De libro Judith}{Jdt. 8: 1-4}{Jdt}{8}{}
\reading{09H4F3L2}{09H4F3L2}{}{Jdt. 8: 5-8}{}{}{}
\reading{09H4F3L3}{09H4F3L3}{}{Jdt. 8: 9-12}{}{}{}
\feast{09H4F4}
        {Mercredi}{Propre du Temps}{Quatrième semaine de septembre}{3}{}
        {}{}{}{}{}
\reading{09H4F4L1}{09H4F4L1}{De libro Judith}{Jdt. 10: 1-4}{Jdt}{10}{}
\reading{09H4F4L2}{09H4F4L2}{}{Jdt. 10: 10-12}{}{}{}
\reading{09H4F4L3}{09H4F4L3}{}{Jdt. 10: 17-19, 21-23}{}{}{}
\feast{09H4F5}
        {Jeudi}{Propre du Temps}{Quatrième semaine de septembre}{3}{}
        {}{}{}{}{}
\rubric{Si le mois de septembre n'a que quatre semaines, on emploie les lectures du cinquième dimanche de septembre; sinon, celles qui suivent.}
\reading{09H4F5L1}{09H4F5L1}{De libro Judith}{Jdt. 12: 10-14}{Jdt}{12}{}
\reading{09H4F5L2}{09H4F5L2}{}{Jdt. 13: 1-5}{Jdt}{13}{}
\reading{09H4F5L3}{09H4F5L3}{}{Jdt. 13: 6-10}{}{}{}
\feast{09H4F6}
        {Vendredi}{Propre du Temps}{Quatrième semaine de septembre}{3}{}
        {}{}{}{}{}
\rubric{Si le mois de septembre n'a que quatre semaines, on emploie les lectures du vendredi de la cinquième semaine de septembre; sinon, celles qui suivent.}
\reading{09H4F6L1}{09H4F6L1}{De libro Judith}{Jdt. 15: 1-3}{Jdt}{15}{}
\reading{09H4F6L2}{09H4F6L2}{}{Jdt. 15: 4-6}{}{}{}
\reading{09H4F6L3}{09H4F6L3}{}{Jdt. 15: 8-10}{}{}{}
\feast{09H4F7}
        {Samedi}{Propre du Temps}{Quatrième semaine de septembre}{3}{}
        {}{}{}{}{}
\rubric{Si le mois de septembre n'a que quatre semaines, on emploie les lectures du samedi de la cinquième semaine de septembre; sinon, celles qui suivent.}
\reading{09H4F7L1}{09H4F7L1}{De libro Judith}{Jdt. 16: 18-19}{Jdt}{16}{}
\reading{09H4F7L2}{09H4F7L2}{}{Jdt. 16: 20-22}{}{}{}
\reading{09H4F7L3}{09H4F7L3}{}{Jdt. 16: 23-25}{}{}{}
\feast{09H5F1}
        {Cinquième dimanche de septembre}{Propre du Temps}{Cinquième semaine de septembre}{2}{}
        {}{}{}{}{}
\reading{09H5F1N1L1}{09H5F1N1L1}{Incipit liber Esther}{Est. 1: 1-4}{Est}{1}{}
\reading{09H5F1N1L2}{09H5F1N1L2}{}{Est. 1: 5-6}{}{}{}
\reading{09H5F1N1L3}{09H5F1N1L3}{}{Est. 1: 7-9}{}{}{}
\feast{09H5F2}
        {Lundi}{Propre du Temps}{Cinquième semaine de septembre}{3}{}
        {}{}{}{}{}
\reading{09H5F2L1}{09H5F2L1}{De libro Esther}{Est. 2: 5-7}{Est}{2}{}
\reading{09H5F2L2}{09H5F2L2}{}{Est. 2: 8-11}{}{}{}
\reading{09H5F2L3}{09H5F2L3}{}{Est. 2: 15-17}{}{}{}
\feast{09H5F3}
        {Mardi}{Propre du Temps}{Cinquième semaine de septembre}{3}{}
        {}{}{}{}{}
\reading{09H5F3L1}{09H5F3L1}{De libro Esther}{Est. 3: 1-3}{Est}{3}{}
\reading{09H5F3L2}{09H5F3L2}{}{Est. 3: 4-6}{}{}{}
\reading{09H5F3L3}{09H5F3L3}{}{Est. 3: 6-7}{}{}{}
\feast{09H5F4}
        {Mercredi}{Propre du Temps}{Cinquième semaine de septembre}{3}{}
        {}{}{}{}{}
\reading{09H5F4L1}{09H5F4L1}{De libro Esther}{Est. 4: 1-5}{Est}{4}{}
\reading{09H5F4L2}{09H5F4L2}{}{Est. 4: 6-11}{}{}{}
\reading{09H5F4L3}{09H5F4L3}{}{Est. 4: 12-17}{}{}{}
\feast{09H5F5}
        {Jeudi}{Propre du Temps}{Cinquième semaine de septembre}{3}{}
        {}{}{}{}{}
\reading{09H5F5L1}{09H5F5L1}{De libro Esther}{Est. 5: 1-5}{Est}{5}{}
\reading{09H5F5L2}{09H5F5L2}{}{Est. 5: 9-13}{}{}{}
\reading{09H5F5L3}{09H5F5L3}{}{Est. 5: 14}{}{}{}
\feast{09H5F6}
        {Vendredi}{Propre du Temps}{Cinquième semaine de septembre}{3}{}
        {}{}{}{}{}
\reading{09H5F6L1}{09H5F6L1}{De libro Esther}{Est. 6: 1-5}{Est}{6}{}
\reading{09H5F6L2}{09H5F6L2}{}{Est. 6: 6-9}{}{}{}
\reading{09H5F6L3}{09H5F6L3}{}{Est. 6: 10-13}{}{}{}
\feast{09H5F7}
        {Samedi}{Propre du Temps}{Cinquième semaine de septembre}{3}{}
        {}{}{}{}{}
\reading{09H5F7L1}{09H5F7L1}{De libro Esther}{Est. 7: 1-4}{Est}{7}{}
\reading{09H5F7L2}{09H5F7L2}{}{Est. 7: 5-7}{}{}{}
\reading{09H5F7L3}{09H5F7L3}{}{Est. 7: 8-10}{}{}{}
\feast{10H1F1}
        {Premier dimanche d’octobre}{Propre du Temps}{Première semaine d’octobre}{2}{}
        {}{}{}{}{}
\reading{10H1F1N1L1}{10H1F1N1L1}{Incipit liber primus Machabæórum}{1 M. 1: 1-6}{1 M}{1}{}
\reading{10H1F1N1L2}{10H1F1N1L2}{}{1 M. 1: 7-10}{}{}{}
\reading{10H1F1N1L3}{10H1F1N1L3}{}{1 M. 1: 11-15}{}{}{}
\feast{10H1F2}
        {Lundi}{Propre du Temps}{Première semaine d’octobre}{3}{}
        {}{}{}{}{}
\reading{10H1F2L1}{10H1F2L1}{De libro primo Machabæórum}{1 M. 1: 16-19}{1 M}{1}{}
\reading{10H1F2L2}{10H1F2L2}{}{1 M. 1: 20-22}{}{}{}
\reading{10H1F2L3}{10H1F2L3}{}{1 M. 1: 23-28}{}{}{}
\feast{10H1F3}
        {Mardi}{Propre du Temps}{Première semaine d’octobre}{3}{}
        {}{}{}{}{}
\reading{10H1F3L1}{10H1F3L1}{De libro primo Machabæórum}{1 M. 2: 1-6}{1 M}{2}{}
\reading{10H1F3L2}{10H1F3L2}{}{1 M. 2: 6-10}{}{}{}
\reading{10H1F3L3}{10H1F3L3}{}{1 M. 2: 14-16}{}{}{}
\feast{10H1F4}
        {Mercredi}{Propre du Temps}{Première semaine d’octobre}{3}{}
        {}{}{}{}{}
\reading{10H1F4L1}{10H1F4L1}{De libro primo Machabæórum}{1 M. 2: 19-22}{1 M}{2}{}
\reading{10H1F4L2}{10H1F4L2}{}{1 M. 2: 23-26}{}{}{}
\reading{10H1F4L3}{10H1F4L3}{}{1 M. 2: 27-30}{}{}{}
\feast{10H1F5}
        {Jeudi}{Propre du Temps}{Première semaine d’octobre}{3}{}
        {}{}{}{}{}
\reading{10H1F5L1}{10H1F5L1}{De libro primo Machabæórum}{1 M. 2: 49-54}{1 M}{2}{}
\reading{10H1F5L2}{10H1F5L2}{}{1 M. 2: 55-63}{}{}{}
\reading{10H1F5L3}{10H1F5L3}{}{1 M. 2: 64-69}{}{}{}
\feast{10H1F6}
        {Vendredi}{Propre du Temps}{Première semaine d’octobre}{3}{}
        {}{}{}{}{}
\reading{10H1F6L1}{10H1F6L1}{De libro primo Machabæórum}{1 M. 2: 70; 3: 1-3, 5-6}{1 M}{2}{3}
\reading{10H1F6L2}{10H1F6L2}{}{1 M. 3: 7-12}{}{}{}
\reading{10H1F6L3}{10H1F6L3}{}{1 M. 3: 25-28}{}{}{}
\feast{10H1F7}
        {Samedi}{Propre du Temps}{Première semaine d’octobre}{3}{}
        {}{}{}{}{}
\reading{10H1F7L1}{10H1F7L1}{De libro primo Machabæórum}{1 M. 3: 42-45}{1 M}{3}{}
\reading{10H1F7L2}{10H1F7L2}{}{1 M. 3: 46-53}{}{}{}
\reading{10H1F7L3}{10H1F7L3}{}{1 M. 3: 54-60}{}{}{}
\feast{10H2F1}
        {Deuxième dimanche d’octobre}{Propre du Temps}{Deuxième semaine d’octobre}{2}{}
        {}{}{}{}{}
\reading{10H2F1N1L1}{10H2F1N1L1}{De libro primo Machabæórum}{1 M. 4: 36-40}{1 M}{4}{}
\reading{10H2F1N1L2}{10H2F1N1L2}{}{1 M. 4: 41-46}{}{}{}
\reading{10H2F1N1L3}{10H2F1N1L3}{}{1 M. 4: 47-51}{}{}{}
\feast{10H2F2}
        {Lundi}{Propre du Temps}{Deuxième semaine d’octobre}{3}{}
        {}{}{}{}{}
\reading{10H2F2L1}{10H2F2L1}{De libro primo Machabæórum}{1 M. 4: 52-55}{1 M}{4}{}
\reading{10H2F2L2}{10H2F2L2}{}{1 M. 4: 56-59}{}{}{}
\reading{10H2F2L3}{10H2F2L3}{}{1 M. 4: 60-61}{}{}{}
\feast{10H2F3}
        {Mardi}{Propre du Temps}{Deuxième semaine d’octobre}{3}{}
        {}{}{}{}{}
\reading{10H2F3L1}{10H2F3L1}{De libro primo Machabæórum}{1 M. 5: 1-5}{1 M}{5}{}
\reading{10H2F3L2}{10H2F3L2}{}{1 M. 5: 6-9}{}{}{}
\reading{10H2F3L3}{10H2F3L3}{}{1 M. 5: 10-13}{}{}{}
\feast{10H2F4}
        {Mercredi}{Propre du Temps}{Deuxième semaine d’octobre}{3}{}
        {}{}{}{}{}
\reading{10H2F4L1}{10H2F4L1}{De libro primo Machabæórum}{1 M. 5: 55-58}{1 M}{5}{}
\reading{10H2F4L2}{10H2F4L2}{}{1 M. 5: 59-62}{}{}{}
\reading{10H2F4L3}{10H2F4L3}{}{1 M. 5: 63-67}{}{}{}
\feast{10H2F5}
        {Jeudi}{Propre du Temps}{Deuxième semaine d’octobre}{3}{}
        {}{}{}{}{}
\reading{10H2F5L1}{10H2F5L1}{De libro primo Machabæórum}{1 M. 6: 1-6}{1 M}{6}{}
\reading{10H2F5L2}{10H2F5L2}{}{1 M. 6: 6-9}{}{}{}
\reading{10H2F5L3}{10H2F5L3}{}{1 M. 6: 9-13}{}{}{}
\feast{10H2F6}
        {Vendredi}{Propre du Temps}{Deuxième semaine d’octobre}{3}{}
        {}{}{}{}{}
\reading{10H2F6L1}{10H2F6L1}{De libro primo Machabæórum}{1 M. 7: 1, 4-7}{1 M}{7}{}
\reading{10H2F6L2}{10H2F6L2}{}{1 M. 7: 8-11}{}{}{}
\reading{10H2F6L3}{10H2F6L3}{}{1 M. 7: 12-17}{}{}{}
\feast{10H2F7}
        {Samedi}{Propre du Temps}{Deuxième semaine d’octobre}{3}{}
        {}{}{}{}{}
\reading{10H2F7L1}{10H2F7L1}{De libro primo Machabæórum}{1 M. 8: 1-4}{1 M}{8}{}
\reading{10H2F7L2}{10H2F7L2}{}{1 M. 8: 17-22}{}{}{}
\reading{10H2F7L3}{10H2F7L3}{}{1 M. 8: 23-27}{}{}{}
\feast{10H3F1}
        {Troisième dimanche d’octobre}{Propre du Temps}{Troisième semaine d’octobre}{2}{}
        {}{}{}{}{}
\reading{10H3F1N1L1}{10H3F1N1L1}{De libro primo Machabæórum}{1 M. 9: 1-6}{1 M}{9}{}
\reading{10H3F1N1L2}{10H3F1N1L2}{}{1 M. 9: 7-11}{}{}{}
\reading{10H3F1N1L3}{10H3F1N1L3}{}{1 M. 9: 12-20}{}{}{}
\feast{10H3F2}
        {Lundi}{Propre du Temps}{Troisième semaine d’octobre}{3}{}
        {}{}{}{}{}
\reading{10H3F2L1}{10H3F2L1}{De libro primo Machabæórum}{1 M. 9: 28-32}{1 M}{9}{}
\reading{10H3F2L2}{10H3F2L2}{}{1 M. 9: 33-36}{}{}{}
\reading{10H3F2L3}{10H3F2L3}{}{1 M. 9: 37-40}{}{}{}
\feast{10H3F3}
        {Mardi}{Propre du Temps}{Troisième semaine d’octobre}{3}{}
        {}{}{}{}{}
\reading{10H3F3L1}{10H3F3L1}{De libro primo Machabæórum}{1 M. 12: 1-4}{1 M}{12}{}
\reading{10H3F3L2}{10H3F3L2}{}{1 M. 12: 5-8}{}{}{}
\reading{10H3F3L3}{10H3F3L3}{}{1 M. 12: 9-11}{}{}{}
\feast{10H3F4}
        {Mercredi}{Propre du Temps}{Troisième semaine d’octobre}{3}{}
        {}{}{}{}{}
\reading{10H3F4L1}{10H3F4L1}{De libro primo Machabæórum}{1 M. 12: 39-43}{1 M}{12}{}
\reading{10H3F4L2}{10H3F4L2}{}{1 M. 12: 44-47}{}{}{}
\reading{10H3F4L3}{10H3F4L3}{}{1 M. 12: 48-52}{}{}{}
\feast{10H3F5}
        {Jeudi}{Propre du Temps}{Troisième semaine d’octobre}{3}{}
        {}{}{}{}{}
\reading{10H3F5L1}{10H3F5L1}{De libro primo Machabæórum}{1 M. 13: 1-6}{1 M}{13}{}
\reading{10H3F5L2}{10H3F5L2}{}{1 M. 13: 7-13}{}{}{}
\reading{10H3F5L3}{10H3F5L3}{}{1 M. 13: 14-19}{}{}{}
\feast{10H3F6}
        {Vendredi}{Propre du Temps}{Troisième semaine d’octobre}{3}{}
        {}{}{}{}{}
\reading{10H3F6L1}{10H3F6L1}{De libro primo Machabæórum}{1 M. 14: 16-20}{1 M}{14}{}
\reading{10H3F6L2}{10H3F6L2}{}{1 M. 14: 20-23}{}{}{}
\reading{10H3F6L3}{10H3F6L3}{}{1 M. 14: 24-26}{}{}{}
\feast{10H3F7}
        {Samedi}{Propre du Temps}{Troisième semaine d’octobre}{3}{}
        {}{}{}{}{}
\reading{10H3F7L1}{10H3F7L1}{De libro primo Machabæórum}{1 M. 16: 14-17}{1 M}{16}{}
\reading{10H3F7L2}{10H3F7L2}{}{1 M. 16: 18-21}{}{}{}
\reading{10H3F7L3}{10H3F7L3}{}{1 M. 16: 22-24}{}{}{}
\feast{10H4F1}
        {Quatrième dimanche d’octobre}{Propre du Temps}{Quatrième semaine d’octobre}{2}{}
        {}{}{}{}{}
\rubric{Si on célèbre la fête du Christ-Roi, ces lectures sont transférées au premier jour de la semaine où on emploie l'Écriture courante.}
\reading{10H4F1N1L1}{10H4F1N1L1}{Incipit liber secúndus Machabæórum}{2 M. 1: 1-6}{2 M}{1}{}
\reading{10H4F1N1L2}{10H4F1N1L2}{}{2 M. 1: 18-19}{}{}{}
\reading{10H4F1N1L3}{10H4F1N1L3}{}{2 M. 1: 20-22}{}{}{}
\feast{10H4F2}
        {Lundi}{Propre du Temps}{Quatrième semaine d’octobre}{3}{}
        {}{}{}{}{}
\reading{10H4F2L1}{10H4F2L1}{De libro secúndo Machabæórum}{2 M. 2: 1-3}{2 M}{2}{}
\reading{10H4F2L2}{10H4F2L2}{}{2 M. 2: 4-6}{}{}{}
\reading{10H4F2L3}{10H4F2L3}{}{2 M. 2: 7-9}{}{}{}
\feast{10H4F3}
        {Mardi}{Propre du Temps}{Quatrième semaine d’octobre}{3}{}
        {}{}{}{}{}
\reading{10H4F3L1}{10H4F3L1}{De libro secúndo Machabæórum}{2 M. 3: 1-4}{2 M}{3}{}
\reading{10H4F3L2}{10H4F3L2}{}{2 M. 3: 5-8}{}{}{}
\reading{10H4F3L3}{10H4F3L3}{}{2 M. 3: 9-12}{}{}{}
\feast{10H4F4}
        {Mercredi}{Propre du Temps}{Quatrième semaine d’octobre}{3}{}
        {}{}{}{}{}
\reading{10H4F4L1}{10H4F4L1}{De libro secúndo Machabæórum}{2 M. 3: 23-25}{2 M}{3}{}
\reading{10H4F4L2}{10H4F4L2}{}{2 M. 3: 26-29}{}{}{}
\reading{10H4F4L3}{10H4F4L3}{}{2 M. 3: 32-34}{}{}{}
\feast{10H4F5}
        {Jeudi}{Propre du Temps}{Quatrième semaine d’octobre}{3}{}
        {}{}{}{}{}
\rubric{Si le mois d’octobre n'a que quatre semaines, on emploie les lectures du cinquième dimanche d'octobre, et si elles sont empêchées, on les transfère au lendemain ou au surlendemain, en en conservant l'ordre; sinon, on emploie celles qui suivent.}
\reading{10H4F5L1}{10H4F5L1}{De libro secúndo Machabæórum}{2 M. 4: 1-5}{2 M}{4}{}
\reading{10H4F5L2}{10H4F5L2}{}{2 M. 4: 6-9}{}{}{}
\reading{10H4F5L3}{10H4F5L3}{}{2 M. 4: 10-11}{}{}{}
\feast{10H4F6}
        {Vendredi}{Propre du Temps}{Quatrième semaine d’octobre}{3}{}
        {}{}{}{}{}
\rubric{Si le mois d’octobre n'a que quatre semaines, on emploie les lectures du lundi de la cinquième semaine d'octobre, sauf si celles du cinquième dimanche sont transférées à ce jour, et si elles sont empêchées, on les transfère au lendemain, en en conservant l'ordre; sinon, on emploie celles qui suivent.}
\reading{10H4F6L1}{10H4F6L1}{De libro secúndo Machabæórum}{2 M. 5: 1-4}{2 M}{5}{}
\reading{10H4F6L2}{10H4F6L2}{}{2 M. 5: 5-7}{}{}{}
\reading{10H4F6L3}{10H4F6L3}{}{2 M. 5: 8-10}{}{}{}
\feast{10H4F7}
        {Samedi}{Propre du Temps}{Quatrième semaine d’octobre}{3}{}
        {}{}{}{}{}
\rubric{Si le mois d’octobre n'a que quatre semaines, on emploie les lectures du mardi de la cinquième semaine d'octobre, sauf si des précédentes lectures sont transférées à ce jour; sinon, on emploie celles qui suivent.}
\reading{10H4F7L1}{10H4F7L1}{De libro secúndo Machabæórum}{2 M. 6: 1-4}{2 M}{6}{}
\reading{10H4F7L2}{10H4F7L2}{}{2 M. 6: 5-9}{}{}{}
\reading{10H4F7L3}{10H4F7L3}{}{2 M. 6: 10-12}{}{}{}
\feast{10H5F1}
        {Cinquième dimanche d’octobre}{Propre du Temps}{Cinquième semaine d’octobre}{2}{}
        {}{}{}{}{}
\rubric{Si on célèbre la fête du Christ-Roi, ces lectures sont transférées au lendemain.}
\reading{10H5F1N1L1}{10H5F1N1L1}{De libro secúndo Machabæórum}{2 M. 6: 18-22}{2 M}{6}{}
\reading{10H5F1N1L2}{10H5F1N1L2}{}{2 M. 6: 23-28}{}{}{}
\reading{10H5F1N1L3}{10H5F1N1L3}{}{2 M. 7: 1-5}{2 M}{7}{}
\feast{10H5F2}
        {Lundi}{Propre du Temps}{Cinquième semaine d’octobre}{3}{}
        {}{}{}{}{}
\rubric{Si on a célébré la fête du Christ-Roi, ces lectures sont transférées au lendemain.}
\reading{10H5F2L1}{10H5F2L1}{De libro secúndo Machabæórum}{2 M. 7: 7-12}{2 M}{7}{}
\reading{10H5F2L2}{10H5F2L2}{}{2 M. 7: 13-19}{}{}{}
\reading{10H5F2L3}{10H5F2L3}{}{2 M. 7: 20-23}{}{}{}
\feast{10H5F3}
        {Mardi}{Propre du Temps}{Cinquième semaine d’octobre}{3}{}
        {}{}{}{}{}
\rubric{Si on a célébré la fête du Christ-Roi, ces lectures sont transférées au lendemain.}
\reading{10H5F3L1}{10H5F3L1}{De libro secúndo Machabæórum}{2 M. 7: 24-27}{2 M}{7}{}
\reading{10H5F3L2}{10H5F3L2}{}{2 M. 7: 28-33}{}{}{}
\reading{10H5F3L3}{10H5F3L3}{}{2 M. 7: 34-41}{}{}{}
\feast{10H5F4}
        {Mercredi}{Propre du Temps}{Cinquième semaine d’octobre}{3}{}
        {}{}{}{}{}
\rubric{Si on a célébré la fête du Christ-Roi, ces lectures sont empêchées par un transfert, et omises.}
\reading{10H5F4L1}{10H5F4L1}{De libro secúndo Machabæórum}{2 M. 8: 10-14}{2 M}{8}{}
\reading{10H5F4L2}{10H5F4L2}{}{2 M. 8: 16-19}{}{}{}
\reading{10H5F4L3}{10H5F4L3}{}{2 M. 8: 21-28}{}{}{}
\feast{10H5F5}
        {Jeudi}{Propre du Temps}{Cinquième semaine d’octobre}{3}{}
        {}{}{}{}{}
\reading{10H5F5L1}{10H5F5L1}{De libro secúndo Machabæórum}{2 M. 9: 1-4}{2 M}{9}{}
\reading{10H5F5L2}{10H5F5L2}{}{2 M. 9: 5-7}{}{}{}
\reading{10H5F5L3}{10H5F5L3}{}{2 M. 9: 8-10}{}{}{}
\feast{10H5F6}
        {Vendredi}{Propre du Temps}{Cinquième semaine d’octobre}{3}{}
        {}{}{}{}{}
\reading{10H5F6L1}{10H5F6L1}{De libro secúndo Machabæórum}{2 M. 10: 1-5}{2 M}{10}{}
\reading{10H5F6L2}{10H5F6L2}{}{2 M. 10: 24-27}{}{}{}
\reading{10H5F6L3}{10H5F6L3}{}{2 M. 10: 28-32}{}{}{}
\feast{10H5F7}
        {Samedi}{Propre du Temps}{Cinquième semaine d’octobre}{3}{}
        {}{}{}{}{}
\reading{10H5F7L1}{10H5F7L1}{De libro secúndo Machabæórum}{2 M. 15: 7-11}{2 M}{15}{}
\reading{10H5F7L2}{10H5F7L2}{}{2 M. 15: 12-16}{}{}{}
\reading{10H5F7L3}{10H5F7L3}{}{2 M. 15: 17-19}{}{}{}
\feast{11H1F1}
        {Premier dimanche de novembre}{Propre du Temps}{Première semaine de novembre}{2}{}
        {}{}{}{}{}
\reading{11H1F1N1L1}{11H1F1N1L1}{Incipit liber Ezechiélis Prophétæ}{Ez. 1: 1-4}{Ez}{1}{}
\reading{11H1F1N1L2}{11H1F1N1L2}{}{Ez. 1: 5-9}{}{}{}
\reading{11H1F1N1L3}{11H1F1N1L3}{}{Ez. 1: 10-12}{}{}{}
\feast{11H1F2}
        {Lundi}{Propre du Temps}{Première semaine de novembre}{3}{}
        {}{}{}{}{}
\reading{11H1F2L1}{11H1F2L1}{De Ezechiéle Prophéta}{Ez. 2: 2-5}{Ez}{2}{}
\reading{11H1F2L2}{11H1F2L2}{}{Ez. 2: 6-7}{}{}{}
\reading{11H1F2L3}{11H1F2L3}{}{Ez. 2: 8-10}{}{}{}
\feast{11H1F3}
        {Mardi}{Propre du Temps}{Première semaine de novembre}{3}{}
        {}{}{}{}{}
\reading{11H1F3L1}{11H1F3L1}{De Ezechiéle Prophéta}{Ez. 3: 1-4}{Ez}{3}{}
\reading{11H1F3L2}{11H1F3L2}{}{Ez. 3: 5-9}{}{}{}
\reading{11H1F3L3}{11H1F3L3}{}{Ez. 3: 10-13}{}{}{}
\feast{11H1F4}
        {Mercredi}{Propre du Temps}{Première semaine de novembre}{3}{}
        {}{}{}{}{}
\reading{11H1F4L1}{11H1F4L1}{De Ezechiéle Prophéta}{Ez. 7: 1-4}{Ez}{7}{}
\reading{11H1F4L2}{11H1F4L2}{}{Ez. 7: 5-9}{}{}{}
\reading{11H1F4L3}{11H1F4L3}{}{Ez. 7: 10-13}{}{}{}
\feast{11H1F5}
        {Jeudi}{Propre du Temps}{Première semaine de novembre}{3}{}
        {}{}{}{}{}
\reading{11H1F5L1}{11H1F5L1}{De Ezechiéle Prophéta}{Ez. 13: 1-6}{Ez}{13}{}
\reading{11H1F5L2}{11H1F5L2}{}{Ez. 13: 7-10}{}{}{}
\reading{11H1F5L3}{11H1F5L3}{}{Ez. 13: 11-14}{}{}{}
\feast{11H1F6}
        {Vendredi}{Propre du Temps}{Première semaine de novembre}{3}{}
        {}{}{}{}{}
\reading{11H1F6L1}{11H1F6L1}{De Ezechiéle Prophéta}{Ez. 15: 1-5}{Ez}{15}{}
\reading{11H1F6L2}{11H1F6L2}{}{Ez. 15: 6-8}{}{}{}
\reading{11H1F6L3}{11H1F6L3}{}{Ez. 16: 1-5}{Ez}{16}{}
\feast{11H1F7}
        {Samedi}{Propre du Temps}{Première semaine de novembre}{3}{}
        {}{}{}{}{}
\reading{11H1F7L1}{11H1F7L1}{De Ezechiéle Prophéta}{Ez. 19: 1-7}{Ez}{19}{}
\reading{11H1F7L2}{11H1F7L2}{}{Ez. 19: 8-11}{}{}{}
\reading{11H1F7L3}{11H1F7L3}{}{Ez. 19: 12-14}{}{}{}
\feast{11H2F1}
        {Deuxième dimanche de novembre}{Propre du Temps}{Deuxième semaine de novembre}{2}{}
        {}{}{}{}{}
\rubric{Si le mois de novembre n'a que quatre semaines avant le premier dimanche de l'Avent, la deuxième semaine de novembre est omise, et on emploie, ce jour et les suivants, les lectures de la troisième semaine.}
\reading{11H2F1N1L1}{11H2F1N1L1}{De Ezechiéle Prophéta}{Ez. 21: 6-10}{Ez}{21}{}
\reading{11H2F1N1L2}{11H2F1N1L2}{}{Ez. 21: 11-16}{}{}{}
\reading{11H2F1N1L3}{11H2F1N1L3}{}{Ez. 21: 17-20}{}{}{}
\feast{11H2F2}
        {Lundi}{Propre du Temps}{Deuxième semaine de novembre}{3}{}
        {}{}{}{}{}
\reading{11H2F2L1}{11H2F2L1}{De Ezechiéle Prophéta}{Ez. 33: 1-5}{Ez}{33}{}
\reading{11H2F2L2}{11H2F2L2}{}{Ez. 33: 6-8}{}{}{}
\reading{11H2F2L3}{11H2F2L3}{}{Ez. 33: 9-11}{}{}{}
\feast{11H2F3}
        {Mardi}{Propre du Temps}{Deuxième semaine de novembre}{3}{}
        {}{}{}{}{}
\reading{11H2F3L1}{11H2F3L1}{De Ezechiéle Prophéta}{Ez. 34: 1-4}{Ez}{34}{}
\reading{11H2F3L2}{11H2F3L2}{}{Ez. 34: 5-9}{}{}{}
\reading{11H2F3L3}{11H2F3L3}{}{Ez. 34: 10-12}{}{}{}
\feast{11H2F4}
        {Mercredi}{Propre du Temps}{Deuxième semaine de novembre}{3}{}
        {}{}{}{}{}
\reading{11H2F4L1}{11H2F4L1}{De Ezechiéle Prophéta}{Ez. 40: 1-2}{Ez}{40}{}
\reading{11H2F4L2}{11H2F4L2}{}{Ez. 40: 3-4}{}{}{}
\reading{11H2F4L3}{11H2F4L3}{}{Ez. 40: 5-6}{}{}{}
\feast{11H2F5}
        {Jeudi}{Propre du Temps}{Deuxième semaine de novembre}{3}{}
        {}{}{}{}{}
\reading{11H2F5L1}{11H2F5L1}{De Ezechiéle Prophéta}{Ez. 41: 1-3}{Ez}{41}{}
\reading{11H2F5L2}{11H2F5L2}{}{Ez. 41: 4-6}{}{}{}
\reading{11H2F5L3}{11H2F5L3}{}{Ez. 41: 7-10}{}{}{}
\feast{11H2F6}
        {Vendredi}{Propre du Temps}{Deuxième semaine de novembre}{3}{}
        {}{}{}{}{}
\reading{11H2F6L1}{11H2F6L1}{De Ezechiéle Prophéta}{Ez. 43: 1-5}{Ez}{43}{}
\reading{11H2F6L2}{11H2F6L2}{}{Ez. 43: 6-8}{}{}{}
\reading{11H2F6L3}{11H2F6L3}{}{Ez. 43: 9-11}{}{}{}
\feast{11H2F7}
        {Samedi}{Propre du Temps}{Deuxième semaine de novembre}{3}{}
        {}{}{}{}{}
\reading{11H2F7L1}{11H2F7L1}{De Ezechiéle Prophéta}{Ez. 47: 1-2}{Ez}{47}{}
\reading{11H2F7L2}{11H2F7L2}{}{Ez. 47: 3-5}{}{}{}
\reading{11H2F7L3}{11H2F7L3}{}{Ez. 47: 6-9}{}{}{}
\feast{11H3F1}
        {Troisième dimanche de novembre}{Propre du Temps}{Troisième semaine de novembre}{2}{}
        {}{}{}{}{}
\reading{11H3F1N1L1}{11H3F1N1L1}{Incipit liber Daniélis Prophétæ}{Dn. 1: 1-4}{Dn}{1}{}
\reading{11H3F1N1L2}{11H3F1N1L2}{}{Dn. 1: 5-9}{}{}{}
\reading{11H3F1N1L3}{11H3F1N1L3}{}{Dn. 1: 10-15}{}{}{}
\feast{11H3F2}
        {Lundi}{Propre du Temps}{Troisième semaine de novembre}{3}{}
        {}{}{}{}{}
\reading{11H3F2L1}{11H3F2L1}{De Daniéle Prophéta}{Dn. 2: 31-35}{Dn}{2}{}
\reading{11H3F2L2}{11H3F2L2}{}{Dn. 2: 36-40}{}{}{}
\reading{11H3F2L3}{11H3F2L3}{}{Dn. 2: 41-44}{}{}{}
\feast{11H3F3}
        {Mardi}{Propre du Temps}{Troisième semaine de novembre}{3}{}
        {}{}{}{}{}
\reading{11H3F3L1}{11H3F3L1}{De Daniéle Prophéta}{Dn. 3: 14-15}{Dn}{3}{}
\reading{11H3F3L2}{11H3F3L2}{}{Dn. 3: 16-19}{}{}{}
\reading{11H3F3L3}{11H3F3L3}{}{Dn. 3: 21-24}{}{}{}
\feast{11H3F4}
        {Mercredi}{Propre du Temps}{Troisième semaine de novembre}{3}{}
        {}{}{}{}{}
\reading{11H3F4L1}{11H3F4L1}{De Daniéle Prophéta}{Dn. 4: 16-19}{Dn}{4}{}
\reading{11H3F4L2}{11H3F4L2}{}{Dn. 4: 20-22}{}{}{}
\reading{11H3F4L3}{11H3F4L3}{}{Dn. 4: 22-25}{}{}{}
\feast{11H3F5}
        {Jeudi}{Propre du Temps}{Troisième semaine de novembre}{3}{}
        {}{}{}{}{}
\reading{11H3F5L1}{11H3F5L1}{De Daniéle Prophéta}{Dn. 5: 1-6}{Dn}{5}{}
\reading{11H3F5L2}{11H3F5L2}{}{Dn. 5: 13-17}{}{}{}
\reading{11H3F5L3}{11H3F5L3}{}{Dn. 5: 25-30; 6: 1}{Dn}{6}{}
\feast{11H3F6}
        {Vendredi}{Propre du Temps}{Troisième semaine de novembre}{3}{}
        {}{}{}{}{}
\reading{11H3F6L1}{11H3F6L1}{De Daniéle Prophéta}{Dn. 6: 12-16}{Dn}{6}{}
\reading{11H3F6L2}{11H3F6L2}{}{Dn. 6: 17-21}{}{}{}
\reading{11H3F6L3}{11H3F6L3}{}{Dn. 6: 22-25}{}{}{}
\feast{11H3F7}
        {Samedi}{Propre du Temps}{Troisième semaine de novembre}{3}{}
        {}{}{}{}{}
\reading{11H3F7L1}{11H3F7L1}{De Daniéle Prophéta}{Dn. 9: 1-5}{Dn}{9}{}
\reading{11H3F7L2}{11H3F7L2}{}{Dn. 9: 21-24}{}{}{}
\reading{11H3F7L3}{11H3F7L3}{}{Dn. 9: 25-27}{}{}{}
\feast{11H4F1}
        {Quatrième dimanche de novembre}{Propre du Temps}{Quatrième semaine de novembre}{2}{}
        {}{}{}{}{}
\reading{11H4F1N1L1}{11H4F1N1L1}{Incipit liber Osée Prophétæ}{Os. 1: 1-3}{Os}{1}{}
\reading{11H4F1N1L2}{11H4F1N1L2}{}{Os. 1: 4-7}{}{}{}
\reading{11H4F1N1L3}{11H4F1N1L3}{}{Os. 1: 8-9; 2: 1-2}{Os}{2}{}
\feast{11H4F2}
        {Lundi}{Propre du Temps}{Quatrième semaine de novembre}{3}{}
        {}{}{}{}{}
\reading{11H4F2L1}{11H4F2L1}{De Osée Prophéta}{Os. 4: 1-3}{Os}{4}{}
\reading{11H4F2L2}{11H4F2L2}{}{Os. 4: 4-6}{}{}{}
\reading{11H4F2L3}{11H4F2L3}{}{Os. 4: 7-10}{}{}{}
\feast{11H4F3}
        {Mardi}{Propre du Temps}{Quatrième semaine de novembre}{3}{}
        {}{}{}{}{}
\reading{11H4F3L1}{11H4F3L1}{Incipit Joël Prophéta}{Jl. 1: 1-4}{Jl}{1}{}
\reading{11H4F3L2}{11H4F3L2}{}{Jl. 1: 5-7}{}{}{}
\reading{11H4F3L3}{11H4F3L3}{}{Jl. 1: 8-11}{}{}{}
\feast{11H4F4}
        {Mercredi}{Propre du Temps}{Quatrième semaine de novembre}{3}{}
        {}{}{}{}{}
\reading{11H4F4L1}{11H4F4L1}{De Joéle Prophéta}{Jl. 4: 1-3}{Jl}{4}{}
\reading{11H4F4L2}{11H4F4L2}{}{Jl. 4: 4-7}{}{}{}
\reading{11H4F4L3}{11H4F4L3}{}{Jl. 4: 8-12}{}{}{}
\feast{11H4F5}
        {Jeudi}{Propre du Temps}{Quatrième semaine de novembre}{3}{}
        {}{}{}{}{}
\reading{11H4F5L1}{11H4F5L1}{Incipit Amos Prophéta}{Am. 1: 1-2}{Am}{1}{}
\reading{11H4F5L2}{11H4F5L2}{}{Am. 1: 3-5}{}{}{}
\reading{11H4F5L3}{11H4F5L3}{}{Am. 1: 6-8}{}{}{}
\feast{11H4F6}
        {Vendredi}{Propre du Temps}{Quatrième semaine de novembre}{3}{}
        {}{}{}{}{}
\reading{11H4F6L1}{11H4F6L1}{Incipit Abdías Prophéta}{Abd. 1: 1-4}{Abd}{1}{}
\reading{11H4F6L2}{11H4F6L2}{}{Abd. 1: 5-7}{}{}{}
\reading{11H4F6L3}{11H4F6L3}{}{Abd. 1: 8-11}{}{}{}
\feast{11H4F7}
        {Samedi}{Propre du Temps}{Quatrième semaine de novembre}{3}{}
        {}{}{}{}{}
\reading{11H4F7L1}{11H4F7L1}{Incipit Jonas Prophéta}{Jon. 1: 1-4}{Jon}{1}{}
\reading{11H4F7L2}{11H4F7L2}{}{Jon. 1: 5-7}{}{}{}
\reading{11H4F7L3}{11H4F7L3}{}{Jon. 1: 8-12}{}{}{}
\feast{11H5F1}
        {Cinquième dimanche de novembre}{Propre du Temps}{Cinquième semaine de novembre}{2}{}
        {}{}{}{}{}
\reading{11H5F1N1L1}{11H5F1N1L1}{Incipit Michǽas Prophéta}{Mi. 1: 1-3}{Mi}{1}{}
\reading{11H5F1N1L2}{11H5F1N1L2}{}{Mi. 1: 4-6}{}{}{}
\reading{11H5F1N1L3}{11H5F1N1L3}{}{Mi. 1: 7-9}{}{}{}
\feast{11H5F2}
        {Lundi}{Propre du Temps}{Cinquième semaine de novembre}{3}{}
        {}{}{}{}{}
\reading{11H5F2L1}{11H5F2L1}{Incipit Nahum Prophéta}{Na. 1: 1-4}{Na}{1}{}
\reading{11H5F2L2}{11H5F2L2}{}{Na. 1: 4-6}{}{}{}
\reading{11H5F2L3}{11H5F2L3}{}{Na. 1: 7-10}{}{}{}
\feast{11H5F3}
        {Mardi}{Propre du Temps}{Cinquième semaine de novembre}{3}{}
        {}{}{}{}{}
\reading{11H5F3L1}{11H5F3L1}{Incipit Hábacuc Prophéta}{Ha. 1: 1-4}{Ha}{1}{}
\reading{11H5F3L2}{11H5F3L2}{}{Ha. 1: 5-7}{}{}{}
\reading{11H5F3L3}{11H5F3L3}{}{Ha. 1: 8-10}{}{}{}
\feast{11H5F4}
        {Mercredi}{Propre du Temps}{Cinquième semaine de novembre}{3}{}
        {}{}{}{}{}
\reading{11H5F4L1}{11H5F4L1}{Incipit Sophonías Prophéta}{So. 1: 1-3}{So}{1}{}
\reading{11H5F4L2}{11H5F4L2}{}{So. 1: 4-6}{}{}{}
\reading{11H5F4L3}{11H5F4L3}{}{So. 1: 7-9}{}{}{}
\feast{11H5F5}
        {Jeudi}{Propre du Temps}{Cinquième semaine de novembre}{3}{}
        {}{}{}{}{}
\reading{11H5F5L1}{11H5F5L1}{Incipit Aggǽus Prophéta}{Ag. 1: 1-2}{Ag}{1}{}
\reading{11H5F5L2}{11H5F5L2}{}{Ag. 1: 3-6}{}{}{}
\reading{11H5F5L3}{11H5F5L3}{}{Ag. 1: 7-10}{}{}{}
\feast{11H5F6}
        {Vendredi}{Propre du Temps}{Cinquième semaine de novembre}{3}{}
        {}{}{}{}{}
\reading{11H5F6L1}{11H5F6L1}{Incipit Zacharías Prophéta}{Za. 1: 1-3}{Za}{1}{}
\reading{11H5F6L2}{11H5F6L2}{}{Za. 1: 4-5}{}{}{}
\reading{11H5F6L3}{11H5F6L3}{}{Za. 1: 6}{}{}{}
\feast{11H5F7}
        {Samedi}{Propre du Temps}{Cinquième semaine de novembre}{3}{}
        {}{}{}{}{}
\reading{11H5F7L1}{11H5F7L1}{Incipit Malachías Prophéta}{Ml. 1: 1-4}{Ml}{1}{}
\reading{11H5F7L2}{11H5F7L2}{}{Ml. 1: 5-7}{}{}{}
\reading{11H5F7L3}{11H5F7L3}{}{Ml. 1: 8-11}{}{}{}

\feast{CS}
        {Communs}{Communs}{Communs}{1}{}
        {}{}{}{}{}
\addcontentsline{toc}{chapter}{Communs}
\feast{VIAP}
        {Aux vigiles des apôtres}{Communs}{Aux vigiles des apôtres}{2}{}
        {}{}{}{}{}
\rubric{Lectures à l'Homéliaire.}
\feast{COAP}
        {Commun des apôtres}{Communs}{Commun des apôtres}{2}{}
        {}{}{}{}{}
\addcontentsline{toc}{section}{des apôtres}
\reading{COAPN1L1}{COAPN1L1}{De Epístola prima beáti Pauli Apóstoli ad Corínthios}{1 Co. 4: 1-5}{1 Co}{4}{}
\reading{COAPN1L2}{COAPN1L2}{}{1 Co. 4: 6-9}{}{}{}
\reading{COAPN1L3}{COAPN1L3}{}{1 Co. 4: 10-15}{}{}{}
\feast{COEV}
        {Commun des évangélistes}{Communs}{Commun des évangélistes}{2}{}
        {}{}{}{}{}
\addcontentsline{toc}{section}{des évangélistes}
\reading{COEVN1L1}{11H1F1N1L1}{Incipit liber Ezechiélis Prophétæ}{Ez. 1: 1-4}{Ez}{1}{}
\reading{COEVN1L2}{11H1F1N1L2}{}{Ez. 1: 5-9}{}{}{}
\reading{COEVN1L3}{11H1F1N1L3}{}{Ez. 1: 10-12}{}{}{}
\feast{COUM}
        {Commun d’un martyr hors du temps pascal}{Communs}{Commun d’un martyr}{2}{}
        {}{}{}{}{}
\addcontentsline{toc}{section}{des martyrs}
\rubric{Les lectures qui suivent sont employées lors de la fête d'un martyr pontife.
	Lors de la fête d'un martyr non pontife,
	on emploie les lectures \normaltext{Ainsi donc, frères, nous avons une dette}, p.\ \pageref{COPMN1L1}.}
\reading{COUMN1L1}{P2F4L1}{De Áctibus Apostolórum}{Ac. 20: 17-24}{Ac}{20}{}
\reading{COUMN1L2}{P2F4L2}{}{Ac. 20: 25-31}{}{}{}
\reading{COUMN1L3}{P2F4L3}{}{Ac. 20: 32-38}{}{}{}
\feast{COPM}
        {Commun de plusieurs martyrs hors du temps pascal}{Communs}{Commun de plusieurs martyrs}{2}{}
        {}{}{}{}{}
\reading{COPMN1L1}{COPMN1L1}{De Epístola beáti Pauli Apóstoli ad Romános}{Rm. 8: 12-19}{Rm}{8}{}
\reading{COPMN1L2}{COPMN1L2}{}{Rm. 8: 28-34}{}{}{}
\reading{COPMN1L3}{COPMN1L3}{}{Rm. 8: 35-39}{}{}{}
\feast{COPO}
        {Commun des confesseurs pontifes}{Communs}{Commun des confesseurs pontifes}{2}{}
        {}{}{}{}{}
\addcontentsline{toc}{section}{des confesseurs pontifes}
\reading{COPON1L1}{E5F2L1}{De Epístola prima beáti Pauli Apóstoli ad Timótheum}{1 Tm. 3: 1-7}{1 Tm}{3}{}
\reading{COPON1L2}{COPON1L2}{De Epístola ad Titum}{Tt. 1: 7-11}{Tt}{1}{}
\reading{COPON1L3}{COPON1L3}{}{Tt. 2: 1-8}{Tt}{2}{}
\feast{PMCP}
        {Lectures pour plusieurs confesseurs et pontifes}{Communs}{Commun des confesseurs pontifes}{3}{}
        {}{}{}{}{}
\rubric{On emploie ces lectures lors de la fête de plusieurs confesseurs.}
\reading{PMCPN1L1}{PMCPN1L1}{De libro Ecclesiástici}{Si. 44: 1-5}{Si}{44}{}
\reading{PMCPN1L2}{PMCPN1L2}{}{Si. 44: 6-9}{}{}{}
\reading{PMCPN1L3}{PMCPN1L3}{}{Si. 44: 10-15}{}{}{}
\feast{CODO}
        {Commun des docteurs}{Communs}{Commun des docteurs}{2}{}
        {}{}{}{}{}
\addcontentsline{toc}{section}{des docteurs}
\rubric{Ces lectures sont employées, que le docteur de l'Église que l'on fête soit pontife ou non.}
\reading{CODON1L1}{CODON1L1}{De libro Ecclesiástici}{Si. 39: 1-4}{Si}{39}{}
\reading{CODON1L2}{CODON1L2}{}{Si. 39: 5-7}{}{}{}
\reading{CODON1L3}{CODON1L3}{}{Si. 39: 8-10}{}{}{}
\feast{CONP}
        {Commun des confesseurs non-pontifes}{Communs}{Commun des confesseurs non-pontifes}{2}{}
        {}{}{}{}{}
\addcontentsline{toc}{section}{des confesseurs non-pontifes}
\rubric{Ces lectures sont également employées pour la fête d'un abbé.}
\reading{CONPN1L1}{CONPN1L1}{De libro Ecclesiástici}{Si. 31: 8-11}{Si}{31}{}
\reading{CONPN1L2}{CONPN1L2}{}{Si. 32: 14-16, 24; 33: 1-3}{Si}{32}{33}
\reading{CONPN1L3}{CONPN1L3}{}{Si. 34: 14-20}{Si}{34}{}
\rubric{Pour la fête de plusieurs confesseurs non-pontifes,
	on emploie les lectures \normaltext{Faisons l’éloge de ces hommes}, p.\ \pageref{PMCPN1L1}.}
\feast{ALCN}
        {Autres lectures pour les confesseurs non-pontifes}{Communs}{Commun des confesseurs non-pontifes}{3}{}
        {}{}{}{}{}
\rubric{Ces lectures sont employées là où il est indiqué ainsi dans le propre des saints.}
\reading{ALCNN1L1}{ALCNN1L1}{De libro Sapiéntiæ}{Sg. 4: 7-14}{Sg}{4}{}
\reading{ALCNN1L2}{ALCNN1L2}{}{Sg. 4: 14-19}{}{}{}
\reading{ALCNN1L3}{ALCNN1L3}{}{Sg. 4: 19-20; 5: 1-5}{Sg}{5}{}
\feast{MUVX}
        {Commun des vierges}{Communs}{Commun des vierges}{2}{}
        {}{}{}{}{}
\addcontentsline{toc}{section}{des vierges}
\reading{MUVXN1L1}{MUVXN1L1}{De Epístola prima beáti Pauli Apóstoli ad Corínthios}{1 Co. 7: 25-31}{1 Co}{7}{}
\reading{MUVXN1L2}{MUVXN1L2}{}{1 Co. 7: 32-35}{}{}{}
\reading{MUVXN1L3}{MUVXN1L3}{}{1 Co. 7: 36-40}{}{}{}
\feast{VXAL}
        {Autres lectures pour les vierges}{Communs}{Commun des vierges}{3}{}
        {}{}{}{}{}
\rubric{Ces lectures sont employées là où il est indiqué ainsi dans le propre des saints, surtout pour les vierges martyres.}
\reading{VXALN1L1}{VXALN1L1}{De libro Ecclesiástici}{Si. 51: 1-6}{Si}{51}{}
\reading{VXALN1L2}{VXALN1L2}{}{Si. 51: 6-8}{}{}{}
\reading{VXALN1L3}{VXALN1L3}{}{Si. 51: 9-12}{}{}{}
\feast{MUNX}
        {Commun des saintes femmes}{Communs}{Commun des saintes femmes}{2}{}
        {}{}{}{}{}
\addcontentsline{toc}{section}{des saintes femmes}
\rubric{Pour la fête d'une sainte martyre, on emploie les lectures \normaltext{Je veux te rendre grâce}, p.\ \pageref{VXALN1L1}.
	Pour une sainte qui n'est ni vierge ni martyre, on emploie les lectures suivantes.}
\reading{MUNXN1L1}{MUNXN1L1}{De Parábolis Salomónis}{Pr. 31: 10-17}{Pr}{31}{}
\reading{MUNXN1L2}{MUNXN1L2}{}{Pr. 31: 18-24}{}{}{}
\reading{MUNXN1L3}{MUNXN1L3}{}{Pr. 31: 25-31}{}{}{}
\feast{CDED}
        {Commun de la Dédicace}{Communs}{Commun de la Dédicace}{2}{}
        {}{}{}{}{}
\addcontentsline{toc}{section}{de la Dédicace}
\reading{CDEDN1L1}{CDEDN1L1}{De libro secundo Paralipómenon}{2 Ch. 7: 1-5}{2 Ch}{7}{}
\reading{CDEDN1L2}{CDEDN1L2}{}{2 Ch. 7: 6-9}{}{}{}
\reading{CDEDN1L3}{CDEDN1L3}{}{2 Ch. 7: 11-16}{}{}{}
\rubric{Pendant l'Octave et le jour Octave, lectures scripturaires de l'Écriture courante; à défaut, lectures scripturaires de la fête.}
\feast{CBMV}
        {Commun de la Bienheureuse Vierge Marie}{Communs}{Commun de la Bienheureuse Vierge Marie}{2}{}
        {}{}{}{}{}
\addcontentsline{toc}{section}{de la Bienheureuse Vierge Marie}
\reading{CBMVN1L1}{08H1F4L3}{De Parábolis Salomónis}{Pr. 8: 12-17}{Pr}{8}{}
\reading{CBMVN1L2}{CBMVN1L2}{}{Pr. 8: 18-25}{}{}{}
\reading{CBMVN1L3}{CBMVN1L3}{}{Pr. 8: 34-36; 9: 1-5}{Pr}{9}{}
\feast{CSMS}
        {Office de Sainte Marie le samedi}{Communs}{Office de Sainte Marie le samedi}{2}{}
        {}{}{}{}{}
\rubric{Le samedi, hors des temps privilégiés, s’il n'est pas fêté de fête semi-double ou double, on emploie pour la troisième lecture de l'unique nocturne, au lieu de l'Écriture courante, une lecture qui figure à sa place dans les Sermons \& Légendes.}
\feast{OPBM}
        {Petit Office de la Vierge Marie, en dehors de l'Avent}{Communs}{Petit Office de la Vierge Marie}{2}{}
        {}{}{}{}{}
\addcontentsline{toc}{section}{Petit Office de la Vierge}
\reading{OPBML1}{OPBML1}{}{Si. 24: 7-8}{Si}{24}{}
\reading{OPBML2}{OPBML2}{}{Si. 24: 10-12}{}{}{}
\reading{OPBML3}{OPBML3}{}{Si. 24: 13-15}{}{}{}
\feast{OPADV}
        {Petit Office de la Vierge Marie, pendant l’Avent}{Communs}{Petit Office de la Vierge Marie}{2}{}
        {}{}{}{}{}
\reading{OPADVL1}{OPADVL1}{}{Lc. 1: 26-28}{Lc}{1}{}
\reading{OPADVL2}{OPADVL2}{}{Lc. 1: 29-33}{}{}{}
\reading{OPADVL3}{OPADVL3}{}{Lc. 1: 34-38}{}{}{}

\cleartoleftpage{}
\feast{ODEF}
        {Office des défunts}{Communs}{Office des défunts}{2}{}
        {}{}{}{}{}
\addcontentsline{toc}{section}{Office des défunts}
\rubric{On peut toujours célébrer les trois nocturnes. En dehors des jours où la célébration des trois nocturnes est obligatoire, on emploie le premier le dimanche, le lundi et le jeudi, le deuxième le mardi et le vendredi, et le troisième le mercredi et le samedi. Les lectures se chantent sans absolution, ni bénédiction, ni titre, ni conclusion.}
\begin{paracol}[1]*{2}
\intermediatetitle{Premier nocturne}
\switchcolumn
\nocturn{1}
\switchcolumn*
\reading{ODEFN1L1}{ODEFN1L1}{}{Jb. 7: 16-21}{Jb}{7}{}
\switchcolumn
\reading{ODEFN1L1lat}{ODEFN1L1lat}{}{\null}{}{}{}
\switchcolumn*
\reading{ODEFN1L2}{ODEFN1L2}{}{Jb. 10: 1-7}{Jb}{10}{}
\switchcolumn
\reading{ODEFN1L2lat}{ODEFN1L2lat}{}{\null}{}{}{}
\switchcolumn*
\reading{ODEFN1L3}{ODEFN1L3}{}{Jb. 10: 8-12}{}{}{}
\switchcolumn
\reading{ODEFN1L3lat}{ODEFN1L3lat}{}{\null}{}{}{}
\newpage
\switchcolumn*
\intermediatetitle{Deuxième nocturne}
\switchcolumn
\nocturn{2}
\switchcolumn*
\reading{ODEFN2L1}{ODEFN2L1}{}{Jb. 13: 22-28}{Jb}{13}{}
\switchcolumn
\reading{ODEFN2L1lat}{ODEFN2L1lat}{}{\null}{}{}{}
\switchcolumn*
\reading{ODEFN2L2}{ODEFN2L2}{}{Jb. 14: 1-6}{Jb}{14}{}
\switchcolumn
\reading{ODEFN2L2lat}{ODEFN2L2lat}{}{\null}{}{}{}
\switchcolumn*
\reading{ODEFN2L3}{ODEFN2L3}{}{Jb. 14: 13-16}{}{}{}
\switchcolumn
\reading{ODEFN2L3lat}{ODEFN2L3lat}{}{\null}{}{}{}
\newpage
\switchcolumn*
\intermediatetitle{Troisième nocturne}
\switchcolumn
\nocturn{3}
\switchcolumn*
\reading{ODEFN3L1}{ODEFN3L1}{}{Jb. 17: 1-3, 11-15}{Jb}{17}{}
\switchcolumn
\reading{ODEFN3L1lat}{ODEFN3L1lat}{}{\null}{}{}{}
\switchcolumn*
\reading{ODEFN3L2}{ODEFN3L2}{}{Jb. 19: 20-27}{Jb}{19}{}
\switchcolumn
\reading{ODEFN3L2lat}{ODEFN3L2lat}{}{\null}{}{}{}
\switchcolumn*
\reading{ODEFN3L3}{ODEFN3L3}{}{Jb. 10: 18-22}{Jb}{10}{}
\switchcolumn
\reading{ODEFN3L3lat}{ODEFN3L3lat}{}{\null}{}{}{}
\end{paracol}

\feast{PRSA}
        {Propre des Saints}{Propre des Saints}{Propre des Saints}{1}{}
        {}{}{}{}{}
\addcontentsline{toc}{chapter}{Propre des Saints}
\rubric{À toutes les fêtes à trois nocturnes du Seigneur, de la Bienheureuse Vierge Marie, des Anges, de saint Jean Baptiste, de saint Joseph, des Apôtres, des Évangélistes, et aux fêtes de première et deuxième classe, les lectures scripturaires sont prises au propre, ou, à défaut, au commun, selon les rubriques. À toutes les autres fêtes à trois nocturnes, les lectures scripturaires sont de l'Écriture courante, soit celles du jour, soit d'autres anticipées ou transférées, selon les rubriques. En l'absence de lectures scripturaires de l'Écriture courante, les lectures scripturaires sont prises au commun, sauf mention contraire; toujours en l'absence de lectures scripturaires de l'Écriture courante et de lectures propres, pour les docteurs, les lectures sont prises au commun des docteurs, que ceux-ci soient pontifes, abbés, martyrs, ou simples confesseurs. Pour le jour Octave dans le cas d'une Octave simple, et pour les jours dans l'Octave dans le cas d'une Octave commune, les lectures scripturaires sont de l'Écriture courante, sauf si elles sont propres.}

\rubric{Dans la suite ne figurent que les saints qui ont des lectures scripturaires propres,
	et ceux qui nécessitent des rubriques spécifiques,
	particulièrement ceux dont les communs contiennent plusieurs choix de lectures scripturaires.}

\feast{0118}
        {Chaire de Saint Pierre, apôtre, à Rome}{Propre des Saints}{Janvier}{2}{18 janvier}
        {}{}{Pierre et Paul, apôtres!Chaire de Saint Pierre à Rome}{}{}
\reading{0118N1L1}{P5F1N1L1}{Incipit Epístola prima beáti Petri Apóstoli}{1 P. 1: 1-5}{1 P}{1}{}
\reading{0118N1L2}{0222N1L2}{}{1 P. 1: 6-9}{}{}{}
\reading{0118N1L3}{0222N1L3}{}{1 P. 1: 10-12}{}{}{}
\feast{0121}
        {Sainte Agnès, vierge et martyre}{Propre des Saints}{Janvier}{2}{21 janvier}
        {}{}{Agnès, vierge et martyre}{}{}
\rubric{On emploie les lectures \normaltext{Je veux te rendre grâce}, p.\ \pageref{VXALN1L1}, au lieu de l'Écriture courante.}
\feast{0125}
        {Conversion de Saint Paul, apôtre}{Propre des Saints}{Janvier}{2}{25 janvier}
        {}{}{Pierre et Paul, apôtres!Conversion de Saint Paul}{}{}
\reading{0125N1L1}{0125N1L1}{De Actibus Apostolórum}{Ac. 9: 1-5}{Ac}{9}{}
\reading{0125N1L2}{0125N1L2}{}{Ac. 9: 6-9}{}{}{}
\reading{0125N1L3}{0125N1L3}{}{Ac. 9: 10-16}{}{}{}
\feast{0126}
        {Saint Polycarpe, évêque et martyr}{Propre des Saints}{Janvier}{2}{26 janvier}
        {}{}{Polycarpe, évêque et martyr}{}{}
\rubric{Si les lectures scripturaires doivent être prises au commun,
	lectures \normaltext{Depuis Milet}, p.\ \pageref{COUMN1L1}.}
\feast{0130}
        {Sainte Martine, vierge et martyre}{Propre des Saints}{Janvier}{2}{30 janvier}
        {}{}{Martine, vierge et martyre}{}{}
\rubric{Si les lectures scripturaires doivent être prises au commun,
        lectures \normaltext{Je veux te rendre grâce}, p.\ \pageref{VXALN1L1}.}
\feast{0202}
        {Purification de la Bienheureuse Vierge Marie}{Propre des Saints}{Février}{2}{2 février}
        {}{}{Marie, la Très Sainte Vierge!Purification}{}{}
\reading{0202N1L1}{0202N1L1}{De libro Éxodi}{Ex. 13: 1-3, 11-13}{Ex}{13}{}
\reading{0202N1L2}{0202N1L2}{De libro Levítici}{Lv. 12: 1-5}{Lv}{12}{}
\reading{0202N1L3}{0202N1L3}{}{Lv. 12: 6-8}{}{}{}
\feast{0205}
        {Sainte Agathe, vierge et martyre}{Propre des Saints}{Février}{2}{5 février}
        {}{}{Agathe, vierge et martyre}{}{}
\rubric{On emploie les lectures \normaltext{Je veux te rendre grâce}, p.\ \pageref{VXALN1L1}, au lieu de l'Écriture courante.}
\feast{0211}
        {Apparition de la Vierge Immaculée}{Propre des Saints}{Février}{2}{11 février}
        {}{}{Marie, la Très Sainte Vierge!Apparition de la Vierge Immaculée}{}{}
\reading{0211N1L1}{08H1F4L3}{De Parábolis Salomónis}{Pr. 8: 12-17}{Pr}{8}{}
\reading{0211N1L2}{CBMVN1L2}{}{Pr. 8: 18-25}{}{}{}
\reading{0211N1L3}{CBMVN1L3}{}{Pr. 8: 34-36; 9: 1-5}{Pr}{9}{}
\feast{0212}
        {Les Sept Saints Fondateurs de l’ordre des Servites, confesseurs}{Propre des Saints}{Février}{2}{12 février}
        {}{}{Sept Fondateurs de l’ordre des Servites, confesseurs}{}{}
\rubric{Si les lectures scripturaires doivent être prises au commun,
        lectures \normaltext{Faisons l’éloge de ces hommes}, p.\ \pageref{PMCPN1L1}.}
\feast{0222}
        {Chaire de Saint Pierre, apôtre, à Antioche}{Propre des Saints}{Février}{2}{22 février}
        {}{}{Pierre et Paul, apôtres!Chaire de Saint Pierre à Antioche}{}{}
\reading{0222N1L1}{P5F1N1L1}{Incipit Epístola prima beáti Petri Apóstoli}{1 P. 1: 1-5}{1 P}{1}{}
\reading{0222N1L2}{0222N1L2}{}{1 P. 1: 6-9}{}{}{}
\reading{0222N1L3}{0222N1L3}{}{1 P. 1: 10-12}{}{}{}
\feast{0224b}
        {Vigile de Saint Matthias, apôtre}{Propre des Saints}{Février}{2}{24 février, les années bissextiles en dehors du Carême}
        {}{}{Matthias, apôtre!Vigile}{}{}
\rubric{Lectures à l'Homéliaire.}
\feast{0224}
        {Saint Matthias, apôtre}{Propre des Saints}{Février}{2}{24 février, le 25 les années bissextiles}
        {}{}{Matthias, apôtre}{}{}
\reading{0224N1L1}{0224N1L1}{De Actibus Apostolórum}{Ac. 1: 15-18}{Ac}{1}{}
\reading{0224N1L2}{0224N1L2}{}{Ac. 1: 19-22}{}{}{}
\reading{0224N1L3}{0224N1L3}{}{Ac. 1: 23-26}{}{}{}
\feast{0227}
        {Saint Gabriel de la Vierge des Douleurs, confesseur}{Propre des Saints}{Février}{2}{27 février, le 28 les années bissextiles}
        {}{}{Gabriel de la Vierge des Douleurs, confesseur}{}{}
\rubric{Si les lectures scripturaires doivent être prises au commun,
        lectures \normaltext{Même s’il meurt}, p.\ \pageref{ALCNN1L1}.}
\feast{0304}
        {Saint Casimir, confesseur}{Propre des Saints}{Mars}{2}{4 mars}
        {}{}{Casimir, confesseur}{}{}
\rubric{Si les lectures scripturaires doivent être prises au commun,
        lectures \normaltext{Même s’il meurt}, p.\ \pageref{ALCNN1L1}.}
\feast{0306}
        {Saintes Perpétue et Félicité, Martyres}{Propre des Saints}{Mars}{2}{6 mars}
        {}{}{Perpétue et Félicité, Martyres}{}{}
\rubric{En l'absence d'Écriture courante,
        lectures \normaltext{Je veux te rendre grâce}, p.\ \pageref{VXALN1L1}.}
\feast{0319}
        {Saint Joseph, époux de la Bienheureuse Vierge Marie}{Propre des Saints}{Mars}{2}{19 mars}
        {}{}{Joseph, époux de la Bienheureuse Vierge Marie}{}{}
\reading{0319N1L1}{0319N1L1}{De libro Génesis}{Gn. 39: 1-5}{Gn}{39}{}
\reading{0319N1L2}{0319N1L2}{}{Gn. 41: 37-40}{Gn}{41}{}
\reading{0319N1L3}{0319N1L3}{}{Gn. 41: 41-44}{}{}{}
\feast{0321}
        {Saint Benoît, abbé}{Propre des Saints}{Mars}{2}{21 mars}
        {}{}{Benoît, abbé}{}{}
\rubric{Lectures \normaltext{Faisons l’éloge de ces hommes}, p.\ \pageref{PMCPN1L1}.}
\feast{0324}
        {Saint Gabriel, archange}{Propre des Saints}{Mars}{2}{24 mars}
        {}{}{Gabriel, archange}{}{}
\reading{0324N1L1}{0324N1L1}{De Daniéle Prophéta}{Dn. 9: 20-23}{Dn}{9}{}
\reading{0324N1L2}{0324N1L2}{}{Dn. 9: 24-25}{}{}{}
\reading{0324N1L3}{0324N1L3}{}{Dn. 9: 26-27}{}{}{}
\feast{0325}
        {Annonciation de la Bienheureuse Vierge Marie}{Propre des Saints}{Mars}{1}{25 mars}
        {}{}{Marie, la Très Sainte Vierge!Annonciation}{}{}
\reading{0325N1L1}{A1F7L3}{De Isaía Prophéta}{Is. 7: 10-15}{Is}{7}{}
\reading{0325N1L2}{0325N1L2}{}{Is. 11: 1-5}{Is}{11}{}
\reading{0325N1L3}{A4F1N1L1}{}{Is. 35: 1-7}{Is}{35}{}
\feast{Q5F6b}
        {Notre Dame des Sept Douleurs}{Propre des Saints}{Mars}{2}{Vendredi de la Passion}
        {}{}{Marie, la Très Sainte Vierge!Notre Dame des Sept Douleurs}{}{}
\reading{Q5F6bN1L1}{Q5F6bN1L1}{De Isaía Prophéta}{Is. 53: 1-5}{Is}{53}{}
\reading{Q5F6bN1L2}{Q5F6bN1L2}{}{Is. 53: 6-9}{}{}{}
\reading{Q5F6bN1L3}{Q5F6bN1L3}{}{Is. 53: 10-12}{}{}{}
\feast{0411}
        {Saint Léon le Grand, pape, confesseur et docteur de l’Église}{Propre des Saints}{Avril}{2}{11 avril}
        {}{}{Léon le Grand, pape, confesseur et docteur de l’Église}{}{}
\reading{0411N1L1}{P5F1N1L1}{Incipit Epístola prima beáti Petri Apóstoli}{1 P. 1: 1-5}{1 P}{1}{}
\reading{0411N1L2}{P5F1N1L2}{}{1 P. 1: 6-12}{}{}{}
\reading{0411N1L3}{P5F1N1L3}{}{1 P. 1: 13-21}{}{}{}
\feast{0413}
        {Saint Herménégilde, martyr}{Propre des Saints}{Avril}{2}{13 avril}
        {}{}{Herménégilde, martyr}{}{}
\rubric{En l'absence d'Écriture courante, ou si les lectures scripturaires doivent être prises au commun,
	lectures \normaltext{Ainsi donc, frères, nous avons une dette}, p.\ \pageref{COPMN1L1}.}
\feast{0414}
        {Saint Justin, martyr}{Propre des Saints}{Avril}{2}{14 avril}
        {}{}{Justin, martyr}{}{}
\rubric{En l'absence d'Écriture courante, ou si les lectures scripturaires doivent être prises au commun,
	lectures \normaltext{Ainsi donc, frères, nous avons une dette}, p.\ \pageref{COPMN1L1}.}
\feast{0425}
        {Saint Marc, évangéliste}{Propre des Saints}{Avril}{2}{25 avril}
        {}{}{Marc, évangéliste}{}{}
\rubric{Lectures \normaltext{La trentième année}, p.\ \pageref{COEVN1L1}.}
\feast{P2F4b}
        {Saint Joseph, patron de l’Église Universelle}{Propre des Saints}{Avril}{2}{Mercredi de la deuxième semaine après Pâques}
        {}{}{Joseph!Patron de l’Église Universelle}{}{}
\reading{P2F4bN1L1}{P2F4bN1L1}{De libro Génesis}{Gn. 39: 1-6}{Gn}{39}{}
\reading{P2F4bN1L2}{P2F4bN1L2}{}{Gn. 41: 37-43}{Gn}{41}{}
\reading{P2F4bN1L3}{P2F4bN1L3}{}{Gn. 41: 44-49}{}{}{}
\rubric{Pendant l'Octave et le jour Octave, lectures scripturaires de l'Écriture courante.}
\feast{0501}
        {Saints Philippe et Jacques, apôtres}{Propre des Saints}{Mai}{2}{1\ier mai (usage ancien)}
        {}{}{Philippe et Jacques, apôtres}{}{}
\rubric{Si l'Écriture courante est de l'épître de saint Jacques, on l'emploie. Sinon, on emploie les lectures qui suivent.}
\reading{0501N1L1}{P4F1N1L1}{Incipit Epístola cathólica beáti Jacóbi Apóstoli}{Jc. 1: 1-6}{Jc}{1}{}
\reading{0501N1L2}{P4F1N1L2}{}{Jc. 1: 6-11}{}{}{}
\reading{0501N1L3}{P4F1N1L3}{}{Jc. 1: 12-16}{}{}{}
\feast{0501b}
        {Saint Joseph, artisan}{Propre des Saints}{Mai}{2}{1\ier mai (usage récent)}
        {}{}{Joseph!Aartisan}{}{}
\reading{0501bN1L1}{0501bN1L1}{De libro Génesis}{Gn. 1: 27-28, 31; 2: 1-3}{Gn}{1}{2}
\reading{0501bN1L2}{0501bN1L2}{}{Gn. 2: 7-9, 15}{}{}{}
\reading{0501bN1L3}{0501bN1L3}{}{Gn. 3: 17-19, 23-24}{Gn}{3}{}
\feast{0503}
        {Invention de la Sainte Croix}{Propre des Saints}{Mai}{2}{3 mai}
        {}{}{Notre-Seigneur Jésus-Christ!Croix!Invention}{}{}
\reading{0503N1L1}{0503N1L1}{De Epístola beáti Pauli Apóstoli ad Gálatas}{Ga. 3: 10-14}{Ga}{3}{}
\reading{0503N1L2}{0503N1L2}{De Epístola ad Philippénses}{Ph. 2: 5-11}{Ph}{2}{}
\reading{0503N1L3}{0503N1L3}{De Epístola ad Colossénses}{Col. 2: 9-15}{Col}{2}{}
\feast{0506}
        {Saint Jean devant la Porte latine, apôtre et évangéliste}{Propre des Saints}{Mai}{2}{6 mai}
        {}{}{Jean, apôtre et évangéliste!devant la Porte latine}{}{}
\rubric{Si l'Écriture courante est des épîtres de saint Jean, ou de l'Apocalypse, on l'emploie. Sinon, on emploie les lectures qui suivent.}
\reading{0506N1L1}{P6F1N1L1}{Incipit Epístola prima beáti Joánnis Apóstoli}{1 Jn. 1: 1-5}{1 Jn}{1}{}
\reading{0506N1L2}{P6F1N1L2}{}{1 Jn. 1: 6-10}{}{}{}
\reading{0506N1L3}{P6F1N1L3}{}{1 Jn. 2: 1-6}{1 Jn}{2}{}
\feast{0508}
        {Apparition de Saint Michel, archange}{Propre des Saints}{Mai}{2}{8 mai}
        {}{}{Michel, archange!Apparition}{}{}
\reading{0508N1L1}{0508N1L1}{De Daniéle Prophéta}{Dn. 7: 9-11}{Dn}{7}{}
\reading{0508N1L2}{0508N1L2}{}{Dn. 10: 4-8}{Dn}{10}{}
\reading{0508N1L3}{0508N1L3}{}{Dn. 10: 9-14}{}{}{}
\feast{0511}
        {Saints Philippe et Jacques, apôtres}{Propre des Saints}{Mai}{2}{11 mai (usage récent)}
        {}{}{Philippe et Jacques, apôtres}{}{}
\rubric{Tout comme pour la fête ancienne, p.\ \pageref{0501}.}
\feast{0531b}
        {La Bienheureuse Vierge Marie, reine}{Propre des Saints}{Mai}{2}{31 mai (usage plus récent)}
        {}{}{Marie, la Très Sainte Vierge!Reine}{}{}
\reading{0531bN1L1}{1011N1L1}{De Libro Ecclesiástici}{Si. 24: 3-7}{Si}{24}{}
\reading{0531bN1L2}{0531bN1L2}{}{Si. 24: 9-12}{}{}{}
\reading{0531bN1L3}{1007N1L3}{}{Si. 24: 18-22}{}{}{}
\feast{0611}
        {Saint Barnabé, apôtre}{Propre des Saints}{Juin}{2}{11 juin}
        {}{}{Barnabé, apôtre}{}{}
\reading{0611N1L1}{0611N1L1}{De Actibus Apostolórum}{Ac. 13: 43-47}{Ac}{13}{}
\reading{0611N1L2}{0611N1L2}{}{Ac. 13: 48-52}{}{}{}
\reading{0611N1L3}{0611N1L3}{}{Ac. 14: 1-3}{Ac}{14}{}
\feast{0623}
        {Vigile de la Nativité de Saint Jean Baptiste}{Propre des Saints}{Juin}{2}{23 juin}
        {}{}{Jean Baptiste!Nativité!Vigile}{}{}
\rubric{Lectures à l'Homéliaire.}
\feast{0624}
        {Nativité de Saint Jean Baptiste}{Propre des Saints}{Juin}{2}{24 juin}
        {}{}{Jean Baptiste!Nativité}{}{}
\reading{0624N1L1}{0624N1L1}{Incipit liber Jeremíæ Prophétæ}{Jr. 1: 1-5}{Jr}{1}{}
\reading{0624N1L2}{0624N1L2}{}{Jr. 1: 6-10}{}{}{}
\reading{0624N1L3}{0624N1L3}{}{Jr. 1: 17-19}{}{}{}
\feast{0626}
        {Saints Jean et Paul, martyrs}{Propre des Saints}{Juin}{2}{26 juin}
        {}{}{Jean et Paul, martyrs}{}{}
\rubric{Si les lectures scripturaires doivent être prises au commun,
	lectures \normaltext{Ainsi donc, frères, nous avons une dette}, p.\ \pageref{COPMN1L1}.}
\feast{0628b}
        {Vigile des Saints Pierre et Paul, apôtres}{Propre des Saints}{Juin}{2}{28 juin (usage plus récent)}
        {}{}{Pierre et Paul, apôtres!Vigile}{}{}
\rubric{Si on ne fête pas saint Irénée le même jour, lectures à l'Homéliaire.}
\feast{0629}
        {Saints Pierre et Paul, apôtres}{Propre des Saints}{Juin}{2}{29 juin}
        {}{}{Pierre et Paul, apôtres}{}{}
\reading{0629N1L1}{0629N1L1}{De Actibus Apostolórum}{Ac. 3: 1-5}{Ac}{3}{}
\reading{0629N1L2}{0629N1L2}{}{Ac. 3: 6-10}{}{}{}
\reading{0629N1L3}{0629N1L3}{}{Ac. 3: 11-16}{}{}{}
\rubric{Pendant l'Octave et le jour Octave, lectures scripturaires de l'Écriture courante, sauf quand elles sont propres.}
\feast{0630}
        {Commémoraison de Saint Paul, apôtre}{Propre des Saints}{Juin}{2}{30 juin}
        {}{}{Pierre et Paul, apôtres!Commémoraison de Saint Paul}{}{}
\reading{0630N1L1}{0630N1L1}{De Actibus Apostolórum}{Ac. 13: 1-4}{Ac}{13}{}
\reading{0630N1L2}{0630N1L2}{}{Ac. 13: 5-8}{}{}{}
\reading{0630N1L3}{0630N1L3}{}{Ac. 13: 9-13}{}{}{}
\rubric{Si on fête le 1\ier juillet l'Octave de la Nativité de saint Jean Baptiste et non le Très Précieux Sang du Seigneur,
	lectures scripturaires de l'Écriture courante.}
\feast{0701}
        {Fête du Très Précieux Sang du Seigneur}{Propre des Saints}{Juillet}{2}{1\ier juillet}
        {}{}{Notre-Seigneur Jésus-Christ!Précieux Sang}{}{}
\reading{0701N1L1}{0701N1L1}{De Epístola beáti Pauli Apóstoli ad Hebrǽos}{He. 9: 11-15}{He}{9}{}
\reading{0701N1L2}{0701N1L2}{}{He. 9: 16-22}{}{}{}
\reading{0701N1L3}{0701N1L3}{}{He. 10: 19-24}{He}{10}{}
\feast{0702}
        {Visitation de la Bienheureuse Vierge Marie}{Propre des Saints}{Juillet}{2}{2 juillet}
        {}{}{Marie, la Très Sainte Vierge!Visitation}{}{}
\reading{0702N1L1}{0702N1L1}{De Cánticis canticórum}{Ct. 2: 1-7}{Ct}{2}{}
\reading{0702N1L2}{0702N1L2}{}{Ct. 2: 8-13}{}{}{}
\reading{0702N1L3}{0702N1L3}{}{Ct. 2: 13-17}{}{}{}
\feast{0707}
        {Saints Cyrille et Méthode, évêques et confesseurs}{Propre des Saints}{Juillet}{2}{7 juillet}
        {}{}{Cyrille et Méthode, évêques et confesseurs}{}{}
\rubric{Si les lectures scripturaires doivent être prises au commun,
        lectures \normaltext{Faisons l’éloge de ces hommes}, p.\ \pageref{PMCPN1L1}.}
\feast{0716}
        {Mémoire de Notre Dame du Mont Carmel}{Propre des Saints}{Juillet}{2}{16 juillet}
        {}{}{Marie, la Très Sainte Vierge!Notre Dame du Mont Carmel}{}{}
\rubric{Lectures \normaltext{Moi, la Sagesse}, p.\ \pageref{CBMVN1L1}.}
\feast{0722}
        {Sainte Marie Madeleine, pénitente}{Propre des Saints}{Juillet}{2}{22 juillet}
        {}{}{Marie Madeleine, pénitente}{}{}
\reading{0722N1L1}{0722N1L1}{De Cánticis canticórum}{Ct. 3: 1-4}{Ct}{3}{}
\reading{0722N1L2}{0722N1L2}{}{Ct. 8: 1-4}{Ct}{8}{}
\reading{0722N1L3}{0722N1L3}{}{Ct. 8: 5-7}{}{}{}
\feast{0724}
        {Vigile de Saint Jacques, apôtre}{Propre des Saints}{Juillet}{2}{24 juillet (usage ancien)}
        {}{}{Jacques le Majeur, apôtre!Vigile}{}{}
\rubric{Si on ne fête pas sainte Christine le même jour, lectures à l'Homéliaire.}
\feast{0725}
        {Saint Jacques le Majeur, apôtre}{Propre des Saints}{Juillet}{2}{25 juillet}
        {}{}{Jacques le Majeur, apôtre}{}{}
\rubric{Lectures \normaltext{Que l’on nous regarde}, p.\ \pageref{COAPN1L1}.}
\feast{0726}
        {Sainte Anne, mère de la Bienheureuse Vierge Marie}{Propre des Saints}{Juillet}{2}{26 juillet}
        {}{}{Anne, mère de la Bienheureuse Vierge Marie}{}{}
\rubric{Lectures \normaltext{Une femme parfaite}, p.\ \pageref{MUNXN1L1}.}
\feast{0728}
        {Saints Nazaire et Celse, martyrs, Victor I\ier, pape et martyr, et Innocent I\ier, pape}{Propre des Saints}{Juillet}{2}{28 juillet}
        {}{}{Nazaire et Celse, martyrs, Victor I\ier, pape et martyr, et Innocent I\ier, pape}{}{}
\rubric{Si les lectures scripturaires doivent être prises au commun,
	lectures \normaltext{Ainsi donc, frères, nous avons une dette}, p.\ \pageref{COPMN1L1}.}
\feast{0801}
        {Saint Pierre aux Liens, apôtre}{Propre des Saints}{Août}{2}{1\ier août}
        {}{}{Pierre et Paul, apôtres!Saint Pierre aux Liens}{}{}
\reading{0801N1L1}{0801N1L1}{De Actibus Apostolórum}{Ac. 12: 1-5}{Ac}{12}{}
\reading{0801N1L2}{0801N1L2}{}{Ac. 12: 6-8}{}{}{}
\reading{0801N1L3}{0801N1L3}{}{Ac. 12: 9-11}{}{}{}
\feast{0803}
        {Invention de Saint Étienne, protomartyr}{Propre des Saints}{Août}{2}{3 août}
        {}{}{Etienne@Étienne, protomartyr!Invention}{}{}
\reading{0803N1L1}{0803N1L1}{De Actibus Apostolórum}{Ac. 7: 51-54}{Ac}{7}{}
\reading{0803N1L2}{0803N1L2}{}{Ac. 7: 55-59}{}{}{}
\reading{0803N1L3}{0803N1L3}{}{Ac. 7: 60; 8: 1-2}{Ac}{8}{}
\feast{0805}
        {Dédicace de Sainte Marie aux Neiges}{Propre des Saints}{Août}{2}{5 août}
        {}{}{Marie, la Très Sainte Vierge!Dédicace de Sainte Marie aux Neiges}{}{}
\rubric{Lectures \normaltext{Moi, la Sagesse}, p.\ \pageref{CBMVN1L1}.}
\feast{0806}
        {Transfiguration du Seigneur}{Propre des Saints}{Août}{2}{6 août}
        {}{}{Notre-Seigneur Jésus-Christ!Transfiguration}{}{}
\reading{0806N1L1}{0806N1L1}{De Epístola secúnda beáti Petri Apóstoli}{2 P. 1: 10-14}{2 P}{1}{}
\reading{0806N1L2}{0806N1L2}{}{2 P. 1: 15-17}{}{}{}
\reading{0806N1L3}{0806N1L3}{}{2 P. 1: 18-21}{}{}{}
\feast{0809b}
        {Vigile de Saint Laurent, martyr}{Propre des Saints}{Août}{2}{9 août}
        {}{}{Laurent, martyr!Vigile}{}{}
\rubric{Si on ne fête pas saint Jean-Marie Vianney le même jour, lectures à l'Homéliaire.}
\feast{0810}
        {Saint Laurent, martyr}{Propre des Saints}{Août}{2}{10 août}
        {}{}{Laurent, martyr}{}{}
\reading{0810N1L1}{VXALN1L1}{De libro Ecclesiástici}{Si. 51: 1-6}{Si}{51}{}
\reading{0810N1L2}{0810N1L2}{}{Si. 51: 6-9}{}{}{}
\reading{0810N1L3}{0810N1L3}{}{Si. 51: 10-12}{}{}{}
\feast{0814}
        {Vigile de l’Assomption de la Bienheureuse Vierge Marie}{Propre des Saints}{Août}{2}{14 août}
        {}{}{Marie, la Très Sainte Vierge!Assomption!Vigile}{}{}
\rubric{Lectures à l'Homéliaire.}
\feast{0815}
        {Assomption de la Bienheureuse Vierge Marie}{Propre des Saints}{Août}{1}{15 août}
        {}{}{Marie, la Très Sainte Vierge!Assomption}{}{}
\reading{0815N1L1}{1208N1L3}{De libro Génesis}{Gn. 3: 9-15}{Gn}{3}{}
\reading{0815N1L2}{0815N1L2}{}{1 Co. 15: 20-26}{1 Co}{15}{}
\reading{0815N1L3}{0815N1L3}{}{1 Co. 15: 53-57}{}{}{}
\feast{0815b}
        {Assomption de la Bienheureuse Vierge Marie}{Propre des Saints}{Août}{3}{15 août (usage ancien)}
        {}{}{Marie, la Très Sainte Vierge!Assomption!(usage ancien)}{}{}
\reading{0815bN1L1}{0908N1L1}{Incípiunt Cántica canticórum}{Ct. 1: 1-4}{Ct}{1}{}
\reading{0815bN1L2}{0908N1L2}{}{Ct. 1: 5-9}{}{}{}
\reading{0815bN1L3}{0908N1L3}{}{Ct. 1: 10-16}{}{}{}
\rubric{Pendant l'Octave et le jour Octave, lectures scripturaires propres, comme ci-dessous, sauf aux fêtes d'autres saints, où elles sont prises à l'Écriture courante.}
\feast{0816}
        {Saint Joachim, confesseur, père de la Bienheureuse Vierge Marie}{Propre des Saints}{Août}{2}{16 août}
        {}{}{Joachim, confesseur, père de la Bienheureuse Vierge Marie}{}{}
\rubric{Lectures \normaltext{Heureux le riche}, p.\ \pageref{CONPN1L1}.}
\feast{0818}
        {Quatrième jour dans l’Octave de l’Assomption}{Propre des Saints}{Août}{3}{18 août}
        {}{}{}{}{}
\reading{0818N1L1}{0818N1L1}{De Cánticis canticórum}{Ct. 4: 7-9, 12}{Ct}{4}{}
\reading{0818N1L2}{0818N1L2}{}{Ct. 6: 4, 9-10; 8: 5}{Ct}{6}{8}
\reading{0818N1L3}{0818N1L3}{}{Ct. 8: 6-7}{}{}{}
\feast{0818b}
        {Quatrième jour dans l’Octave de l’Assomption}{Propre des Saints}{Août}{3}{18 août (usage ancien)}
        {}{}{}{}{}
\reading{0818bN1L1}{0818bN1L1}{De Cánticis canticórum}{Ct. 4: 1-4}{Ct}{4}{}
\reading{0818bN1L2}{0818bN1L2}{}{Ct. 4: 7-10}{}{}{}
\reading{0818bN1L3}{0818bN1L3}{}{Ct. 4: 11-15}{}{}{}
\feast{0822}
        {Fête du Cœur Immaculé de la Bienheureuse Vierge Marie}{Propre des Saints}{Août}{2}{22 août}
        {}{}{Marie, la Très Sainte Vierge!Cœur Immaculé}{}{}
\rubric{Lectures \normaltext{Moi, la Sagesse}, p.\ \pageref{CBMVN1L1}.}
\feast{0822b}
        {Octave de l’Assomption}{Propre des Saints}{Août}{2}{22 août (usage ancien)}
        {}{}{Marie, la Très Sainte Vierge!Assomption!Octave}{}{}
\reading{0822bN1L1}{0822bN1L1}{De Cánticis canticórum}{Ct. 8: 5-6}{Ct}{8}{}
\reading{0822bN1L2}{0822bN1L2}{}{Ct. 8: 7-9}{}{}{}
\reading{0822bN1L3}{0822bN1L3}{}{Ct. 8: 10-14}{}{}{}
\feast{0824}
        {Saint Barthélemy, apôtre}{Propre des Saints}{Août}{2}{24 août}
        {}{}{Barthélemy, apôtre}{}{}
\rubric{Lectures \normaltext{Que l’on nous regarde}, p.\ \pageref{COAPN1L1}.}
\feast{0829}
        {Décollation de Saint Jean Baptiste}{Propre des Saints}{Août}{2}{29 août}
        {}{}{Jean Baptiste!Décollation}{}{}
\reading{0829N1L1}{0624N1L1}{Incipit liber Jeremíæ Prophétæ}{Jr. 1: 1-5}{Jr}{1}{}
\reading{0829N1L2}{0624N1L2}{}{Jr. 1: 6-10}{}{}{}
\reading{0829N1L3}{0624N1L3}{}{Jr. 1: 17-19}{}{}{}
\feast{0908}
        {Nativité de la Bienheureuse Vierge Marie}{Propre des Saints}{Septembre}{2}{8 septembre}
        {}{}{Marie, la Très Sainte Vierge!Nativité}{}{}
\reading{0908N1L1}{0908N1L1}{Incípiunt Cántica canticórum}{Ct. 1: 1-5}{Ct}{1}{}
\reading{0908N1L2}{0908N1L2}{}{Ct. 1: 6-10}{}{}{}
\reading{0908N1L3}{0908N1L3}{}{Ct. 1: 11-17}{}{}{}
\feast{0912}
        {Fête du Saint Nom de Marie}{Propre des Saints}{Septembre}{2}{12 septembre}
        {}{}{Marie, la Très Sainte Vierge!Saint Nom}{}{}
\rubric{Lectures \normaltext{Moi, la Sagesse}, p.\ \pageref{CBMVN1L1}.}
\feast{0914}
        {Exaltation de la Sainte Croix}{Propre des Saints}{Septembre}{2}{14 septembre}
        {}{}{Notre-Seigneur Jésus-Christ!Croix!Exaltation}{}{}
\reading{0914N1L1}{0914N1L1}{De libro Númeri}{Nb. 21: 1-3}{Nb}{21}{}
\reading{0914N1L2}{0914N1L2}{}{Nb. 21: 4-6}{}{}{}
\reading{0914N1L3}{0914N1L3}{}{Nb. 21: 7-9}{}{}{}
\feast{0915}
        {Notre Dame des Sept Douleurs}{Propre des Saints}{Septembre}{2}{15 septembre}
        {}{}{Marie, la Très Sainte Vierge!Notre Dame des Sept Douleurs}{}{}
\reading{0915N1L1}{0915N1L1}{De Jeremía Prophéta}{Lam. 1: 2, 20-21}{Lam}{1}{}
\reading{0915N1L2}{0915N1L2}{}{Lam. 2: 13, 15-16}{Lam}{2}{}
\reading{0915N1L3}{0915N1L3}{}{Lam. 2: 17-18}{}{}{}
\feast{0917}
        {Impression des Saints Stigmates de Saint François, confesseur}{Propre des Saints}{Septembre}{2}{17 septembre}
        {}{}{François d’Assise, confesseur!Impression des Saints Stigmates}{}{}
\reading{0917N1L1}{0917N1L1}{De Epístola beáti Pauli Apóstoli ad Gálatas}{Ga. 5: 25-26; 6: 1-6}{Ga}{5}{6}
\reading{0917N1L2}{0917N1L2}{}{Ga. 6: 7-13}{}{}{}
\reading{0917N1L3}{0917N1L3}{}{Ga. 6: 14-18}{}{}{}
\feast{0918}
        {Saint Joseph de Cupertino, confesseur}{Propre des Saints}{Septembre}{2}{18 septembre}
        {}{}{Joseph de Cupertino, confesseur}{}{}
\reading{0918N1L1}{0918N1L1}{De Epístola secúnda beáti Pauli Apóstoli ad Corínthios}{2 Co. 4: 6-11}{2 Co}{4}{}
\reading{0918N1L2}{0918N1L2}{}{2 Co. 5: 1-8}{2 Co}{5}{}
\reading{0918N1L3}{0918N1L3}{}{2 Co. 12: 1-9}{2 Co}{12}{}
\feast{0921}
        {Saint Matthieu, apôtre et évangéliste}{Propre des Saints}{Septembre}{2}{21 septembre}
        {}{}{Matthieu, apôtre et évangéliste}{}{}
\rubric{Lectures \normaltext{La trentième année}, p.\ \pageref{COEVN1L1}.}
\feast{0924}
        {Notre Dame de la Merci}{Propre des Saints}{Septembre}{2}{24 septembre}
        {}{}{Marie, la Très Sainte Vierge!Notre Dame de la Merci}{}{}
\rubric{Lectures \normaltext{Moi, la Sagesse}, p.\ \pageref{CBMVN1L1}.}
\feast{0928}
        {Saint Wenceslas, duc, martyr}{Propre des Saints}{Septembre}{2}{28 septembre}
        {}{}{Wenceslas, duc, martyr}{}{}
\rubric{Si les lectures scripturaires doivent être prises au commun,
	lectures \normaltext{Ainsi donc, frères, nous avons une dette}, p.\ \pageref{COPMN1L1}.}
\feast{0929}
        {Dédicace de Saint Michel, archange}{Propre des Saints}{Septembre}{2}{29 septembre}
        {}{}{Michel, archange!Dédicace}{}{}
\reading{0929N1L1}{0508N1L1}{De Daniéle Prophéta}{Dn. 7: 9-11}{Dn}{7}{}
\reading{0929N1L2}{0508N1L2}{}{Dn. 10: 4-8}{Dn}{10}{}
\reading{0929N1L3}{0508N1L3}{}{Dn. 10: 9-14}{}{}{}
\feast{1002}
        {Les Saints Anges gardiens}{Propre des Saints}{Octobre}{2}{2 octobre}
        {}{}{Anges gardiens}{}{}
\reading{1002N1L1}{1002N1L1}{De libro Éxodi}{Ex. 23: 20-23}{Ex}{23}{}
\reading{1002N1L2}{1002N1L2}{De Zacharía Prophéta}{Za. 1: 7-11}{Za}{1}{}
\reading{1002N1L3}{1002N1L3}{}{Za. 2: 1-5}{Za}{2}{}
\feast{1004}
        {Saint François d’Assise, confesseur}{Propre des Saints}{Octobre}{2}{4 octobre}
        {}{}{François d’Assise, confesseur}{}{}
\rubric{Si les lectures scripturaires doivent être prises au commun,
        lectures \normaltext{Même s’il meurt}, p.\ \pageref{ALCNN1L1}.}
\feast{1007}
        {Notre Dame du Rosaire}{Propre des Saints}{Octobre}{2}{7 octobre}
        {}{}{Marie, la Très Sainte Vierge!Notre Dame du Rosaire}{}{}
\reading{1007N1L1}{1007N1L1}{De Libro Ecclesiástici}{Si. 24: 7-12}{Si}{24}{}
\reading{1007N1L2}{1007N1L2}{}{Si. 24: 13-16}{}{}{}
\reading{1007N1L3}{1007N1L3}{}{Si. 24: 18-22}{}{}{}
\feast{1011}
        {Maternité de la Bienheureuse Vierge Marie}{Propre des Saints}{Octobre}{2}{11 octobre}
        {}{}{Marie, la Très Sainte Vierge!Maternité}{}{}
\reading{1011N1L1}{1011N1L1}{De libro Ecclesiástici}{Si. 24: 3-7}{Si}{24}{}
\reading{1011N1L2}{1011N1L2}{}{Si. 24: 8-12}{}{}{}
\reading{1011N1L3}{1011N1L3}{}{Si. 24: 13-17}{}{}{}
\feast{1018}
        {Saint Luc, évangéliste}{Propre des Saints}{Octobre}{2}{18 octobre}
        {}{}{Luc, évangéliste}{}{}
\rubric{Lectures \normaltext{La trentième année}, p.\ \pageref{COEVN1L1}.}
\feast{1024}
        {Saint Raphaël, archange}{Propre des Saints}{Octobre}{2}{24 octobre}
        {}{}{Raphaël, archange}{}{}
\reading{1024N1L1}{09H3F5L1}{De libro Tobíæ}{Tb. 12: 1-4}{Tb}{12}{}
\reading{1024N1L2}{1024N1L2}{}{Tb. 12: 5-13}{}{}{}
\reading{1024N1L3}{1024N1L3}{}{Tb. 12: 14-22}{}{}{}
\feast{1027}
        {Vigile des Saints Simon et Jude, apôtres}{Propre des Saints}{Octobre}{2}{27 octobre}
        {}{}{Simon et Jude, apôtres!Vigile}{}{}
\rubric{Lectures à l'Homéliaire.}
\feast{1028}
        {Saints Simon et Jude, apôtres}{Propre des Saints}{Octobre}{2}{28 octobre}
        {}{}{Simon et Jude, apôtres}{}{}
\reading{1028N1L1}{P6F7N1L1}{Incipit Epístola cathólica beáti Judæ Apóstoli}{Jd. 1: 1-4}{Jd}{1}{}
\reading{1028N1L2}{P6F7N1L2}{}{Jd. 1: 5-8}{}{}{}
\reading{1028N1L3}{P6F7N1L3}{}{Jd. 1: 9-13}{}{}{}
\feast{1031}
        {Vigile de la fête de tous les saints}{Propre des Saints}{Octobre}{2}{31 octobre (le 30 si le 31 tombe un dimanche)}
        {}{}{Toussaint!Vigile}{}{}
\rubric{Lectures à l'Homéliaire.}
\feast{10H6F1}
        {Fête du Christ-Roi}{Propre des Saints}{Octobre}{2}{Dernier dimanche d’octobre}
        {}{}{Notre-Seigneur Jésus-Christ!Roi}{}{}
\reading{10H6F1N1L1}{10H6F1N1L1}{De Epístola beáti Pauli Apóstoli ad Colossénses}{Col. 1: 3-8}{Col}{1}{}
\reading{10H6F1N1L2}{10H6F1N1L2}{}{Col. 1: 9-17}{}{}{}
\reading{10H6F1N1L3}{10H6F1N1L3}{}{Col. 1: 18-23}{}{}{}
\feast{1101}
        {Fête de tous les saints}{Propre des Saints}{Novembre}{1}{1\ier novembre}
        {}{}{Toussaint}{}{}
\reading{1101N1L1}{1101N1L1}{De libro Apocalýpsis beáti Joánnis Apóstoli}{Ap. 4: 2-8}{Ap}{4}{}
\reading{1101N1L2}{1101N1L2}{}{Ap. 5: 1-8}{Ap}{5}{}
\reading{1101N1L3}{1101N1L3}{}{Ap. 5: 9-14}{}{}{}
\rubric{Pendant l'Octave et le jour Octave, lectures scripturaires de l'Écriture courante, sauf mention contraire.}
\feast{1102}
        {Commémoraison de tous les fidèles défunts}{Propre des Saints}{Novembre}{2}{2 novembre}
        {}{}{Toussaint!Commémoraison des défunts}{}{}
\rubric{Les lectures des trois nocturnes se chantent sans absolution, ni bénédiction, ni conclusion, et celles du premier nocturne seulement, sans titre.}
\intermediatetitle{Premier nocturne}
\reading{1102N1L1}{ODEFN1L1}{}{Jb. 7: 16-21}{Jb}{7}{}
\reading{1102N1L2}{ODEFN2L2}{}{Jb. 14: 1-6}{Jb}{14}{}
\reading{1102N1L3}{ODEFN3L2}{}{Jb. 19: 20-27}{Jb}{19}{}
\intermediatetitle{Troisième nocturne}
\reading{1102N3L1}{1102N3L1}{De Epístola prima beáti Pauli Apóstoli ad Corínthios}{1 Co. 15: 12-22}{1 Co}{15}{}
\reading{1102N3L2}{1102N3L2}{}{1 Co. 15: 35-44}{}{}{}
\reading{1102N3L3}{1102N3L3}{}{1 Co. 15: 51-58}{}{}{}
\feast{1109}
        {Dédicace de l’Archibasilique du Très Saint Sauveur}{Propre des Saints}{Novembre}{2}{9 novembre}
        {}{}{Notre-Seigneur Jésus-Christ!Dédicace de l’Archibasilique du Très Saint Sauveur}{}{}
\reading{1109N1L1}{1109N1L1}{De libro Apocalýpsis beáti Joánnis Apóstoli}{Ap. 21: 9-11}{Ap}{21}{}
\reading{1109N1L2}{1109N1L2}{}{Ap. 21: 12-15}{}{}{}
\reading{1109N1L3}{1109N1L3}{}{Ap. 21: 16-18}{}{}{}
\feast{1111}
        {Saint Martin, évêque et confesseur}{Propre des Saints}{Novembre}{2}{11 novembre}
        {}{}{Martin, évêque et confesseur}{}{}
\reading{1111N1L1}{E5F2L1}{De Epístola prima beáti Pauli Apóstoli ad Timótheum}{1 Tm. 3: 1-7}{1 Tm}{3}{}
\reading{1111N1L2}{COPON1L2}{De Epístola ad Titum}{Tt. 1: 7-11}{Tt}{1}{}
\reading{1111N1L3}{COPON1L3}{}{Tt. 2: 1-8}{Tt}{2}{}
\feast{1118}
        {Dédicace des Basiliques des Saints Pierre et Paul}{Propre des Saints}{Novembre}{2}{18 novembre}
        {}{}{Pierre et Paul, apôtres!Dédicace des Basiliques}{}{}
\reading{1118N1L1}{1118N1L1}{De libro Apocalýpsis beáti Joánnis Apóstoli}{Ap. 21: 18-20}{Ap}{21}{}
\reading{1118N1L2}{1118N1L2}{}{Ap. 21: 21-23}{}{}{}
\reading{1118N1L3}{1118N1L3}{}{Ap. 21: 24-27}{}{}{}
\feast{1121}
        {Présentation de la Bienheureuse Vierge Marie au temple}{Propre des Saints}{Novembre}{2}{21 novembre}
        {}{}{Marie, la Très Sainte Vierge!Présentation au temple}{}{}
\rubric{Lectures \normaltext{Moi, la Sagesse}, p.\ \pageref{CBMVN1L1}.}
\feast{1122}
        {Sainte Cécile, vierge et martyre}{Propre des Saints}{Novembre}{2}{22 novembre}
        {}{}{Cécile, vierge et martyre}{}{}
\reading{1122N1L1}{MUVXN1L1}{De Epístola prima beáti Pauli Apóstoli ad Corínthios}{1 Co. 7: 25-31}{1 Co}{7}{}
\reading{1122N1L2}{MUVXN1L2}{}{1 Co. 7: 32-35}{}{}{}
\reading{1122N1L3}{MUVXN1L3}{}{1 Co. 7: 36-40}{}{}{}
\feast{1125}
        {Sainte Catherine, vierge et martyre}{Propre des Saints}{Novembre}{2}{25 novembre}
        {}{}{Catherine, vierge et martyre}{}{}
\rubric{Si les lectures scripturaires doivent être prises au commun,
        lectures \normaltext{Je veux te rendre grâce}, p.\ \pageref{VXALN1L1}.}
\feast{1129b}
        {Vigile de Saint André, apôtre}{Propre des Saints}{Novembre}{2}{29 novembre}
        {}{}{André, apôtre!Vigile}{}{}
\rubric{Cette vigile est omise pendant l'Avent. Si elle est célébrée, lectures à l'Homéliaire.}
\feast{1130}
        {Saint André, apôtre}{Propre des Saints}{Novembre}{2}{30 novembre}
        {}{}{André, apôtre}{}{}
\reading{1130N1L1}{1130N1L1}{De Epístola beáti Pauli Apóstoli ad Romános}{Rm. 10: 4-9}{Rm}{10}{}
\reading{1130N1L2}{1130N1L2}{}{Rm. 10: 10-15}{}{}{}
\reading{1130N1L3}{1130N1L3}{}{Rm. 10: 16-21}{}{}{}
\feast{1204}
        {Saint Pierre Chrysologue, évêque et docteur de l’Église}{Propre des Saints}{Décembre}{2}{4 décembre}
        {}{}{Pierre Chrysologue, évêque et docteur de l’Église}{}{}
        \rubric{Si les lectures ne sont pas de l'Écriture courante, lectures \normaltext{Voici une parole}, p.\ \pageref{COPON1L1}.}
\feast{1207}
        {Saint Ambroise, évêque et docteur de l’Église}{Propre des Saints}{Décembre}{2}{7 décembre}
        {}{}{Ambroise, évêque et docteur de l’Église}{}{}
        \rubric{Si les lectures ne sont pas de l'Écriture courante, lectures \normaltext{Voici une parole}, p.\ \pageref{COPON1L1}.}
\feast{1208}
        {Immaculée Conception\\de la Bienheureuse Vierge Marie}{Propre des Saints}{Décembre}{1}{8 décembre}
        {}{}{Marie, la Très Sainte Vierge!Immaculée Conception}{}{}
\reading{1208N1L1}{1208N1L1}{De libro Génesis}{Gn. 3: 1-5}{Gn}{3}{}
\reading{1208N1L2}{1208N1L2}{}{Gn. 3: 6-8}{}{}{}
\reading{1208N1L3}{1208N1L3}{}{Gn. 3: 9-15}{}{}{}
\rubric{Pendant l'Octave et le jour Octave, lectures scripturaires de l'Écriture courante.}
\feast{1214}
        {Septième jour dans l’Octave de l’Immaculée Conception}{Propre des Saints}{Décembre}{3}{14 décembre}
        {}{}{}{}{}
\rubric{Si ce jour tombe le mecredi des Quatre-Temps et qu’il n’y a pas de lectures de l'Écriture courante transférées à ce jour,
	lectures \normaltext{Moi, la Sagesse}, p.\ \pageref{CBMVN1L1}.}
\feast{1215}
        {Octave de l’Immaculée Conception}{Propre des Saints}{Décembre}{2}{15 décembre}
        {}{}{Marie, la Très Sainte Vierge!Immaculée Conception!Octave}{}{}
\rubric{Si ce jour tombe le mecredi des Quatre-Temps et qu’il n’y a pas de lectures de l'Écriture courante transférées à ce jour,
	lectures \normaltext{Le serpent était le plus rusé}, p.\ \pageref{1208N1L1}.}
\feast{1221}
        {Saint Thomas, apôtre}{Propre des Saints}{Décembre}{2}{21 décembre}
        {}{}{Thomas, apôtre}{}{}
\rubric{Lectures \normaltext{Que l’on nous regarde}, p.\ \pageref{COAPN1L1}.}
\newpage
\addcontentsline{toc}{chapter}{Index}
\addcontentsline{toc}{section}{des livres bibliques}
\cprintindex{L}{Index des livres bibliques}
\addcontentsline{toc}{section}{des jours liturgiques}
\cprintindex{F}{Index des jours liturgiques}
\tableofcontents
\thispagestyle{empty}

\end{document}