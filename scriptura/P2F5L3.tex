Félix, qui avait une connaissance approfondie de ce qui concerne le Chemin du Seigneur,
	ajourna l’audience en disant:
	«Quand le commandant Lysias descendra de Jérusalem,
	je rendrai une sentence sur votre affaire.»
Il donna l’ordre au centurion de garder Paul en détention avec un régime adouci,
	et sans empêcher les siens de lui rendre des services.
Quelques jours plus tard, Félix vint avec sa femme Drusille, qui était juive.
Il envoya chercher Paul et l’écouta parler de la foi au Christ Jésus.
Mais quand l’entretien porta sur la justice,
	la maîtrise de soi et le jugement à venir,
	Félix fut pris de peur et déclara:
	«Pour le moment, retire-toi; je te rappellerai à une prochaine occasion.»
Il n’en espérait pas moins que Paul lui donnerait de l’argent;
	c’est pourquoi il l’envoyait souvent chercher pour parler avec lui.
Deux années s’écoulèrent;
	Félix reçut comme successeur Porcius Festus.
Voulant accorder une faveur aux Juifs, Félix avait laissé Paul en prison.
