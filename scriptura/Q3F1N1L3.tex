Mais Roubène les entendit, et voulut le sauver de leurs mains.
	Il leur dit: «Ne touchons pas à sa vie.»
Et il ajouta: «Ne répandez pas son sang:
	jetez-le dans cette citerne du désert, mais ne portez pas la main sur lui.»
Il voulait le sauver de leurs mains et le ramener à son père.
Dès que Joseph eut rejoint ses frères,
	ils le dépouillèrent de sa tunique, la tunique de grand prix qu’il portait,
	ils se saisirent de lui et le jetèrent dans la citerne, qui était vide et sans eau.
Ils s’assirent ensuite pour manger.
En levant les yeux, ils virent une caravane d’Ismaélites qui venait de Galaad.
Leurs chameaux étaient chargés d’aromates, de baume et de myrrhe
	qu’ils allaient livrer en Égypte.
Alors Juda dit à ses frères:
	«Quel profit aurions-nous à tuer notre frère et à dissimuler sa mort?
	Vendons-le plutôt aux Ismaélites et ne portons pas la main sur lui,
	car il est notre frère, notre propre chair.»
Ses frères l’écoutèrent.
Des marchands madianites qui passaient par là retirèrent Joseph de la citerne,
	ils le vendirent pour vingt pièces d’argent aux Ismaélites,
	et ceux-ci l’emmenèrent en Égypte.
