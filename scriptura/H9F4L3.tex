Le jour où il revint, il se retira dans cette chambre pour y coucher.
Élisée dit à Guéhazi, son serviteur:
	«Appelle notre Sunamite!»
	Guéhazi l’appela, et elle se tint devant lui.
Élisée reprit: «Dis-lui donc:
	Voici que tu t’es donné beaucoup de peine pour nous.
Que peut-on faire pour toi?
Faut-il parler pour toi au roi ou au chef de l’armée?»
Mais elle répondit: «Je vis tranquille au milieu des miens.»
Puis il dit à son serviteur:
		«Que peut-on faire pour cette femme?»
Le serviteur répondit: «Hélas, elle n’a pas de fils, et son mari est âgé.»
Élisée lui dit: «Appelle-la.»
	Le serviteur l’appela et elle se présenta à la porte.
Élisée lui dit: «À cette même époque, au temps fixé pour la naissance,
	tu tiendras un fils dans tes bras.»
Mais elle dit: «Non, mon seigneur, homme de Dieu,
	ne dis pas de mensonge à ta servante.»
Or, la femme devint enceinte et, l’année suivante, à la même époque,
	elle enfanta un fils, au moment prédit par Élisée.
