Bacchidès, lui, se tenait à l’aile droite.
Les bataillons se mirent en marche sur deux fronts, au son des trompettes.
Les hommes de Judas sonnèrent, eux aussi, des trompettes,
	et la terre fut ébranlée par le vacarme des armées.
	Le combat fit rage du matin jusqu’au soir.
Judas vit que Bacchidès se tenait sur la droite avec la partie la plus forte de son armée.
Entouré de tous les guerriers les plus ardents,
	il réussit à enfoncer l’aile droite et la poursuivit jusqu’à la montagne d’Azôt.
En voyant la déroute de l’aile droite,
	ceux de l’aile gauche se rabattirent sur les pas de Judas et de ses compagnons,
	les prenant à revers.
Le combat devint acharné et il y eut beaucoup de victimes de part et d’autre.
C’est alors que Judas succomba, lui aussi.
	Tous les autres s’enfuirent.
Jonathan et Simon emportèrent leur frère Judas.
	Ils l’ensevelirent dans le tombeau de ses pères, à Modine.
Tout Israël le pleura et se lamenta sur lui pendant de nombreux jours.
