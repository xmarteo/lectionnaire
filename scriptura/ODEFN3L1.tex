Mon souffle s’épuise, mes jours s’éteignent;
	pour moi le cimetière!
Ne suis-je pas objet de raillerie,
	l’œil tenu éveillé par leurs provocations?
Dépose donc ma caution près de toi:
	qui d’autre accepterait un gage de ma main?
Mes jours ont passé, brisés sont mes plans, les désirs de mon cœur.
On veut faire de la nuit le jour;
	face aux ténèbres, on prétend que la lumière est proche.
Si je dois espérer le séjour des morts comme demeure,
	étendre dans les ténèbres ma couche,
	appeler la fosse «mon père», la vermine «ma mère et ma sœur»,
	où donc est mon espoir?
	Mon espérance, qui l’entrevoit?
