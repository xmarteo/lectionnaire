Il faut que le responsable de communauté soit sans reproche,
	puisqu’il est l’intendant de Dieu;
	il ne doit être ni arrogant, ni coléreux,
	ni buveur, ni brutal, ni avide de profits malhonnêtes;
	mais il doit être accueillant, ami du bien, raisonnable, juste, saint, maître de lui.
Il doit être attaché à la parole digne de foi, celle qui est conforme à la doctrine,
	pour être capable d’exhorter en donnant un enseignement solide,
	et aussi de réfuter les opposants.
Car il y a beaucoup de réfractaires, des gens au discours inconsistant,
	des marchands d’illusion, surtout parmi ceux qui viennent du judaïsme.
Il faut fermer la bouche à ces gens qui, pour faire des profits malhonnêtes,
	bouleversent des maisons entières, en enseignant ce qu’il ne faut pas.
