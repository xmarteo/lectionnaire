Quand arriva le jour de la Pentecôte, au terme des cinquante jours,
	ils se trouvaient réunis tous ensemble.
Soudain un bruit survint du ciel comme un violent coup de vent:
	la maison où ils étaient assis en fut remplie tout entière.
Alors leur apparurent des langues qu’on aurait dites de feu,
	qui se partageaient,
	et il s’en posa une sur chacun d’eux.
Tous furent remplis d’Esprit Saint:
	ils se mirent à parler en d’autres langues,
	et chacun s’exprimait selon le don de l’Esprit.
Or, il y avait, résidant à Jérusalem,
	des Juifs religieux, venant de toutes les nations sous le ciel.
Lorsque ceux-ci entendirent la voix qui retentissait,
	ils se rassemblèrent en foule.
Ils étaient en pleine confusion
	parce que chacun d’eux entendait dans son propre dialecte ceux qui parlaient.
Dans la stupéfaction et l’émerveillement, ils disaient:
	«Ces gens qui parlent ne sont-ils pas tous Galiléens?
	Comment se fait-il que chacun de nous les entende dans son propre dialecte,
	sa langue maternelle?»
