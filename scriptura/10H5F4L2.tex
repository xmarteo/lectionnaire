Judas Maccabée rassembla ceux qui l’entouraient: ils étaient au nombre de six mille.
Il les exhortait à ne pas être frappés de crainte devant l’ennemi,
	à ne pas s’inquiéter du très grand nombre des païens
		qui marchaient contre eux injustement,
	mais à combattre noblement.
Il les encourageait à garder devant les yeux
		l’outrage criminel commis par ces gens contre le Lieu saint,
	les tourments infligés à la ville ravagée,
	et même la ruine des institutions ancestrales.
«Ceux-là, disait-il, mettent leur confiance
		à la fois dans leurs armes et dans leurs actions téméraires.
	Mais nous, nous mettons notre confiance en Dieu tout-puissant,
	qui peut d’un seul signe de tête
	renverser aussi bien ceux qui marchent contre nous, que le monde tout entier».
Il leur rappela en outre les cas de protection divine
		qui avaient eu lieu en faveur de leurs ancêtres,
	notamment sous le règne de Sennakérib,
	lorsque cent quatre-vingt-cinq mille hommes avaient péri.
