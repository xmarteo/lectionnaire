De son côté, Judas Maccabée gardait une confiance inébranlable
	et le plein espoir d’obtenir la protection du Seigneur.
Il exhortait ses compagnons à ne pas redouter l’approche des païens,
	mais à garder présents à l’esprit les secours déjà reçus du Ciel
	et à compter maintenant aussi sur la victoire qui leur viendrait du Tout-Puissant.
En les encourageant par des mots de la Loi et des Prophètes,
	en leur rappelant aussi les combats qu’ils avaient déjà soutenus,
	il les remplit d’une détermination nouvelle.
Ayant ainsi réveillé leur ardeur,
	il acheva de les stimuler en leur montrant combien les païens
		avaient manqué à leur parole et méprisé leurs serments.
Il arma ainsi chacun d’eux,
	non pas de la sécurité que donnent les boucliers et les lances,
	mais du réconfort qu’on trouve dans les paroles de bon conseil.
Il leur raconta en outre un songe digne de foi, une sorte de vision qui les réjouit tous.
