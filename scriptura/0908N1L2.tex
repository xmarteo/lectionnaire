Ne regardez pas à ma peau noire:
	c’est le soleil qui m’a brunie.
Les fils de ma mère se sont fâchés contre moi:
	ils m’ont mise à garder les vignes.
	Ma vigne, la mienne, je ne l’ai pas gardée.
Raconte-moi, bien-aimé de mon âme,
	où tu mènes paître tes brebis, où tu les couches aux heures de midi,
	que je n’aille plus m’égarer vers les troupeaux de tes compagnons.
Si tu ne le sais pas, ô belle entre les femmes,
	va dehors sur les traces du troupeau
	et mène paître tes jeunes chèvres vers les tentes des bergers.
Cavale attelée aux chars de Pharaon, ainsi tu m’apparais, ô mon amie!
	Quel charme, tes joues entre tes boucles, ton cou entre les perles!
