Je te le renvoie, lui qui est comme mon cœur.
Je l’aurais volontiers gardé auprès de moi, pour qu’il me rende des services en ton nom,
	à moi qui suis en prison à cause de l’Évangile.
Mais je n’ai rien voulu faire sans ton accord,
	pour que tu accomplisses ce qui est bien, non par contrainte mais volontiers.
S’il a été éloigné de toi pendant quelque temps,
	c’est peut-être pour que tu le retrouves définitivement,
	non plus comme un esclave, mais, mieux qu’un esclave, comme un frère bien-aimé:
	il l’est vraiment pour moi, combien plus le sera-t-il pour toi,
	aussi bien humainement que dans le Seigneur.
Si donc tu estimes que je suis en communion avec toi, accueille-le comme si c’était moi.
S’il t’a fait du tort ou s’il te doit quelque chose, mets cela sur mon compte.
Moi, Paul, j’écris ces mots de ma propre main.
