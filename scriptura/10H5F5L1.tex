01 À cette époque, Antiocos était piteusement revenu des régions de Perse.

02 Il s’était en effet rendu dans la ville de Persépolis; il y avait entrepris de piller le temple et d’opprimer la ville. Mais la foule s’était soulevée, en ayant recours aux armes, si bien qu’Antiocos fut mis en déroute par les habitants du pays, et contraint d’opérer une retraite honteuse.

03 Comme il se trouvait près d’Ecbatane, il apprit ce qui était arrivé à Nicanor et aux gens de Timothée.

04 Transporté de fureur, il résolut de faire payer aux Juifs l’injure infligée par ceux qui avaient causé sa fuite. Pour ce motif, il ordonna au conducteur de lancer le char en avant, et de continuer sans répit jusqu’à la fin du voyage. En réalité, la sentence du Ciel était déjà sur lui. Il avait dit, en effet, dans son arrogance: «Arrivé à Jérusalem, je ferai de cette ville la fosse commune des Juifs.»
