Mais si tu fais le mal, alors, vis dans la crainte. En effet, ce n’est pas pour rien que l’autorité détient le glaive. Car elle est au service de Dieu : en faisant justice, elle montre la colère de Dieu envers celui qui fait le mal.

05 C’est donc une nécessité d’être soumis, non seulement pour éviter la colère, mais encore pour obéir à la conscience.

06 C’est pour cette raison aussi que vous payez des impôts : ceux qui les perçoivent sont des ministres de Dieu quand ils s’appliquent à cette tâche.

07 Rendez à chacun ce qui lui est dû : à celui-ci l’impôt, à un autre la taxe, à celui-ci le respect, à un autre l’honneur.
