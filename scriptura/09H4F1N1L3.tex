Mais tous les habitants de toute la terre
		méprisèrent la parole de Nabucodonosor, roi des Assyriens,
	et ne se rangèrent pas à ses côtés pour combattre.
Ils renvoyèrent donc ses messagers les mains vides, et sans honneurs.
Nabucodonosor fut pris d'une violente fureur contre toute cette partie de la terre.
	Il jura par son trône et son diadème de châtier et d'anéantir par l'épée
	tous les territoires de Cilicie, de Damascène et de Syrie,
	ainsi que tous les habitants de la région de Moab, les fils d'Ammone, toute la Judée
	et tous les habitants d'Égypte jusqu'aux abords du territoire des deux mers.
La dix-huitième année, le vingt-deuxième jour du premier mois,
	il fut question, dans la demeure de Nabucodonosor, roi des Assyriens,
	du châtiment qu'il exercerait sur toute la terre, comme il l'avait dit.
Il convoqua tous les officiers de sa maison et les grands de sa cour,
	tint avec eux un conseil secret
	et, de sa propre bouche, il voua totalement la terre à la malédiction.
