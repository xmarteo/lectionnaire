Le Christ est venu, grand prêtre des biens à venir.
	Par la tente plus grande et plus parfaite,
	celle qui n’est pas œuvre de mains humaines
		et n’appartient pas à cette création,
	il est entré une fois pour toutes dans le sanctuaire,
	en répandant, non pas le sang de boucs et de jeunes taureaux,
	mais son propre sang.
De cette manière, il a obtenu une libération définitive.
S’il est vrai qu’une simple aspersion avec le sang de boucs et de taureaux,
		et de la cendre de génisse,
	sanctifie ceux qui sont souillés, leur rendant la pureté de la chair,
	le sang du Christ fait bien davantage,
	car le Christ, poussé par l’Esprit éternel,
	s’est offert lui-même à Dieu comme une victime sans défaut;
	son sang purifiera donc notre conscience des actes qui mènent à la mort,
	pour que nous puissions rendre un culte au Dieu vivant.
Voilà pourquoi il est le médiateur d’une alliance nouvelle,
		d’un testament nouveau:
	puisque sa mort a permis le rachat
		des transgressions commises sous le premier Testament,
	ceux qui sont appelés peuvent recevoir l’héritage éternel jadis promis.
