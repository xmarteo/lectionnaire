Le grand prêtre Joakim, qui résidait alors à Jérusalem,
	écrivit aux habitants de Béthulie et de Bétomesthaïm,
	ville située en face d'Esdrelon et de la plaine de Dothaïne,
	pour leur dire de bloquer les cols de la région montagneuse,
	seule voie d'accès vers la Judée.
Il leur serait facile, en effet, d'arrêter ceux qui passeraient,
	car le passage étroit ne se laissait franchir que par deux hommes à la fois.
Les fils d'Israël agirent selon les ordres du grand prêtre Joakim.
Avec une ardeur soutenue, tous les hommes d'Israël crièrent vers Dieu;
	avec une ardeur soutenue, ils s'humilièrent.
