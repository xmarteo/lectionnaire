À cette époque, le roi Hérode Agrippa
		se saisit de certains membres de l’Église pour les mettre à mal.
	Il supprima Jacques, frère de Jean, en le faisant décapiter.
Voyant que cette mesure plaisait aux Juifs, il décida aussi d’arrêter Pierre.
	C’était les jours des Pains sans levain.
Il le fit appréhender, emprisonner,
	et placer sous la garde de quatre escouades de quatre soldats;
	il voulait le faire comparaître devant le peuple après la Pâque.
Tandis que Pierre était ainsi détenu dans la prison,
	l’Église priait Dieu pour lui avec insistance.
