
09 Arrivé à Jérusalem, il fut reçu avec bienveillance par le grand prêtre et par la ville. Il communiqua ce dont on l’avait informé et il exposa la raison de sa présence. Il désirait savoir, en effet, si tout cela correspondait bien à la réalité.

10 Le grand prêtre lui expliqua que le trésor contenait les dépôts de veuves et d’orphelins,

11 ainsi qu’une somme appartenant à Hyrcan, fils de Tobie, personnage qui occupait une situation très élevée. Contrairement aux allégations fallacieuses de l’impie Simon, l’ensemble ne comprenait que quatre cents talents d’argent et deux cents talents d’or.

12 D’ailleurs, ajouta-t-il, il était absolument inconcevable de léser ceux qui avaient mis leur confiance dans la sainteté de ce lieu, dans le caractère sacré et inviolable du Temple vénéré dans le monde entier.