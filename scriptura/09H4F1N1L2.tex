C’était l’an douze du règne de Nabucodonosor, roi des Assyriens à Ninive la grande ville.
Il fit la guerre au roi Arphaxad dans la Grande Plaine,
	c’est-à-dire la plaine située sur le territoire de Ragau.
Tous les habitants de la région montagneuse se rallièrent à lui,
	ainsi que tous ceux des vallées de l’Euphrate, du Tigre et de l’Hydaspe,
	et ceux de la plaine soumise au roi d’Élam, Ariok.
Nabucodonosor, roi des Assyriens,
	envoya aussi des messagers à tous les habitants de la Perside;
	et aux habitants de Cilicie, de Damascène, du Liban et de l’Anti-Liban;
	et à tous les habitants du littoral, aux peuples du Carmel et du Galaad,
	de la Haute-Galilée et de la grande plaine d’Esdrelon,
	à tous ceux de Samarie et de ses villes;
	et au-delà du Jourdain jusqu’à Jérusalem;
	et au-delà du Torrent d’Égypte, jusqu’au-dessus de Tanis et de Memphis,
	à tous les habitants de l’Égypte jusqu’aux abords du territoire de l’Éthiopie.
