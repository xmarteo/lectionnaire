Cher Théophile, dans mon premier livre,
	j’ai parlé de tout ce que Jésus a fait et enseigné,
	depuis le moment où il commença,
	jusqu’au jour où il fut enlevé au ciel, après avoir, par l’Esprit Saint,
	donné ses instructions aux Apôtres qu’il avait choisis.
C’est à eux qu’il s’est présenté vivant après sa Passion;
	il leur en a donné bien des preuves, puisque, pendant quarante jours,
	il leur est apparu et leur a parlé du royaume de Dieu.
Au cours d’un repas qu’il prenait avec eux,
	il leur donna l’ordre de ne pas quitter Jérusalem,
	mais d’y attendre que s’accomplisse la promesse du Père.
Il déclara: «Cette promesse, vous l’avez entendue de ma bouche:
	alors que Jean a baptisé avec l’eau,
	vous, c’est dans l’Esprit Saint que vous serez baptisés d’ici peu de jours.»
Ainsi réunis, les Apôtres l’interrogeaient:
	«Seigneur, est-ce maintenant le temps où tu vas rétablir le royaume pour Israël?»
Jésus leur répondit: «Il ne vous appartient pas de connaître les temps et les moments
	que le Père a fixés de sa propre autorité.
Mais vous allez recevoir une force quand le Saint-Esprit viendra sur vous;
	vous serez alors mes témoins à Jérusalem,
	dans toute la Judée et la Samarie, et jusqu’aux extrémités de la terre.»
