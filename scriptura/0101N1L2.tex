
09 Cette béatitude-là concerne-t-elle seulement ceux qui ont la circoncision, ou bien aussi ceux qui ne l’ont pas ? Nous disons, en effet : « C’est pour sa foi qu’il a été accordé à Abraham d’être juste. »

10 Et quand cela lui fut-il accordé ? Après la circoncision ? ou avant ? Non pas après, mais avant.

11 Et il reçut le signe de la circoncision comme la marque de la justice obtenue par la foi avant d’être circoncis. De cette façon, il est le père de tous ceux qui croient sans avoir la circoncision, pour qu’à eux aussi, il soit accordé d’être justes ;

12 et il est également le père des circoncis, ceux qui non seulement ont la circoncision, mais qui marchent aussi sur les traces de la foi de notre père Abraham avant sa circoncision.
