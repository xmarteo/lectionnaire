01 LA TROISIEME ANNEE du règne de Joakim, roi de Juda, Nabucodonosor, roi de Babylone, arriva devant Jérusalem et l’assiégea.

02 Le Seigneur livra entre ses mains Joakim, roi de Juda, ainsi qu’une partie des objets de la Maison de Dieu. Il les emporta au pays de Babylone, et les déposa dans le trésor de ses dieux.

03 Le roi ordonna à Ashpénaz, chef de ses eunuques, de faire venir quelques jeunes Israélites de race royale ou de famille noble.

04 Ils devaient être sans défaut corporel, de belle figure, exercés à la sagesse, instruits et intelligents, pleins de vigueur, pour se tenir à la cour du roi et apprendre l’écriture et la langue des Chaldéens.
