24 Antiocos s’imagina qu’on le méprisait, et soupçonna que ce discours contenait des insultes. Il se mit à exhorter le plus jeune, le dernier survivant. Bien plus, il lui promettait avec serment de le rendre à la fois riche et très heureux s’il abandonnait les usages de ses pères: il en ferait son ami et lui confierait des fonctions publiques.

25 Comme le jeune homme n’écoutait pas, le roi appela la mère, et il l’exhortait à conseiller l’adolescent pour le sauver.

26 Au bout de ces longues exhortations, elle consentit à persuader son fils.

27 Elle se pencha vers lui, et lui parla dans la langue de ses pères, trompant ainsi le cruel tyran: «Mon fils, aie pitié de moi: je t’ai porté neuf mois dans mon sein, je t’ai allaité pendant trois ans, je t’ai nourri et élevé jusqu’à l’âge où tu es parvenu, j’ai pris soin de toi.»
