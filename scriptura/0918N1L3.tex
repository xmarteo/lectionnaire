01 Faut-il se vanter ? Ce n’est pas utile. J’en viendrai pourtant aux visions et aux révélations reçues du Seigneur.

02 Je sais qu’un fidèle du Christ, voici quatorze ans, a été emporté jusqu’au troisième ciel – est-ce dans son corps ? je ne sais pas ; est-ce hors de son corps ? je ne sais pas ; Dieu le sait – ;

03 mais je sais que cet homme dans cet état-là – est-ce dans son corps, est-ce sans son corps ? je ne sais pas, Dieu le sait –

04 cet homme-là a été emporté au paradis et il a entendu des paroles ineffables, qu’un homme ne doit pas redire.

05 D’un tel homme, je peux me vanter, mais pour moi-même, je ne me vanterai que de mes faiblesses.

06 En fait, si je voulais me vanter, ce ne serait pas folie, car je ne dirais que la vérité. Mais j’évite de le faire, pour qu’on n’ait pas de moi une idée plus favorable qu’en me voyant ou en m’écoutant.

07 Et ces révélations dont il s’agit sont tellement extraordinaires que, pour m’empêcher de me surestimer, j’ai reçu dans ma chair une écharde, un envoyé de Satan qui est là pour me gifler, pour empêcher que je me surestime.

08 Par trois fois, j’ai prié le Seigneur de l’écarter de moi.

09 Mais il m’a déclaré : « Ma grâce te suffit, car ma puissance donne toute sa mesure dans la faiblesse. » C’est donc très volontiers que je mettrai plutôt ma fierté dans mes faiblesses, afin que la puissance du Christ fasse en moi sa demeure.