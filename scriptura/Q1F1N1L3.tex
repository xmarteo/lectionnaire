Grande est l’assurance que j’ai devant vous,
	grande est ma fierté à votre sujet,
	je me sens pleinement réconforté,
	je déborde de joie au milieu de toutes nos détresses.
En fait, à notre arrivée en Macédoine,
	dans notre faiblesse nous n’avons pas eu le moindre répit
	mais nous étions dans la détresse à tout moment:
	au-dehors, des conflits, et au-dedans, des craintes.
Pourtant, Dieu, lui qui réconforte les humbles,
	nous a réconfortés par la venue de Tite,
	et non seulement par sa venue,
		mais par le réconfort qu’il avait trouvé chez vous:
	il nous a fait part de votre grand désir de nous revoir,
	de votre désolation, de votre zèle pour moi,
	et cela m’a donné encore plus de joie.
En effet, même si je vous ai attristés par ma lettre, je ne le regrette pas;
	et même si j’ai pu le regretter
	–car je vois bien que cette lettre vous a attristés,
		au moins pour un moment–,
	je me réjouis maintenant, non de ce que vous avez été attristés,
	mais parce que cette tristesse vous a conduits au repentir.
