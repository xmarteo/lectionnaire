Judas et ses frères virent que la menace s’aggravait
	et que des armées campaient sur leur territoire.
Ils apprirent aussi que le roi avait ordonné de faire périr leur peuple jusqu’au dernier.
Ils se dirent entre eux: «Relevons notre peuple de sa ruine.
	Combattons pour notre peuple et pour le Lieu saint!»
Afin d’être prête au combat,
	la communauté se réunit pour prier, pour implorer miséricorde et pitié.
Jérusalem est inhabitée, comme un désert.
	De ses enfants, aucun n’entre, aucun ne sort.
	On a foulé aux pieds le sanctuaire.
Des étrangers occupent la citadelle, des païens s’y installent.
	En Jacob, les cris de joie se sont tus, le son des flûtes et des harpes s’est éteint.
