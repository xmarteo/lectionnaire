L’autel était recouvert d’offrandes non conformes aux lois et illicites.
Il n’était possible ni de célébrer le sabbat, ni d’observer les fêtes de nos pères,
	ni simplement de se déclarer juif.
Chaque mois, le jour anniversaire de la naissance du roi,
	on était contraint par une amère nécessité de prendre part à un repas sacrilège.
Et lors des fêtes dionysiaques,
	on était forcé de suivre, couronné de lierre, le cortège en l’honneur de Dionysos.
Un décret fut promulgué, à l’instigation de Ptolémée,
	pour que, dans les villes grecques du voisinage,
	on tienne la même conduite à l’égard des Juifs,
	on organise des repas sacrilèges,
	et que l’on égorge ceux qui ne choisiraient pas d’adopter les coutumes grecques.
Tout ceci laissait entrevoir l’imminence de la détresse.
