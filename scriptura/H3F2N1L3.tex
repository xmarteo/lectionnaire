Samuel exposa au peuple le droit de la royauté;
	il l’écrivit dans un livre qu’il déposa devant le Seigneur.
Puis Samuel renvoya tout le peuple, chacun chez soi.
Saül aussi s’en alla chez lui, à Guibéa.
	Les hommes de valeur, dont Dieu avait touché le cœur, partirent avec lui.
Quant aux vauriens, ils dirent:
	«Comment celui-là nous sauverait-il?»
Ils le méprisèrent et ne lui apportèrent pas d’offrandes.
	Mais Saül fit comme s’il n’avait rien entendu.
