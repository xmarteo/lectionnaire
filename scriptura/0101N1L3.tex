
13 Car ce n’est pas en vertu de la Loi que la promesse de recevoir le monde en héritage a été faite à Abraham et à sa descendance, mais en vertu de la justice obtenue par la foi.

14 En effet, si l’on devient héritier par la Loi, alors la foi est sans contenu, et la promesse, abolie.

15 Car la Loi aboutit à la colère de Dieu, mais là où il n’y a pas de Loi, il n’y a pas non plus de transgression.

16 Voilà pourquoi on devient héritier par la foi: c’est une grâce, et la promesse demeure ferme pour tous les descendants d’Abraham, non pour ceux qui se rattachent à la Loi seulement, mais pour ceux qui se rattachent aussi à la foi d’Abraham, lui qui est notre père à tous.

17 C’est bien ce qui est écrit: J’ai fait de toi le père d’un grand nombre de nations. Il est notre père devant Dieu en qui il a cru, Dieu qui donne la vie aux morts et qui appelle à l’existence ce qui n’existe pas.