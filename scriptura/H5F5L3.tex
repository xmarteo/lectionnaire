David fut irrité de ce que le Seigneur avait ouvert une brèche parmi les siens
		en frappant Ouzza,
	et on appela ce lieu Pèrès-Ouzza (c’est-à-dire: Brèche-d’Ouzza),
	nom qu’il a gardé jusqu’à ce jour.
David eut peur du Seigneur, ce jour-là, et dit:
	«Comment l’arche du Seigneur pourrait-elle entrer chez moi?»
David renonça donc à transférer chez lui, dans la Cité de David, l’arche du Seigneur,
	mais il la dévia vers la maison d’Obed-Édom, le Guittite.
L’arche du Seigneur resta pendant trois mois dans la maison d’Obed-Édom, le Guittite,
	et le Seigneur bénit Obed-Édom ainsi que toute sa maison.
On rapporta au roi David:
	«Le Seigneur a béni la maison d’Obed-Édom et tout ce qui lui appartient,
	à cause de l’arche de Dieu.»
David partit alors et fit monter l’arche de Dieu
		de la maison d’Obed-Édom jusqu’à la Cité de David,
	au milieu des cris de joie.
