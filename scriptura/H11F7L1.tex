Sédécias avait vingt et un ans lorsqu’il devint roi, et il régna onze ans à Jérusalem.
	Sa mère s’appelait Hamoutal, fille de Jérémie; elle était de Libna.
Il fit ce qui est mal aux yeux du Seigneur, tout comme avait fait Joakim.
C’est à cause de la colère du Seigneur qu’il en fut ainsi à Jérusalem et en Juda,
	jusqu’à ce qu’il les rejette loin de sa face.
Mais Sédécias se révolta contre le roi de Babylone.
La neuvième année du règne de Sédécias, le dixième jour du dixième mois,
	Nabucodonosor, roi de Babylone, vint attaquer Jérusalem avec toute son armée;
	il établit son camp devant la ville qu’il entoura d’un ouvrage fortifié.
La ville fut assiégée jusqu’à la onzième année du règne de Sédécias.
Le neuvième jour du quatrième mois, la famine était devenue terrible dans la ville
	et les gens du pays n’avaient plus de pain.
