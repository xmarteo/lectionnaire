Sous la conduite du Seigneur,
	Judas Maccabée et ses compagnons reconquirent le Temple et la ville de Jérusalem.
Ils détruisirent les autels païens édifiés sur la place publique par les étrangers,
	ainsi que leurs lieux de culte.
Après avoir purifié le Temple, ils bâtirent un nouvel autel.
Après deux ans d’interruption, ils offrirent des sacrifices,
	en prenant le feu obtenu par le moyen de pierres à feu.
Ils brûlèrent de l’encens, allumèrent les lampes et placèrent les pains de l’offrande.
Cela fait, ils se jetèrent à plat ventre
	et supplièrent le Seigneur de ne plus les faire tomber dans de tels malheurs,
	mais de les corriger lui-même avec modération s’il leur arrivait encore de pécher,
	et de ne pas les livrer aux païens blasphémateurs et barbares.
Or, c’est au jour anniversaire de la profanation du Temple par les étrangers,
		qu’eut lieu la purification du Temple,
	le même jour, à savoir le vingt-cinq du même mois, le mois de Kisléou.
