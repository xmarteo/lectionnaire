Le vingt-cinquième jour du neuvième mois, c’est-à-dire le mois de Kisléou,
	en l’année 148, de grand matin,
	les prêtres offrirent le sacrifice prescrit par la Loi
	sur le nouvel autel qu’ils avaient construit.
On fit la dédicace de l’autel au chant des hymnes,
	au son des cithares, des harpes et des cymbales.
C’était juste l’anniversaire du jour où les païens l’avaient profané.
Le peuple entier se prosterna la face contre terre pour adorer,
	puis ils bénirent le Ciel qui avait fait aboutir leur effort.
