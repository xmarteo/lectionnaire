
27 Alors, y a-t-il de quoi s'enorgueillir ? Absolument pas. Par quelle loi ? Par celle des œuvres que l’on pratique ? Pas du tout. Mais par la loi de la foi.

28 En effet, nous estimons que l’homme devient juste par la foi, indépendamment de la pratique de la loi de Moïse.

29 Ou bien, Dieu serait-il seulement le Dieu des Juifs ? N’est-il pas aussi le Dieu des nations ? Bien sûr, il est aussi le Dieu des nations,

30 puisqu’il n’y a qu’un seul Dieu : il rendra justes en vertu de la foi ceux qui ont reçu la circoncision, et aussi, au moyen de la foi, ceux qui ne l’ont pas reçue.

31 Sommes-nous en train d’abolir la Loi au moyen de la foi ? Pas du tout ! Au contraire, nous confirmons la Loi.