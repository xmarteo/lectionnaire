Comme il ne pouvait l’emporter sur Onias,
	il alla trouver Apollonios, fils de Thraséas,
	qui était à cette époque le gouverneur militaire de Cœlé-Syrie et de Phénicie.
Il le mit au courant du fait que le trésor de Jérusalem regorgeait de richesses inouïes,
	au point qu’on ne pouvait en calculer la somme,
	et qu’elles étaient sans proportion avec le budget requis pour les sacrifices.
Il ajouta qu’il était possible de les faire tomber en la possession du roi.
Lors d’une entrevue avec le roi,
	Apollonios l’informa des richesses dont on lui avait dénoncé l’existence.
Le roi désigna Héliodore, qui était à la tête de ses affaires.
	Il l’envoya avec l’ordre de procéder à l’enlèvement des richesses indiquées.
Aussitôt, Héliodore fit le voyage,
	officiellement pour visiter les villes de Cœlé-Syrie et de Phénicie,
	mais en fait pour exécuter le mandat du roi.
