Après avoir ainsi galvanisé ses compagnons,
	et les avoir rendus prêts à mourir pour les lois et la patrie,
	Judas divisa l’armée en quatre corps.
À la tête de chaque unité, il plaça ses frères Simon, Joseph et Jonathan,
	chacun ayant quinze cents hommes sous ses ordres,
	et en outre Éléazar.
Il fit la lecture du Livre saint, puis donna pour mot d’ordre:
		«Secours de Dieu!»;
	il prit personnellement la tête du premier détachement
	et engagea le combat avec Nicanor.
Le Tout-Puissant s’étant fait leur allié, ils égorgèrent plus de neuf mille ennemis,
	blessèrent et mutilèrent la plus grande partie des soldats de Nicanor
	et les mirent tous en fuite.
Ils prirent aussi l’argent de ceux qui étaient venus pour les acheter.
Après les avoir poursuivis assez loin, ils revinrent sur leurs pas, pressés par l’heure,
	car c’était la veille du sabbat,
		motif pour lequel ils ne s’attardèrent pas à courir derrière eux.
Quand ils eurent ramassé les armes et enlevé le butin des ennemis,
	ils célébrèrent le sabbat:
	ils multiplièrent les bénédictions et les louanges au Seigneur qui les avait sauvés,
	qui avait fixé à ce jour la première manifestation de sa miséricorde à leur égard.
Après le sabbat, ils distribuèrent une part du butin aux victimes de la persécution,
	aux veuves et aux orphelins;
	ils partagèrent le reste entre eux et leurs enfants.
