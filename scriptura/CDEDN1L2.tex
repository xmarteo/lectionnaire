Les prêtres se tenaient à leur poste,
	ainsi que les lévites avec les instruments de musique du Seigneur,
	ceux qu’avait faits le roi David, afin de rendre grâce au Seigneur
		«car éternel est son amour!»,
	comme au temps où, par eux, David psalmodiait.
Les prêtres, en face d’eux, sonnaient de la trompette, et tout Israël se tenait debout.
Salomon consacra le milieu de la cour qui était devant la Maison du Seigneur.
C’est là, en effet, qu’il offrit les holocaustes et les graisses des sacrifices de paix,
	car l’autel de bronze que Salomon avait fait ne pouvait pas contenir l’holocauste,
	l’offrande de céréales et les graisses.
Salomon --- et tout Israël avec lui ---
	célébra, en ce temps-là, la fête pendant sept jours.
Ce fut un très grand rassemblement depuis l’Entrée-de-Hamath jusqu’au Torrent d’Égypte.
On avait célébré la dédicace de l’autel pendant sept jours;
	le huitième jour, on fit une réunion solennelle, et la fête dura encore sept jours.
