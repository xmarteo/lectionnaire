
12 Il ne faut donc pas que le péché règne dans votre corps mortel et vous fasse obéir à ses désirs.

13 Ne présentez pas au péché les membres de votre corps comme des armes au service de l’injustice; au contraire, présentez-vous à Dieu comme des vivants revenus d’entre les morts, présentez à Dieu vos membres comme des armes au service de la justice.

14 Car le péché n’aura plus de pouvoir sur vous: en effet, vous n’êtes plus sujets de la Loi, vous êtes sujets de la grâce de Dieu.

15 Alors? Puisque nous ne sommes pas soumis à la Loi mais à la grâce, allons-nous commettre le péché? Pas du tout.

16 Ne le savez-vous pas? Celui à qui vous vous présentez comme esclaves pour lui obéir, c’est de celui-là, à qui vous obéissez, que vous êtes esclaves: soit du péché, qui mène à la mort, soit de l’obéissance à Dieu, qui mène à la justice.

17 Mais rendons grâce à Dieu: vous qui étiez esclaves du péché, vous avez maintenant obéi de tout votre cœur au modèle présenté par l’enseignement qui vous a été transmis.

18 Libérés du péché, vous êtes devenus esclaves de la justice.