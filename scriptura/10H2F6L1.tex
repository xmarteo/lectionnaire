En l’an 151 de l’empire grec,
	Démétrios, fils de Séleucos, quitta Rome
	et se rendit avec une poignée d’hommes dans une ville du littoral,
		où il inaugura son règne.
Démétrios s’assit sur le trône royal.
Alors, tous les hommes sans foi ni loi que l’on pouvait trouver en Israël
	se rendirent auprès de lui, sous la conduite d’Alkime,
	qui convoitait la charge de grand prêtre.
Ils se mirent à accuser leur propre peuple devant le roi, en disant:
	«Judas et ses frères ont fait périr tous tes amis;
	ils nous ont dispersés hors de notre pays.
Envoie donc maintenant un homme de confiance:
	qu’il vienne voir tous les ravages
		dont Judas s’est rendu coupable envers nous et envers le domaine du roi,
	et qu’il les punisse, lui, ses frères et tous leurs auxiliaires.»
