Le septième jour du cinquième mois,
		la dix-neuvième année du règne de Nabucodonosor, roi de Babylone,
	Nabouzardane, commandant de la garde, au service du roi de Babylone,
	fit son entrée à Jérusalem.
Il incendia la maison du Seigneur et la maison du roi;
	il incendia toutes les maisons de Jérusalem, --- toutes les maisons des notables.
Toutes les troupes chaldéennes qui étaient avec lui abattirent les remparts de Jérusalem.
Nabouzardane déporta tout le peuple resté dans la ville,
	les déserteurs qui s’étaient ralliés au roi de Babylone,
	bref, toute la population.
Il laissa seulement une partie du petit peuple de la campagne,
	pour avoir des vignerons et des laboureurs.
Les colonnes de bronze qui se trouvaient dans la maison du Seigneur,
	les bases et la Mer de bronze qui se trouvaient dans la maison du Seigneur,
	les Chaldéens les brisèrent et en emportèrent le bronze à Babylone.
