
13 Je ne veux pas vous le laisser ignorer, frères: j’ai bien souvent eu l’intention de venir chez vous, et j’en ai été empêché jusqu’à maintenant; je pensais obtenir chez vous quelque fruit comme chez les autres nations païennes.

14 J’ai des devoirs envers tous: Grecs et non-Grecs, savants et ignorants;

15 de là cet élan qui me pousse à vous annoncer l’Évangile à vous aussi qui êtes à Rome.

16 En effet, je n’ai pas honte de l’Évangile, car il est puissance de Dieu pour le salut de quiconque est devenu croyant, le Juif d’abord, et le païen.

17 Dans cet Évangile se révèle la justice donnée par Dieu, celle qui vient de la foi et conduit à la foi, comme il est écrit: Celui qui est juste par la foi, vivra.

18 Or la colère de Dieu se révèle du haut du ciel contre toute impiété et contre toute injustice des hommes qui, par leur injustice, font obstacle à la vérité.

19 En effet, ce que l’on peut connaître de Dieu est clair pour eux, car Dieu le leur a montré clairement.