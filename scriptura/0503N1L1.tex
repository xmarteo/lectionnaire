Quant à ceux qui se réclament de la pratique de la Loi,
	ils sont tous sous la menace d’une malédiction, car il est écrit:
	Maudit soit celui qui ne s’attache pas à mettre en pratique
		tout ce qui est écrit dans le livre de la Loi.
Il est d’ailleurs clair que par la Loi personne ne devient juste devant Dieu,
	car, comme le dit l’Écriture, celui qui est juste par la foi, vivra,
	et la Loi ne procède pas de la foi, mais elle dit:
	Celui qui met en pratique les commandements vivra à cause d’eux.
Quant à cette malédiction de la Loi, le Christ nous en a rachetés
	en devenant, pour nous, objet de malédiction, car il est écrit:
	Il est maudit, celui qui est pendu au bois du supplice.
Tout cela pour que la bénédiction d’Abraham
		s’étende aux nations païennes dans le Christ Jésus,
	et que nous recevions, par la foi, l’Esprit qui a été promis.
