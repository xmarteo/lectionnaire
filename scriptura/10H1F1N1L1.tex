Alexandre, fils de Philippe, de Macédoine, quitta le pays des Grecs
	pour affronter Darius, le roi des Perses et des Mèdes.
Après l’avoir vaincu, il régna à sa place; auparavant il régnait déjà sur le monde grec.
Il livra de multiples batailles, s’empara de nombreuses forteresses
	et fit périr les rois de la région.
Il poussa jusqu’aux extrémités de la terre et ramassa le butin d’une multitude de nations.
Devant lui, la terre resta muette.
	Son cœur s’exalta à l’extrême.
Il rassembla une armée très puissante,
	soumit des provinces, des nations et des souverains, qui durent lui payer l’impôt.
Après quoi, il fut contraint de s’aliter et comprit qu’il allait mourir.
Il convoqua ses auxiliaires les plus illustres, élevés avec lui depuis le jeune âge
	et, de son vivant, il partagea entre eux son royaume.
