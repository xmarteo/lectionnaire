Après ces paroles, tandis que les Apôtres le regardaient,
	il s’éleva, et une nuée vint le soustraire à leurs yeux.
Et comme ils fixaient encore le ciel où Jésus s’en allait,
	voici que, devant eux, se tenaient deux hommes en vêtements blancs,
	qui leur dirent: «Galiléens,
	pourquoi restez-vous là à regarder vers le ciel?
	Ce Jésus qui a été enlevé au ciel d’auprès de vous,
	viendra de la même manière que vous l’avez vu s’en aller vers le ciel.»
Alors, ils retournèrent à Jérusalem
		depuis le lieu-dit «mont des Oliviers» qui en est proche,
	la distance de marche ne dépasse pas ce qui est permis le jour du sabbat.
À leur arrivée,
	ils montèrent dans la chambre haute où ils se tenaient habituellement;
	c’était Pierre, Jean, Jacques et André,
	Philippe et Thomas, Barthélemy et Matthieu,
	Jacques fils d’Alphée, Simon le Zélote, et Jude fils de Jacques.
Tous, d’un même cœur, étaient assidus à la prière,
	avec des femmes, avec Marie la mère de Jésus, et avec ses frères.
