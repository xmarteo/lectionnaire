En ce temps-là, Arphaxad, roi des Mèdes à Ecbatane,
	entoura cette ville d'un mur d'enceinte en pierres de taille,
	donnant au rempart une hauteur de soixante-dix coudées et une largeur de cinquante.
Sur les portes, il dressa des tours
	de cent coudées de haut sur soixante de large à leurs fondations;
	les portes elles-mêmes s'élevaient à soixante-dix coudées sur quarante de large,
	pour permettre les sorties de ses forces d'élite
		et des fantassins en ordre de bataille.
