Nicanor se proposait d’acquitter pour le roi
		le tribut de deux mille talents dû aux Romains,
	au moyen de la vente des Juifs que l’on ferait prisonniers.
Il s’empressa donc d’envoyer aux villes de la côte
		l’invitation de venir acheter des esclaves juifs,
	promettant d’en livrer quatre-vingt-dix pour un talent;
	il ne s’attendait pas au jugement qui devait s’ensuivre pour lui,
		de la part du Tout-Puissant.
La nouvelle de l’avance de Nicanor parvint à Judas.
Quand celui-ci eut averti ses compagnons de l’approche de l’armée ennemie,
	les lâches et ceux qui manquaient de confiance dans le jugement de Dieu
	s’enfuirent de tous côtés et gagnèrent d’autres lieux.
D’autres mettaient en vente tout ce qui leur restait,
	et priaient le Seigneur de les délivrer de l’impie Nicanor.
