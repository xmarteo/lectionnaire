Les impies seront dans la douleur, et leur souvenir périra.
Les impies viendront, tout tremblants, quand on fera le compte de leurs péchés,
	et leurs crimes se dresseront contre eux pour les accuser.
Alors le juste se tiendra debout, plein d’assurance,
	en présence de ceux qui l’ont opprimé, de ceux qui méprisaient sa peine.
À sa vue, ils seront pris d’une peur épouvantable,
	sidérés de le voir sauvé contre toute attente;
	saisis par le remords, ils se diront entre eux,
		la gorge serrée, incapables de reprendre souffle:
	«Le voilà, celui que nous tournions jadis en ridicule!
Nous en faisions la cible de nos sarcasmes, fous que nous étions!
	Nous trouvions absurde sa manière de vivre et infâme sa mort!
Pourquoi est-il compté parmi les fils de Dieu?
Pourquoi partage-t-il le sort des saints?»
