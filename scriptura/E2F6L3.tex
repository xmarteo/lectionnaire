C’est donc très volontiers que je mettrai plutôt ma fierté dans mes faiblesses, afin que la puissance du Christ fasse en moi sa demeure.

10 C’est pourquoi j’accepte de grand cœur pour le Christ les faiblesses, les insultes, les contraintes, les persécutions et les situations angoissantes. Car, lorsque je suis faible, c’est alors que je suis fort.

11 Me voilà devenu insensé: c’est vous qui m’y avez forcé! J’aurais dû plutôt être recommandé par vous; en effet, je n’ai été en rien inférieur à ces super-apôtres, quoique je ne sois rien.