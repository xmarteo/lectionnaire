Il voyait bien, en effet, que sans une intervention royale,
	il était impossible de rétablir la paix publique,
	et que Simon ne mettrait pas un terme à sa folie.
Après la mort de Séleucos et l’accès au trône d’Antiocos surnommé Épiphane,
	Jason, le frère d’Onias, usurpa la charge de grand prêtre.
Au cours d’une entrevue avec le roi,
	il lui promit trois cent soixante talents d’argent de l’impôt,
	ainsi que quatre-vingts talents prélevés sur quelque autre revenu.
Il s’engageait en outre à payer cent cinquante autres talents,
	si on lui accordait le pouvoir de fonder un gymnase
		et un lieu d’éducation pour les jeunes gens,
	et de recenser les partisans de l’hellénisme à Jérusalem.
