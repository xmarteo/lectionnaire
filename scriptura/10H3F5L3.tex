Apprenant que Simon avait pris la relève de son frère Jonathan
		et qu’il s’apprêtait à lui livrer bataille,
	Tryphon lui envoya des messagers pour lui dire:
«Ton frère Jonathan doit de l’argent au trésor royal,
	en raison des fonctions qu’il exerçait;
	c’est pour cela que nous le retenons captif.
Envoie donc maintenant une somme de cent talents d’argent et deux de ses fils en otages,
	de peur qu’une fois relâché, il ne se dresse contre nous.
	À cette condition, nous le relâcherons.»
Simon se rendit bien compte que ces paroles étaient trompeuses,
	mais il envoya chercher l’argent et les jeunes enfants,
	de peur de s’attirer une grande hostilité de la part du peuple.
On aurait pu dire:
	«C’est parce que Simon ne lui a pas envoyé l’argent et les jeunes enfants,
	que Jonathan a péri.»
Simon envoya donc les deux jeunes enfants et la somme de cent talents,
	mais Tryphon renia sa parole et ne relâcha pas Jonathan.
