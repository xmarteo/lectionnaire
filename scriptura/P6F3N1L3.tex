Celui qui proclame que Jésus est le Fils de Dieu,
	Dieu demeure en lui, et lui en Dieu.
Et nous, nous avons reconnu l’amour que Dieu a pour nous,
	et nous y avons cru.
Dieu est amour:
	qui demeure dans l’amour demeure en Dieu, et Dieu demeure en lui.
Voici comment l’amour atteint, chez nous, sa perfection:
	avoir de l’assurance au jour du jugement;
	comme Jésus, en effet, nous ne manquons pas d’assurance en ce monde.
Il n’y a pas de crainte dans l’amour,
	l’amour parfait bannit la crainte;
	car la crainte implique un châtiment,
	et celui qui reste dans la crainte n’a pas atteint la perfection de l’amour.
Quant à nous, nous aimons parce que Dieu lui-même nous a aimés le premier.
Si quelqu’un dit: «J’aime Dieu», alors qu’il a de la haine contre son frère,
	c’est un menteur.
En effet, celui qui n’aime pas son frère, qu’il voit,
	est incapable d’aimer Dieu, qu’il ne voit pas.
Et voici le commandement que nous tenons de lui:
	celui qui aime Dieu, qu’il aime aussi son frère.
