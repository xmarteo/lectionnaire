Lorsque le premier fut mort de cette manière, on amena le deuxième pour le torturer.
On lui arracha la peau de la tête avec les cheveux, puis on lui demanda:
	«Mangeras-tu, plutôt que d’être châtié dans ton corps, membre par membre?»
Mais il répondit, dans la langue de ses pères: «Non!»
C’est pourquoi lui aussi subit aussitôt les mêmes sévices que le premier.
Au moment de rendre le dernier soupir, il dit:
	«Tu es un scélérat, toi qui nous arraches à cette vie présente,
	mais puisque nous mourons par fidélité à ses lois,
	le Roi du monde nous ressuscitera pour une vie éternelle.»
Après cela, le troisième fut mis à la torture.
Il tendit la langue aussitôt qu’on le lui ordonna
	et il présenta les mains avec intrépidité,
	en déclarant avec noblesse:
	«C’est du Ciel que je tiens ces membres,
	mais à cause de ses lois je les méprise,
	et c’est par lui que j’espère les retrouver.»
Le roi et sa suite furent frappés de la grandeur d’âme de ce jeune homme
	qui comptait pour rien les souffrances.
