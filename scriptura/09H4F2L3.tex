Eux-mêmes, leurs femmes, leurs tout-petits et leurs troupeaux,
	mirent des sacs sur leurs reins.
Tous les Israélites de Jérusalem, hommes, femmes et enfants,
	se jetèrent sur le sol devant le Temple, la tête couverte de cendres,
	et déployèrent leurs sacs devant le Seigneur.
Ils enveloppèrent d’un sac l’autel lui-même
	et crièrent d’un seul cœur vers le Dieu d’Israël,
	le suppliant ardemment de ne pas livrer leurs tout-petits à la razzia,
	leurs femmes au rapt, les villes de leur héritage à l’anéantissement,
	le Lieu saint à la profanation et à l’outrage pour la plus grande joie des nations.
Le peuple observa un jeûne pendant de nombreux jours, dans toute la Judée et à Jérusalem,
	devant le Lieu saint du Seigneur souverain de l’univers.
Le grand prêtre Joakim, tous les prêtres qui se tenaient devant le Seigneur,
		et ceux qui assuraient le service liturgique,
	les reins enveloppés de toile à sac, offraient l’holocauste perpétuel.
