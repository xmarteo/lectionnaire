17 Judas choisit Eupolème, fils de Jean, fils d’Akkôs, et Jason, fils d’Éléazar ; il les envoya à Rome pour conclure amitié et alliance.

18 Il espérait ainsi que les Romains, voyant le joug de servitude imposé à Israël par le royaume des Grecs, l’en délivreraient.

19 Ces hommes se rendirent donc à Rome. Au bout d’un très long voyage, ils entrèrent au sénat et prirent la parole. Ils dirent :

20 « Judas, celui que l’on surnomme Maccabée, ainsi que ses frères et tout le peuple des Juifs nous ont envoyés pour conclure avec vous une alliance de paix, afin d’être inscrits au nombre de vos alliés et amis. »

21 Cette affaire parut bonne aux yeux des Romains.

22 Voici la copie de la lettre qu’ils gravèrent sur des tables de bronze et qu’ils envoyèrent à Jérusalem pour y être un mémorial de paix et d’alliance.