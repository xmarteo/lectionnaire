Josias avait huit ans lorsqu’il devint roi,
	et il régna trente et un ans à Jérusalem.
Le nom de sa mère était Yedida, fille d’Adaya, originaire de Bosqath.
Il fit ce qui est droit aux yeux du Seigneur,
	en tout il marcha sur le chemin de David, son ancêtre;
	il ne s’en écarta ni à droite ni à gauche.
Or, la dix-huitième année du règne de Josias,
	le roi envoya le secrétaire Shafane, fils d’Açalyahou, fils de Meshoullam,
	à la maison du Seigneur, en disant:
	«Monte chez Helcias, le grand-prêtre,
	et qu’il compte l’argent apporté à la maison du Seigneur,
	celui que les gardiens du seuil ont recueilli de la part du peuple.
Que cet argent soit remis entre les mains des maîtres d’œuvre,
	préposés à la maison du Seigneur.
