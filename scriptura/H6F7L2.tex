Absalom se retrouva par hasard en face des serviteurs de David.
Il montait un mulet, et le mulet s’engagea sous la ramure d’un grand térébinthe.
La tête d’Absalom se prit dans les branches, et il resta entre ciel et terre,
	tandis que le mulet qui était sous lui continuait d’avancer.
Quelqu’un l’aperçut et avertit Joab:
	«Je viens de voir Absalom suspendu dans un térébinthe.»
Joab dit à l’homme qui l’avait averti:
	«Tu l’as vu!
		Pourquoi donc ne l’as-tu pas frappé et abattu sur place?
	J’aurais dû alors te donner dix pièces d’argent et une ceinture.»
L’homme répondit à Joab:
	«Même si je soupesais maintenant, dans la paume de mes mains, mille pièces d’argent,
	je ne porterais pas la main sur le fils du roi,
	car nous avons entendu de nos oreilles l’ordre que le roi vous a donné
	à toi, à Abishaï et à Ittaï:
	“Par égard pour moi, veillez sur le jeune Absalom!”»
