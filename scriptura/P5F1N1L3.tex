C’est pourquoi, après avoir disposé votre intelligence pour le service,
	restez sobres, mettez toute votre espérance dans la grâce
		que vous apporte la révélation de Jésus Christ.
Comme des enfants qui obéissent,
	cessez de vous conformer aux convoitises d’autrefois,
	quand vous étiez dans l’ignorance,
	mais, à l’exemple du Dieu saint qui vous a appelés,
	devenez saints, vous aussi, dans toute votre conduite,
	puisqu’il est écrit: «Vous serez saints, car moi, je suis saint.»
Si vous invoquez comme Père
		celui qui juge impartialement chacun selon son œuvre,
	vivez donc dans la crainte de Dieu,
	pendant le temps où vous résidez ici-bas en étrangers.
Vous le savez: ce n’est pas par des biens corruptibles, l’argent ou l’or,
	que vous avez été rachetés de la conduite superficielle
		héritée de vos pères;
	mais c’est par un sang précieux,
	celui d’un agneau sans défaut et sans tache, le Christ.
Dès avant la fondation du monde, Dieu l’avait désigné d’avance
	et il l’a manifesté à la fin des temps à cause de vous.
C’est bien par lui que vous croyez en Dieu,
	qui l’a ressuscité d’entre les morts et qui lui a donné la gloire;
	ainsi vous mettez votre foi et votre espérance en Dieu.
