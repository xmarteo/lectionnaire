«S’il passe à côté de moi, je ne le vois pas; s’il me frôle, je ne m’en aperçois pas.
S’il s’empare d’une proie, qui donc lui fera lâcher prise,
	qui donc osera lui demander: “Que fais-tu là?”
Dieu ne retient pas sa colère: sous ses pieds se prosternent les auxiliaires de Rahab.
Et moi, je prétendrais lui répliquer!
	je chercherais des arguments contre lui!
Même si j’ai raison, à quoi bon me défendre?
Je ne puis que demander grâce à mon juge.
Même s’il répond quand je fais appel, je ne suis pas sûr qu’il écoute ma voix,
	lui qui dans la tempête m’écrase et multiplie sans raison mes blessures.»
