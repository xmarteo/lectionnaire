Ce Melkisédek était roi de Salem, prêtre du Dieu très-haut;
	il vint à la rencontre d’Abraham
		quand celui-ci rentrait de son expédition contre les rois;
	il le bénit, et Abraham lui remit le dixième de tout ce qu’il avait pris.
D’abord, Melkisédek porte un nom qui veut dire «roi de justice»;
	ensuite, il est roi de Salem, c’est-à-dire roi «de paix»,
	et à son sujet on ne parle ni de père ni de mère, ni d’ancêtres,
	ni d’un commencement d’existence ni d’une fin de vie;
	cela le fait ressembler au Fils de Dieu: il demeure prêtre pour toujours.
