Les gens du pays prirent alors Joakaz, fils de Josias;
	ils lui donnèrent l’onction et le firent roi à la place de son père.
Joakaz avait vingt-trois ans lorsqu’il devint roi, et il régna trois mois à Jérusalem.
	Sa mère s’appelait Hamoutal, fille de Jérémie; elle était de Libna.
Il fit ce qui est mal aux yeux du Seigneur tout comme avaient fait ses pères.
Le pharaon Néko le mit aux fers à Ribla, au pays de Hamath,
	pour qu’il ne règne plus à Jérusalem.
	Et il imposa au pays un tribut de cent talents d’argent et d’un talent d’or.
Le pharaon Néko fit roi Élyakim, fils de Josias, à la place de Josias son père
	et il changea son nom en celui de Joakim.
Quant à Joakaz, il le prit et l’emmena en Égypte, où celui-ci mourut.
