Les Philistins rassemblèrent leurs armées pour la guerre;
	ils se rassemblèrent à Soko de Juda
	et ils établirent leur camp entre Soko et Azéqa, à Éfès-Dammim.
Saül et les hommes d’Israël se rassemblèrent
	et établirent leur camp dans le Val du Térébinthe,
	puis se rangèrent en ordre de bataille face aux Philistins.
Les Philistins se tenaient sur la montagne d’un côté,
	Israël se tenait sur la montagne de l’autre côté;
	entre eux il y avait la vallée.
Alors sortit des rangs philistins un champion qui s’appelait Goliath.
	Originaire de Gath, il mesurait six coudées et un empan.
Il avait un casque de bronze sur la tête, il était revêtu d’une cuirasse à écailles;
	la cuirasse pesait cinq mille sicles de bronze.
Il avait des jambières de bronze et un javelot de bronze entre les épaules.
Le bois de sa lance était comme le rouleau d’un métier à tisser,
	et sa pointe pesait six cents sicles de fer.
	Et devant lui marchait le porte-bouclier.
