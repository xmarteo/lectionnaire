Le roi lui dit: «Celui qui te parle de supprimer le meurtrier, tu me l’amèneras.
	Il cessera de s’en prendre à toi.»
Elle poursuivit: «Que le roi daigne donc en appeler au Seigneur son Dieu,
	de peur que le vengeur du sang n’augmente les ravages
		et qu’on ne fasse disparaître mon fils.»
Il répondit:
	«Par le Seigneur vivant, pas un seul cheveu de ton fils ne tombera à terre.»
La femme reprit:
	«Que ta servante puisse encore dire un mot à mon seigneur le roi.»
Il lui dit: «Parle!»
Alors la femme poursuivit:
	«Pourquoi nourris-tu un tel projet à l’encontre du peuple de Dieu?
Avec le serment qu’il vient de prononcer, le roi se déclare lui-même coupable
	en ne laissant pas revenir celui qu’il a banni.
À coup sûr, nous mourrons,
	et nous sommes comme l’eau qui s’écoule sur la terre sans être recueillie.
Mais Dieu n’enlève pas la vie,
	et il forme des projets pour que le banni ne soit plus banni loin de sa présence.»
