01 Les habitants de la Ville sainte jouissaient d’une paix totale; on y observait au mieux les lois, grâce à la piété du grand prêtre Onias et à sa haine du mal.

02 À cette époque, les rois eux-mêmes en vinrent à honorer le Lieu saint et à rehausser la gloire du Temple par les dons les plus magnifiques.

03 Séleucos, roi d’Asie, couvrait lui-même de ses revenus personnels toutes les dépenses exigées par la liturgie des sacrifices.

04 Or, un certain Simon, de la tribu de Bilga, qui avait été nommé administrateur du Temple, se trouva en désaccord avec le grand prêtre, au sujet de la surveillance des marchés de la ville.
