Éléazar était l’un des scribes les plus éminents.
C’était un homme très âgé, et de très belle allure.
	On voulut l’obliger à manger du porc en lui ouvrant la bouche de force.
Préférant avoir une mort prestigieuse plutôt qu’une vie abjecte,
	il marchait de son plein gré vers l’instrument du supplice,
	après avoir recraché cette viande,
	comme on doit le faire
		quand on a le courage de rejeter ce qu’il n’est pas permis de manger,
	même par amour de la vie.
Ceux qui étaient chargés de ce repas sacrilège le connaissaient de longue date.
Ils le prirent à part
	et lui conseillèrent de faire apporter des viandes dont l’usage était permis,
	et qu’il aurait préparées lui-même.
Il n’aurait qu’à faire semblant de manger les chairs de la victime pour obéir au roi;
	en agissant ainsi, il échapperait à la mort et serait traité avec humanité
	grâce à la vieille amitié qu’il avait pour eux.
