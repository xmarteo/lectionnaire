Le roi Antiocos parcourait le Haut Pays.
Il apprit alors qu’il y avait en Perse une ville, Élymaïs,
	fameuse par ses richesses, son argent et son or;
	son temple, extrêmement riche,
		contenait des casques en or, des cuirasses et des armes,
	laissés là par Alexandre, fils de Philippe et roi de Macédoine,
	qui régna le premier sur les Grecs.
Antiocos arriva, et il tenta de prendre la ville et de la piller,
	mais il n’y réussit pas, parce que les habitants avaient été informés de son projet.
Ils lui résistèrent et livrèrent bataille,
	si bien qu’il prit la fuite et battit en retraite, accablé de chagrin,
	pour retourner à Babylone.
Il était encore en Perse
	quand on vint lui annoncer la déroute des troupes qui avaient pénétré en Judée;
Lysias, en particulier, qui avait été envoyé avec un important matériel,
	avait fait demi-tour devant les Juifs.
