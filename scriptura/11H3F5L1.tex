Le roi Balthazar donna un somptueux festin pour les grands du royaume au nombre de mille,
	et il se mit à boire du vin en leur présence.
Excité par le vin, il fit apporter les vases d’or et d’argent
		que son père Nabucodonosor avait enlevés au temple de Jérusalem;
	il voulait y boire, avec ses grands, ses épouses et ses concubines.
On apporta donc les vases d’or enlevés du temple, de la Maison de Dieu à Jérusalem,
	et le roi, ses grands, ses épouses et ses concubines s’en servirent pour boire.
Après avoir bu, ils entonnèrent la louange de leurs dieux d’or et d’argent,
		de bronze et de fer, de bois et de pierre.
Soudain on vit apparaître, en face du candélabre, les doigts d’une main d’homme
	qui se mirent à écrire sur la paroi de la salle du banquet royal.
Lorsque le roi vit cette main qui écrivait,
	il changea de couleur, et son esprit se troubla.
