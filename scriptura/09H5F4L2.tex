
06 Hatak se rendit auprès de Mardochée, sur la place de la ville, en face de la porte du roi.

07 Mardochée l’informa de tout ce qui lui était arrivé, et du montant de la somme d’argent qu’Amane avait proposé de verser au trésor royal, en échange de l’extermination des Juifs.

08 Il lui remit une copie de l’édit promulgué à Suse pour les anéantir. Il chargea Hatak de le montrer à Esther, pour qu’elle soit informée. Il enjoignait à la reine d’aller chez le roi pour implorer sa grâce et plaider devant lui la cause de son peuple.

09 Hatak revint et rapporta à Esther les paroles de Mardochée.

10 Elle ordonna à Hatak de lui répondre :

11 « Tous les serviteurs du roi et les habitants des provinces royales savent bien que, pour quiconque, homme ou femme, qui se rend auprès du roi dans la cour intérieure sans avoir été convoqué, il n’y a qu’une seule loi : la mort. Sauf celui auquel le roi tend son sceptre d’or : il a la vie sauve. Moi-même, cela fait trente jours que je n’ai pas été convoquée chez le roi. »
