Alors Paul, debout au milieu de l’Aréopage, fit ce discours:
	«Athéniens, je peux observer que vous êtes, en toutes choses,
	des hommes particulièrement religieux.
En effet, en me promenant et en observant vos monuments sacrés,
	j’ai même trouvé un autel avec cette inscription:
	“Au dieu inconnu.”
Or, ce que vous vénérez sans le connaître, voilà ce que, moi, je viens vous annoncer.
Le Dieu qui a fait le monde et tout ce qu’il contient,
	lui qui est Seigneur du ciel et de la terre,
	n’habite pas des sanctuaires faits de main d’homme;
	il n’est pas non plus servi par des mains humaines,
	comme s’il avait besoin de quoi que ce soit,
	lui qui donne à tous la vie, le souffle et tout le nécessaire.
À partir d’un seul homme, il a fait tous les peuples
	pour qu’ils habitent sur toute la surface de la terre,
	fixant les moments de leur histoire et les limites de leur habitat;
	Dieu les a faits pour qu’ils le cherchent
	et, si possible, l’atteignent et le trouvent,
	lui qui, en fait, n’est pas loin de chacun de nous.»
