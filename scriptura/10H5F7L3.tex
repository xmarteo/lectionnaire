
17 Ce très beau discours de Judas eut le pouvoir d’inciter à la vertu et de donner aux jeunes une âme virile. Rassurés, les Juifs résolurent de ne pas se retrancher dans un camp, mais de prendre héroïquement l’offensive et de décider de l’issue de la bataille en se jetant dans la mêlée avec toute leur bravoure. Car c’étaient la ville, les lieux saints et le Temple qui étaient menacés.

18 En effet, la crainte qu’ils avaient pour leurs femmes et leurs enfants, ainsi que pour leurs frères et leurs proches, comptait moins que la plus grande et la première des craintes, celle qu’ils éprouvaient pour le Temple consacré.

19 L’angoisse de ceux qui avaient été laissés en ville n’était pas moins grande : ils tremblaient pour l’attaque menée en rase campagne.