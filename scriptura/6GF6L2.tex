Toute la terre avait alors la même langue et les mêmes mots.
Au cours de leurs déplacements du côté de l’orient,
	les hommes découvrirent une plaine en Mésopotamie, et s’y établirent.
Ils se dirent l’un à l’autre:
	«Allons! fabriquons des briques et mettons-les à cuire!»
Les briques leur servaient de pierres, et le bitume, de mortier.
Ils dirent: «Allons! bâtissons-nous une ville,
	avec une tour dont le sommet soit dans les cieux.
	Faisons-nous un nom,
	pour ne pas être disséminés sur toute la surface de la terre.»
