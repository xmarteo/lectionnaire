1 Salomon dominait sur tous les royaumes, depuis l’Euphrate – le fleuve –, sur le pays des Philistins, et jusqu’à la frontière de l’Égypte. Ils acquittèrent un tribut et servirent Salomon tous les jours de sa vie.

02 Les vivres de Salomon comprenaient, pour chaque jour, trente quintaux de semoule et soixante quintaux de farine,

03 dix bœufs gras, vingt bœufs de pâturage, cent moutons ; de plus, des cerfs, des gazelles, des chevreuils et des volailles engraissées.

04 Car il était maître des régions en deçà du fleuve Euphrate, depuis Tifsa jusqu’à Gaza : maître de tous les rois des régions en deçà du Fleuve. Et il avait la paix sur toutes ses frontières alentour.
