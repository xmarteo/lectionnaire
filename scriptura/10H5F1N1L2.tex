
23 Mais il fit un beau raisonnement, bien digne de son âge, du rang que lui donnait sa vieillesse, du respect que lui valaient ses cheveux blancs, de sa conduite irréprochable depuis l’enfance, et surtout digne de la législation sainte établie par Dieu. Il s’exprima en conséquence, demandant qu’on l’envoyât sans tarder au séjour des morts :

24 « Une telle comédie est indigne de mon âge. Car beaucoup de jeunes gens croiraient qu’Éléazar, à quatre-vingt-dix ans, adopte la manière de vivre des étrangers.

25 À cause de cette comédie, par ma faute, ils se laisseraient égarer eux aussi ; et moi, pour un misérable reste de vie, j’attirerais sur ma vieillesse la honte et le déshonneur.

26 Même si j’évite, pour le moment, le châtiment qui vient des hommes, je n’échapperai pas, vivant ou mort, aux mains du Tout-Puissant.

27 C’est pourquoi, en quittant aujourd’hui la vie avec courage, je me montrerai digne de ma vieillesse

28 et, en choisissant de mourir avec détermination et noblesse pour nos vénérables et saintes lois, j’aurai laissé aux jeunes gens le noble exemple d’une belle mort. » Sur ces mots, il alla tout droit au supplice.