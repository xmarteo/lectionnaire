Le Seigneur envoya contre Joakim des bandes de Chaldéens,
	et des bandes venant d’Aram, de Moab et d’Ammone.
Il les envoya contre Juda pour l’anéantir,
	conformément à la parole que le Seigneur avait prononcée
	par l’intermédiaire de ses serviteurs les prophètes.
Cela se produisit en Juda,
	uniquement par ordre du Seigneur qui voulait l’écarter loin de sa face,
	à cause des péchés de Manassé, en tout ce qu’il avait fait.
De même, le Seigneur ne voulut pas lui pardonner
		pour le sang innocent qu’il avait répandu,
	celui dont il avait rempli Jérusalem.
