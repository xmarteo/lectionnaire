La troisième année du règne d’Osée, fils d’Éla, roi d’Israël,
	Ézékias, fils d’Acaz, roi de Juda, devint roi.
Il avait vingt-cinq ans lorsqu’il devint roi, et il régna vingt-neuf ans à Jérusalem.
	Le nom de sa mère était Abi, fille de Zacharie.
Il fit ce qui est droit aux yeux du Seigneur, tout comme avait fait David, son ancêtre.
C’est lui qui supprima les lieux sacrés, brisa les stèles, coupa le Poteau sacré
	et mit en pièces le serpent de bronze que Moïse avait fabriqué;
	car jusqu’à ces jours-là les fils d’Israël brûlaient de l’encens devant lui;
	on l’appelait Nehoushtane.
C’est dans le Seigneur, le Dieu d’Israël, qu’Ézékias mit sa confiance.
