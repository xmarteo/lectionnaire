Il y avait à Césarée un homme du nom de Corneille,
	centurion de la cohorte appelée Italique.
C’était quelqu’un de grande piété qui craignait Dieu,
	lui et tous les gens de sa maison;
	il faisait de larges aumônes au peuple juif et priait Dieu sans cesse.
Vers la neuvième heure du jour,
	il eut la vision très claire d’un ange de Dieu qui entrait chez lui
	et lui disait: «Corneille!»
Celui-ci le fixa du regard et, saisi de crainte, demanda:
	«Qu’y a-t-il, Seigneur?»
L’ange lui répondit: «Tes prières et tes aumônes sont montées devant Dieu
	pour qu’il se souvienne de toi.
Et maintenant, envoie des hommes à Jaffa
	et fais venir un certain Simon surnommé Pierre:
	il est logé chez un autre Simon qui travaille le cuir
	et dont la maison est au bord de la mer.»
Après le départ de l’ange qui lui avait parlé,
	il appela deux de ses domestiques
	et l’un des soldats attachés à son service, un homme de grande piété.
Leur ayant tout expliqué, il les envoya à Jaffa.
