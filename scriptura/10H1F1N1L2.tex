Alexandre avait régné douze ans quand il mourut.
Alors, ceux qu’il avait mis en fonction exercèrent le pouvoir, chacun dans sa région.
Après sa mort, ils portèrent tous le diadème,
	et leurs fils après eux, durant de longues années.
	Et ils multiplièrent les malheurs sur la terre.
De leur descendance surgit un homme de péché,
	Antiocos Épiphane, fils du roi Antiocos le Grand.
Il avait séjourné à Rome comme otage, et il devint roi en l’année 137 de l’empire grec.
