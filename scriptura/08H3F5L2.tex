La Sagesse elle-même a veillé sur celui qui fut façonné le premier,
	créé seul, le père du monde;
	puis elle l’arracha à sa propre faute,
	et lui donna la force de dominer toute chose.
Or un homme injuste, pris de colère, se détourna d’elle et périt de cette rage fratricide;
	à cause de lui, la terre fut submergée par le déluge,
	mais la Sagesse, de nouveau, la sauva
		en pilotant le juste sur un simple morceau de bois.
Lorsque les nations, unanimes dans le mal, furent confondues,
	c’est elle qui reconnut le juste, le garda irréprochable devant Dieu,
	et assez fort pour surmonter sa tendresse envers son enfant.
