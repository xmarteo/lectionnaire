Grande était la sagesse de Salomon, plus que la sagesse de tous les fils de l’Orient,
	plus que toute la sagesse de l’Égypte.
Il fut le plus sage des hommes, plus encore qu’Étane l’Ézrahite,
	et que Hémane, Kalkol et Darda, les fils de Mahol:
	son nom était connu de toutes les nations d’alentour.
Il prononça trois mille proverbes et composa des chants au nombre de mille cinq.
Il parla des arbres, depuis le cèdre du Liban jusqu’à l’hysope qui pousse sur le mur;
	il parla des quadrupèdes et des oiseaux, des reptiles et des poissons.
Et l’on venait de tous les peuples pour entendre la sagesse de Salomon,
	on venait de la part de tous les rois de la terre
		qui avaient entendu parler de sa sagesse.
