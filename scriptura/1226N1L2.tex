La parole de Dieu était féconde,
	le nombre des disciples se multipliait fortement à Jérusalem,
	et une grande foule de prêtres juifs parvenaient à l’obéissance de la foi.
Étienne, rempli de la grâce et de la puissance de Dieu,
	accomplissait parmi le peuple des prodiges et des signes éclatants.
Intervinrent alors certaines gens de la synagogue dite des Affranchis,
	ainsi que des Cyrénéens et des Alexandrins,
	et aussi des gens originaires de Cilicie et de la province d’Asie.
Ils se mirent à discuter avec Étienne,
	mais sans pouvoir résister à la sagesse et à l’Esprit
		qui le faisaient parler.
Ceux qui écoutaient ce discours avaient le cœur exaspéré
	et grinçaient des dents contre Étienne.
