Festus, voulant accorder une faveur aux Juifs, s’adressa à Paul:
	«Veux-tu monter à Jérusalem
	pour y être jugé sur cette affaire en ma présence?»
Paul répondit: «Je suis ici devant le tribunal impérial:
	c’est là qu’il me faut être jugé.
	Je ne suis coupable de rien contre les Juifs,
	comme toi-même tu t’en rends fort bien compte.
Si donc je suis coupable et si j’ai fait quelque chose qui mérite la mort,
	je ne refuse pas de mourir.
Mais s’il ne reste rien des accusations que ces gens-là portent contre moi,
	personne ne peut leur faire la faveur de me livrer à eux.
	J’en appelle à l’empereur.»
Alors, après avoir conféré avec son conseil, Festus déclara:
	«Tu en as appelé à l’empereur, tu iras devant l’empereur.»
