«En voici le texte: Mené, Mené, Teqèl, Ou-Pharsine.
Et voici l’interprétation de ces mots:
	Mené (c’est-à-dire “compté”):
	Dieu a compté les jours de ton règne et y a mis fin;
	Teqèl (c’est-à-dire “pesé”):
	tu as été pesé dans la balance, et tu as été trouvé trop léger;
	Ou-Pharsine (c’est-à-dire “partagé”):
	ton royaume a été partagé et donné aux Mèdes et aux Perses.»
Alors, Balthazar ordonna de revêtir Daniel de pourpre,
	de lui mettre au cou un collier d’or
	et de proclamer qu’il deviendrait le troisième personnage du royaume.
Cette nuit-là, Balthazar, le roi des Chaldéens, fut tué.
Darius le Mède reçut le royaume. Il avait soixante-deux ans.
