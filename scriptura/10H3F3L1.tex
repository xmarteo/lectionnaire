Voyant que les circonstances lui étaient favorables,
	Jonathan choisit des hommes qu’il envoya à Rome
	pour confirmer et renouveler l’amitié avec les Romains.
À Sparte également, ainsi qu’en d’autres régions,
	il envoya des lettres rédigées dans le même sens.
Ces hommes se rendirent donc à Rome.
Ils entrèrent au sénat et s’y exprimèrent en ces termes:
	«Le grand prêtre Jonathan et la nation des Juifs
	nous ont envoyés renouveler l’amitié et l’alliance avec eux, comme par le passé.»
On leur donna des lettres pour les autorités locales,
	recommandant de les escorter en paix vers le pays de Juda.
