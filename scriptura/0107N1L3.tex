Ses enfants n’avaient pas encore été mis au monde,
	et n’avaient donc fait ni bien ni mal;
	or, afin que demeure le projet de Dieu qui relève de son choix
	et ne dépend pas des œuvres mais de celui qui appelle,
	il fut dit à cette femme: L’aîné servira le plus jeune,
	comme il est écrit: J’ai aimé Jacob, je n’ai pas aimé Ésaü.
Que dire alors? Y a-t-il de l’injustice en Dieu?
	Pas du tout!
En effet, il dit à Moïse:
	À qui je fais miséricorde, je ferai miséricorde;
	pour qui j’ai de la tendresse, j’aurai de la tendresse.
Il ne s’agit donc pas du vouloir ni de l’effort humain,
	mais de Dieu qui fait miséricorde.
