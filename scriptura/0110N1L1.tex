Accueillez celui qui est faible dans la foi, sans critiquer ses raisonnements.

02 L’un, à cause de sa foi, s’autorise à manger de tout ; l’autre, étant faible, ne mange que des légumes.

03 Que celui qui mange ne méprise pas celui qui ne mange pas, et que celui qui ne mange pas ne juge pas celui qui mange, car Dieu l’a accueilli, lui aussi.

04 Toi, qui es-tu pour juger le serviteur d’un autre ? Qu’il tienne debout ou qu’il tombe, cela regarde son maître à lui. Mais il sera debout, car son maître, le Seigneur, a le pouvoir de le faire tenir debout.
