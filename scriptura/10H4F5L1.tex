Simon, dont il a déjà été question,
	était devenu celui qui dénonçait les richesses de sa patrie.
Il continuait à calomnier Onias,
	en prétendant que c’était lui qui avait assailli Héliodore et provoqué ce malheur.
Il osait présenter le bienfaiteur de la ville,
	le protecteur des gens de sa nation, l’ardent défenseur des lois,
	comme un conspirateur contre les affaires publiques.
Cette haine alla si loin
	que même des meurtres furent commis par l’un des partisans de Simon.
Considérant l’importance de cette rivalité,
	et constatant qu’Apollonios, fils de Ménesthée,
		gouverneur militaire de Cœlé-Syrie et de Phénicie,
	ne faisait qu’accroître la malveillance de Simon,
	Onias se rendit chez le roi, non comme accusateur de ses concitoyens,
	mais pour veiller à l’intérêt général de tout le peuple et de chacun en particulier.
