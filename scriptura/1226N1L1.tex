En ces jours-là, comme le nombre des disciples augmentait,
	les frères de langue grecque récriminèrent contre ceux de langue hébraïque,
	parce que les veuves de leur groupe étaient désavantagées
		dans le service quotidien.
Les Douze convoquèrent alors l’ensemble des disciples et leur dirent:
	«Il n’est pas bon que nous délaissions la parole de Dieu
		pour servir aux tables.
Cherchez plutôt, frères, sept d’entre vous,
	des hommes qui soient estimés de tous,
		remplis d’Esprit Saint et de sagesse,
		et nous les établirons dans cette charge.
En ce qui nous concerne,
	nous resterons assidus à la prière et au service de la Parole.»
Ces propos plurent à tout le monde, et l’on choisit:
	Étienne, homme rempli de foi et d’Esprit Saint,
	Philippe, Procore, Nicanor, Timon, Parménas
	et Nicolas, un converti au judaïsme, originaire d’Antioche.
On les présenta aux Apôtres,
	et après avoir prié, ils leur imposèrent les mains.
