Lorsque celui-ci fut mort, le quatrième frère fut soumis aux mêmes sévices.
Sur le point d’expirer, il parla ainsi:
	«Mieux vaut mourir par la main des hommes,
		quand on attend la résurrection promise par Dieu,
	tandis que toi, tu ne connaîtras pas la résurrection pour la vie.»
On amena aussitôt le cinquième pour le tourmenter.
Fixant les yeux sur le roi, il dit:
	«Tout mortel que tu es, tu as autorité sur les hommes et tu fais ce que tu veux.
	Ne t’imagine pas pour autant que notre race soit abandonnée de Dieu.
Mais toi, attends: tu verras combien sa puissance est grande
	et de quelle manière il sévira contre toi-même et ta descendance!»
Après celui-là, on amena le sixième, et lorsqu’il fut sur le point de mourir, il dit:
	«Ne te fais pas de vaine illusion:
	c’est à cause de nous-mêmes que nous endurons ces souffrances,
	pour avoir péché contre notre propre Dieu.
	De là viennent ces malheurs surprenants.
Mais toi, ne va pas croire que tu resteras impuni,
	pour avoir entrepris de faire la guerre à Dieu.»
