Pour Esther, fille d’Abihaïl, l’oncle de ce Mardochée qui l’avait adoptée comme fille,
	quand vint le tour de se rendre chez le roi,
	elle ne demanda rien d’autre que ce qu’avait indiqué Hégué,
		l’eunuque du roi, gardien des femmes.
Et Esther gagnait la bienveillance de tous ceux qui la voyaient.
Esther fut amenée au roi Assuérus, au palais royal,
	au dixième mois, qui est le mois de Téveth, la septième année de son règne.
	Et le roi la préféra à toutes les autres femmes.
Elle gagna sa bienveillance et sa faveur plus que toutes les autres jeunes filles.
	Il mit sur sa tête la couronne royale et la fit reine à la place de Vasti.
