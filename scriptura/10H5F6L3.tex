Aux premières lueurs de l’aurore, on engagea la lutte de part et d’autre.
Les uns avaient pour gage de succès et de victoire, en plus de leur valeur,
	le recours au Seigneur;
	les autres prenaient leur fureur pour guide des combats.
Au plus fort de la bataille,
	les adversaires virent apparaître, sortant du ciel sur des chevaux harnachés d’or,
	cinq hommes magnifiques qui se mirent à la tête des Juifs.
Plaçant Judas Maccabée au milieu d’eux et le protégeant de leur armement,
	ils le rendaient invulnérable.
Mais sur les adversaires, ils lançaient des flèches et des éclairs,
	si bien que ceux-ci, bouleversés et aveuglés, se dispersaient en pleine panique.
Vingt mille cinq cents furent égorgés, en plus de six cents cavaliers.
Timothée lui-même se réfugia dans une forteresse appelée Gazara,
	importante citadelle que commandait Chéréas.
