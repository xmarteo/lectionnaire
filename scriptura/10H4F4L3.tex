Le grand prêtre, dans la crainte que le roi ne soupçonne les Juifs
		d’avoir commis un mauvais coup contre Héliodore,
	offrit un sacrifice pour le salut de cet homme.
Or, tandis qu’il accomplissait le sacrifice d’expiation,
	les mêmes jeunes gens, revêtus des mêmes habits,
	apparurent une seconde fois à Héliodore.
Ils se tinrent près de lui et lui dirent:
	«Rends pleinement grâce à Onias, le grand prêtre,
	car c’est à cause de lui que le Seigneur t’a accordé la grâce de vivre.
Toi qui as été flagellé par le Ciel,
	proclame devant tous le pouvoir grandiose de Dieu.»
Après ces paroles, ils disparurent.
