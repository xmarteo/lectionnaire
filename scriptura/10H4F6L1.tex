Vers cette époque, Antiocos se mit à préparer sa deuxième expédition contre l’Égypte.
Or, il arriva que dans la ville de Jérusalem tout entière, pendant près de quarante jours,
	apparurent des cavaliers courant dans les airs, vêtus de robes brodées d’or,
	des troupes armées disposées en cohorte, des glaives dégainés,
	des escadrons de cavalerie en ordre de bataille,
	des attaques et des charges lancées de part et d’autre,
	des mouvements de boucliers, des forêts de piques, des projectiles volants,
	un éclat fulgurant d’armures d’or et des cuirasses en tout genre.
Aussi, tous priaient pour que cette apparition soit de bon augure.
