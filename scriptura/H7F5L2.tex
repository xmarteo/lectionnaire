Juda et Israël habitèrent en sécurité, chacun sous sa vigne et sous son figuier,
	de Dane jusqu’à Bershéba, durant toute la vie de Salomon.
Salomon avait quarante mille stalles pour les chevaux de ses chars,
	et douze mille chevaux de selle.
Les préfets mentionnés plus haut
		approvisionnaient le roi Salomon et ceux qui partageaient la table du roi,
	un mois chacun, à tour de rôle:
	ils ne les laissaient manquer de rien.
Quant à l’orge et au fourrage pour les chevaux et les attelages,
	chacun les apportait, sur ordre, là où séjournait le roi.
Dieu donna à Salomon une sagesse et une intelligence très grandes,
	et un cœur aussi vaste que le sable au bord de la mer.
