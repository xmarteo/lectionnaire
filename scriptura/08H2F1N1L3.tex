Moi, Qohèleth, j’étais roi d’Israël à Jérusalem.
J’ai pris à cœur de rechercher et d’explorer, grâce à la sagesse,
	tout ce qui se fait sous le ciel;
	c’est là une rude besogne que Dieu donne aux fils d’Adam
		pour les tenir en haleine.
J’ai vu tout ce qui se fait et se refait sous le soleil.
	Eh bien! Tout cela n’est que vanité et poursuite du vent.
Ce qui est courbé ne se redresse pas et ce qui manque ne peut être compté.
J’ai réfléchi et je me disais:
	C’est moi qui ai fait grandir et progresser la sagesse
		plus que tous mes prédécesseurs à Jérusalem.
	J’ai approfondi la sagesse et le savoir.
J’avais à cœur de connaître la sagesse,
	de connaître aussi la sottise et la folie.
