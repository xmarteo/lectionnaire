«Prospérité aux Romains et à la nation des Juifs, sur mer et sur terre, à jamais!
	Loin d’eux l’épée et l’ennemi!
Mais si une guerre menace Rome la première,
	ou l’un de ses alliés en tout lieu où s’exerce sa domination,
	la nation des Juifs combattra avec elle de tout cœur, selon les exigences du moment.
Aux combattants, on ne donnera rien,
	on ne fournira ni blé, ni armes, ni argent, ni vaisseaux.
	Ainsi en a décidé Rome.
Ils tiendront leurs engagements sans rien recevoir en échange.
De même, si une guerre touche d’abord la nation des Juifs,
	les Romains combattront avec elle de toute leur âme, selon les exigences du moment.»
