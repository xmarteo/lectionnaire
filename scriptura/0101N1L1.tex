Que dirons-nous alors d’Abraham,
	notre ancêtre selon la chair?
	Qu’a-t-il obtenu?
Si Abraham était devenu un homme juste par la pratique des œuvres,
	il aurait pu en tirer fierté, mais pas devant Dieu.
Or, que dit l’Écriture?
	Abraham eut foi en Dieu, et il lui fut accordé d’être juste.
Si quelqu’un accomplit un travail,
	son salaire ne lui est pas accordé comme un don gratuit, mais comme un dû.
Au contraire, si quelqu’un, sans rien accomplir,
		a foi en Celui qui rend juste l’homme impie,
	il lui est accordé d’être juste par sa foi.
C’est ainsi que le psaume de David proclame heureux
		l’homme à qui Dieu accorde d’être juste,
	indépendamment de la pratique des œuvres:
	Heureux ceux dont les offenses ont été remises, et les péchés, effacés.
	Heureux l’homme dont le péché n’est pas compté par le Seigneur.
