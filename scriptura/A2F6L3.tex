Deuil pour le vin nouveau: la vigne a dépéri!
	Tous ceux qui avaient le cœur en fête se lamentent.
Elle a cessé, l’allégresse des tambourins; il a pris fin, le joyeux vacarme;
	elle a cessé, l’allégresse des cithares!
Ils ne boiront plus de vin en chantant;
	la boisson forte est amère aux buveurs.
La cité-du-néant est en ruine,
	chaque maison est fermée, nul ne peut y entrer.
Dans la rue, on réclame du vin;
	toute joie a disparu; l’allégresse est bannie du pays.
Il ne reste de la ville que désolation:
	sa porte est brisée, fracassée.
Au cœur du pays, au milieu des populations,
	il en sera comme à la cueillette des olives, comme au grappillage après la vendange.
Ceux qui restent élèvent la voix, ils crient de joie;
	du côté de la mer, on célèbre la grandeur du Seigneur;
	au pays de la lumière, on glorifie le Seigneur
	et, dans les îles de la mer, le nom du Seigneur, Dieu d’Israël.
Depuis les limites de la terre nous entendons des hymnes:
	«Honneur à Dieu le juste!»
