Ainsi, deux femmes furent déférées en justice pour avoir fait circoncire leurs enfants.
On suspendit leurs nourrissons à leurs seins
	et on les traîna publiquement à travers la ville,
	avant de les précipiter du haut des remparts.
D’autres étaient accourus ensemble dans les cavernes voisines,
	pour y célébrer en cachette le septième jour.
On les dénonça à Philippe, et ils furent tous brûlés,
	car ils s’étaient gardés de se défendre eux-mêmes,
	par respect pour la sainteté du jour.
Je recommande donc à ceux qui liront ce livre
		de ne pas se laisser décourager par ces événements,
	mais de penser que ces châtiments ont eu lieu
	non pour la ruine, mais pour l’éducation de notre race.
