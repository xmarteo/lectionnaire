01 L’homme, né de la femme, vit peu de jours, rassasié de tourments ;

02 comme fleur, il germe et se fane ; tel une ombre, il fuit sans s’arrêter.

03 Et toi, Dieu, c’est sur lui que tu fixes ton regard, c’est moi que tu obliges à comparaître avec toi !

04 Qui tirera le pur de l’impur ? Personne !

05 Puisque ses jours sont décrétés, que tu as décidé du nombre de ses mois, et fixé sa limite, infranchissable,

06 détourne de lui ton regard, et laisse-le, jusqu’à ce que, tel un salarié, il s’acquitte de sa journée !