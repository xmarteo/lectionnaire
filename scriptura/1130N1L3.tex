
16 Et pourtant, tous n’ont pas obéi à la Bonne Nouvelle. Isaïe demande en effet : Qui a cru, Seigneur, en nous entendant parler ?

17 Or la foi naît de ce que l’on entend ; et ce que l’on entend, c’est la parole du Christ.

18 Alors, je pose la question : n’aurait-on pas entendu ? Mais si, bien sûr ! Un psaume le dit : Sur toute la terre se répand leur message, et leurs paroles, jusqu’aux limites du monde.

19 Je pose encore la question : Israël n’aurait-il pas compris ? Moïse, le premier, dit : Je vais vous rendre jaloux par une nation qui n’en est pas une, par une nation stupide je vais vous exaspérer.

20 Et Isaïe a l’audace de dire : Je me suis laissé trouver par ceux qui ne me cherchaient pas, je me suis manifesté à ceux qui ne me demandaient rien.

21 Et à propos d’Israël, il dit : Tout le jour, j’ai tendu les mains vers un peuple qui refuse de croire et qui conteste.