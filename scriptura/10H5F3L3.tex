
34 «Et toi, impie, le plus infâme de tous les hommes, ne t’enfle pas d’orgueil sans raison en te berçant d’espoirs incertains, alors que tu portes la main sur les serviteurs du Ciel,

35 car tu n’as pas encore échappé au jugement du Dieu tout-puissant qui voit tout !

36 Nos frères, maintenant, ont supporté une épreuve passagère, pour une vie intarissable : ils sont tombés à cause de l’alliance de Dieu. Mais toi, par le jugement de Dieu, tu recevras le juste châtiment de ton arrogance.

37 Quant à moi, comme mes frères, je me livre corps et âme pour les lois de nos pères, en suppliant Dieu de se montrer bientôt favorable à la nation et de t’amener, par des épreuves et des fléaux, à confesser que lui seul est Dieu.

38 Je prie aussi pour que la colère du Tout-Puissant, justement déchaînée sur l’ensemble de notre race, prenne fin avec ma mort et celle de mes frères. »

39 Fou de rage, exaspéré par la moquerie, le roi s’acharna contre ce dernier plus cruellement encore que contre les autres.

40 Le jeune homme mourut donc, pur de toute souillure, mettant toute sa confiance dans le Seigneur.

41 Enfin, après tous ses fils, la mère mourut la dernière.