Or, quand il y a testament,
	il est nécessaire que soit constatée la mort de son auteur.
Car un testament ne vaut qu’après la mort,
	il est sans effet tant que son auteur est en vie.
C’est pourquoi le premier Testament lui-même n’a pas été inauguré
	sans que soit utilisé du sang.
Lorsque Moïse eut proclamé chaque commandement à tout le peuple
		conformément à la Loi,
	il prit le sang des veaux et des boucs avec de l’eau,
	de la laine écarlate et de l’hysope,
	et il en aspergea le livre lui-même et tout le peuple,
	en disant: Ceci est le sang de l’Alliance que Dieu a prescrite pour vous.
Puis il aspergea de même avec le sang
	la tente et tous les objets du service liturgique.
D’après la Loi, on purifie presque tout avec du sang,
	et s’il n’y a pas de sang versé, il n’y a pas de pardon.
