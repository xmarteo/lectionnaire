Elle s’en alla, elle se mit à manger, et son visage n’était plus le même.
Le lendemain, Elcana et les siens se levèrent de bon matin.
	Après s’être prosternés devant le Seigneur, ils s’en retournèrent chez eux, à Rama.
Elcana s’unit à Anne sa femme, et le Seigneur se souvint d’elle.
Anne conçut et, le temps venu, elle enfanta un fils;
	elle lui donna le nom de Samuel (c’est-à-dire: Dieu exauce)
	car, disait-elle: «Je l’ai demandé au Seigneur.»
Elcana, son mari, monta au sanctuaire avec toute sa famille
	pour offrir au Seigneur le sacrifice annuel
	et s’acquitter du vœu pour la naissance de l’enfant.
	Mais Anne n’y monta pas.
Elle dit à son mari: «Quand l’enfant sera sevré, je l’emmènerai:
	il sera présenté au Seigneur, et il restera là pour toujours.»
