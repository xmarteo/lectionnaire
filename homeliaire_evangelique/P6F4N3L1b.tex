Il faut d’abord nous demander pourquoi,
		à la naissance du Seigneur, des anges apparurent,
	et pourquoi cependant l’Écriture ne rapporte pas
		qu’ils fussent vêtus de blanc,
	tandis que nous lisons que les anges
		qui furent envoyés lorsque le Seigneur monta au ciel,
	avaient des vêtements blancs.
Voici en effet ce qui est écrit:
	«Tandis que les Apôtres le regardaient,
	il s’éleva, et une nuée vint le soustraire à leurs yeux.
Et comme ils fixaient encore le ciel où Jésus s’en allait,
	voici que, devant eux, se tenaient deux hommes en vêtements blancs.»
Les habits blancs nous indiquent la joie et la fête de l’âme.
Pourquoi donc, à la naissance du Seigneur,
	les anges n’apparaissent-ils pas dans des vêtements blancs,
	tandis qu’à son Ascension, ils apparaissent ainsi?
C’est que ce fut une grande réjouissance pour les anges
	lorsque l’Homme-Dieu entra dans le ciel.
À la naissance du Seigneur, la divinité semblait humiliée;
	lors de son Ascension, l’humanité fut exaltée.
	Les habits blancs conviennent mieux à l’élévation qu’à l’humiliation.
