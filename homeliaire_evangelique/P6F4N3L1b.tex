Il faut d’abord nous demander pourquoi, à la naissance du Seigneur, des Anges apparurent, et pourquoi cependant l’Écriture ne rapporte pas qu’ils fussent vêtus de blanc, tandis que nous lisons que les Anges qui furent envoyés lorsque le Seigneur monta au ciel, avaient des vêtements blancs. Voici en effet ce qui est écrit : « Eux le voyant, il s’éleva, et une nuée le déroba à leurs yeux. Et comme ils le regardaient allant au ciel, voilà que deux hommes se présentèrent devant eux, avec des vêtements blancs » [38]. Les habits blancs nous marquent la joie de l’esprit et la solennité. Qu’est-ce donc à dire que, le Seigneur étant né, les Anges n’apparaissent pas dans des vêtements blancs, et que, le Seigneur montant au ciel, ils apparaissent ainsi, sinon que ce fut une grande réjouissance pour les Anges lorsque l’Homme-Dieu entra dans le ciel ? A la naissance du Seigneur, la divinité semblait humiliée ; lors de son ascension, l’humanité fut exaltée. Les habits blancs conviennent mieux à l’élévation qu’à l’humiliation.