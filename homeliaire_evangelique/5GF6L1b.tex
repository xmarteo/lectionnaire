Mais moi, je vous dit : Aimez vos ennemis, faites du bien à ceux qui vous haïssent. Beaucoup, jugeant des préceptes de Dieu selon leur faiblesse, et non pas selon l’énergie des Saints, pensent que leurs prescriptions sont impossibles et disent qu’il suffit à la vertu de ne pas haïr les ennemis j que d’ailleurs le précepte de les aimer dépasserait ce que l’humaine nature peut porter. Il faut savoir cependant que le Christ ne prescrit pas des choses impossibles, mais des actes de perfection. David l’a fait pour Saül et Absalon. Le martyr Étienne, lui aussi, a prié pour ses ennemis qui le lapidaient, et Paul a souhaité être anathème pour ses persécuteurs. C’est ce que Jésus a fait et enseigné, quand il a dit : Père, pardonnez-leur, ils ne savent ce qu'ils font.
