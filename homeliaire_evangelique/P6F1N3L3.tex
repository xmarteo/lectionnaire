Ensuite il ajouta, comme conséquemment, ces paroles que nous avons entrepris d’expliquer aujourd’hui. « Mais lorsque sera venu le Paraclet que je vous enverrai du Père, l’Esprit de vérité qui procède du Père, il rendra témoignage de moi. Et vous aussi, vous rendrez témoignage, parce que, dès le commencement, vous êtes avec moi » [54]. Quel rapport ces paroles ont-elles avec ce qu’il vient de dire : « Mais maintenant ils ont vu mes œuvres ; et ils ont haï et moi et mon Père, mais c’est afin que s’accomplisse la parole qui est écrite dans leur loi : Ils m’ont haï gratuitement » [55]. Quand le Paraclet est venu, cet Esprit de vérité a-t-il convaincu par un témoignage plus évident ceux qui avaient vu ses œuvres et le haïssaient encore ? Il a fait bien plus ; car, en se manifestant à eux, il a converti à la foi, qui opère par la charité, plusieurs de ceux qui avaient vu, et dont la haine persévérait encore.