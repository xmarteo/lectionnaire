«Il rendra donc témoignage de moi, et vous aussi vous rendrez témoignage.»
Car la charité de Dieu,
	«répandue dans vos cœurs par l’Esprit Saint qui vous sera donné,»
	vous inspirera le courage de rendre ce témoignage.
C’est ce courage qui a manqué à Pierre,
	lorsque, effrayé par la question d’une servante,
	il ne put rendre un témoignage véritable
	et fut entraîné par la crainte à renier trois fois son Maître,
	malgré la promesse qu’il avait faite.
Or il n’y a pas de crainte dans la charité;
	mais la charité parfaite chasse la crainte.
Ainsi, avant la passion du Seigneur,
	la crainte servile de Pierre fut interrogée par une femme esclave;
	mais après la Résurrection du Seigneur,
	son amour libre fut interrogé par l’auteur même de la liberté.
Aussi dans le premier cas fut-il troublé,
	dans le second, il était tranquille;
	là il renia celui qu’il avait aimé,
	ici il aimait celui qu’il avait renié.
Cependant, même à ce moment, son amour était encore faible et borné,
	jusqu’à ce que le Saint-Esprit l’ait fortifié et dilaté.
