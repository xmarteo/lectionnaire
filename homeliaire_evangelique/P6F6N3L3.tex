« Il rendra donc témoignage de moi, et vous aussi vous rendrez témoignage ». « Car la charité de Dieu, répandue dans vos cœurs par l’Esprit-Saint qui vous sera donné », vous inspirera le courage de rendre ce témoignage. C’est ce courage qui a fait défaut à Pierre, lorsque, effrayé de la question d’une servante, il ne put rendre un témoignage véritable et fut entraîné par la crainte à renier trois fois son Maître, malgré la promesse qu’il avait faite. Or « il n’y a point de crainte dans la charité ; mais la charité parfaite chasse la crainte ». Ainsi, avant la passion du Seigneur, la crainte servile de Pierre fut interrogée par une femme esclave ; mais après la résurrection du Seigneur, c’est l’amour d’un cœur libre qui est interrogé par l’auteur même de la liberté. Aussi dans le premier cas fut-il troublé, dans le second, il était dans le calme ; alors il renia celui qu’il avait aimé, en ce moment il aimait celui qu’il avait renié. Cependant après la résurrection son amour même était encore faible et borné, il le fut jusqu’à ce que le Saint-Esprit le fortifiât et l’agrandît.