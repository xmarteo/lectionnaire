 La leçon du saint Évangile qui vient d'être lue à vos oreilles pose à l’esprit une question, mais indique en la posant la puissance de la vertu de discrétion. On peut en effet se demander pourquoi Pierre qui était pêcheur avant sa conversion est retourné à la pêche après sa conversion, alors que la Vérité dit : Personne de ceux qui mettent la main à la charrue et regardent en arrière n'est apte au royaume de Dieu. Pourquoi a-t-il repris ce qu’il avait abandonné ? Mais si l’on consulte la vertu de discrétion, il apparaît bien vite que certainement ce ne fut pas un péché de reprendre après la conversion une occupation qui, avant la conversion, se pratiquait sans péché.