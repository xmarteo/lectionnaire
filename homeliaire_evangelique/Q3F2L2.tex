Ce n’est donc pas sans motifs
		que le Sauveur s’excuse de n’avoir accompli dans sa patrie
	aucun prodige de sa puissance;
	il ne voulait pas que personne pensât que l’amour dû à la patrie
	était un sentiment méprisable.
Car il ne pouvait pas ne point aimer ses concitoyens,
	celui qui devait aimer tous les hommes;
	mais ceux-là mêmes manquent à l’amour de la patrie, par leur envie.
Je vous le dis en vérité: il y avait beaucoup de veuves aux jours d’Élie.
Ce n’est pas que ces jours aient appartenu à Elie,
	mais c’étaient les jours pendant lesquels Élie opéra des prodiges,
	ou qu’Élie utilisait pour ceux
		qui dans ses œuvres apercevaient la lumière de la grâce spirituelle
	et se convertissaient au Seigneur.
Et c’est pourquoi le ciel était ouvert
		à ceux qui distinguaient les mystères éternels et divins,
	et il était fermé, et c’était la famine,
	quand, sur la terre, il n’y avait plus aucun désir de connaître la divinité.
	Mais nous avons parlé de cela plus amplement,
		quand nous écrivions au sujet des veuves.
