En effet, leur étonnement et les paroles qu’il leur inspire, donnent lieu au Seigneur de leur révéler une vérité profonde, bien digne d’être soigneusement méditée et expliquée. Que répond donc le Seigneur à ceux qui s’étonnaient qu’il sût les écritures sans les avoir apprises ? « Ma doctrine n’est pas de moi, mais de celui qui m’a envoyé. » Voici une première profondeur, car ce peu de paroles semble renfermer une contradiction. En effet, il ne dit pas : Cette doctrine n’est pas la mienne ; mais il dit : « Ma doctrine n’est pas de moi. » Si cette doctrine n’est pas de vous, comment est-elle la vôtre ? Et si elle est la vôtre, comment se fait-il qu’elle ne vienne pas de vous ? Vous dites pourtant l’un et l’autre : « c’est ma doctrine », et, « elle n’est pas de moi. »
