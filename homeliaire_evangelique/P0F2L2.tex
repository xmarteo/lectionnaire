 Il engagea donc la conversation, leur reprocha la dureté de leur intelligence, leur expliqua les mystères de la sainte Écriture qui le concernaient; et cependant, parce que dans leurs coeurs il restait un étranger pour leur foi, il feignit d’aller plus loin. Feindre en effet, (fingere) c’est, disons-nous, composer une figure. De là vient que nous appelons « figulos » ceux qui composent des figures d’argile. La pure Vérité n’a rien fait par duplicité ; mais, dans son corps, elle s’est montrée à eux telle qu’elle était en eux dans leur âme. Il fallait éprouver si ceux qui ne l’aimaient pas encore comme Dieu, pourraient du moins l’aimer sous l’aspect d’un étranger.