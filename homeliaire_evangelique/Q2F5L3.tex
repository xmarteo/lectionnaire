Il se servait de la connaissance de la loi, non pour la charité,
	mais pour sa vanité, comme gonflé par les richesses reçues;
	et parce que ses paroles découlaient de sa science,
	elles étaient comme les miettes qui tombaient de sa table.
Au contraire, pour le pauvre gisant à terre,
	les chiens léchaient ses blessures.
Parfois, dans le langage sacré,
	on entend volontiers sous le nom de chiens, les prédicateurs.
En effet, la langue des chiens guérit la blessure qu’elle lèche;
	aussi les saints docteurs,
		tandis qu’ils nous apprennent à reconnaître le péché,
	touchent, comme avec leur langue, la blessure de notre âme.
