Quelqu’un demandera : Comment Lazare pouvait-il être une figure du pécheur, et être aimé ainsi par le Seigneur ? Que celui-là écoute Jésus-Christ, disant : « Je ne suis pas venu appeler les justes, mais les pécheurs. » Si Dieu n’aimait pas les pécheurs, il ne serait pas descendu du ciel sur la terre. « Or Jésus, entendant cela, leur dit : Cette maladie ne va pas à la mort, mais elle est pour la gloire de Dieu, afin que le Fils de Dieu en soit glorifié. » Cette glorification du Fils de Dieu n’a pas augmenté sa gloire, mais elle nous a été utile. Il dit donc : « Cette maladie ne va pas à la mort », parce que la mort même de Lazare n’allait point à la mort, mais bien plutôt au miracle qui devait s’accomplir pour amener les hommes à croire en Jésus-Christ, et à éviter la véritable mort. Considérez ici comment notre Seigneur donne une preuve indirecte de sa divinité, contre ceux qui nient que le Fils soit Dieu lui-même.
