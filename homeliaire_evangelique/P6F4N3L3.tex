C’est à bon droit que le Seigneur est appelé oiseau, puisqu’il a soutenu dans les airs un corps de chair. Quiconque n’a pas cru qu’il est monté au ciel, a ignoré le chemin de l’oiseau. C’est de cette solennité que le Psalmiste a dit : « Votre magnificence est élevée au-dessus des cieux » [45]. C’est de cette solennité qu’il dit aussi : « Dieu est monté au milieu des acclamations de joie, et le Seigneur au son de la trompette » [46]. C’est de cette solennité qu’il dit encore : « Montant au ciel, il a conduit une captivité captive, il a donné des dons aux hommes » [47]. En effet, montant au ciel, il a conduit une captivité captive, puisqu’il a anéanti notre corruption, par la vertu de son incorruption. Il a donné des dons aux hommes, car en leur envoyant d’en haut l’Esprit-Saint, il a accordé à l’un la parole de sagesse, à un autre la parole de science, à un autre la grâce de force, à un autre la grâce de guérison, à un autre diverses espèces de langues, à un autre l’interprétation des discours.