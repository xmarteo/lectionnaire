C’est à bon droit que le Seigneur est appelé oiseau,
	puisqu’il a soutenu dans les airs un corps de chair.
Quiconque n’a pas cru qu’il est monté au ciel,
	a ignoré le chemin de l’oiseau.
C’est de cette solennité que le Psalmiste a dit:
	«Votre magnificence est élevée au-dessus des cieux.»
C’est de cette solennité qu’il dit aussi:
	«Dieu est monté au milieu des acclamations de joie,
	et le Seigneur au son de la trompette.»
C’est de cette solennité qu’il dit encore:
	«Montant au ciel, il a conduit la captivité captive,
	il a donné des dons aux hommes.»
En effet, montant au ciel, il a conduit la captivité captive,
	puisqu’il a anéanti toute corruption, par la vertu de son incorruption.
Il a donné des dons aux hommes, car en leur envoyant d’en haut l’Esprit Saint,
	il a accordé à l’un la parole de sagesse,
	à un autre la parole de science,
	à un autre la grâce de force,
	à un autre la grâce de guérison,
	à un autre le don de diverses langues,
	à un autre l’interprétation des discours.
