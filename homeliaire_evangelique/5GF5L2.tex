Si en effet les choses se sont ainsi passées, comment sera-t-il vrai, le récit de Matthieu disant : Un centurion s'approcha de lut, alors qu’il ne s’est pas approché en personne, mais qu’il a envoyé ses amis ? Ce ne sera vrai que si, avec une attention diligente, nous comprenons que Matthieu ne s’est pas tellement écarté de nos façons habituelles de parler. Car non seulement nous avons coutume de dire que quelqu’un s’approche, avant même qu’il soit arrivé au lieu dont nous disons qu’il s’est approché, puisque nous disons qu’il s’est peu ou beaucoup approché du lieu où il désire arriver, mais nous disons même très souvent que c’est chose faîte, quand on y arrive par un ami, sans même voir celui qui est touché et dont la faveur nous est nécessaire. Cette manière de dire est d’usage si courant que le vulgaire donne le nom d’arrivistes à ceux qui, possédant 1 ’art de l’intrigue, atteignent, par l’intermédiaire de personnes convenablement choisies, les esprits de certains puissants personnages qui paraissent inaccessibles d’autre façon.
