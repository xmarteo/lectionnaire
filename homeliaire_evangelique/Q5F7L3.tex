 La foule le saluait donc par ces acclamations:
	«Hosanna! béni celui qui vient au nom du Seigneur comme roi d’Israël.»
Quelle torture l’esprit envieux des princes des Juifs ne devait-il pas souffrir
	lorsqu’une si grande multitude acclamait le Christ comme son roi?
Mais qu’était-ce pour le Seigneur que d’être roi d’Israël?
	Était-ce quelque chose de grand pour le roi des siècles,
	de devenir roi des hommes?
Le Christ ne fut pas roi d’Israël pour exiger des tributs,
	armer de fer des bataillons et dompter visiblement ses ennemis,
	mais il est roi d’Israël parce qu’il gouverne les âmes,
	parce qu’il veille sur elles pour l’éternité,
	parce qu’il conduit au royaume des Cieux ceux qui croient en lui,
	qui espèrent en lui et qui l’aiment.
