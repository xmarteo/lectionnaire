Ordinairement, qui veut être cru commence par prouver qu’il est digne de foi. Aussi l’Ange, annonçant à Marie le mystère qui va s’accomplir, lui dit, pour la convaincre, qu’une femme stérile et avancée en âge est devenue mère, et il lui fait ainsi comprendre que Dieu peut tout ce qui lui plaît. Dès que Marie eut appris cette nouvelle, elle se dirigea vers les montagnes. Ce n’est pas qu’elle fût incrédule à l’oracle de l’Ange, ou qu’elle doutât de la réalité de la mission du messager qui lui était envoyé, ni même qu’elle hésitât sur l’exemple qui lui avait été donné ; mais Marie était heureuse de voir les désirs de sa cousine réalisés ; elle voulut remplir, en la visitant, un pieux devoir, et partit en toute hâte, car la joie la transportait. Déjà pleine de Dieu, où se dirigerait-elle avec tant d’empressement, sinon vers des régions plus élevées ? La grâce du Saint-Esprit ne connaît pas de lenteur dans les efforts qu’elle inspire.