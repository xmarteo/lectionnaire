 « Vous me chercherez, et vous ne me trouverez pas, et là où je suis vous ne pouvez venir. » Ces paroles sont déjà une prédiction de sa résurrection ; les Juifs, en effet, n’ont pas voulu le reconnaître lorsqu’il était présent au milieu d’eux, et ils le cherchèrent ensuite lorsqu’ils virent la multitude qui croyait en lui. En effet, il s’opéra de grands prodiges au temps de la résurrection et de l’ascension du Seigneur. Les disciples firent alors des miracles éclatants, mais ce fut lui qui les accomplit par eux comme il en avait opéré par lui-même, car il leur avait dit : « Vous ne pouvez rien faire sans moi. » Lorsque le boiteux qui était assis à la porte du temple se leva à la voix de Pierre, se tint sur ses pieds et marcha, tous furent dans l’admiration : alors le Prince des Apôtres leur adressa la parole, et leur déclara que s’il avait guéri cet homme ce n’était point en vertu de son propre pouvoir, mais que c’était par la puissance de celui qu’ils avaient mis à mort. Beaucoup, touchés de componction, lui dirent : « Que ferons-nous ? »