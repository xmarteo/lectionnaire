 Donc, en cette fête qui est sienne et nôtre, l’Ange est apparu en vêtements blancs, parce que tandis que nous sommes ramenés au monde d’en haut par la résurrection du Seigneur, les dommages faits à la patrie céleste sont réparés, Mais écoutons comment il parle aux femmes venues au tombeau : Ne vous effrayez pas, comme s’il disait ouvertement : « Qu’ils s’effraient, ceux qui n’aiment pas la visite des citoyens d’en haut ; qu’ils craignent, ceux qui, oppressés par leurs désirs charnels, désespèrent de pouvoir arriver à la société de ces esprits célestes. Mais vous, pourquoi craindriez-vous, vous qui voyez vos concitoyens ? C’est pourquoi saint Matthieu, décrivant lui aussi l’apparition de l’Ange, dit : Son aspect était comme celui de Véclair et ses vêtements comme la neige. Dans l’éclair en effet, il y a l’éclat terrifiant qui fait craindre, mais dans la neige, la caresse de la blancheur.