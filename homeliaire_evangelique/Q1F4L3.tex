De plus, le signe de Jonas,
	en même temps qu’une figure de la passion du Seigneur,
	est aussi l’attestation des graves péchés commis par les Juifs.
On peut y remarquer à la fois,
	et l’oracle prophétique de la majesté et la note de miséricorde.
Car l’exemple des Ninivites dénonce le châtiment et manifeste le remède.
D’où, même pour les Juifs,
	le devoir de ne pas désespérer du pardon, s’ils veulent faire pénitence.
