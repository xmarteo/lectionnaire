Notre Seigneur, Fils unique du Père et coéternel avec lui,
	ayant pris la forme d’un esclave,
	pouvait, en cette forme d’esclave, prier en silence, s’il le fallait;
	mais il a voulu se présenter en suppliant devant son Père,
	comme pour se rappeler qu’il est notre docteur.
C’est pourquoi la prière qu’il a faite pour nous, il nous l’a fait connaître;
	car l’édification des disciples ressort
		non seulement des leçons que leur donne un si grand maître,
	mais encore de la prière qu’il adresse pour eux à son Père.
Et si elle fut l’édification des témoins qui devaient entendre ces paroles,
	elle l’est assurément aussi de nous-mêmes qui devions en lire le récit.
