 Considérons maintenant quels sont ceux qui, après avoir été convaincus de péché, furent chassés du paradis et relégués dans une demeure vulgaire que je compare à ce bourg. Et voyez de quelle manière la Vie rappelle ceux que la mort avait exilés. Nous lisons dans saint Matthieu que le Fils de Dieu envoya délier un ânon et une ânesse, afin que, comme l’un et l’autre sexe avaient été chassés du paradis en la personne de nos premiers parents, il montrât par le symbole de ces deux animaux, qu’il venait rappeler les deux sexes. Il semble que l’ânesse figurait Ève coupable, et l’ânon désignait la généralité du peuple gentil : c’est pourquoi le Sauveur s’assit sur le petit de l’ânesse. Il est dit justement que personne n’avait encore monté cet ânon, parce que personne avant le Christ n’avait appelé les peuples de la gentilité à entrer dans l’Église. On lit en effet dans saint Marc : « Vous trouverez un ânon lié, sur lequel aucun homme ne s’est encore assis. »