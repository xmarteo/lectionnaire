A quel temps le Précurseur de notre Rédempteur reçut le mandat de prêcher, nous le trouvons indiqué par la double mention que fait l’Évangile, et du chef de l’empire Romain et des rois de la Judée. Il venait annoncer Celui qui allait en racheter quelques-uns d’entre les Juifs et un plus grand nombre d’entre les Gentils : voilà pourquoi on précise l’époque de sa prédication, en citant et un empereur des Gentils et les princes des Juifs. La Gentilité devait être rassemblée, tandis que la nation juive allait être dispersée, en punition de sa perfidie, cela aussi nous est indiqué par la mention faite des chefs du pouvoir civil : un seul, dit l’Évangile, dominait dans la République Romaine, tandis que plusieurs princes commandaient dans la Judée, divisée en quatre parties.