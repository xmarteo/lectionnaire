Le moment où le précurseur de notre Rédempteur reçut le mandat de prêcher,
	est indiqué par la double mention du chef de l’empire Romain,
		et des rois de la Judée.
Il venait annoncer celui qui allait racheter quelques-uns parmi les Juifs
	et un grand nombre d’entre les Gentils:
	voilà pourquoi on précise l’époque de sa prédication,
	en citant un empereur des Gentils et les princes des Juifs.
La Gentilité devait être rassemblée,
	tandis que la Judée allait être dispersée, en punition de son incrédulité,
	comme l’indique la mention faite des chefs du pouvoir civil:
	un seul dominait dans l’Empire romain,
	tandis que plusieurs princes commandaient dans la Judée,
	divisée en quatre parties.
