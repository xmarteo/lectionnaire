Il y avait aussi beaucoup de lépreux en Judée, au temps du prophète Élisée;
	et aucun d’eux ne fut guéri sinon Naaman le Syrien.
Évidemment, la parole salutaire du Seigneur nous donne un enseignement
	et nous exhorte à rendre un culte à la divinité,
	puisque personne n’a paru guéri et délivré
		de cette maladie qui tache le corps,
	sinon celui qui a cherché sa guérison
		dans l’accomplissement du devoir religieux.
Car ce n’est point aux âmes endormies, mais à celles qui veillent,
	que sont accordés les bienfaits divins.
Nous avons dit, dans un autre livre,
	au sujet de cette veuve vers laquelle Élie fut adressé,
	qu’elle était une figure prophétique de l’Église.
Un peuple forma le noyau de l’Église,
	pour que le suivît cet autre peuple formé des nations étrangères.
Cet autre peuple était jusqu’alors lépreux,
	ce peuple était jusqu’alors souillé,
	avant d’être baptisé dans le fleuve mystique;
	ce même peuple, après les mystères du baptême,
	purifié de ses taches corporelles et spirituelles, n’est plus un lépreux,
	mais il est devenu une vierge immaculée et sans rides.
