 Considérez, mes très chers frères, la mansuétude de Dieu. Le Sauveur était venu effacer les péchés du monde, et il disait : « Qui de vous me convaincra de péché ? » Il ne dédaigne pas de montrer par le raisonnement qu’il n’est pas un pécheur, lui qui, par la vertu de sa divinité, avait le pouvoir de justifier les pécheurs. Les paroles qui suivent sont vraiment terribles : « Celui qui est de Dieu écoute les paroles de Dieu. Et si vous ne les écoutez peint c’est que vous n’êtes point de Dieu. » Si donc celui qui est de Dieu entend les paroles de Dieu, et si au contraire celui qui n’est pas de Dieu ne peut les entendre, que chacun se demande si l’oreille de son cœur perçoit les paroles de Dieu, et il connaîtra à qui il appartient. La Vérité ordonne de désirer la patrie céleste, de fouler aux pieds les désirs de la chair, de fuir la gloire du monde, de ne point convoiter le bien d’autrui, et de donner généreusement ce que l’on possède.