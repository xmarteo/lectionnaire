Ce que vous venez d’entendre est un grand mystère. Demande à un homme : Es-tu chrétien ? Il te répond : Je ne le suis pas. Tu lui demandes encore : Es-tu païen ou juif ? S’il te répond : Je ne le suis pas ; tu continues de l’interroger : Es-tu catéchumène ou fidèle ? S’il te répond : Catéchumène, il a été oint, mais non encore lavé. Comment a-t-il été oint ? Interroge-le, et il te répondra. Demande-lui en qui il croit ? Par cela même qu’il est catéchumène, il te dira : Je crois au Christ. Je m’adresse maintenant aux fidèles et aux catéchumènes. Qu’ai-je dit de la salive et de la boue ? Que le Verbe s’est fait chair. C’est ce qui est enseigné aux catéchumènes ; mais il ne leur suffit pas d’avoir été oints : qu’ils se hâtent vers le bain salutaire, s’ils recherchent la lumière.
