 Le Christ est peut-être lui-même aussi cette montagne. Quel autre que lui, produirait en effet une telle moisson d’oliviers ? non pas de ces oliviers qui ploient sous l’abondance de leurs fruits, mais de ceux qui prouvent leur fécondité en communiquant aux nations la plénitude du Saint-Esprit. Il est celui par qui nous montons et vers qui nous montons. Il est la porte et il est la voie ; il est la porte qui s’ouvre et il est celui qui l’ouvre, la porte à laquelle frappent ceux qui veulent entrer, et le Dieu qu’adorent ceux qui ont mérité d’entrer. Jésus était donc dans un bourg, et il y avait un ânon lié auprès de sa mère ; cet ânon, il ne pouvait être détaché que sur l’ordre du Seigneur. La main d’un Apôtre le délie. Telles sont les actions, telle est la vie, telle est la grâce. Soyez donc tels vous aussi, que vous puissiez délivrer ceux qui sont liés.