« Et il vint dans toute la région du Jourdain, prêchant le baptême de pénitence pour la rémission des péchés. » Il est évident pour tous les lecteurs, que Jean n’a pas seulement prêché le baptême de la pénitence, mais qu’il le donna aussi à plusieurs : cependant il n’a pas pu donner son baptême en rémission des péchés, car la rémission des péchés ne nous est accordée que par le seul baptême du Christ. Aussi, faut-il remarquer qu’il est dit : « prêchant le baptême de la pénitence pour la rémission des péchés » ; car, ne pouvant donner le baptême qui remet les péchés, il le prêchait. De sorte que, comme la parole de sa prédication était l’avant-coureur de la Parole incarnée du Père, de même son baptême, par lequel les péchés ne pouvaient être remis, fut l’avant-coureur du baptême de pénitence qui remet les péchés.