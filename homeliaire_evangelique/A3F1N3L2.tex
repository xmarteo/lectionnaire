Mais le souvenir d’une parole de notre Rédempteur tirée d’un autre passage
	soulève dans cette lecture un problème complexe.
En effet, dans un autre passage,
	le Seigneur, interrogé par ses disciples sur la venue d’Élie,
	répondit: «Élie est déjà venu;
	au lieu de le reconnaître, ils lui ont fait tout ce qu’ils ont voulu.»
Et, si vous voulez le savoir, c’est Jean qui est Élie.
Jean, cependant, étant interrogé,
	dit: «Je ne suis pas Élie.»
Comment se fait-il, mes frères, que la Vérité affirme une chose
	et que le Prophète de la Vérité la nie?
Car il y a opposition complète entre ces expressions:
	«Il est» et «Je ne suis pas».
Comment donc est-il le Prophète de la Vérité,
	s’il n’est pas d’accord avec les paroles de cette même Vérité?
