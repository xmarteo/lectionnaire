Mais, quand revient à l’esprit une autre parole de notre Rédempteur, les expressions que nous venons de lire soulèvent une question très compliquée. En effet, dans un autre endroit, le Seigneur, interrogé par ses disciples au sujet de l’avènement d’Élie, répondit : « Élie est déjà venu, et ils ne l’ont pas connu ; mais ils lui ont fait tout ce qu’ils ont voulu : et, si vous voulez le savoir, Jean lui-même est Élie. » — Jean, cependant, étant interrogé, dit : « Je ne suis point Élie. » Comment se fait-il, mes frères, que la Vérité affirme une chose et que le Prophète de la Vérité la nie ? Car il y a opposition complète entre ces expressions : « Il est », et, « Je ne suis pas. » Comment donc est-il le Prophète de la Vérité, fil n’est pas d’accord avec les paroles de cette même Vérité ?