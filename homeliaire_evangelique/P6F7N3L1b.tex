Notre Seigneur, en disant : « Je prierai mon Père et il vous donnera un autre Paraclet », fait voir que lui-même est aussi un Paraclet. Paraclet se traduit en latin par advocatus (avocat) ; or, il a été dit du Christ : « Nous avons pour avocat auprès du Père, Jésus-Christ le juste » [34]. Le Sauveur déclare que le monde ne peut recevoir l’Esprit-Saint, dans le même sens où il a été dit : « La prudence de la chair est ennemie de Dieu ; car elle n’est pas soumise à la loi et ne peut l’être » [35]. C’est comme si nous disions : L’injustice ne peut être la justice. Par ces mots « le monde ». il désigne ici ceux qui sont pleins de l’amour du monde, amour qui ne vient pas du Père. C’est pourquoi, à l’amour de ce monde, que nous avons tant de peine à diminuer et à détruire en nous, est opposé « l’amour de Dieu, que répand dans nos cœurs l’Esprit-Saint, qui nous a été donné »