En effet, ils lient de lourds fardeaux, qu'on ne peut porter, et les placent sur les épaules des hommes, alors qu'ils ne veulent pas les remuer du bout de leur doigt. Cela se dit d’une façon générale contre tous les maîtres qui commandent des choses difficiles et ne font pas les plus faciles. Mais il faut noter que les mots : épaules, fardeaux, doigts, et liens avec lesquels ces fardeaux sont liés, doivent s’entendre au sens spirituel. Mais ils font toutes leurs actions pour être vus des hommes. Donc, quiconque fait quoi que ce soit pour être vu des hommes est un scribe et un pharisien.
