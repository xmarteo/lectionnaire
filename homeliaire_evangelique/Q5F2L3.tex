 Ils se voyaient sous le poids d’un crime énorme d’impiété,
	ayant mis à mort celui qu’ils auraient dû respecter et adorer;
	et il leur semblait impossible d’expier leur crime:
	crime énorme, en effet, dont la vue les jetait dans le désespoir;
	mais ils ne devaient pas désespérer,
	puisque le Seigneur suspendu à la croix avait daigné prier pour eux,
	en disant: «Mon Père, pardonnez-leur, car ils ne savent ce qu’ils font.»
Parmi un grand nombre d’hommes qui lui étaient étrangers,
	Jésus mourant distinguait ceux qui lui appartenaient,
	et il demandait le pardon de ceux qui l’insultaient encore;
	car il ne considérait pas que les hommes le faisaient mourir,
	mais bien qu’il mourait pour eux.