 On peut aussi se demander pourquoi, tandis que les disciples peinaient en mer, le Seigneur, après sa résurrection, s’est tenu sur le rivage ; alors qu’avant sa résurrection il a marché sur les flots en présence de ses disciples. On peut rapidement en savoir la raison si l’on considère ce qui alors la motivait. Qu’est-ce que la mer représente en effet, si ce n’est le siècle présent qui se précipite dans le vacarme des catastrophes et les vagues de la vie corruptible ? Et que figure la solidité du rivage, si ce n’est la perpétuité du repos éternel ? Donc, puisque les disciples étaient encore dans les flots de la vie mortelle, ils peinaient en mer ; mais parce que notre Rédempteur avait déjà dépassé l’état de corruption de la chair, après sa résurrection, il se tenait sur le rivage.