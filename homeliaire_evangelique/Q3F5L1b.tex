Voyez la clémence du Seigneur, notre Sauveur. Il ne s’émeut pas d’indignation, ni ne s’offense du crime, ni ne se froisse de l’injure, et il n’abandonne pas le pays des Juifs. Bien plus, oublieux de l’injure et se souvenant de sa clémence, il cherche à gagner le coeur de ce peuple infidèle, tantôt en l’instruisant, tantôt en délivrant les possédés, tantôt en guérissant les malades. Et c’est avec raison que saint Luc, après avoir présenté un homme délivré de l’esprit mauvais, continue par le récit de la guérison d’une femme. En effet, le Seigneur était venu pour guérir Pun et l’autre sexe ; mais il devait guérir d’abord celui qui avait été créé le premier, puis ne pas laisser de côté celle qui avait péché par légèreté d’esprit plus que par perversité.
