 Afin que les hommes ne s’imaginassent point que Lazare était un fantôme et n’avait pas été vraiment ressuscité, il était du nombre de ceux qui se trouvaient à table ; il était vivant, il parlait, il prenait part au festin : la vérité se manifestait ainsi au grand jour, et l’incrédulité des Juifs se trouvait confondue. Jésus était donc à table avec Lazare et les autres, et Marthe, une des sœurs de Lazare, les servait. « Or Marie », l’autre sœur de Lazare, « prit une livre d’un nard pur de grand prix, elle en oignit les pieds de Jésus, et les essuya avec ses cheveux, et la maison fut remplie de l’odeur du parfum. » Vous avez entendu le récit du fait, cherchons le mystère qu’il renferme.