Et ses disciples s’approchant, lui faisaient cette prière : Renvoyez-la, car elle crie après nous. L, es disciples, en ce temps-là encore, ignoraient les mystères du Seigneur, et c’était, ou par pitié qu’ils intercédaient en faveur de la chananéenne (qu’un autre évangéliste appelle syro-phénicienne), ou par désir d’écarter son im- portunité, car elle le poursuivait de cris de plus en plus pressants, comme un médecin insensible. Mais lui, leur fit cette réponse : Je n'ai été envoyé qu'aux brebis perdues de la maison d'Israël. Non qu’il n’ait pas été envoyé aussi aux Gentils, mais parce qu’il a été envoyé d’abord à Israël : ceux-ci refusant l’Évangile, le passage aux Gentils serait ainsi justifié.
