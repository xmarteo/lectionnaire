Marie-Madeleine, qui avait été une pécheresse connue comme telle dans la cité,
	lava de ses larmes les souillures de sa faute,
	en aimant la vérité;
	et elle réalisa cette parole de la Vérité:
	«Beaucoup de péchés lui ont été remis, parce qu’elle a beaucoup aimé.»
Celle qui était restée froide en péchant,
	devint ensuite de feu, par son ardent amour.
Car étant venue au tombeau et n’y ayant pas trouvé le corps du Seigneur,
	elle le crut enlevé et l’annonça aux disciples;
	ceux-ci vinrent, ils virent
		et pensèrent qu’il en était comme cette femme le leur avait dit.
Et c’est alors que l’Écriture dit d’eux:
	«Les disciples s’en allèrent donc chez eux;»
	puis elle ajoute:
	«Mais Marie se tenait près du tombeau, au dehors, et elle pleurait.»
