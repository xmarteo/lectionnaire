Souvenez-vous que je vous ai proposé le nombre trente-huit au sujet de ce malade. Je veux vous expliquer pourquoi ce nombre de trente-huit appartient plutôt à la maladie qu’à la santé. Donc, comme je le disais, la charité accomplit la loi ; à la plénitude de la loi dans toutes les œuvres se rapporte le nombre de quarante. Or, à l’égard de la charité, on nous recommande deux préceptes : Tu aimeras le Seigneur ton Dieu, de tout ton cœur, de toute ton âme, de tout ton esprit, et tu aimeras le prochain comme toi-même. En ces deux commandements tient toute la loi et les prophètes. C’est à juste titre que la veuve remit tous ses biens, soit deux oboles, en offrande à Dieu ; à juste titre aussi quel’hôtelier reçut deux deniers pour guérir le blessé frappé par les brigands ; à juste titre que Jésus passa deux jours chez les Samaritains pour les confirmer dans la charité. Quand donc ce nombre deux esc signalé pour accompagner quelque bonne œuvre, c’est surtout pour nous recommander la charité au double précepte. Mais si le nombre quarante marque la perfection de la loi et si l’on n’observe pleinement la loi que par le double précepte de la charité, pourquoi s’étonner s’il languissait, l’homme à qui manquait le nombre deux pour atteindre quarante ?
