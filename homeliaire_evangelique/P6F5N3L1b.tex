Habacuc, lui aussi, parle de la gloire de l’ascension du Christ quand il dit : « Le soleil s’est élevé, et la lune s’est tenue en son rang ». Qui est désigné sous le nom de soleil sinon le Seigneur, et que signifie la lune, sinon l’Église ? Jusqu’à ce que le Seigneur monte au ciel, son Église sainte a redouté de toutes façons l’hostilité du monde, mais après avoir été fortifiée par son ascension, elle a prêché ouvertement ce qu’elle avait cru en secret. Le soleil s’est donc élevé, et la lune s’est tenue en son rang, parce que, dès que le Seigneur eut gagné le ciel, son Église sainte a grandi dans l’autorité de la prédication.