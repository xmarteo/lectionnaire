Ce prodige a donc été présenté à nos sens pour élever notre esprit ; il a été placé sous nos yeux pour exercer notre intelligence. Alors, admirant le Dieu invisible à travers ses œuvres visibles, élevés jusqu’à la foi et purifiés par la foi, nous désirerons même voir l’Invisible en personne ; cet Invisible que nous connaissons à partir des choses visibles. Et pourtant, il ne suffit pas de considérer cela dans les miracles du Christ. Demandons aux miracles eux-mêmes ce qu’ils nous disent du Christ ; en effet, si nous les comprenons, ils ont leur langage. Car le Christ en soi est la Parole de Dieu, l’action de la Parole aussi est parole pour nous.
