 Il faut admirer comment une nature corporelle s’est glissée à travers un corps impénétrable : son entrée est invisible, son aspect est visible ; elle est facile à toucher, difficile à comprendre. Finalement, les disciples troublés pensaient qu’ils voyaient un esprit. Et voilà pourquoi le Seigneur voulant nous montrer ce qu’était la résurrection : « Palpez, dit-il, et voyez, car un esprit n’a pas de chair et d’os comme vous voyez que j’en ai. » Ce n’est donc pas en raison d’une nature incorporelle, mais à cause d’une qualité due à la résurrection, qu’il a pénétré sans voie d’accès dans un lieu clos. Car ce qui se touche est corps ; ce qui se palpe est corps.