« Le monde ne peut donc recevoir cet Esprit, parce qu’il ne le voit pas et ne le connaît point ». L’amour mondain est dépourvu de ces yeux invisibles au moyen desquels on peut voir l’Esprit-Saint, qui ne peut être vu que d’une manière invisible. « Mais vous, dit notre Seigneur, vous le connaîtrez, parce qu’il demeurera au milieu de vous et qu’il sera en vous ». Il sera en eux pour y demeurer ; il ne demeurera pas au milieu d’eux pour y être ; car le fait d’être en un lieu est antérieur à celui d’y demeurer. Mais afin que les disciples n’entendissent pas ces paroles : « Il demeurera au milieu de vous », d’un séjour visible, comme celui que fait d’ordinaire un hôte chez celui qui lui donne l’hospitalité, il a expliqué ces paroles : « Il demeurera au milieu de vous », en ajoutant : « Il sera en vous ».