Voyons ce qu’il a voulu signifier dans ce paralytique,
	que par respect pour la mystique de l’unité,
	il a daigné guérir seul, parmi tant de malades.
Il a trouvé dans les années de ce paralytique un chiffre de maladie:
	Il était malade depuis trente-huit ans.
Comment ce nombre a-t-il plus de rapport avec la maladie qu’avec la santé?
	Expliquons cela avec quelque détail.
Soyez attentifs, je vous prie;
	le Seigneur nous aidera, moi, à parler comme il convient,
	vous, à comprendre suffisamment.
Le nombre quarante nous est recommandé
	comme consacré par une certaine perfection.
Je pense que votre charité le sait:
	les divines Écritures l’attestent très souvent,
	vous savez bien que le jeûne est consacré par ce nombre.
Car Moïse a jeûné quarante jours, Élie autant;
	et notre Seigneur et Sauveur Jésus-Christ
		a observé lui-même ce nombre du jeûne.
Moïse représente la Loi, Ëlie représente les prophètes,
	le Seigneur représente l’Évangile.
C’est pourquoi tous trois apparurent sur la montagne
		où Jésus se montra à ses disciples,
	dans le rayonnement éclatant de son visage et de ses vêtements.
Il apparut au milieu, entre Moïse et Élie,
	comme si l’Évangile recevait le témoignage de la Loi et des Prophètes.
