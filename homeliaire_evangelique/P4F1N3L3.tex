Que veulent donc dire ces paroles:
	«Mais je ne vous ai pas dit ces choses dès le commencement,
	parce que j’étais avec vous»,
	si ce n’est que les déclarations qu’il leur fait ici
		au sujet du Saint-Esprit,
	à savoir que celui-ci viendrait à eux
	et rendrait témoignage au moment où ils auraient à souffrir tous ces maux,
	il ne les leur avait pas faites dès le commencement,
	parce qu’il était avec eux?
Ce consolateur ou cet avocat
	(car le mot grec «Paraclet» veut dire l’un et l’autre)
	n’était donc nécessaire qu’après le départ du Christ;
	il ne leur en avait point parlé dès le commencement,
	lorsqu’il était avec eux,
	parce qu’il les consolait lui-même par sa présence.
