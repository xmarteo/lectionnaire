Il faut aussi remarquer qu’on lit dans l’Écriture qu’Élie monta dans un char, de sorte qu’on voit clairement que, purement homme, il avait besoin d’un secours étranger. Ce secours lui fut donné par les Anges et montré par eux, parce qu’il ne pouvait même pas s’élever par lui-même dans le ciel aérien, étant encore chargé de l’infirmité de sa nature. Quant à notre Rédempteur, nous ne lisons pas qu’il fut enlevé dans un char ou par les Anges, car celui qui avait fait toutes choses, qui est certes au-dessus d’elles, s’élevait par sa propre puissance. Il retournait donc là où il était, il revenait là où il n’avait cessé de demeurer, parce qu’au moment où il montait au ciel dans son humanité, il contenait pareillement par sa divinité, et le ciel et la terre.