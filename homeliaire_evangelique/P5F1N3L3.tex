«Jusqu’à présent, dit notre Seigneur,
	vous n’avez rien demandé en mon nom.
	Demandez et vous recevrez, afin que votre joie soit parfaite.»
Cette joie qu’il appelle une joie parfaite,
	n’est pas une joie charnelle, mais une joie spirituelle,
	et quand elle sera si grande qu’on ne pourra plus rien y ajouter,
	alors, sans le moindre doute, elle sera parfaite.
Nous devons donc demander au nom du Christ
	ce qui tend à nous procurer cette joie
	si nous comprenons bien la nature de la grâce divine,
	si nous aspirons vraiment à la vie bienheureuse.
Demander toute autre chose, c’est ne rien demander:
	non pas qu’il n’existe absolument autre chose,
	mais parce qu’en comparaison d’un si grand bien,
	tout ce que l’on peut désirer d’autre n’est rien.
