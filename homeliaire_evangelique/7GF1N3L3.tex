Le Seigneur ne cesse donc en aucun temps temps
	d’envoyer des ouvriers pour cultiver sa vigne,
	c’est-à-dire pour instruire son peuple.
Par les Patriarches d’abord,
	ensuite par les Docteurs de la Loi et les Prophètes
		et enfin par les Apôtres,
	cultivant les mœurs de son peuple, il a travaillé,
	comme par le moyen d’ouvriers, à la culture de sa vigne;
	mais cela n’empêche pas que tous ceux qui, avec une foi droite,
	se sont appliqués et ont exhorté à faire le bien,
	ne puissent être considérés aussi,
		chacun dans sa mesure et à un certain degré,
	comme les ouvriers de cette vigne.
Ceux de la première heure
	ainsi que ceux de la troisième, de la sixième et de la neuvième,
	désignent l’ancien peuple hébreu qui, depuis le commencement du monde,
	s’efforçant, en la personne de ses saints,
		de servir Dieu avec une foi droite,
	n’a pour ainsi dire pas cessé de travailler à la culture de la vigne.
Mais à la onzième heure, les Gentils sont appelés,
	c’est à eux que s’adressent ces paroles:
	Pourquoi êtes-vous ici tout le jour sans rien faire?
