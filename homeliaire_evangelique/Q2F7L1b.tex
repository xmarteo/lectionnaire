Vous voyez que le divin patrimoine est accordé à ceux qui le demandent.
Ne croyez pas qu’il y ait faute chez le père,
	parce qu’il a accédé au désir de son plus jeune fils.
Aucun âge n’est trop faible pour le royaume de Dieu;
	et la foi n’est pas alourdie par le poids des années.
En demandant sa part, le jeune homme se jugeait capable de la gérer;
	et plût à Dieu, cependant, qu’il ne se fût pas éloigné de son père,
	il n’eût pas connu l’inexpérience de son âge.
C’est après qu’il eut déserté la maison paternelle et fut parti à l’étranger,
	qu’il commença à manquer.
On a bien raison de dire qu’il a dissipé son patrimoine,
	celui qui s’est éloigné de l’Église.
