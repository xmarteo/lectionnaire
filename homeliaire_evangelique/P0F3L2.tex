 C’est bien dans notre corps que nous ressusciterons ; car ce corps est semé corps animal et ressuscite corps spirituel ; mais celui-ci est plus subtil, celui-là plus grossier parce qu’encore engagé dans la condition de notre déchéance terrestre. Comment en effet ne serait-ce pas un corps, ce qui garde les marques des blessures, les traces des cicatrices que le Seigneur offrit à palper ? En quoi, non seulement il confirme notre foi, mais il excite notre dévotion, ayant mieux aimé emporter au ciel les blessures reçues pour nous, que les faire disparaître, pour montrer à Dieu le Père le prix de notre affranchissement. C’est ainsi que le Père le met à sa droite, embrassant le trophée de notre salut ; c’est ainsi qu’il nous a montré a les Martyrs avec la couronne de leurs cicatrices.