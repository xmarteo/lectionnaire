Les faits surprenants et merveilleux de la vie de notre Seigneur Jésus-Christ, sont à la fois des œuvres et des paroles, des œuvres, parce que ces faits se sont réellement passés, des paroles, parce qu’ils sont des signes. Si donc nous réfléchissons à la signification de ce miracle, nous verrons que l’aveugle représente, le genre humain. Cette cécité a été chez le premier homme le résultat du péché, et il nous a communiqué à tous, non seulement le germe de la mort, mais encore celui de l’iniquité. Si la cécité est l’infidélité, si l’illumination est la foi, quel est celui que le Christ a trouvé fidèle au moment de sa venue sur la terre, puisque l’Apôtre, né de la race des Prophètes, dit lui-même : « Nous étions autrefois par nature enfants de colère, comme tous les autres. » Si nous étions enfants de colère, nous étions aussi enfants de la vengeance, enfants du châtiment, enfants de la géhenne. Comment l’étions-nous par nature, si ce n’est que, par le péché du premier homme, le vice est passé pour nous comme en nature ? Si le vice est de venu pour nous une seconde nature, tout homme naît aveugle, quant à son âme.
