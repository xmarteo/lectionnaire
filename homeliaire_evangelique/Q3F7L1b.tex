Jésus s’en alla sur la montagne des oliviers, au mont fécond en fruits, au mont de 1’huile, au mont de Fonction. Car où convenait-il que le Christ enseignât, sinon sur le mont des Oliviers ? En effet, le nom de Christ vient de Chrisma ; mais le mot grec Chrisma signifie Onction en latin. Or le Christ nous a oints, parce qu’il a fait de nous des lutteurs contre le diable. Et de grand matin9 il revint dans le temple et tout le peuple vint vers lui ; et s’étant assis il les enseignait, et l’on ne mettait pas la main sur lui, parce qu’il ne daignait pas encore subir sa passion. Remarquez maintenant dans quelles circonstances les ennemis du Seigneur éprouvèrent sa mansuétude.
