Ces paroles du Seigneur: «Je suis la lumière du monde»,
	me semblent claires pour ceux qui ont les yeux
		à l’aide desquels on devient participant de cette lumière;
	mais ceux qui n’ont d’autres yeux que ceux du corps
	s’étonnent que notre Seigneur Jésus-Christ ait dit:
	«Je suis la lumière du monde.»
Peut-être même en est-il qui se disent intérieurement:
	Le Seigneur Jésus serait-il peut-être ce soleil
	qui fixe la durée du jour par l’alternative de son lever et de son coucher?
	Il n’a pas manqué d’hérétiques pour soulever cette opinion.
Les Manichéens ont cru que ce soleil visible aux yeux corporels,
	exposé à nos regards,
	et dont la lumière non seulement brille indifféremment pour tous les hommes,
	mais éclaire même les animaux,
	était le Christ, le Seigneur.
