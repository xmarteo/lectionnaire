Mais quoique les derniers devoirs rendus aux morts aient enlevé toute espérance de vie, et que les corps des défunts gisent déjà près du tombeau, cependant, à la parole de Dieu, les cadavres ressuscitent aussitôt, la voix leur revient, un fils est rendu à sa mère, il est rappelé du tombeau, arraché du sépulcre. Quel est pour toi ce tombeau, sinon les mauvaises habitudes ? Ton tombeau, c’est ta déloyauté ; ton gosier est un sépulcre : « C’est un sépulcre ouvert que leur gosier », d’où sont proférées des paroles de mort. Le Christ te délivre de ce sépulcre ; tu sortiras de ce tombeau si tu écoutes la parole de Dieu. Et s’il est un péché grave que tu ne puisses laver toi-même par les larmes de la pénitence, que l’Église ta mère pleure pour toi, elle qui intervient en faveur de chacun de ses enfants, comme une mère veuve pour son fils unique, car elle est pleine de compassion et éprouve une douleur spirituelle qui lui est propre, lorsqu’elle voit ses enfants entraînés à leur perte par des vices mortels.
