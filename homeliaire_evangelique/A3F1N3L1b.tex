Les paroles qu’on vient de nous lire, mes très chers frères, portent notre attention sur l’humilité de saint Jean. Lui, dont la vertu était si grande qu’on avait pu croire qu’il était le Christ, il préféra demeurer simplement et inébranlablement en son propre rôle et ne pas être vainement élevé dans l’opinion des hommes au-dessus de lui-même. Car il le déclara et ne le nia point ; il le proclama : « Je ne suis pas, moi, le Christ. » En disant : « Je ne le suis pas », il a clairement nié qu’il fût ce qu’il n’était pas ; mais il n’a pas nié être ce qu’il était, afin que, parlant selon la vérité, il devînt membre de celui dont il ne voulait pas usurper fallacieusement le nom. Parce qu’il ne veut pas chercher à prendre le nom de Christ, il est fait membre du Christ. Tandis qu’il s’étudie à reconnaître humblement sa propre faiblesse, il mérite de participer véritablement à la grandeur du Christ.