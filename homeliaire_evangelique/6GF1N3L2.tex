En expliquant ce qu’il venait de dire,
	Jésus-Christ nous a fait connaître qu’il avait parlé dans un sens figuré,
	de manière à nous donner pleine assurance,
	quand notre faiblesse nous découvrirait les figures
		que renferment ses paroles.
Qui, en effet, m’eût jamais cru,
	si j’eusse dit de moi-même que les épines signifient les richesses;
	d’autant que celles-là piquent tandis que celles-ci délectent?
Et néanmoins les richesses sont des épines,
	car elles déchirent l’esprit par les piqûres des soucis qu’elles donnent;
	et lorsqu’elles nous entraînent jusqu’au péché,
	elles nous font pour ainsi dire une blessure sanglante.
Aussi est-ce justement qu’en cet endroit,
	selon le témoignage d’un autre Évangéliste,
	le Seigneur ne les appelle pas simplement richesses,
		mais richesses fallacieuses.
