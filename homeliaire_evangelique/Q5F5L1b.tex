 Quand je réfléchis à la pénitence de Marie-Madeleine, j’ai plus envie de pleurer que de parler. Est-il quelqu’un dont le cœur, fût-il de pierre, ne sera pas attendri par les larmes de cette pécheresse et porté ainsi à imiter son repentir ? Elle considéra ce qu’elle avait fait par le passé et ne voulut point mettre de retard à ce qu’elle ferait pour le réparer. Elle entra dans la salle où les conviés étaient à table, elle vint sans être invitée, et pendant le repas elle offrit aux regards le spectacle de ses larmes. Voyez quelle douleur la consume, elle ne rougit point de pleurer, et cela au milieu d’un festin.