Mes bien-aimés, la lecture de l’Évangile
	qui, par les oreilles de notre corps,
		a sollicité l’écoute intérieure de nos âmes,
	nous appelle à l’intelligence d’un grand mystère.
Nous y parviendrons plus facilement, au souffle de la grâce de Dieu,
	si nous ramenons notre attention aux faits racontés un peu plus haut.
Le Sauveur du genre humain, Jésus-Christ, établissait cette foi
	qui rappelle les impies à la justice et les morts à la vie,
	et qui, par les avertissements de son enseignement
		et par le caractère miraculeux de ses œuvres,
	amenait les disciples à cette conviction
	que le Christ était à la fois le fils unique de Dieu et le fils de l’homme.
Car l’une de ces convictions sans l’autre ne servait pas au salut
	et il y avait égal péril à croire
	que le Seigneur Jésus-Christ fut ou Dieu seulement, sans être homme,
	ou simplement homme, sans être Dieu,
	puisqu’il fallait confesser en même temps l’une et l’autre vérité.
Car à Dieu était unie une véritable humanité,
	de même, à l’homme, la vraie divinité.
