Ce démoniaque, d’après Matthieu, était non seulement muet, mais le récit le dit aussi aveugle, et il fut si bien guéri par le Seigneur qu’il parla et qu’il vit. Trois prodiges ont donc été accomplis en même temps dans un seul homme : l’aveugle voit, le muet parle, le possédé est délivré du démon. Ce qui se fit alors était seulement corporel ; mais cela s’accomplit aussi chaque jour, lors de la conversion des croyants, si bien qu'après l’expulsion du démon, les lumières de la foi apparaissent, et les bouches auparavant muettes s’ouvrent ensuite aux louanges de Dieu. Mais quelques uns de ceux qui étaient là décrièrent : C’est par Bêelzébub, prince des démons3 qu’il chasse les démons. Ceux qui disaient ces choses n’étaient point des gens du peuple, mais les pharisiens et les scribes qui le calomniaient, comme l’attestent les autres Évangélistes.
