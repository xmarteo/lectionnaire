Elle ne se montre pas médiocre, l’envie qui se manifeste ici, puisque, oublieuse de la charité en usage entre concitoyens, elle transforme en haine violente les raisons d’aimer. En même temps cet exemple et ce jugement montrent que tu attendrais vainement le secours de la miséricorde céleste, si tu te montrais jaloux des fruits de la vertu du prochain. Le Seigneur, en effet, méprise les envieux ; et à ceux qui poursuivent de leur jalousie les bienfaits divins chez autrui, il refuse les miracles de sa puissance. En effet, les actes humains du Seigneur sont l’image de ses actes divins ; et ses opérations invisibles nous sont manifestées par celles qui sont visibles.
