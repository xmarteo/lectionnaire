L’Esprit-Saint se voit donc d’une manière invisible ; et s’il n’est pas en nous, nous ne pouvons en avoir la connaissance. C’est ainsi que nous voyons aussi en nous-mêmes notre propre conscience. Nous voyons bien le visage d’un autre, et nous ne pouvons voir le nôtre ; au contraire, nous voyons notre propre conscience, et nous ne voyons pas celle d’autrui. Mais notre conscience ne peut jamais exister qu’en nous-mêmes,~

tandis que l’Esprit-Saint peut être sans nous. Il nous est donné afin qu’il soit aussi en nous ; et, s’il n’est point en nous, il nous est impossible de le voir, de le connaître, comme il doit être vu et connu. Après avoir promis le Saint-Esprit, notre Seigneur ne voulant pas qu’on pût croire qu’il donnerait ce divin Paraclet pour le remplacer lui-même et qu’il cesserait ainsi d’être avec ses disciples, ajouta : « Je ne vous laisserai point orphelins ; je viendrai à vous » [44]. Non content donc de nous avoir rendus les fils adoptifs de son Père, et d’avoir voulu que nous ayons pour Père, par un effet de la grâce, celui qui est son Père par nature, le Fils de Dieu fait preuve encore lui-même à notre égard d’une tendresse en quelque sorte paternelle, lorsqu’il dit : « Je ne vous laisserai point orphelins »