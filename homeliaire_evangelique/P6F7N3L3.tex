L’Esprit Saint se voit donc d’une manière invisible;
	et s’il n’est pas en nous, nous ne pouvons le connaître.
C’est ainsi que nous voyons en nous-mêmes notre propre conscience.
Nous voyons bien le visage d’un autre,
	et nous ne pouvons voir le nôtre;
	au contraire, nous voyons notre propre conscience,
	et nous ne voyons pas celle d’autrui.
Mais notre conscience ne peut jamais exister qu’en nous-mêmes,
	tandis que l’Esprit Saint peut être aussi sans nous.
Il nous est donné afin qu’il soit aussi en nous;
	et, s’il n’est pas en nous,
		il nous est impossible de le voir et de le connaître,
	comme il doit être vu et connu.
Après avoir promis le Saint-Esprit,
	notre Seigneur, ne voulant pas qu’on puisse croire
		qu’il donnerait ce divin Paraclet pour le remplacer lui-même,
	et qu’il cesserait ainsi d’être avec ses disciples,
	ajouta: «Je ne vous laisserai pas orphelins; je viendrai à vous.»
Non content d’avoir fait de nous les fils adoptifs de son Père,
	et d’avoir voulu que nous ayons par la grâce le même Père
		qui est son Père par nature,
	il nous montre lui-même une tendresse en quelque sorte paternelle,
	lorsqu’il dit: «Je ne vous laisserai pas orphelins.»
