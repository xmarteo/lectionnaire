Si donc le Seigneur ne voulait pas même laisser vendre dans le temple les choses qu’il voulait qu'on y offrît, sans doute à cause de la passion d’avarice et de fraude qui est habituellement le péché propre des marchands, quelle n’est pas, pensez-vous, la sévérité de la peine qu’il infligerait à ceux qu’il trouverait là occupés à rire ou à bavarder ou adonnés à quelqu’autre vice ? Car si le Seigneur ne souffre pas qu’on traite dans sa maison des affaires temporelles qui peuvent se traiter licitement ailleurs, combien plus mériteront-ils la colère du del, ceux qui font dans les édifices consacrés à Dieu ce qu’il n’est nulle part permis de faire ? Mais, puisque l’Esprit-Saint est apparu en forme de co- bombe au-dessus du Seigneur, c’est à juste titre que les colombes figurent les dons du Saint-Esprit. Or aujourd’hui, dans le temple de Dieu, qui sont les vendeurs de colombes, sinon ceux qui, dans l’Église, acceptent d’être payés pour l’imposition des mains par laquelle le Saint-Esprit est donné du ciel ?
