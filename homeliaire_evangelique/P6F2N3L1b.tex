«Le Seigneur Jésus, après leur avoir parlé, s’éleva au ciel,
	où il est assis à la droite de Dieu.»
Nous apprenons dans l’Ancien Testament qu’Élie fut enlevé au ciel.
	Mais une chose est le ciel aérien, autre chose le ciel éthéré.
Le ciel aérien est proche de la terre,
	aussi nous disons: les oiseaux du ciel,
	parce que nous les voyons voltiger dans l’air.
Élie a été enlevé dans le ciel aérien
	pour être conduit aussitôt dans quelque région retirée de la terre,
	où il pût vivre déjà dans un grand repos du corps et de l’âme,
	jusqu’à ce qu’il revienne à la fin du monde,
	et paie sa dette à la mort.
Car il a retardé sa mort, il n’y a pas échappé.
Mais notre Rédempteur, lui, l’a vaincue, parce qu’il ne l’a pas retardée;
	il l’a anéantie en ressuscitant,
	et il a fait éclater la gloire de sa Résurrection en montant au ciel.
