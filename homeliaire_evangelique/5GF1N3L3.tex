Il est à noter que l’aveugle est illuminé,
	quand on dit que Jésus approche de Jéricho.
Or Jéricho signifie lune;
	mais dans le langage de la Sainte Écriture,
	la lune représente la déficience de la chair,
	car en décroissant chaque mois,
		elle indique la vie défaillante de notre corps mortel.
Ainsi donc, quand notre Créateur approche de Jéricho,
	l’aveugle revient à la lumière,
	parce qu’au jour où la divinité a pris avec elle notre chair défaillante,
	le genre humain a reçu la lumière qu’il avait perdue.
Car, du fait que Dieu accepte de souffrir l’humain,
	l’homme est élevé jusqu’au divin.
C’est à bon droit que l’aveugle nous est représenté
	assis le long du chemin et mendiant,
	car la Vérité nous dit elle-même: Je suis la voie.
