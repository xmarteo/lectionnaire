Ne peut-on pas résoudre cette difficulté,
	en disant que les autres Évangélistes font observer
		que sa Passion était proche,
	au moment où il parlait ainsi?
Il ne leur avait donc pas dit ces choses dès le commencement,
	lorsqu’il était avec eux,
	puisqu’il ne les leur dit qu’au moment de s’éloigner d’eux
		et de retourner à son Père.
Ainsi donc, même selon ces Évangélistes,
	se trouve confirmée la vérité de ces paroles du Sauveur:
	«Je ne vous ai pas dit ces choses dès le commencement.»
Mais que penser de la véracité de l’évangile selon saint Matthieu,
	qui rapporte que ces prédictions ont été faites par le Seigneur,
	non seulement à la veille de sa Passion,
	lorsqu’il allait célébrer la Pâque avec ses disciples,
	mais dès le commencement,
	à l’endroit où les douze sont désignés par leur nom
	et sont envoyés aux œuvres divines?
