Mais, si il quelqu’un est tombé dans de grands péchés, il doit d’autant plus se retrancher ce qui est permis, qu’il se souvient d’avoir commis des actions défendues. Et, en effet, les fruits des bonnes œuvres ne doivent pas être pareils en celui qui a peu péché, et en celui qui a péché beaucoup ; ou bien en celui qui n’a jamais commis de crimes, celui qui en a commis quelques-uns, et celui qui en a commis un grand nombre. Ces paroles donc : « Faites de dignes fruits de pénitence, » sont un appel à la conscience de chacun, l’invitant à acquérir par la pénitence un trésor de bonnes œuvres d’autant plus grand, qu’il s’est causé de plus grands dommages par le péché.