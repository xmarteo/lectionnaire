Vous avez entendu, frères très chers, dans la lecture de l’Évangile,
	une parole qui vous instruit;
	vous avez appris aussi à quel danger nous sommes exposés.
Celui, en effet, qui est bon,
	non par un don accidentel, mais par l’essence de sa nature,
	vous dit: «Je suis le Bon Pasteur.»
Et nous donnant le modèle de cette bonté, pour que nous l’imitions,
	il ajoute: «Le Bon Pasteur donne sa vie pour ses brebis.»
Il a fait ce qu’il a enseigné;
	il nous donne l’exemple de ce qu’il a ordonné.
Le Bon Pasteur a donné sa vie pour ses brebis,
	afin de convertir en nourriture, dans notre sacrement,
		son Corps et son Sang,
	et d’en rassasier les brebis qu’il avait rachetées.
