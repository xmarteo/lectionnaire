Or, voici les prodiges qui accompagneront ceux qui auront cru : ils chasseront les démons en mon nom ; ils parleront des langues nouvelles ; ils prendront les serpents, et s’ils boivent quelque poison mortel, il ne leur nuira point ; ils imposeront les mains sur les malades et ceux-ci seront guéris. » Quoi, mes frères, parce que vous n’opérez pas ces miracles, ne croyez-vous pas ? Ils étaient nécessaires au commencement de l’Église, car pour faire croître la multitude des croyants, il fallait la nourrir par des miracles ; de même que lorsque nous plantons des arbustes, nous les arrosons jusqu’à ce que nous les voyions avoir bien repris ; mais s’ils ont fixé leurs racines, nous cessons de les arroser. Saint Paul dit à ce propos : « Le don des langues n’est pas un signe pour les fidèles, mais pour les infidèles »