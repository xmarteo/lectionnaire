 « Le Lendemain, une foule nombreuse qui était venue pour la fête, ayant appris que Jésus venait à Jérusalem, prit des rameaux de palmiers, et alla au-devant de lui, criant : Hosanna, béni celui qui vient au nom du Seigneur, comme roi d’Israël. » Les rameaux de palmiers sont les louanges et l’emblème de la victoire : le Seigneur devait en effet vaincre la mort en mourant lui-même, et triompher par le trophée de la croix, du démon, prince de la mort. Selon quelques interprètes qui connaissent la langue hébraïque, Hosanna est une parole de supplication qui exprime plutôt un sentiment du cœur qu’une pensée déterminée ; tels sont les mots qu’on appelle interjections dans la langue latine ; ainsi dans la douleur nous nous écrions : hélas ! ou dans la joie : ah !