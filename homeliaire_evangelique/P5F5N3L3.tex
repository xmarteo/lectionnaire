Écoutons ce que le Sauveur commande à ses disciples,
	après leur avoir reproché leur endurcissement:
	«Allez dans tout l’univers, et prêchez l’Évangile à toute créature.»
Est-ce à dire, mes frères,
	que le saint Évangile dût être annoncé aux choses inanimées,
	ou aux animaux dépourvus de raison,
	et que ce soit à leur sujet que cette parole ait été dite aux disciples:
	«Prêchez à toute créature»?
Mais c’est l’homme qui est désigné ici par ces mots: «toute créature»:
	l’homme a, en effet, quelque chose de toute créature.
L’être lui est commun avec les pierres, la vie avec les arbres,
	la sensibilité avec les animaux, et l’intelligence avec les anges.
Si donc l’homme a quelque chose de commun avec toute créature,
	on peut dire, en quelque sorte, que l’homme est toute créature,
	et par conséquent l’Évangile est prêché à toute créature,
	lorsqu’il est prêché à l’homme seul.
