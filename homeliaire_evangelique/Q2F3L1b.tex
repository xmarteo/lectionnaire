Quoi de plus doux, quoi de plus bénin que la conduite du Seigneur?
	Il est tenté par les pharisiens, leurs embûches sont brisées;
	car, selon le Psalmiste,
		leurs coups sont devenus des flèches de petits enfants.
Mais, néanmoins, pour la dignité du sacerdoce et du rang,[TODO] drôle d'expression "rang"
	il engage le peuple à leur rester soumis,
	en tenant compte, non de leurs actes, mais de leur enseignement.
Or, quand Jésus dit:
	Les scribes et les pharisiens sont assis sur la chaire de Moïse,
	par chaire, il veut dire l’enseignement de la loi.
Et donc ce qui est dit dans le Psaume:
	Il ne s’est point assis dans la chaire de pestilence;
	et ailleurs: Il renversa les chaires de ceux qui vendaient des colombes,
	nous devons le prendre dans le sens d’enseignement.
