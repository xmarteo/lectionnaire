C'est donc à bon droit qu’il est dit en saint Matthieu que le père de famille entoura sa vigne d'une haie ; c’est-à-dire qu’il la fortifia du soutien de la protection divine, afin qu’elle ne fût pas facilement accessible aux incursions des bêtes spirituelles. Et il y creusa un pressoir. Comment entendre ce pressoir, sinon peut-être qu’il y a des Psaumes intitulés pour les pressoirs, pour la raison que dans ceux-ci, les mystères de la passion du Seigneur y bouillonnent à déborder, à la manière du moût en fermentation sous l’action de l’Esprit-Saint ? C’est pourquoi on croyait pris de boisson ceux que l’Esprit-Saint remplissait. Le Seigneur a donc aussi creusé un pressoir dans sa vigne, afin que le fruit intérieur de la grappe mystérieuse y découlât, sous l’action de l’esprit qui s’y infuse.
