Il est manifeste que ces préceptes
		dirigent toute notre intention vers les joies intérieures;
	de peur qu’en cherchant la récompense, nous nous conformions à ce siècle
	et perdions la promesse d’une béatitude
		d’autant plus solide et plus ferme qu’elle est plus intérieure,
	en laquelle Dieu nous a choisis
		pour devenir conformes à l’image de son Fils.
Mais, en ce chapitre, il faut surtout remarquer
	que ce n’est pas seulement dans l’éclat de la pompe des choses corporelles,
	mais même aussi dans le négligé de la tenue de deuil,
	qu’il peut y avoir de la jactance,
	et cela avec d’autant plus de péril
		que l’on se couvre du prétexte du service de Dieu.
