Or, tandis que les hommes dont l’espoir est en Dieu ne doivent pas rendre le mal, même pour le mal ; ceux-ci rendaient le mal pour le bien. C’est pourquoi le Seigneur leur dit cette sentence qu’il connaissait d’avance, à savoir qu’ils mourraient dans leur péché. Ensuite il ajouta : Où je vais, vous ne pouvez venir. Cela, il le dit aussi à ses disciples dans un autre endroit ; et cependant il ne leur dit pas, vous mourrez dans votre péché. Pourquoi donc leur dit-il comme à ceux-ci : Où je vais3 vous ne pouvez venir. Il ne leur enleva pas l’espérance, mais il prédit le délai. Quand, en effet, le Seigneur parlait ainsi aux disciples, ils ne pouvaient point alors venir où lui-même allait, mais ils devaient y venir plus tard ; au contraire, ils n’y viendraient jamais, ceux auxquels il a dit par avance : Vous mourrez dans votre péché.
