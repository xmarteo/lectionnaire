 Il leur dit : Paix à vous. Comme le Père m’a envoyé, moi aussi je vous envoie. C ’est-à-dire, comme le Père, Dieu, m’a envoyé Dieu, moi, homme, je vous envoie hommes. Le Père a envoyé le Fils qui, pour racheter le genre humain, a résolu de s’incarner, et il a voulu que ce Fils vînt au monde pour souffrir ; et cependant il aimait le Fils qu’il a envoyé souffrir. Ceux qu’il a choisis comme Apôtres, le Seigneur en vérité ne les envoie pas aux joies du monde, mais comme luimême a été envoyé, il lesenvoie dans le monde pour souffrir. Donc, puisque le Fils est à la fois aimé du Père, et cependant envoyé à la souffrance ; ainsi les disciples sont à la fois aimés du Seigneur, et cependant envoyés dans le monde pour souffrir. Aussi est-ce en toute vérité qu’il est dit : Comme le Père m'a envoyé, moi aussi je vous envoie. C’est-à-dire : l’amour dont je vous aime, en vous envoyant parmi les pièges des persécuteurs, c’est l’amour dont le Père m’a aimé, en me faisant venir pour supporter la souffrance. 