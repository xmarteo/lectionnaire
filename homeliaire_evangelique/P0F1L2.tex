Mais il nous faut noter
	ce que signifie le fait que l’Ange apparaisse assis à droite.
Que signifie en effet la gauche si ce n’est la vie présente,
	et que signifie la droite si ce n’est la vie éternelle?
Aussi, dans le Cantique des Cantiques, il est écrit:
	Sa gauche est sous ma tête et, de sa droite, il m’enlacera.
Puisque notre Rédempteur avait dépassé la corruption de la vie présente,
	il était donc bien juste que l’Ange qui venait annoncer sa vie éternelle,
	fût assis à droite.
Il apparut vêtu d’une robe blanche,
	parce qu’il était messager des joies de notre fête.
La blancheur du vêtement signale en effet
		le splendide rayonnement de notre solennité.
Dirons-nous la nôtre, ou la sienne?
	Mais pour mieux dire, c’est la sienne et la nôtre.
Cette résurrection de notre Rédempteur fut tout à la fois,
	et notre fête, parce qu’elle nous a ramenés à notre immortalité,
	et la fête des Anges, parce qu’en nous rappelant au ciel,
		elle a complété leur nombre.
