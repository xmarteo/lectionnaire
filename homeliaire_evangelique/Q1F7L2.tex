Donc, pour confirmer la salutaire connaissance de cette foi, le Sauveur avait demandé à ses disciples ce qu’eux-mêmes, au milieu des diverses opinions des autres, croyaient ou pensaient de lui. C’est alors que l’apôtre Pierre, s’élevant par une révélation du Père céleste au-dessus des choses coiporelles et dépassant les choses humaines, vit des yeux de l’esprit le Fils du Dieu vivant et confessa la glohe de la divinité, ne s’arrêtant pas à la considération de la substance de la chair et du sang. Par la sublimité de cette foi, il plut tellement au Seigneur que, gratifié de la félicité de la béatitude, il reçut la fermeté sacrée de la pierre inviolable, fondement sur lequell’Église saurait dominer les puissances de l’enfer et les lois de la mort, et la promesse qu5 en n’ impor te quelle cause, soit pour délier, soit pour lier, ne seraient ratifiées au ciel que les décisions du jugement de Pierre.
