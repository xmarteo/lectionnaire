L’effort des disciples en train de ramer et le vent qui leur était contraire
	symbolisent les labeurs variés de la sainte Église
	qui, au milieu des flots d’un monde hostile
		et des souffles des esprits mauvais,
	s’efforce d’arriver au repos de la patrie céleste,
	comme à la sûre solidité de la côte.
Le texte dit avec raison que la barque était au milieu de la mer,
	et Jésus seul sur le rivage,
	car parfois l’Église est non seulement éprouvée,
	mais encore souillée par les persécutions des Gentils,
	si bien qu’il semblerait que son Rédempteur
		l’a totalement abandonnée pour le moment.
