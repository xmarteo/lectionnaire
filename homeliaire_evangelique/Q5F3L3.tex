 Car il n’est point vrai que le Christ soit dans le chef sans être dans le corps ; il est tout entier dans le chef et dans le corps de son Église. Ce qui donc s’attribue à ses membres, il le faut attribuer à lui-même ; mais tout ce qui lui convient à lui, ne convient pas pour cela à ses membres. Si ses membres n’étaient pas lui-même, il n’aurait pas dit à Saul : « Pourquoi me persécutes-tu ? » Car ce n’était pas lui en personne que Saul persécutait sur la terre : c’étaient ses membres, c’est-à-dire ses fidèles. Il n’a point cependant voulu dire mes saints, mes serviteurs, ou ce qui est plus honorable encore, mes frères ; mais il dit : moi ; c’est-à-dire mes membres, dont je suis le chef.