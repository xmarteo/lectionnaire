 Et puisque notre discours nous y a conduits, considérons par quelle grâce, d’après Jean, les Apôtres ont cru et se sont réjouis alors que, selon Luc, ils sont réprimandés comme incrédules ; d’après Jean, ils auraient reçu l’ordre de rester dans la ville jusqu'à ce qu'ils aient revêtus de la force d'en haut. Il me semble que Jean, en tant qu’Apôtre, a touché aux réalités les plus importantes et les plus élevées, tandis que le récit de Luc est mieux enchaîné et plus humain. Luc a suivi le circuit de l’histoire, Jean a fait un tableau d’ensemble. Et comme nous ne pouvons pas douter de la parole de Jean qui rend témoignage des faits auxquels il a assisté, et dont le témoignage est vrai *, pas plus qu’il n’est juste de porter une accusation de négligence ou de mensonge contre Luc qui a mérité d’être Évangéliste, nous pensons que les deux témoignages sont vrais, et ne se séparent ni par la variété des affirmations ni par la diversité des personnes. Car Luc, après avoir dit que les Apôtres n’ont pas cru, nous montre ensuite qu’ils ont cru ; si nous ne considérons que ses premières affirmations, il y a contradiction, mais dans les suivantes, l’accord est évident.