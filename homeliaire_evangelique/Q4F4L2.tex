Le Seigneur est venu : Qu’a-t-il fait ? Il a voulu attirer notre attention sur un grand mystère. « Il cracha à terre et il fit de la boue avec sa salive », parce que le Verbe s’est fait chair, « et il oignit les yeux de l’aveugle. » Les yeux de cet homme étaient couverts de cette boue, et il ne voyait pas encore. Le Sauveur l’envoya à la piscine qui porte le nom de Siloé. L’Évangéliste a cru devoir nous faire remarquer le nom de cette piscine et il nous dit « qu’on l’interprète par Envoyé. » Vous savez déjà qui a été envoyé. S’il n’avait pas été envoyé, nul d’entre nous n’eût été délivré du péché. L’aveugle lava donc ses yeux dans cette piscine dont le nom signifie Envoyé : il fut baptisé dans le Christ. Si donc le Sauveur l’a baptisé en quelque sorte lorsqu’il lui rendit la vue, on peut dire qu’il l’avait fait catéchumène quand il oignit ses yeux.
