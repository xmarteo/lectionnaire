Le Seigneur, dans le discours qu’il tint à ses disciples après la cène, aux approches de sa passion, alors qu’il allait partir et les quitter quant à sa présence corporelle, quoiqu’il dût néanmoins rester avec tous les siens par sa présence spirituelle jusqu’à la consommation des siècles ; le Seigneur Jésus, dans ce discours, les exhorta à supporter les persécutions des impies, qu’il désignait sous le nom de monde. Du sein de ce même monde il avait tiré ses disciples : il le leur déclara, afin qu’ils sussent que la grâce de Dieu les faisait ce qu’ils se trouvaient être, tandis que leurs vices les avaient rendus ce qu’ils étaient auparavant.