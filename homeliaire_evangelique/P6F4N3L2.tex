Les Anges durent donc paraître vêtus de blanc au moment de l’ascension, parce que celui qui dans sa naissance, apparut Dieu humilié, se montra, dans son ascension, homme glorieusement élevé. Mais nous devons surtout considérer, très chers frères, en cette solennité, qu’en ce jour fut déchiré l’acte écrit de notre condamnation, et modifiée la sentence de notre corruption. Car cette même nature à laquelle il a été dit : « Tu es terre, et tu iras en terre » [41], est allée aujourd’hui au ciel. C’est en considération de cette élévation de notre chair, que le bienheureux Job donna au Sauveur le nom figuratif d’oiseau. Voyant que les Juifs ne comprendraient pas le mystère de son ascension, le bienheureux Job proféra une sentence figurative au sujet de leur infidélité, en disant : « Ce peuple ignora le chemin de l’oiseau »