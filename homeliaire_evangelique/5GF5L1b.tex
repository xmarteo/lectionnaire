Voyons si, au sujet du serviteur du centurion, Matthieu et Luc sont d’accord, Car Matthieu dit : Un centurion s'approcha de lui en disant : Mon serviteur est couché, paralysé, dans ma maison. A cela semble s’opposer ce que dit Luc : Et ayant entendu parler de Jésus, il lui envoya des anciens d'entre les Juifs pour le prier de venir et de guérir son serviteur. Et ceux-ci, arrivés près de Jésus, le priaient avec insistance, lui disant ; « Il est digne que tu lui accordes cette faveur, car il aime notre nation et il nous a bâti lui- même une synagogue. » Jésus s'en allait donc avec eux et comme il n'était plus loin de la maison., le centurion lui envoya des amis lui dire : « Seigneur, ne vous donnez pas tant de peine, car je ne suis pas digne que vous entriez sous mon toit. »
