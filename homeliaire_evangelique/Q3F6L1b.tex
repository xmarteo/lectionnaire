Voici que commencent les mystères.
	En effet, ce n’est pas en vain que Jésus est fatigué.
Car ce n’est pas en vain que la puissance de Dieu est fatiguée.
Et il n’est pas fatigué en vain, celui par qui les fatigués sont recréés;
	il n’est pas non plus fatigué en vain,
	celui dont l’abandon nous fatigue et dont la présence nous raffermit.
Et cependant, Jésus est fatigué; il est fatigué du chemin et il s’assied.
Il s’assied près d’un puits,
	et c’est à la sixième heure du jour que, fatigué, il s’assied.
Toutes ces circonstances signifient quelque chose,
	veulent indiquer quelque chose;
	elles nous rendent attentifs et nous encouragent à frapper.
Qu’il ouvre donc lui-même et à nous et à vous,
	celui qui a daigné y exhorter en disant:
	Frappez et il vous sera ouvert.
