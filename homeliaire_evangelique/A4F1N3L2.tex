Mais, puisque nous avons déjà péché, et que nous sommes enveloppés dans de mauvaises habitudes invétérées, qu’il nous dise ce que nous devons faire pour pouvoir fuir la colère à venir. Le voici : « Faites donc de dignes fruits de pénitence. » Il faut remarquer, dans ces paroles, que l’ami de l’Époux nous avertit de faire, non seulement des fruits de pénitence, mais de dignes fruits de pénitence. Car, c’est autre chose que de faire un fruit de pénitence, et de faire un digne fruit de pénitence. Pour bien parler de ces fruits de pénitence, il faut savoir que quiconque n’a rien commis d’illicite, a le droit d’user des choses licites : et ainsi, en s’exerçant dans les œuvres de piété, il lui est libre d’user, s’il le veut, des choses du monde.