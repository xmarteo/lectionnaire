Ce n’est donc pas chose inconcevable que, pour dire le fait du centurion abordant le Seigneur par l’intermédiaire de ses amis, Matthieu ait pu dire sous une forme abrégée que le vulgaire peut comprendre : Un cênturiôH Rapprocha de lui. Bien plus, il ne faut pas considérer négligemment la profondeur de cette locution mystique du saint Évangile, qui rappelle ce qui est écrit dans le Psaume : Approchez-vous de lui et soyez illuminés. La foi du centurion, qui l’a fait s’approcher de Jésus, a été si hautement louée par le Seigneur qu’il en a dit : Je n9ai pas encore trouvé si grande foi en Israël. De là vient que l’Évangéliste avisé a voulu nous dire que le centurion s’était approché plus près de Jésus que les amis par lesquels il avait envoyé son message.
