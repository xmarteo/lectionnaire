 Mais ce que la Vérité dit des Juifs dignes d’être réprouvés,
	ces hommes condamnables le montrent eux-mêmes par leurs œuvres d’iniquité;
	voici en effet ce qu’on lit après:
	«Les Juifs lui répondirent, et lui dirent:
	Ne disons-nous pas avec raison que tu es un Samaritain,
	et qu’un démon est en toi?»
Écoutez ce que repartit le Seigneur, après avoir reçu un tel outrage:
	«II n’y a pas de démon en moi;
	mais j’honore mon Père, et vous, vous me déshonorez.»
Le mot Samaritain signifie gardien,
	et le Sauveur est véritablement lui-même ce gardien dont le Psalmiste a dit:
	«Si le Seigneur ne garde une cité, inutilement veille celui qui la garde»;
	et ce gardien auquel il est dit dans Isaïe:
	«Garde, où en est la nuit? garde, où en est la nuit?»
Voilà pourquoi le Seigneur ne voulut pas répondre: Je ne suis pas un Samaritain,
	et dit seulement: «II n’y a pas de démon en moi.»
Deux choses lui avaient été reprochées:
	il nia l’une, et convint de l’autre par son silence.
