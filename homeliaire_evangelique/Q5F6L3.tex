 « Mais l’un d’eux, nommé Caïphe, qui était le Pontife de cette année-là, leur dit : Vous n’y entendez rien, et vous ne pensez pas qu’il vous est avantageux qu’un seul homme meure pour le peuple, et non pas que toute la nation périsse. Or, il ne dit pas cela de lui-même ; mais étant le Pontife de cette année-là, il prophétisa. » Nous apprenons ici que même les hommes méchants peuvent, par l’esprit de prophétie, annoncer les choses à venir. Cependant l’Évangéliste attribue ce dernier fait à un mystère tout divin ; car, dit-il, « il était Pontife », c’est-à-dire grand-prêtre.