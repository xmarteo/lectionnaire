Pour faire pénétrer en nous la vérité de la résurrection du Seigneur, il nous faut aussi remarquer ces paroles de saint Luc : « Mangeant avec eux, il leur commanda de ne pas s’éloigner de Jérusalem. » Et un peu plus loin : « Eux le voyant, il s’éleva, et une nuée le déroba à leurs yeux. » Notez ces paroles, remarquez ces mystères. Après avoir mangé avec eux, il s’éleva ; il mangea et il monta, afin de nous rendre manifeste par l’action d’absorber de la nourriture, la réalité de sa chair. Saint Marc rapporte que le Seigneur, avant de monter au ciel, reprocha à ses disciples la dureté de leur cœur et leur incrédulité. Que remarquer en cela, sinon que le Seigneur adressa des reproches à ses disciples au moment où il les quittait corporellement, afin que ces paroles, dites en se séparant d’eux, restassent plus profondément imprimées dans le cœur de ceux qui les entendaient ?