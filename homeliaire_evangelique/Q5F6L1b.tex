Les Pontifes et les Pharisiens délibéraient entre eux,
	mais ils ne disaient pas: Croyons en lui;
	ces hommes pervers étaient bien plus préoccupés
		de la pensée de nuire à Jésus pour le perdre
	que des moyens d’éviter leur propre perte,
	et cependant ils craignaient et se consultaient.
Ils disaient: «Que faisons-nous, car cet homme opère beaucoup de miracles?
	Si nous le laissons ainsi, tous croiront en lui,
	et les Romains viendront et ruineront notre pays et notre nation.»
Ils craignirent de perdre les biens temporels,
	et ils ne songèrent pas aux biens de la vie éternelle:
	c’est ainsi qu’ils perdirent les uns et les autres.
