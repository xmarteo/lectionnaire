Donc, soit dans la Loi, soit dans les Prophètes, soit dans l’Évangile, le nombre quarante est recommandé pour le jeûne. Mais le grand jeûne complet consiste à s’abstenir des iniquités et des plaisirs illicites du monde ; c’est la perfection du jeûne, que renonçant à Vimpiélé et aux cupidités séculières, nous vivions sobrement, justement et pieusement en ce siècle.A ce jeûne, quelle est la récompense qu’attache l’Apôtre ? Il continue en disant : Attendant cette bienheureuse espérance et Vavènement de la gloire de notre bienheureux Dieu et Sauveur Jésus-Christ. Ainsi, dans ce siècle, nous observons la quarantaine de l’abstinence, quand nous vivons bien, dans le renoncement aux iniquités et aux plaisirs illicites ; mais comme cette abstinence ne sera pas sans récompense, nous attendons cette bienheureuse espérance et Vavènement de la gloire de notre grand Dieu et Sauveur Jésus-Chrisî. Dans cette espérance, lorsque cette espérance sera réalisée, nous recevrons le salaire d'un denier ; car c'est le salaire payé aux ouvriers qui travaillent à la vigne, selon l'Évangile dont je pense que vous vous souvenez, car il ne faut pas tout vous rappeler comme à des commençants sans formation. On paye donc un denier qui tire son nom du nombre dix, ce qui, ajouté à quarante, fait cinquante. C’est pourquoi nous célébrons dans la peine la quarantaine avant Pâques mais dans la joie, comme si nous avions déjà reçu notre salaire, la cinquantaine après Pâques.
