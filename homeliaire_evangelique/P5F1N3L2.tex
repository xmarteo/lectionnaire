Aussi celui qui pense de Jésus-Christ
		ce qui ne doit pas être pensé du Fils unique de Dieu
	ne demande pas en son nom,
	même s’il prononce les lettres et les syllabes du nom du Christ;
	car il prie
		au nom de celui qui est présent à sa pensée au moment de sa prière.
Celui, au contraire, qui pense de Jésus-Christ ce qu’il en doit penser,
	celui-là prie en son nom, et reçoit ce qu’il demande,
	si toutefois il ne demande rien de contraire à son salut éternel:
	il reçoit lorsqu’il est bon pour lui qu’il reçoive.
Il est des grâces qui ne nous sont pas refusées,
	mais seulement différées pour un temps plus opportun.
On doit donc entendre que ces paroles: «Il vous donnera»,
	désignent des bienfaits particuliers à ceux qui les demandent.
Tous les saints, en effet, sont exaucés pour eux-mêmes,
	mais ils ne le sont pas nécessairement pour les autres,
	soit amis, soit ennemis, ou tout autre;
	car notre il n’est pas dit: «Il donnera» en général,
	mais: «Il vous donnera».
