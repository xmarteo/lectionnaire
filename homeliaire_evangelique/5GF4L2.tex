Celui-la donc qui se fait remarquer par un culte immodéré du corps et du vêtement ou par l’éclat d’autres choses est facilement convaincu d’être trop attaché aux pompes du siècle et ne trompe personne par un faux air de sainteté. Quant à celui qui, dans sa profession de vie chrétienne, attire l’attention des regards des hommes par le port inusité de vêtements grossiers et souillés, et fait cela volontairement, sans le subir par nécessité, il faut voir, d’après ses autres œuvres, s’il fait cela par mépris de soins superflus, ou par quelque motif d’ambition. Car le Seigneur nous a commandé de prendre garde aux loups, sous la peau de brebis : C'est à leurs fruits, dit-il que vous les reconnaîtrez.
