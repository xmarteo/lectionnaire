Il est remarquable que le Seigneur, ayant laissé les Juifs, monte au temple,
	lui qui devait habiter dans les cœurs des Gentils.
Car le vrai temple c’est celui où le Seigneur est adoré,
	non selon la lettre, mais en esprit.
Le temple de Dieu, c’est celui qui s’établit,
	non sur une structure de pierres,
		mais sur l’enchaînement des vérités de la foi.
Le Seigneur abandonne donc ceux qui le haïssent
	et il choisit ceux qui doivent l’aimer.
Et voilà pourquoi il vient au mont des oliviers
	planter en sa vertu divine
		ces jeunes plants d’olivier qui ont pour mère la Jérusalem d’en haut.
Sur cette montagne, il est lui-même le céleste jardinier,
	pour que tous ceux qui sont plantés dans la maison de Dieu puissent dire,
		chacun en particulier:
	«Pour moi, je suis comme un olivier
		qui porte du fruit dans la maison de Dieu.»
