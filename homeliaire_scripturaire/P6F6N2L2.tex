C’est donc dans ce corps appartenant à notre nature
	que la mort du Christ nous a donné la vie,
	que sa Résurrection nous a relevés,
	que son Ascension nous a consacrés.
C’est en ce corps, d’une origine identique à la nôtre,
	qu’il a placé dans le Royaume des cieux l’arrhe de notre condition humaine.
Travaillons donc, très chers frères,
	afin que, de même que le Seigneur est monté au ciel aujourd’hui
		avec notre chair,
	ainsi, autant que nous le pouvons, nous montions par notre espérance
	pour le suivre de cœur.
Montons après lui par notre affection, par notre avancement dans la vertu,
	et même au moyen de nos vices et de nos passions.
En effet, si chacun de nous s’efforce de les maîtriser,
	s’accoutume à se tenir debout au-dessus d’eux,
	il s’en fera comme un degré pour monter plus haut.
Ils nous élèveront, s’ils restent au-dessous de nous.
