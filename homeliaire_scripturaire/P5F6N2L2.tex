Ainsi, de même que la Résurrection du Seigneur
		a été le sujet de notre joie dans la fête de Pâques,
	son Ascension au ciel est la cause de notre allégresse présente;
	car nous le célébrons et nous le vénérons à juste titre,
	ce jour où la bassesse de notre nature fut élevée, dans le Christ,
	au-dessus de toutes les armées célestes, de tous les ordres des anges,
	plus haut que toutes les puissances, et jusqu’au trône de Dieu le Père.
C’est par cette économie des œuvres divines
	que nous avons été fondés et élevés;
	ainsi la grâce de Dieu fut-elle rendue plus admirable:
	car, quand fut soustraite au regard des hommes
		cette présence qui nous inspirait une juste révérence,
	la foi n’a pas défailli,
	l’espérance n’a pas hésité,
	et la charité ne s’est pas attiédie.
