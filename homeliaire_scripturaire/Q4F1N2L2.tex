 En un moment de temps, à cause de sa gourmandise, ce peuple que les plus grands prodiges avaient instruit du culte à rendre à Dieu, versa de la façon la plus honteuse dans l’idolâtrie des Égyptiens. Si l’on compare ces deux faits, on peut voir que le jeûne conduit à Dieu, tandis que les délices anéantissent le salut. Qu’est-ce qui corrompit Ésaü et le rendit serviteur de son frère ? N’est-ce pas un seul mets pour lequel il vendit son droit d’aînesse ? Et Samuel ? N’est-ce pas au contraire par le jeûne qu’il fut accordé à la prière de sa mère ? Qu’est-ce donc qui a rendu invincible le très fort Samson, sinon le jeûne avec lequel il fut conçu dans le sein de sa mère ? Le jeûne le conçut, le jeûne le nourrit, le jeûne en fit un homme. Un ange l’a sûrement prescrit à sa mère, l’avertissant de s’abstenir de tout ce qui provient de la vigne, de ne boire ni vin ni boisson fermentée. Le jeûne engendre les prophètes, affermit et fortifie les puissants.