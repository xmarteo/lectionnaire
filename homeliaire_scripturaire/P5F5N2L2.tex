Et c’est ainsi que les bienheureux Apôtres et tous les disciples,
	d’abord effrayés de la mort sur la croix
	et fort hésitants dans leur foi à la Résurrection,
	ont été à ce point affermis par l’évidence de la vérité
	qu’à la vue du Seigneur s’en allant dans les hauteurs du ciel,
	non seulement ils n’ont pas éprouvé de tristesse,
	mais ils ont même été remplis de joie.
Et certes bien grand et ineffable était leur motif de se réjouir,
	quand, en présence d’une sainte multitude,
	on voyait la nature humaine
		monter plus haut en dignité que toutes les créatures célestes,
	pour dépasser les ordres angéliques et s’élever au-dessus des archanges.
Elle ne devait connaître de terme aux sublimités de son élévation
	qu’une fois reçue par le Père éternel,
	associée à sa gloire,
	sur le trône de celui dont elle partage la nature en qualité de Fils.
