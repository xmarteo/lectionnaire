Il convient que tout homme se prépare à l’avènement du Sauveur ; de crainte qu’il ne le trouve livré à la gourmandise, ou embarrassé dans les soucis du siècle. Il est prouvé, par une expérience de tous les jours, que la vivacité de l’esprit s’altère par l’excès du boire, et que l’énergie du cœur est affaiblie par une trop grande quantité d’aliments. Le plaisir de manger peut devenir nuisible, même à la santé du corps, si la raison et la tempérance ne le modèrent, ne résistent à l’attrait, et ne retranchent au plaisir ce qui serait superflu.