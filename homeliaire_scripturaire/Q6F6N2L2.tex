 Vous savez quelles étaient ces assemblées de méchants : c’étaient celles des Juifs ; vous connaissez l’iniquité de cette multitude de pécheurs : elle a consisté dans le dessein formé par eux de faire mourir Notre Seigneur Jésus-Christ. « J’ai opéré sous vos yeux un si grand nombre de bonnes oeuvres : pour laquelle voulez-vous me mettre à mort ? » Il avait supporté patiemment les indiscrets empressements de tous leurs malades, guéri tous leurs infirmes, prêché au milieu d’eux la parole de Dieu ; Il avait mis le doigt sur leurs vices pour leur en inspirer la haine et non pour leur faire détester le médecin, qui voulait leur rendre la santé de l’âme : au lieu de Lui témoigner de la reconnaissance pour tant de guérisons, ils se montrèrent ingrats : à les voir s’emporter contre Lui, on eût dit qu’une fièvre violente leur avait ôté le sens et qu’une sorte de rage les animait à l’égard du bienveillant médecin qui était venu apporter un remède à leurs maux : ils formèrent donc le projet de Le perdre, comme s’ils voulaient s’assurer de ce qu’Il était : un homme, comme les autres, sujet à la mort, ou un homme supérieur aux autres et à l’abri des coups du trépas. Le livre de la Sagesse de Salomon a prédit les paroles qu’ils prononcèrent alors : « Condamnons-Le à mourir d’une mort infâme : éprouvons si ce qu’Il a dit est véritable. S’il est le Fils de Dieu, que Dieu le délivre ! »