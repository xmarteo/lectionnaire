Puisque je dois parler de la patience, frères bien-aimés,
	et vous dire ses avantages et ses convenances,
	par où commencerai-je de préférence,
	puisqu’aussi maintenant je vois que la patience
		vous est nécessaire pour écouter,
	et que vous ne pouvez non plus, sans la patience,
	faire ce que vous entendez et apprenez?
Car enfin la parole et le moyen du salut ne se perçoivent efficacement
	que si l’on entend patiemment ce qui est dit.
Et, parmi toutes les voies de la céleste discipline
	qui dirige notre conduite vers l’acquisition des récompenses divines,
	objet de notre espérance et de notre foi,
	je ne trouve rien de plus utile pour la vie,
	ni de meilleur pour obtenir la gloire,
	que de garder la patience avec un soin extrême,
	nous qui nous attachons aux préceptes du Seigneur
		par le service de la crainte et de la dévotion.
Les philosophes païens aussi font profession de pratiquer cette vertu,
	mais leur patience est aussi fausse que leur sagesse.
Car comment pourrait-il être sage ou patient,
	celui qui ne connaît ni la sagesse, ni la patience de Dieu?
