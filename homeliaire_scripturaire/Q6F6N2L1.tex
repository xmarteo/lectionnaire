 « Vous m’avez protégé contre l’assemblée des méchants, contre la multitude de ceux qui commettent l’iniquité ». Ici, portons nos regards sur notre chef. Beaucoup de martyrs ont pu, à juste titre, se plaindre des procédés des méchants et des pécheurs, mais nul d’entre eux n’a eu à souffrir, de leur part, autant que le Sauveur : en considérant ce qu’Il a enduré, nous comprendrons bien mieux ce qu’ils ont supporté. Il a été protégé contre l’assemblée des méchants : Dieu lui accordait son secours ; Il n’a pas Lui-même abandonné Son corps à la volonté perverse des pécheurs : Fils de Dieu incarné, Fils de Dieu et Fils de l’homme tout ensemble, Fils de Dieu à cause de la substance divine qu’Il possédait, Fils de l’homme, à cause de la forme d’esclave dont Il S’était revêtu, Il le protégeait : car Il avait le pouvoir de donner Sa vie et de la reprendre. Quel mal ses ennemis ont-ils pu Lui faire ? Ils ont fait mourir Son corps mais ils n’ont pu faire mourir Son âme. Veuillez remarquer ceci : c’eût été peu pour Lui d’exciter de bouche ses disciples au martyre : il fallait qu’Il leur prêchât d’exemple.