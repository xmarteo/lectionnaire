De plus, Dieu menace d’exterminer l’homme:
	«J’exterminerai, dit-il, depuis l’homme jusqu’aux animaux,
	depuis le reptile jusqu’aux oiseaux du ciel.»
En quoi les créatures dépourvues de raison avaient- elles offensé Dieu?
	Elles n’avaient point péché, mais comme elles étaient faites pour l’homme,
	il était logique que leur destruction suivît celle de l’hcmme
	à cause de qui elles avaient été créées,
	du moment que celui-ci n’existerait plus pour s’en servir.
Dans un sens plus élevé, ce passage nous manifeste ceci,
	que l’homme possède une intelligence capable de raison.
L’homme se définit en effet un animal vivant, mortel et raisonnable.
Quand ce qu’il y a de meilleur en l’homme vient à s’éteindre en lui,
	le sens s’éteint aussi;
	il n’y a plus rien en lui à sauver,
	lorsque le fondement du salut, qui est la vertu, fait défaut.
