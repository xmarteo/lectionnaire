 « Ils ont aiguisé leurs langues comme une épée. Les dents des enfants des hommes sont comme des armes et des flèches : leur langue est comme une épée perçante ». Ce que le Psalmiste dit ailleurs, nous le retrouvons ici : « Il ont aiguisé leur langue comme une épée ». Que les Juifs ne disent pas : Nous n’avons pas fait mourir le Christ. Car s’ils l’ont traduit au tribunal de Pilate, c’était afin de rejeter sur le gouverneur romain l’odieux de la condamnation du Sauveur et de n’être point eux-mêmes accusés. En effet, lorsque Pilate leur dit : « Faites-le vous-mêmes mourir», ils lui firent cette réponse : « Il ne nous est permis de faire mourir personne ». Leur dessein était donc de faire peser sur un seul, sur le juge, toute la responsabilité de leur crime ; mais pouvaient-ils tromper le souverain Juge ? Ce qu’a fait Pilate pèse donc sur lui dans la proportion de la part qu’il a prise à la perpétration du déicide. Mais, si l’on compare sa conduite à celle des Juifs, il est de beaucoup moins coupable qu’eux. Autant que possible, il insista en sa faveur pour le tirer de leurs mains : dans cette intention, il le fit flageller et le présenta tout ensanglanté à leurs regards. En le soumettant au supplice de la flagellation, ce faible juge n’avait certainement pas la volonté de se déclarer contre Jésus et de lui faire du mal : ce qu’il avait en vue, c’était de donner à leur fureur une sorte de satisfaction ; il s’imaginait qu’en le voyant meurtri de la sorte, ils s’adouciraient un peu et se désisteraient de leur projet homicide. Il suivit donc ce plan de conduite, mais s’apercevant qu’ils persévéraient dans leurs idées sanguinaires, il lava ses mains, vous le savez, et il déclara qu’il n’était pour rien dans la condamnation de cet homme et qu’il était innocent de sa mort. Néanmoins, il le condamna. Il agit contre son gré et tout le monde lui impute l’injustice de cette condamnation ; et ceux qui l’ont forcé à rendre l’inique sentence seraient innocents ! Oh ! non, Pilate a prononcé le verdict ; il a donné l’ordre de crucifier Jésus ; il l’a, en quelque sorte, tué de sa main : mais, en réalité, ô Juifs, c’est vous qui lui avez donné le coup de la mort. Et comment lui avez-vous ôté la vie ? De quel instrument vous êtes-vous servi ? Du glaive de votre langue, car vous l’avez aiguisée comme une épée. Et à quel moment avez-vous frappé votre victime ? C’est lorsque vous vous êtes écriés : « Crucifie-le, crucifie-le ! »