 Le jeûne rend les législateurs sages ; il est pour l’âme la meilleure sauvegarde ; pour le corps, un compagnon sûr ; pour les hommes courageux, une protection et une arme ; pour les athlètes et les combattants, un entraînement. En outre, il écarte les tentations, dispose à la piété, habite avec la sobriété, est artisan de la tempérance. Dans la guerre, il apporte le courage ; dans la paix, il apprend la tranquillité. Il sanctifie le nazir, rend le prêtre parfait, car il n’est pas permis d’aborder le sacrifice sans être à jeun, et cela, non seulement aujourd’hui, en cette adoration sacramentelle et véritable de Dieu, mais même dans celle où, en figure, le sacrifice était offert selon la loi. C’est le jeûne qui rendit Élie digne de contempler sa grande vision ; car, après avoir purifié son âme par un jeûne de quarante jours, il mérite, dans une caverne, de voir Dieu autant que cela est permis à un homme. Lorsque Moïse reçut de nouveau la loi, il observa de nouveau un jeûne. Les Ninivites n’auraient échappé en aucune manière à la destruction qui les menaçait s’ils n’avaient jeûné et fait jeûner avec eux jusqu’aux animaux. Dans le désert, par contre, qui vit fléchir ses membres, sinon ceux qui eurent envie de viandes ?