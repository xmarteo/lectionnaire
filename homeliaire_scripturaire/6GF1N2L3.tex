C'est pour condamner les autres hommes et nous manifester la bonté divine,
	que l’Ecriture nous dit que Noé a trouvé grâce devant Dieu.
On voit en même temps que l’homme juste
	n’est point noirci par les crimes des pécheurs,
	puisque Noé, loin de périr,
	est réservé pour être le père de toute une race.
Il est loué, non pas à cause de la noblesse de sa naissance,
	mais à raison du mérite de sa justice et de sa sainteté.
Ce qui fait la race de l’homme de bien, c’est la noblesse de ses vertus.
Les familles humaines sont ennoblies par la noblesse de leur race,
	celle des âmes leur vient de leurs vertus.
	Une famille est illustre par la splendeur de sa race;
	c’est l’éclat des vertus qui, de sa grâce, illustre les âmes.
