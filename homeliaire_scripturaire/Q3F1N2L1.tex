La vie des Saints est pour les autres une règle de vie.
Aussi trouvons-nous une série ordonnée de récits scripturaires
	dont la lecture nous fait connaître Abraham,
		Isaac, Jacob et d’autres justes,
	pour que nous les suivions en marchant sur leurs traces,
	sur cette sorte de sentier d’innocence ouvert par leur vertu.
Ayant souvent parlé de ces saints personnages,
	voici qu’aujourd’hui se présente l’histoire de Joseph
	qui, remarquable par beaucoup d’autres genres de vertus,
	brille spécialement de l’éclat de la chasteté.
Ainsi donc il est juste qu’après avoir appris auprès d’Abraham
		la promptitude indéfectible de sa foi,
	auprès d’Isaac, l’intégrité d’un esprit sincère,
	auprès de Jacob, la singulière patience de son cœur au milieu de ses peines,
	vous passiez de la considération de ces vertus générales
	à celles d’enseignement plus particuliers.
