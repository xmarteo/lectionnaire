Le Prophète ayant dit:
	«Cherchez le Seigneur et soyez fortifiés, ne cessez de chercher sa face»,
	que personne n’ait la présomption
		de croire avoir trouvé tout ce qu’il cherche ;
	de peur que, cessant d’avancer, il ne renonce aussi à approcher.
Parmi toutes les œuvres de Dieu que l’admiration humaine s’épuise à observer,
	en est-il une qui touche notre âme
	et dépasse en même temps la portée de notre intelligence
	comme la passion du Sauveur ?
Pour délivrer le genre humain des liens formés par une prévarication mortelle,
	le Christ cacha la puissance de sa majesté divine
	au démon qui brûlait d’exercer sa rage,
	et ne lui montra que l’infirmité de notre bassesse humaine.
Si cet ennemi cruel et orgueilleux
	avait pu connaître le dessein de la miséricorde de Dieu,
	il aurait plutôt cherché à adoucir les esprits des Juifs,
	qu’à les enflammer d’une haine injuste;
	de crainte de perdre,
		en poursuivant la liberté de celui qui ne lui devait rien,
	ses droits sur tous ceux que le péché avait rendus ses esclaves.
