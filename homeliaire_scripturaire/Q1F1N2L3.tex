 Donc, mes bien-aimés, à l’entrée de ce temps mystique et plus saintement réservé à la purification des âmes et des corps, ayons soin d’obéir aux préceptes apostoliques, en nous débarrassant de toute souillure de la chair et de l’esprit ; réprimons les luttes qui existent entre l’une et l’autre substance et que Pâme, qui est établie sous le gouvernement de Dieu et doit être la directrice de son corps, conquière la dignité de sa souveraineté, afin que, ne donnant de scandale à personne, nous ne soyons pas exposés aux reproches des contradicteurs. Car nous mériterons la censure du blâme des infidèles et les langues impies s’armeront de nos défauts pour injurier la religion, si les mœurs des jeûneurs ne correspondent pas à la pureté d’une parfaite continence. Ce n’est pas en effet dans la seule abstinence de nourriture que consiste la perfection de notre jeûne, et c’est sans fruit qu’on enlève au corps la nourriture, si l’âme ne se détourne pas de l’iniquité.