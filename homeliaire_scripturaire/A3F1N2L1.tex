Nous vous avertissons publiquement, mes très chers frères, et avec une sollicitude pastorale d’observer le jeûne du dixième mois. Le temps où nous sommes et la coutume de notre dévotion nous y engagent. Par ce jeûne, qu’on célèbre lorsque la récolte de tous les fruits de la terre est terminée, on offre à Dieu, qui nous a donné ces fruits, un très juste sacrifice de continence. En effet, que peut-il y avoir de plus utile que le jeûne ? Par son observance, nous nous approchons de Dieu, et, résistant au démon, nous surmontons les attraits des vices.