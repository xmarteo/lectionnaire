Car, bien que, sans l’âme, la chair ne désirerait rien, et que c’est d’elle qu’elle reçoit la sensibilité, comme elle en reçoit le mouvement, il est cependant du devoir de cette âme de refuser certaines choses à la substance matérielle qui lui est assujettie. Par un jugement intérieur, elle doit tenir ses sens extérieurs éloignés de ce qui ne lui convient pas, afin qu’étant presque constamment détachée des désirs corporels, elle puisse vaquer à l’étude de la sagesse divine dans le palais de l’intelligence, où le bruit des sollicitudes terrestres ne se faisant plus entendre, elle se réjouit dans des méditations saintes, à la pensée des délices éternelles.