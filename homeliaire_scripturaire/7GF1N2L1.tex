Dieu avait menacé l’homme de le punir de mort s’il venait à pécher;
il lui avait fait don du libre arbitre,
	mais tout en le gouvernant par son commandement
	et en lui faisant craindre sa ruine.
Il le plaça dans un jardin de délices, qui n’était que l’ombre de la vie
	et d’où il serait monté à un monde meilleur,
	s’il avait conservé la justice.
Exilé de là, après sa faute,
	le premier homme entraîna dans la mort et la réprobation
		tous ses descendants,
	corrompus en sa personne comme dans leur source,
	de telle sorte que toute la race qui devait naître de lui et de son épouse,
	condamnée comme lui après l’avoir porté au péché,
	naissant par la concupiscence chamelle,
	désobéissante, à l’imitation et en punition de la première désobéissance,
	contracterait la faute originelle
	et serait par elle entraînée à travers diverses erreurs et douleurs,
	jusqu’au supplice sans fin, avec les anges infidèles, ses corrupteurs,
	ses maîtres et les compagnons de son malheureux sort.
