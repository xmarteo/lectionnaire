«Un rejeton sortira de la souche de Jessé.»
Jusqu’au début de la vision, ou fardeau de Babylone,
	que vit Isaïe, fils d’Amots,
	toute cette prophétie se rapporte au Christ.
Nous allons l’expliquer par parties,
	de peur que, proposée et discutée à la fois tout entière,
	elle ne jette la confusion dans la mémoire du lecteur.
Les Juifs prétendent que le rejeton et la fleur, sortis de la souche de Jessé,
	désignent le Seigneur lui-même,
	dont la puissance royale serait indiquée par le rejeton,
	et la beauté figurée par la fleur.
