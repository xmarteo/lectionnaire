Ce que que fit Jacob sur le conseil de sa mère,
	de manière, semble-t-il, à tromper son père,
	considérons-le de près et avec foi.
Ce n’est pas un mensonge, mais un mystère.
Si nous appelons cela mensonge,
	toutes les paraboles et les figures qui ont une signification symbolique
	et ne peuvent se prendre à la lettre,
		mais font entendre une chose par une autre,
	devront être appelées mensonges:
	ce qui est tout à fait inadmissible.
Car qui penserait ainsi devrait encore étendre cette injuste condamnation
	à toutes les expressions figurées qui sont si nombreuses.
Ainsi celle-là même qu’on appelle métaphore
	et qui consiste à transposer un mot
		de son sens propre à un sens qui ne l’est pas,
	pourrait, à ce compte, s’appeler un mensonge.
