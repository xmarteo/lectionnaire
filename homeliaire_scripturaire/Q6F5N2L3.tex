 « Car j’ai vu l’injustice et la contradiction dans leur ville ». Remarque au moins le glorieux éclat de sa croix. Cette croix, que ses ennemis ont insultée, a déjà trouvé place au front des rois ; l’événement a déjà prouvé la puissance de Celui qui y a été attaché ; il a dompté le monde, non par les armes, mais par le bois de sa croix. Ses ennemis avaient cru que ce bois était digne de tous les mépris ; ils s’arrêtaient devant lui, secouaient la tête et disaient : « S’il est le Fils de Dieu, qu’il descende de sa croix ». Cloué à ce bois, il tendait les bras à ce peuple incrédule et ennemi. Si celui qui vit de la foi est juste, celui-là est injuste qui n’a pas la foi. Par le mot « injustice », j’entends donc la perfidie. Le Seigneur voyait donc l’iniquité et la contradiction dans la ville ; il étendait ses bras vers un peuple incrédule et ennemi ; il en attendait patiemment le retour au bien et disait : « Mon Père, pardonnez-leur, car ils ne savent ce qu’ils font ».