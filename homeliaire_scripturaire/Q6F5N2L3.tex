«Car j’ai vu l’iniquité et la contradiction dans leur ville.»
Considère la gloire de sa croix.
Cette croix, que ses ennemis ont insultée, est maintenant au front des rois;
	cet effet en a prouvé la puissance:
	elle a soumis le monde, non par les armes, mais par le bois.
Les ennemis avaient cru que ce bois était digne de mépris;
	ils s’arrêtaient devant lui, hochaient la tête et disaient:
	«S’il est le Fils de Dieu, qu’il descende de sa croix.»
Mais lui tendait les bras à ce peuple incrédule et ennemi.
Si le juste est celui qui vit de la foi,
	celui-là est injuste qui n’a pas la foi.
Par le mot «iniquité», il faut comprendre: incrédulité.
Le Seigneur voyait donc l’iniquité et la contradiction dans la ville;
	il étendait ses bras vers un peuple incrédule et ennemi;
	il en attendait patiemment le retour au bien et disait:
	«Mon Père, pardonnez-leur, car ils ne savent ce qu’ils font.»
