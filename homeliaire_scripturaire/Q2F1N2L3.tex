 Cette signification véritable
	ne peut en aucune façon être correctement qualifiée de mensonge,
	puisqu’elle est vraie en fait et encore quant aux paroles.
Car, lorsque le père demanda: «Qui es-tu, mon fils?»
	Celui-ci répondit: «Je suis Esaü ton premier-né.»
Rapportez cela aux deux jumeaux, cela paraîtra un mensonge;
	rapportez-le à ce que le récit de ces actes et paroles devait signifier,
	il faudra reconnaître ici, dans son corps qui est son Église,
	celui qui, parlant de toute cette histoire, dit:
	Oui, vous verrez Abraham et Isaac et Jacob et toits les prophètes,
		dans le royaume de Dieu;
	mais vous, vous vous verrez chassés dehors.
Et: Il en viendra de l’Orient et de l’Occident et du Nord et du Midi
	et ils prendront place au royaume de Dieu.
Et: Voici que les premiers sont les derniers et les derniers sont les premiers.
C’est ainsi que le frère cadet
	a en quelque sorte enlevé la primogéniture à l’aîné
	et se l’est transférée.
