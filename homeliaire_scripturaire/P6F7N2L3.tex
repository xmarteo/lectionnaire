C’est à cet ennemi que vous avez fait profession de renoncer,
	et dans cette profession, ce n’est pas aux hommes,
	mais à Dieu et à ses anges qui l’inscrivaient,
	que vous avez dit: Je renonce.
N’y renoncez pas seulement en paroles, mais encore dans votre conduite;
	non seulement par le son de votre langue, mais par les actes de votre vie;
	non seulement par le bruit des lèvres, mais par le témoignage des œuvres.
Sachez que vous avez entrepris de combattre un ennemi rusé,
	ancien, et qui semble parfois assoupi;
	qu’après votre renoncement il ne trouve plus en vous de ses œuvres,
	qu’il n’ait pas le droit de vous entraîner dans son esclavage.
Tu es en effet pris sur le fait, tu es démasqué, ô chrétien,
	quand tu fais une chose et que tu en professes une autre,
	fidèle de nom, te démentant dans tes œuvres,
	ne gardant pas la foi de ta promesse;
	tantôt entrant à l’église pour y répandre des prières,
	et un moment après dans les spectacles
	pour mêler impudemment ta voix à celle des histrions.
Qu’as-tu de commun avec les pompes du diable,
	auxquelles tu as renoncé?
