C’est la force des grandes âmes et la lumière des cœurs vraiment fidèles,
	de croire sans hésiter des choses que le regard corporel ne peut atteindre,
	et de fixer ses désirs là où l’on ne peut porter la vue.
Mais d’où cette piété naîtrait-elle dans nos cœurs,
	et comment quelqu’un pourrait-il être justifié par la foi,
	si notre salut ne consistait
		que dans ces choses qui se trouvent placées sous nos regards?
Aussi est-ce pour cela qu’au disciple
		qui semblait douter de la résurrection du Christ,
	s’il ne voyait dans sa chair les traces de ses plaies,
	et s’il ne les constatait par la vue et le toucher,
	le Seigneur dit: «Parce que tu m’as vu, tu as cru;
	Heureux ceux qui croient sans avoir vu.»
