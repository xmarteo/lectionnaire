La Résurrection du Seigneur est notre espérance;
	l’Ascension du Seigneur, notre glorification.
	Nous célébrons aujourd’hui la solennité de l’Ascension.
Si donc nous célébrons l’Ascension du Seigneur avec droiture,
	avec fidélité, avec dévotion, avec sainteté et avec piété,
	montons avec lui et tenons en haut nos cœurs.
Mais, en montant,
	gardons-nous de nous enorgueillir et de présumer de nos mérites,
	comme s’ils nous étaient propres.
Nous devons tenir nos cœurs en haut attachés au Seigneur;
	car le cœur en haut, mais non auprès du Seigneur, c’est l’orgueil;
	le cœur en haut près du Seigneur, c’est le refuge.
Voici, mes frères, un fait surprenant:
	Dieu est élevé, tu t’élèves et il fuit loin de toi;
	tu t’humilies et il descend vers toi.
Pourquoi cela?
	C’est que «le Seigneur est élevé, et il regarde ce qui est bas,
	et ce qui est haut, c’est de loin qu’il le connaît».
Il regarde de près ce qui est humble, pour l’attirer à lui,
	et il regarde de loin ce qui s’élève, c’est-à-dire les superbes,
	pour les abaisser.
