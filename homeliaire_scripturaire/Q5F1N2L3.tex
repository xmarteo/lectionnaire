Comme il convient à tout ce corps de vivre pieusement,
	ainsi l’obligation de porter la croix est-elle de tous les temps;
	ce n’est pas sans raison qu’il est conseillé à chacun de porter sa croix,
	car chacun s’en voit chargé
		d’une manière et dans une mesure qui lui sont propres.
La persécution n’est désignée que par un seul mot,
	mais il existe plus d’une cause de combat,
	et il y a ordinairement plus à craindre
		d’un ennemi qui tend des pièges en secret que d’un adversaire déclaré.
Le bienheureux Job,
	qui avait appris que les biens et les maux se succèdent en ce monde,
	disait avec piété et vérité:
	«N’est-ce pas une tentation que la vie de l’homme sur la terre?».
Ce ne sont pas seulement les douleurs et les supplices du corps
		qui assaillent l’âme fidèle,
	car elle est menacée d’une grave maladie,
	encore que tous les membres demeurent parfaitement sains,
	si elle se laisse amollir par les plaisirs des sens.
Mais comme «la chair convoite contre l’esprit, et l’esprit contre la chair»,
	l’âme raisonnable est munie du secours de la croix du Christ,
	et moyennant ce secours,
		elle ne consent pas aux désirs coupables lorsqu’elle est tentée,
	parce qu’elle est transpercée et attachée
		par les clous de la continence et par la crainte de Dieu.
