 Pour quelle raison, en effet, Joseph eût-il mérité d’être préféré à tous les autres, s’il avait offensé ceux qui l’offensaient ou aimé seulement ceux qui l*aimaient ? Car cela, la plupart le font. Mais ce qui est admirable, c’est d’aimer son ennemi, comme le Sauveur l’enseigne. Vraiment admirable est celui qui a fait cela avant l’Évangile, en sorte qu’offensé il a pardonné, attaqué il a oublié, vendu il n’a pas rendu l'injure, mais au contraire a rendu le bienfait pour l'outrage, choses que l’Évangile nous a enseignées à tous et que nous ne pouvons observer. Étudions donc la jalousie des Saints afin d’imiter leur patience ; et apprenons qu'ils n’étaient point d’une nature supérieure à la nôtre, mais plus soumise au devoir, qu’ils n’ont point ignoré les penchants mauvais, mais les ont réprimés. Si donc la flamme de l’envie a touché même les Saints, combien plus faut-il prendre garde qu’elle ne brûle les pécheurs ?