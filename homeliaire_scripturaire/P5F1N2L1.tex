Comme la sagesse de Dieu ne pouvait pas mourir,
	et comme on ne peut ressusciter que si l’on meurt,
	le Verbe a pris une chair mortelle,
	afin de mourir en cette chair sujette au trépas,
	et d’y ressusciter une fois mort.
La résurrection ne pouvait avoir lieu, en effet, qu’au moyen d’un homme,
	puisqu’il est dit: «Par un homme, la mort;
	par un homme aussi, la résurrection des morts.»
Jésus-Christ donc est ressuscité en tant qu’homme,
	parce qu’il est mort en tant qu’homme:
	il est tout ensemble, et homme ressuscité et Dieu ressuscitant;
	il s’est alors montré homme en ce qui regarde la chair,
	il se montre maintenant Dieu en toutes choses,
	car nous ne le connaissons plus tel qu’il était selon la chair;
	mais sa chair est cause que nous le connaissons
	comme prémices de ceux qui ont fermé les yeux,
	comme premier-né d’entre les morts.
