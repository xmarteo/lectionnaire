Nous n’ignorons pas, mes bien-aimés,
	que le mystère pascal occupe le premier rang
		parmi toutes les solennités chrétiennes.
Notre manière de vivre durant l’année tout entière
	doit, il est vrai, par la réforme de nos mœurs,
	nous disposer à le célébrer d’une manière digne et convenable;
	mais les jours présents exigent au plus haut degré notre dévotion,
	car nous savons qu’ils sont proches de celui où nous célébrons
		le mystère très sublime de la divine miséricorde.
C’est avec raison et par l’inspiration de l’Esprit Saint,
	que les saints Apôtres ont ordonné pour ces jours des jeûnes plus austères,
	afin que par une participation commune à la croix du Christ,
	nous fassions, nous aussi, quelque chose qui nous unisse
		à ce qu’il a fait pour nous.
Comme le dit l’Apôtre:
	«Si nous souffrons avec lui, nous serons glorifiés avec lui.»
Là où il y a participation à la passion du Seigneur,
	on peut regarder comme certaine et assurée
		l’attente du bonheur qu’il a promis.
