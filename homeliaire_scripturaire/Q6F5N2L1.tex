«Mon Dieu, écoutez ma prière et ne méprisez pas ma demande:
	soyez attentif à me secourir et exaucez-moi.»
Ces paroles sont celles d’un homme affligé,
	accablé d’ennuis et de tribulations.
Livré à une épreuve pénible, brûlé du désir d’en être délivré,
	il a recours à la prière.
Il nous reste maintenant à apprendre en quels maux il se trouve plongé;
	et, quand il nous l’aura dit,
	nous devrons reconnaître que nous avons part à son affliction:
	unis dans la souffrance, nous le serons aussi dans la prière.
«J’ai été contristé dans mon épreuve et je suis troublé.»
Contristé, troublé, en quoi? «Dans mon épreuve.»
Il mentionne les méchants qui le font souffrir,
	et cette souffrance, voilà son épreuve.
Ne vous imaginez pas que les méchants sont inutiles en ce monde
	et que Dieu n’en fait rien de bon.
Il accorde la vie aux méchants,
	soit pour leur donner le temps de se convertir,
	soit afin de les faire servir à éprouver les bons.
