 « Mon Dieu, écoutez ma prière et ne méprisez pas ma demande : soyez attentifs à me secourir et exaucez-moi ». Ces paroles sont celles d’un homme affligé, accablé d’ennuis et de tribulations. Livré à une épreuve pénible, brûlé du désir d’en être délivré, il a recours à la prière. Il nous reste maintenant à apprendre en quels maux il se trouve plongé ; et, quand il nous l’aura dit, nous devrons reconnaître que nous avons part à son affliction : unis dans la souffrance, nous le serons aussi dans la prière. « Je suis affligé dans mon exercice et je suis troublé ». Affligé, troublé, en quoi ? « Dans mon exercice » Il va parler des méchants qui le font souffrir et des épreuves qu’ils lui font subir : voilà son exercice. Ne vous imaginez point que les méchants sont inutiles en ce monde et que Dieu ne les emploie pas à opérer le bien. Il accorde la vie aux méchants, soit pour leur donner le temps de se convertir, soit afin de les faire servir à éprouver les bons.