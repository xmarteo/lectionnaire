 Nous savons que Moïse gravit la montagne grâce au jeûne, car il n’aurait pas osé approcher de ce sommet fumant et entrer dans la nuée, s’il n’avait été fortifié par le jeûne. C’est grâce au jeûne qu’il reçut les lois gravées par le doigt de Dieu sur des tables. De même, sur la montagne, le jeûne obtint le don de la loi, mais au pied de cette montagne, la gourmandise fit tomber le peuple dans l’idolâtrie et le souilla de péché. « La foule s’assit pour manger et pour boire ; puis ils se levèrent pour se divertir. » L’effort et la persévérance des quarante jours que le serviteur de Dieu avait passés dans la prière et le jeûne continuels, une seule ivresse du peuple les rendit inutiles et vains. Ces tables, en effet, gravées par le doigt de Dieu, que le jeûne avait accueillies, l’ivresse les brisa : le très saint prophète jugea qu’un peuple plein de vin était indigne de recevoir de Dieu une loi.